% パッケージの読み込み
\usepackage{luatexja}
\usepackage{luatexja-fontspec}
\usepackage{auxhook}
\usepackage{graphicx}
\usepackage{adjustbox}

\usepackage{cite}


%括弧
\usepackage{delimseasy}

%二段組
\usepackage{multicol}
\setlength{\columnseprule}{.5pt} %中央の線
% --- 日本語フォント -------------------------------------------------
\setmainjfont[
  YokoFeatures={JFM=jlreq},
  TateFeatures={JFM=jlreqv},
  BoldFont={Hiragino Kaku Gothic ProN W6},
]{Hiragino Kaku Gothic ProN}

\setsansjfont[
  YokoFeatures={JFM=jlreq},
  TateFeatures={JFM=jlreqv},
  BoldFont={Hiragino Kaku Gothic ProN W6},
]{Hiragino Kaku Gothic ProN}



% 英文フォントの設定
\setmainfont{Latin Modern Roman}[%
  BoldFont=lmroman10-bold,%
  BoldItalicFont=lmroman10-bolditalic,%
  ItalicFont=lmromanslant10-regular,%
  SmallCapsFont=lmromancaps10-regular,%
  SlantedFont=lmromanslant10-regular,%
  FontFace={sb}{n}{lmromandemi10-regular},%
  OpticalSize=0
]
\setsansfont{Latin Modern Sans}[%
  BoldFont=lmsans10-bold,%
  ItalicFont=lmsans10-oblique%
]
\setmonofont{Latin Modern Mono}[%
  BoldFont=lmmonolt10-bold,%
  ItalicFont=lmmono10-italic%
]

% ヘッダーフォントの設定
\renewcommand{\headfont}{\sffamily\bfseries}
\renewcommand{\familydefault}{\sfdefault}


% 色を定義

\AtEndPreamble{
  %ハイパーリンク用
  \usepackage{url}
  \usepackage{hyperref}
  \usepackage{xcolor}
  \definecolor{BlueViolet}{RGB}{105,39,255}
  \definecolor{mylightgray}{HTML}{DDDDDD}
  \definecolor{mydarkgray}{HTML}{777777}
  \hypersetup{
    colorlinks=true,
    citecolor=BlueViolet,
    linkcolor=blue!50!black,
    urlcolor=blue!70!black,
  }
  \usepackage{bookmark}
}




%数式
\usepackage{nccmath,amsmath,amssymb}
\usepackage{mathtools}
\usepackage{empheq} %数式の囲いに使う
\usepackage{bm}
\usepackage[bbsets]{jkmath} %\Nなどをつかえる
\usepackage{amsthm}
\usepackage{color}


%箇条書き
\usepackage[shortlabels]{enumitem}
\setlist[description]{font={\bfseries\sffamily}}


%tcolorbox系
\usepackage[many]{tcolorbox}
\tcbuselibrary{breakable,skins}
\newtcolorbox{hosoibox}[1]{colframe=black,colback=white,coltitle=black,colbacktitle=white,boxrule=0.5pt,arc=0mm,enhanced,attach boxed title to top left={xshift=10mm,yshift=-3mm},boxed title style={frame hidden},title=#1}

\usepackage{framed}                 % ← leftbar と同系列の空きを得る
\newtcolorbox{tbleftline}[1][]{%
  breakable,
  enhanced,            % 内部計算を精密化
  blanker,             % 背景塗りつぶし無し
  left=5mm,
  borderline west={1.1mm}{0pt}{mydarkgray},
  lines before break=0,% ← ここが最重要
  pad at break*=0mm,   % 分割前後余白をゼロ
  toprule at break=0pt,
  bottomrule at break=0pt,
  enlargepage flexible=\baselineskip,
  ignore nobreak=false,% 見出し直後の孤立を防ぐ
  before skip=\topsep,
  after skip=\topsep,
  #1}

\newenvironment{tleftbar}{\begin{tbleftline}\setlength{\parindent}{1\zw}}%
    {\end{tbleftline}}


% leftbar環境の色を変更
\makeatletter
\renewenvironment{leftbar}{%
  \def\FrameCommand{%
    {\color{mydarkgray}\vrule width 3pt}%
    \hspace{10pt}%
    \fboxsep=\FrameSep\colorbox{white}}%
  \MakeFramed{\hsize\hsize\advance\hsize-\width\FrameRestore}%
}{%
  \endMakeFramed%
}
\makeatother
\usepackage{needspace}   % ← 既に読み込んでいなければ追加
% ---------- preamble.tex に追記 ----------
\usepackage{etoolbox}

% -------- 見出しをパッチする --------
\makeatletter
% ① 見出しがはみ出さない程度だけ前もって確保
\pretocmd{\section}{\Needspace{3\baselineskip}}{}{}
\pretocmd{\subsection}{\Needspace{2\baselineskip}}{}{}

% ② 見出し後に 0 幅の空行を入れて \@afterheading を発火させる
\pretocmd{\section}{\leavevmode\par}{}{}
\pretocmd{\subsection}{\leavevmode\par}{}{}
\makeatother


%% ---- ここだけ差し替え ----
%   見出し行 + tproof 先頭行 = 2 行確保で十分
\AtBeginEnvironment{tproof}{\Needspace{1\baselineskip}}
\AtBeginEnvironment{tanswer}{\Needspace{1\baselineskip}}
%% --------------------------

% さらに proof 環境内の固定 6pt を “伸縮 6pt±2pt” にする
\newlength{\elasticproofsep}
\setlength{\elasticproofsep}{6pt plus 2pt minus 2pt}
\pretocmd{\tproof}{\setlength{\topsep}{\elasticproofsep}}{}{}
\pretocmd{\tanswer}{\setlength{\topsep}{\elasticproofsep}}{}{}

\makeatletter
% ---- \part の後スキップを 1.8 行・非伸縮に ----
\patchcmd{\@endpart}%
{3.5ex plus 1ex minus .2ex}{1.8\baselineskip}{}{}

% ---- 「上位→下位」のダブルスキップを防ぐ ----------------
% \@afterheading の \addvspace<伸縮グルー> を 0.8行固定に
\patchcmd{\@afterheading}{\addvspace\@tempskipa}{%
  \vspace{.8\baselineskip}}{}{}
\makeatother



\definecolor{mypurple}{HTML}{9900FF}

\definecolor{applePaper}{HTML}{F5F5F7}
\definecolor{appleInk}{HTML}{1D1D1F}
\definecolor{appleLine}{HTML}{D1D1D6}
\definecolor{appleCard}{HTML}{FFFFFF}


\tcbset{
  appleMonoBase/.style={
      enhanced, breakable,
      colback=appleCard, colframe=appleLine, coltitle=appleInk,
      fonttitle=\sffamily\bfseries,
      boxrule=0.5pt, arc=3pt,
      left=8pt,right=8pt,top=6pt,bottom=6pt,
      attach boxed title to top left={xshift=8pt,yshift=-3pt},
      boxed title style={size=small,interior engine=empty},
      drop shadow={black!6!applePaper}
    }
}



\newtcbtheorem
[auto counter, number within=subsection]%
{theorem}{Theorem}% ←英文タイトル
{appleMonoBase,%
  title={\thetcbcounter.\, #2}}% ←タイトル行 1 行で書く
{th}
% 定理環境の定義% ─── 定理ボックス ───────────────────────────────────
\newtcbtheorem
[use counter from=theorem]%
{proposition}{Proposition}% ←英文タイトル
{appleMonoBase,%
  title={\thetcbcounter.\, #2}}% ←タイトル行 1 行で書く
{pr}
\newtcbtheorem
[use counter from=theorem]%
{corollary}{Corollary}% ←英文タイトル
{appleMonoBase,%
  title={\thetcbcounter.\, #2}}% ←タイトル行 1 行で書く
{co}
\newtcbtheorem
[use counter from=theorem]%
{definition}{Definition}% ←英文タイトル
{appleMonoBase,%
  title={\thetcbcounter.\, #2}}% ←タイトル行 1 行で書く
{de}
\newtcbtheorem
[use counter from=theorem]%
{lemma}{Lemma}% ←英文タイトル
{appleMonoBase,%
  title={\thetcbcounter.\, #2}}% ←タイトル行 1 行で書く
{le}
\newtcbtheorem
[use counter from=theorem]%
{example}{Example}% ←英文タイトル
{appleMonoBase,%
  title={\thetcbcounter.\, #2}}% ←タイトル行 1 行で書く
{ex}


% 参照用コマンド
\newcommand{\thref}[1]{{\sffamily\bfseries Theorem\,\ref{th:#1}}}
\newcommand{\prref}[1]{{\sffamily\bfseries Proposition\,\ref{pr:#1}}}
\newcommand{\coref}[1]{{\sffamily\bfseries Corollary\,\ref{co:#1}}}
\newcommand{\deref}[1]{{\sffamily\bfseries Definition\,\ref{de:#1}}}
\newcommand{\leref}[1]{{\sffamily\bfseries Lemma\,\ref{le:#1}}}
\newcommand{\exref}[1]{{\sffamily\bfseries Example\,\ref{ex:#1}}}


%コラム環境
\colorlet{colexam}{lightgray!60!black}

\newtcolorbox[auto counter,number format=\Roman]{column}{
  empty,
  title={\bfseries\sffamily Column \thetcbcounter}, % カウンタをローマ数字で表示
  attach boxed title to top left,
  boxed title style={
      empty,
      size=minimal,
      toprule=2pt,
      top=4pt,
      left=1cm, % タイトルを右に移動
      overlay={
          % タイトルの上の線を削除
          % \draw[colexam,line width=2pt]
          %   ([yshift=-1pt]frame.north west) -- ([yshift=-1pt]frame.north east);
        }
    },
  coltitle=colexam,
  fonttitle=\Large\bfseries,
  before=\par\medskip\noindent,
  parbox=false,
  boxsep=0pt,
  left=5mm, % 左のマージンを増やしてタイトルを右に移動
  right=3mm,
  top=4pt,
  breakable,
  pad at break*=0mm,
  vfill before first,
  overlay unbroken={
      \draw[colexam,line width=2pt]
      ([xshift=-0.5pt,yshift=10pt]frame.north east)
      -- ([xshift=-0.5pt]frame.south east);
      \draw[colexam,line width=2pt]
      ([xshift=-1pt,yshift=10pt]frame.north west)
      -- ([xshift=-1pt]frame.south west);
    },
  overlay first={
      \draw[colexam,line width=2pt]
      ([xshift=-0.5pt,yshift=10pt]frame.north east)
      -- ([xshift=-0.5pt]frame.south east);
      \draw[colexam,line width=2pt]
      ([xshift=-1pt,yshift=10pt]frame.north west)
      -- ([xshift=-1pt]frame.south west);
    },
  overlay middle={
      \draw[colexam,line width=2pt]
      ([xshift=-0.5pt,yshift=10pt]frame.north east)
      -- ([xshift=-0.5pt]frame.south east);
      \draw[colexam,line width=2pt]
      ([xshift=-1pt,yshift=10pt]frame.north west)
      -- ([xshift=-1pt]frame.south west);
    },
  overlay last={
      \draw[colexam,line width=2pt]
      ([xshift=-0.5pt,yshift=10pt]frame.north east)
      -- ([xshift=-0.5pt]frame.south east);
      \draw[colexam,line width=2pt]
      ([xshift=-1pt,yshift=10pt]frame.north west)
      -- ([xshift=-1pt]frame.south west);
    },%
}


%題名付き四角
\usepackage{ascmac}
\usepackage{fancybox}

%図に使うもの
\usepackage{tikz}
\usetikzlibrary{intersections,calc,arrows.meta}
\usepackage{tikz-3dplot}
\usepackage[
  % ---- 共通 ----
  marginparwidth = 0pt,
  % ---- 上下左右 ----
  top    = 25truemm,   %  ← ここは現状維持
  bottom = 25truemm,   %  ← 欲しい下余白に調整
  left   = 25truemm,
  right  = 25truemm
]{geometry}

\usepackage{bxpapersize}
\usepackage[absolute,overlay]{textpos} %図の配置を好きにする


%画像
\usepackage{wrapfig}
%footnoteの変更
\renewcommand\thefootnote{{\dag}\arabic{footnote}}
\renewcommand{\thempfootnote}{{\dag}\arabic{mpfootnote}}
\interfootnotelinepenalty=10000

\usepackage{oubraces} %overunderbraces

%underbraceの文字数が多いときのためのadunderbrace
\usepackage{ifthen}
\newlength{\wdTempA}
\newlength{\wdTempB}
\newcommand{\adunderbrace}[2]{%
  \settowidth{\wdTempA}{$#1$}%
  \settowidth{\wdTempB}{${\scriptstyle #2}$}%
  \ifthenelse{\wdTempA<\wdTempB}{%
    \hspace*{.5\wdTempA}\hspace*{-.5\wdTempB}%
    \underbrace{#1}_{#2}%
    \hspace*{.5\wdTempA}\hspace*{-.5\wdTempB}%
  }{%
    \underbrace{#1}_{#2}%
  }%
}%
%丸付き文字
\newcommand{\ctext}[1]{\raise0.2ex\hbox{\textcircled{\scriptsize{#1}}}}

\setlength{\abovedisplayskip}{5pt}
\setlength{\belowdisplayskip}{3pt}
%ユーザー定義
\newcommand{\dash}[1]{#1^\prime}
\newcommand{\ddash}[1]{#1^{\prime\prime}}
\newcommand{\dddash}[1]{#1^{\prime\prime\prime}}
\newcommand{\hodash}[2]{#2^{(#1)}}
\renewcommand{\labelenumi}{(\arabic{enumi})}%itemを(数字)に変更
\newcommand{\two}{I\hspace{-1.2pt}I}
\newcommand{\three}{I\hspace{-1.2pt}I\hspace{-1.2pt}I}
\renewcommand{\proofname}{証明}
\DeclareMathOperator{\Ker}{Ker}
\DeclareMathOperator{\sgn}{sgn}


\renewcommand{\leq}{\leqq}
\renewcommand{\geq}{\geqq}
\renewcommand{\le}{\leqq}
\renewcommand{\ge}{\geqq}

\newcommand{\Laplacian}{{\mathop{}\!\mathbin\bigtriangleup}}


\newcommand{\cmd}[1]{\texttt{\symbol{"5C}#1}}% 《》囲みコマンド(\kakko)をシンプル囲みに変更
\newcommand\kakko[1]{\noindent{\setlength{\fboxsep}{3.5pt}\colorbox{gray!25}{\textbf{#1}}}}

\newcommand{\pH}{\ensuremath{\mathrm{pH}}}
%%%〈amsthm 読み込み後〉%%%
\makeatletter
\newlength{\proofindent}
\setlength{\proofindent}{1\zw}   % 好きな字下げ幅

\renewenvironment{proof}[1][\proofname]{%
  \par\pushQED{\qed}%
  \normalfont
  \topsep6\p@\@plus6\p@      % 見出し前後の間隔
  \trivlist
  %── 見出し「証明」(ピリオド無し) ──
  \item[\hskip\labelsep\itshape #1]%
  \setlength{\parindent}{\proofindent}% 本文の字下げ量
  \leavevmode\par                  % ★空の 1 段落を作る
  \ignorespaces                    %   → 次行から本文
}{%
  \popQED\endtrivlist\@endpefalse
}
\makeatother
% ── 「証明」を左バー付きボックスに包む環境 ──────────────────
% 使い方:  \begin{tproof}[任意の見出し] ... \end{tproof}
%          見出しを省略すると従来の \proofname(灰色ラベル「証明」)になる
\newenvironment{tproof}[1][\proofname]{%
  \begin{tleftbar}% 左バー付きボックス
    \begin{proof}[{%
            \normalfont                 %  ← ここでイタリックを解除
            \noindent
            \setlength{\fboxsep}{3.5pt}%
            \colorbox{gray!25}{\sffamily\bfseries #1}%
          }]%
      }{%
    \end{proof}% □ を出力
  \end{tleftbar}%
}


% ──────────────────────────────────────────────
% 「解答」ラベル(tanswer 用)
\newcommand{\answername}{解答}

% ── 左バー付き「解答」環境 ─────────────────────
% 使い方:\begin{tanswer}[任意の見出し] ... \end{tanswer}
%        見出し省略時は \answername(灰色ラベル「解答」)
\newenvironment{tanswer}[1][\answername]{%
  \begin{tleftbar}% ← 左バー付きボックス開始
    \begingroup
    \renewcommand{\qed}{}%  □ マークを無効化
    \begin{proof}[{%
            \normalfont                 %  ← ここでイタリックを解除
            \noindent
            \setlength{\fboxsep}{3.5pt}%
            \colorbox{gray!25}{\sffamily\bfseries #1}%
          }]%
      }{%
    \end{proof}%          ← proof 終了(□ は出ない)
    \endgroup
  \end{tleftbar}%        ← ボックスを閉じる
}



%増減表関連
\newcommand{\ner}{
  \begin{tikzpicture}[scale=0.3,baseline=0.3]
    \draw[->,>=stealth] (0,0) to[bend right=45] (1,1);
  \end{tikzpicture}
}

\newcommand{\nel}{
  \begin{tikzpicture}[scale=0.3,baseline=0.3]
    \draw[->,>=stealth] (0,0) to[bend left=45] (1.2,1);
  \end{tikzpicture}
}

\newcommand{\sel}{
  \begin{tikzpicture}[scale=0.3,baseline=0.3]
    \draw[->,>=stealth] (0,1) to[bend left=45] (1,0);
  \end{tikzpicture}
}

\newcommand{\ser}{
  \begin{tikzpicture}[scale=0.3,baseline=0.3]
    \draw[->,>=stealth] (0,1) to[bend right=45] (1.2,0);
  \end{tikzpicture}
}
\usepackage[pagecolor=white,nopagecolor={none}]{pagecolor} % 背景色を変更するためのパッケージ

\newcommand{\tituloum}[5]{\begin{titlepage}
    \begin{center}
      \pagecolor{white} % 背景色をBlueVioletに設定
      \color{black} % テキストカラーを白に設定

      \vspace*{2\baselineskip}

      \rule{\textwidth}{1.6pt}\vspace*{-\baselineskip}\vspace*{2pt}
      \rule{\textwidth}{0.4pt}

      \vspace{0.75\baselineskip}

      {\huge #1}

      \vspace{0.75\baselineskip}

      \rule{\textwidth}{0.4pt}\vspace*{-\baselineskip}\vspace{3.2pt}
      \rule{\textwidth}{1.6pt}

      \vspace{2\baselineskip}

      #3

      \vspace*{3\baselineskip}


      {\huge #2}

      \vspace{0.5\baselineskip}

      \textit{#4}

      \vfill

      \vspace{0.3\baselineskip}

      #5

    \end{center}
  \end{titlepage}}

\everymath{\displaystyle}

\newcommand{\HRule}[1]{\rule{\linewidth}{#1}}


\AtBeginDocument{\RenewCommandCopy\qty\SI}