

\part*{第5章}
\addcontentsline{toc}{part}{\texorpdfstring{第5章}{第5章}}


\section*{p139:問}
\addcontentsline{toc}{section}{\texorpdfstring{p139:問}{p139:問}}

\kakko{前半}

\begin{tproof}
  $T$をエルミート変換とする.また,$ \lambda \in \mathbb{K}$を$T$の固有値,$\bm{x} \ne \bm{o}$を$T$の固有値$\lambda$に対する固有ベクトルとする.

  このとき,
  \begin{alignat*}{2}
    \lambda (\bm{x},\bm{x}) & = (\lambda \bm{x},\bm{x})            & \quad & \text{(内積の定義)}       \\
                            & = (T \bm{x},\bm{x})                  &       & \text{(固有値の定義)}      \\
                            & = ( \bm{x},T^{\ast} \bm{x})          &       & \text{(p139~式(1))}   \\
                            & = ( \bm{x},T \bm{x})                 &       & \text{($T$はエルミート変換)} \\
                            & = (\bm{x},\lambda \bm{x})            &       & \text{(固有値の定義)}      \\
                            & = \overline{\lambda} (\bm{x},\bm{x}) &       & \text{(内積の定義)}
  \end{alignat*}
  となり,$\bm{x} \ne \bm{o}$より$\lambda = \overline{\lambda}$であるから$ \lambda \in \mathbb{R}$である.
\end{tproof}

\kakko{後半}
\begin{tproof}
  $T$をユニタリ変換とする.また,$ \lambda \in \mathbb{K}$を$T$の固有値,$\bm{x} \ne \bm{o}$を$T$の固有値$\lambda$に対する固有ベクトルとする.

  このとき,
  \begin{alignat*}{2}
    \lambda \overline{\lambda} (\bm{x},\bm{x}) & = (\lambda \bm{x},\lambda \bm{x}) & \quad & \text{(内積の定義)}      \\
                                               & = (T \bm{x},T \bm{x})             &       & \text{(固有値の定義)}     \\
                                               & = (\bm{x},\bm{x})                 &       & \text{($T$はユニタリ変換)}
  \end{alignat*}
  となり,$\bm{x} \ne \bm{o}$より$\abs{\lambda} = 1$である.
\end{tproof}