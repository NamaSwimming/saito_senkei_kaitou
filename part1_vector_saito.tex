
\part*{第1章}
\addcontentsline{toc}{part}{\texorpdfstring{第1章}{第1章}}

\section*{p5:問1}
\addcontentsline{toc}{section}{\texorpdfstring{p5:問1}{p5:問1}}

\begin{tproof}
  線分$\mathrm{PQ}$の中点を$\mathrm{M}$とする.このとき,
  \begin{align*}
    \overrightarrow{\mathrm{OM}} & = \overrightarrow{\mathrm{OP}} + \overrightarrow{\mathrm{PM}} \\
                                 & = \bm{a} + \frac{\bm{b}-\bm{a}}{2}                            \\
                                 & = \frac{\bm{a}+\bm{b}}{2}
  \end{align*}
  である.
\end{tproof}
\section*{p5:問2}
\addcontentsline{toc}{section}{\texorpdfstring{p5:問2}{p5:問2}}
\begin{tproof}
  三角形$\mathrm{PQR}$の重心を$\mathrm{G}$,$\mathrm{PQ}$の中点を$\mathrm{N}$とする.$\mathrm{G}$は線分$\mathrm{RN}$を$2:1$に内分する点なので,
  \begin{align*}
    \overrightarrow{\mathrm{OG}} & = \overrightarrow{\mathrm{OR}} + \frac{2}{3} \overrightarrow{\mathrm{RN}} \\
                                 & = \bm{c}+ \frac{2}{3} \left (\frac{\bm{a}+\bm{b}}{2}-\bm{c} \right)       \\
                                 & = \frac{\bm{a}+\bm{b}+\bm{c}}{3}
  \end{align*}
  である.
\end{tproof}



\section*{p7:問-(上)}
\addcontentsline{toc}{section}{\texorpdfstring{p7:問-(上)}{p7:問-(上)}}


\begin{tanswer}
  求めるベクトルを,$\bm{x}=(x,y,z)$~(ただし$x^2+y^2 +z^2=1$)とおく.
  このとき,内積の定義により,
  \begin{align*}
     & \bm{x} \cdot
    \begin{pmatrix}
      1 \\
      1 \\
      1
    \end{pmatrix}
    =x+y+z= 1 \cdot \sqrt{3} \cdot \cos \frac{\pi}{6} =\frac{3}{2} \\
     & \bm{x} \cdot
    \begin{pmatrix}
      1 \\
      1 \\
      4
    \end{pmatrix}
    =x+y+4z= 1 \cdot 3\sqrt{2} \cdot \cos \frac{\pi}{4} =3
  \end{align*}
  これらの式から,
  \begin{equation*}
    \begin{pmatrix}
      x \\
      y \\
      z
    \end{pmatrix}
    =
    \begin{pmatrix}
      (2 \pm \sqrt{2})/4 \\
      (2 \mp \sqrt{2})/4 \\
      1/2
    \end{pmatrix}
    \quad (\text{複号同順})
  \end{equation*}
  である.
\end{tanswer}
\section*{p7:問-(下)}
\addcontentsline{toc}{section}{\texorpdfstring{p7:問-(下)}{p7:問-(下)}}
\begin{tanswer}
  ここでは,[1.4]の結果を利用する.

  求める三角形の面積を$S$とし.
  \[
    \bm{a}=\overrightarrow{\mathrm{P_1 P_2}}=(x_2-x_1,y_2-y_1,z_2-z_1),\quad \bm{b}=\overrightarrow{\mathrm{P_1 P_3}}=(x_3-x_1,y_3-y_1,z_3-z_1)
  \]
  とおく,
  このとき,
  \begin{align*}
    S & = \frac{1}{2} \sqrt{\norm{\overrightarrow{\mathrm{P_1 P_2}}}^2 \norm{\overrightarrow{\mathrm{P_1 P_3}}}^2 - (\overrightarrow{\mathrm{P_1 P_2}}, \overrightarrow{\mathrm{P_1 P_3}})^2} \\
      & = \frac{1}{2} \sqrt{\norm{\bm{a}}^2 \norm{\bm{b}}^2 - (\bm{a},\bm{b})^2 }                                                                                                             \\
      & = \frac{1}{2} \{ \lbrack (x_2 - x_1)^2+(y_2-y_1)^2+ (z_2 - z_1)^2 \rbrack \lbrack (x_3 - x_1)^2+(y_3-y_1)^2+ (z_3 - z_1)^2 \rbrack                                                    \\
      & \qquad - \lbrack (x_2-x_1)(x_3-x_1)+(y_2 -y_1)(y_3-y_1)+(z_2-z_1)(z_3-z_1)\rbrack ^2\}^{\frac{1}{2}}
  \end{align*}
  である.
\end{tanswer}



\section*{p8:問1}
\addcontentsline{toc}{section}{\texorpdfstring{p8:問1}{p8:問1}}


\subsection*{p8:問1-(イ)}
\addcontentsline{toc}{subsection}{\texorpdfstring{p8:問1-(イ)}{p8:問1-(イ)}}

\begin{tanswer}
  与えられた直線を$l$とする.$l$の方程式に$x=-1$を代入すると,$y=2$となるため,$l$は点$(-1,2)$を通る.
  また,$l$は点$(2,0)$を通るため,
  $l$の方向ベクトルのひとつは,
  \[
    \begin{pmatrix}
      2-(-1) \\
      0-2
    \end{pmatrix}
    =
    \begin{pmatrix}
      3 \\
      -2
    \end{pmatrix}
  \]
  である.よって,$l$のベクトル表示のひとつは,
  $
    \begin{pmatrix}
      x \\
      y
    \end{pmatrix}
    =
    \begin{pmatrix}
      2 \\
      0
    \end{pmatrix}
    +t
    \begin{pmatrix}
      3 \\
      -2
    \end{pmatrix}
    ~(-\infty < t < \infty)$である.
\end{tanswer}


\subsection*{p8:問1-(ロ)}
\addcontentsline{toc}{subsection}{\texorpdfstring{p8:問1-(ロ)}{p8:問1-(ロ)}}

\begin{tanswer}
  与えられた直線を$l'$とする.$l '$の方向ベクトルのひとつは,$
    \begin{pmatrix}
      0 \\
      1
    \end{pmatrix}
  $である.また,$l '$は点$(3,0)$を通るので,そのベクトル表示のひとつは,
  $
    \begin{pmatrix}
      x \\
      y
    \end{pmatrix}
    =
    \begin{pmatrix}
      3 \\
      0
    \end{pmatrix}
    +t
    \begin{pmatrix}
      0 \\
      1
    \end{pmatrix}
    ~(-\infty < t < \infty)
  $となる.
\end{tanswer}

\section*{p8:問2}
\addcontentsline{toc}{section}{\texorpdfstring{p8:問2}{p8:問2}}


\subsection*{p8:問2-(イ)}
\addcontentsline{toc}{subsection}{\texorpdfstring{p8:問2-(イ)}{p8:問2-(イ)}}

\begin{tanswer}
  与えられたベクトル表示から.
  \begin{align*}
    \begin{cases}
      x=1+2t \\
      y=-1+t
    \end{cases}
  \end{align*}
  であるから,
  \begin{align*}
    \begin{cases}
      t=\frac{x-1}{2} \\
      t=y+1
    \end{cases}
  \end{align*}
  である.これから$t$を消去すると,
  \begin{gather*}
    \frac{x-1}{2} = y+1 \\
    \therefore ~ x-2y-3 =0
  \end{gather*}
  である.
\end{tanswer}


\subsection*{p8:問2-(ロ)}
\addcontentsline{toc}{subsection}{\texorpdfstring{p8:問2-(ロ)}{p8:問2-(ロ)}}

\begin{tanswer}
  点$(-1,-2)$を通り,$x$軸に平行な直線を表すから,$y=-2$が求める直線の方程式である.
\end{tanswer}
%

%
%

\section*{p10:問1}
\addcontentsline{toc}{section}{\texorpdfstring{p10:問1}{p10:問1}}

\begin{tanswer}
  \begin{align*}
    \begin{cases}
      x+2y+3z=1 \\
      3x+2y+z=-1
    \end{cases}
  \end{align*}
  から,
  \begin{gather*}
    -2x+2z=2 \\
    \therefore \quad -x+z=1
  \end{gather*}
  である.このとき,$
    \begin{pmatrix}
      x \\
      z
    \end{pmatrix}
    =
    \begin{pmatrix}
      1 \\
      2
    \end{pmatrix}
    ,
    \begin{pmatrix}
      2 \\
      3
    \end{pmatrix}
  $はこれを満たす.このときの$y$の値を計算すると,それぞれ$-3,-5$なので,結局,与えられた直線は2点$(1,-3,2),(2,-5,3)$を通る.
  すなわち,この直線の方向ベクトルのひとつは
  \[
    \begin{pmatrix}
      2  \\
      -5 \\
      3
    \end{pmatrix}
    -
    \begin{pmatrix}
      1  \\
      -3 \\
      2
    \end{pmatrix}
    =
    \begin{pmatrix}
      1  \\
      -2 \\
      1
    \end{pmatrix}
  \]
  である.したがって求めるベクトル表示のひとつは,直線上の任意の位置ベクトルを$\bm{x}$とすると,
  \[
    \bm{x} =
    \begin{pmatrix}
      1  \\
      -3 \\
      2
    \end{pmatrix}
    +t
    \begin{pmatrix}
      1  \\
      -2 \\
      1
    \end{pmatrix}
  \]
  と表せる.
\end{tanswer}

\section*{p10:問2}
\addcontentsline{toc}{section}{\texorpdfstring{p10:問2}{p10:問2}}
\begin{tproof}
  $t$を$0 \le t \le 1$をみたす実数,線分$\mathrm{P_1 P_2}$上の任意の点の位置ベクトルを$\bm{x}$とする.
  このとき,
  \begin{align*}
    \bm{x} & = \overrightarrow{\mathrm{O P_1}}+t\overrightarrow{\mathrm{P_1 P_2}} \\
           & = \bm{x}_1 + t (\bm{x}_2 - \bm{x}_1)                                 \\
           & = (1-t) \bm{x}_1 + t \bm{x}_2
  \end{align*}
  である.$1-t = t_1,~t=t_2$と改めておくと,$t$の定め方から$t_1 \ge 0,~t_2 \ge 0$であり,
  \[
    \bm{x}= t_1 \bm{x}_1 + t_2 \bm{x}_2 ,\quad t_1 + t_2 =1
  \]
  となり,これが証明すべきことであった.
\end{tproof}


\section*{p11:問1}
\addcontentsline{toc}{section}{\texorpdfstring{p11:問1}{p11:問1}}

\begin{tanswer}
  与えられた平面を$(S)$とおく.$(S)$は3点$(-1,0,1),~(2,0,-1),~(0,-1,0)$を通るので,
  \begin{equation*}
    \bm{x}_1=
    \begin{pmatrix}
      -1 \\
      0  \\
      1
    \end{pmatrix}
    ,\quad \bm{x}_2=
    \begin{pmatrix}
      2 \\
      0 \\
      -1
    \end{pmatrix}
    \quad
    \bm{x}_3=
    \begin{pmatrix}
      0  \\
      -1 \\
      0
    \end{pmatrix}
  \end{equation*}
  と改めておくと,
  \begin{gather*}
    \bm{x}_2 - \bm{x}_1 =
    \begin{pmatrix}
      3 \\
      0 \\
      -1
    \end{pmatrix}
    ,\quad
    \bm{x}_3 - \bm{x}_1 =
    \begin{pmatrix}
      1  \\
      -1 \\
      -1
    \end{pmatrix}
  \end{gather*}
  となり,$\bm{x}_2 - \bm{x}_1$と$\bm{x}_3 - \bm{x}_1$は線型独立なので,求めるベクトル表示のひとつは,
  \[
    (S) \colon \bm{x}=
    \begin{pmatrix}
      -1 \\
      0  \\
      1
    \end{pmatrix}
    + t
    \begin{pmatrix}
      3 \\
      0 \\
      -1
    \end{pmatrix}
    +s
    \begin{pmatrix}
      1  \\
      -1 \\
      -1
    \end{pmatrix}
    ~( -\infty < t,~s<\infty)
  \]
\end{tanswer}
%
%
%
%
\section*{p12:問2}
\addcontentsline{toc}{section}{\texorpdfstring{p12:問2}{p12:問2}}

\begin{tanswer}
  \begin{align*}
    \begin{cases}
      x=1+t-s   \\
      y=2-t -2s \\
      z=0+2t+s
    \end{cases}
  \end{align*}
  から$t$と$s$を消去して,
  \[
    x-y-z=-1
  \]
  これが求める直線の方程式である.
\end{tanswer}

\section*{p12:問3}
\addcontentsline{toc}{section}{\texorpdfstring{p12:問3}{p12:問3}}

\begin{tproof}
  \[
    \overrightarrow{\mathrm{OP_1}}=\bm{x}_1,\quad \overrightarrow{\mathrm{OP_2}}=\bm{x}_2,\quad \overrightarrow{\mathrm{OP_3}}=\bm{x}_3
  \]
  とする.このとき,三角形$\mathrm{P_1 P_2 P_3}$上の任意の点の位置ベクトルを$\bm{x}$,$s,t$を$0 \le s,t \le 1$を満たす実数とすると,
  \begin{gather*}
    \bm{x}=\bm{x}_1 + s(\bm{x}_2 - \bm{x}_1) + t (\bm{x}_3 - \bm{x}_1) \\
    \therefore \quad \bm{x} = (1-s-t)\bm{x}_1 + s\bm{x}_2 + t \bm{x}_3
  \end{gather*}
  となり,$1-s-t=t_1,~s=t_2,~t=t_3$と改めて書き直すと,$s,t$の定め方より,$0 \le t_1 ,t_2,t_3 \le 1$であり
  \[
    \bm{x} = t_1\bm{x}_1 + t_2\bm{x}_2 + t_3 \bm{x}_3,\quad t_1+t_2+t_3=1
  \]
  となる.これが証明すべきことであった.
\end{tproof}
%

%
\section*{p13:問1}
\addcontentsline{toc}{section}{\texorpdfstring{p13:問1}{p13:問1}}
%
\begin{tanswer}
  $(S_1)$,$(S_2)$の法線ベクトルをそれぞれ$\bm{x}_1$,$\bm{x}_2$とおくと,
  \begin{gather*}
    \bm{x}_1 =
    \begin{pmatrix}
      1 \\
      1 \\
      2
    \end{pmatrix}
    ,
    \bm{x}_2 =
    \begin{pmatrix}
      3 \\
      3 \\
      0
    \end{pmatrix}
  \end{gather*}
  である.ゆえに,交角を$\theta ~(0 \le \theta \le \frac{\pi}{2})$とすると,
  \[
    \cos \theta = \frac{\bm{x}_1 \cdot \bm{x}_2}{\norm{\bm{x}_1} \norm{\bm{x}_2}}=\frac{3}{3\sqrt{2}}=\frac{1}{\sqrt{2}}
  \]
  であるから,$0 \le \theta \le \frac{\pi}{2}$より$\theta =\frac{\pi}{4}$である.
\end{tanswer}
%
\section*{p13:問2}
\addcontentsline{toc}{section}{\texorpdfstring{p13:問2}{p13:問2}}

\begin{tproof}
  平面$\pi_1$,$\pi_2$を考え,$\pi_1$,$\pi_2$の法線ベクトルをそれぞれ$\bm{n}_1$,$\bm{n}_2$とおく.
  \begin{description}
    \item[$\bm{n}_1$と$\bm{n}_2$が平行なとき] \mbox{}\\
          $\pi_1$に垂直な平面は$\pi_2$にも垂直であり,このような平面を$\pi_3$とすると,
          $\pi_3$は$\bm{n}_1$,$\bm{n}_2$と平行である.よって$\pi_3$と$\pi_1$,$\pi_2$はそれぞれ並行であり,このような平面は確かに存在する.
    \item[$\bm{n}_1$と$\bm{n}_2$が平行でないとき] \mbox{} \\
          $\bm{n}_1 , \bm{n}_2 \ne \bm{0}$は明らかなので,$\bm{n}_3 \coloneqq \bm{n}_1 \times \bm{n}_2$とすると,
          $\bm{n}_3 \ne \bm{0}$である.よって,$\bm{n}_3$は$\pi_1$,$\pi_2$に垂直である.このとき$n_3$を法線ベクトルとする平面を取ればよい.
  \end{description}
  以上の考察により証明された.
\end{tproof}

%
%
%
\section*{p18:問}
\addcontentsline{toc}{section}{\texorpdfstring{p18:問}{p18:問}}

\begin{tproof}
  $A$,$B$,$C$が$2 \times 2$行列の場合を証明する.
  \begin{gather*}
    A=
    \begin{pmatrix}
      a & b \\
      c & d
    \end{pmatrix}
    ,
    B=
    \begin{pmatrix}
      e & f \\
      g & h
    \end{pmatrix}
    ,C=
    \begin{pmatrix}
      i & j \\
      k & l
    \end{pmatrix}
  \end{gather*}
  とし,$A$,$B$,$C$の成分はすべて複素数であるとする.このとき,
  \begin{align*}
    (AB)C & =
    \begin{pmatrix}
      ae+bg & af+bh \\
      ce+dg & cf+dh
    \end{pmatrix}
    \begin{pmatrix}
      i & j \\
      k & l
    \end{pmatrix}
    \\
          & =
    \begin{pmatrix}
      aei +bgi +afk +bhk & aej+bgj+afl+bhl     \\
      cei +dgi+cfk +dhk  & cej +dgj + cfl +dhl
    \end{pmatrix}
  \end{align*}
  となる.他方
  \begin{align*}
    A(BC) & =
    \begin{pmatrix}
      a & b \\
      c & d
    \end{pmatrix}
    \begin{pmatrix}
      ei + fk & ej +fl \\
      gi + hk & gj +hl
    \end{pmatrix}
    \\
          & =
    \begin{pmatrix}
      aei + afk +bgi +bhk  & aej +afl +bgj +bhl   \\
      cei + cfk + dgi +dhk & cej + cfl + dgi +dhl
    \end{pmatrix}
  \end{align*}
  となり,たしかに$(AB)C=A(BC)$である.
\end{tproof}
%

\section*{p19:問1-(上)}
\addcontentsline{toc}{section}{\texorpdfstring{p19:問1-(上)}{p19:問1-(上)}}
%
%
\begin{tproof}
  \[
    \begin{pmatrix}
      -1 & 0  \\
      0  & -1
    \end{pmatrix}
    \begin{pmatrix}
      x \\
      y
    \end{pmatrix}
    =
    \begin{pmatrix}
      -x \\
      -y
    \end{pmatrix}
  \]
  となり,これは明らかに線型変換である.対応する行列は,$
    \begin{pmatrix}
      -1 & 0  \\
      0  & -1
    \end{pmatrix}
  $である.
\end{tproof}

\section*{p19:問2-(上)}
\addcontentsline{toc}{section}{\texorpdfstring{p19:問2-(上)}{p19:問2-(上)}}
\begin{tproof}
  式(15)より,$2 \times 2$行列$A$,$B$とベクトル$\bm{x}$について,
  \begin{align*}
    T_B (T_A (\bm{x})) & = B(A\bm{x})      \\
                       & = (BA) \bm{x}     \\
                       & = T_{BA} (\bm{x})
  \end{align*}
  である.これが証明すべきことであった.
\end{tproof}


\section*{p19:問1-(下)}
\addcontentsline{toc}{section}{\texorpdfstring{p19:問1-(下)}{p19:問1-(下)}}

\begin{tanswer}
  $\bm{x} =
    \begin{pmatrix}
      x \\
      y
    \end{pmatrix}
  $
  とおくと,(17)式より,
  \begin{align*}
    T \bm{x} & = \frac{ax+by}{a^2+b^2} \bm{a} \\
             & =
    \begin{pmatrix}
      a^2x +aby   \\
      ab x + b^2y \\
    \end{pmatrix}
    \\
             & =
    \begin{pmatrix}
      a^2 & ab  \\
      ab  & b^2
    \end{pmatrix}
    \begin{pmatrix}
      x \\
      y
    \end{pmatrix}
    \\
             & =
    \begin{pmatrix}
      a^2 & ab  \\
      ab  & b^2
    \end{pmatrix}
    \bm{x}
  \end{align*}
  であるから,
  \[
    T=\begin{pmatrix}
      a^2 & ab  \\
      ab  & b^2
    \end{pmatrix}
  \]
  となる.
\end{tanswer}
%


\section*{p19:問2-(下)}
\addcontentsline{toc}{section}{\texorpdfstring{p19:問2-(下)}{p19:問2-(下)}}


\subsection*{p19:問2-(下)-(イ)}
\addcontentsline{toc}{subsection}{\texorpdfstring{p19:問2-(下)-(イ)}{p19:問2-(下)-(イ)}}
\begin{tproof}
  $\bm{a}=
    \begin{pmatrix}
      a_1 \\
      a_2
    \end{pmatrix}
  $,
  $\bm{b}=
    \begin{pmatrix}
      b_1 \\
      b_2
    \end{pmatrix}
  $,$\bm{a} \ne \bm{0}$かつ$\bm{b} \ne \bm{0}$とする.
  このとき,
  \begin{align*}
    T \bm{x} & =\frac{(\bm{a},\bm{x})}{(\bm{a},\bm{a})} \bm{a} \\
             & = \frac{a_1 x + a_2 y}{{a_1}^2+{a_2}^2}
    \begin{pmatrix}
      a_1 \\
      a_2
    \end{pmatrix}
    \\
             & =
    \frac{1}{{a_1}^2+{a_2}^2}
    \begin{pmatrix}
      {a_1}^2 & a_1 a_2 \\
      a_1 a_2 & {a_2}^2
    \end{pmatrix}
    \begin{pmatrix}
      x \\
      y
    \end{pmatrix}
  \end{align*}
  となる.つまり,$T=\frac{1}{{a_1}^2+{a_2}^2}
    \begin{pmatrix}
      {a_1}^2 & a_1 a_2 \\
      a_1 a_2 & {a_2}^2
    \end{pmatrix}
  $である.このとき,
  \begin{align*}
    T^2 & = \frac{1}{({a_1}^2+{a_2}^2)^2}
    \begin{pmatrix}
      {a_1}^4 + {a_1}^2 {a_2}^2 & {a_1}^3 a_2 + a_1 {a_2}^3 \\
      {a_1}^3 a_2 + a_1 {a_2}^3 & {a_2}^4 + {a_1}^2 {a_2}^2
    \end{pmatrix}
    \\
        & = \frac{1}{{a_1}^2+{a_2}^2}
    \begin{pmatrix}
      {a_1}^2 & a_1 a_2 \\
      a_1 a_2 & {a_2}^2
    \end{pmatrix}
    =T
  \end{align*}
  となり,$T^2=T$である.$S^2=S$も同様にして示される.
\end{tproof}

\subsection*{p19:問2-(下)-(ロ)}
\addcontentsline{toc}{subsection}{\texorpdfstring{p19:問2-(下)-(ロ)}{p19:問2-(下)-(ロ)}}
\begin{tproof}
  $\bm{a}=
    \begin{pmatrix}
      a_1 \\
      a_2
    \end{pmatrix}
  $,$\bm{b}=
    \begin{pmatrix}
      b_1 \\
      b_2
    \end{pmatrix}
  $とする.このとき,$\bm{a}$と$\bm{b}$が直交することから,
  \begin{gather*}
    \bm{a} \cdot \bm{b}=0 \\
    \therefore ~a_1 b_1 + a_2 b_2 =0
  \end{gather*}
  である.ここで,
  \begin{align*}
    TS & = \frac{1}{({a_1}^2 +{a_2}^2)}
    \begin{pmatrix}
      {a_1}^2 & a_1 a_2 \\
      a_1 a_2 & {a_2}^2
    \end{pmatrix}
    \frac{1}{({b_1}^2 +{b_2}^2)}
    \begin{pmatrix}
      {b_1}^2 & b_1 b_2 \\
      b_1 b_2 & {b_2}^2
    \end{pmatrix}
    \\
       & =\frac{a_1 b_1 + a_2 b_2}{({a_1}^2 +{a_2}^2)({b_1}^2 +{b_2}^2)}
    \begin{pmatrix}
      a_1 b_1 & a_1 b_2 \\
      a_2 b_1 & a_2 b_2
    \end{pmatrix}
    =O                                                                   \\
       & \qquad \qquad  (\because ~ a_1 b_1 + a_2 b_2 =0)
  \end{align*}
  である.同様に$ST$を計算すると,$ST=O$であることもわかり,これで$TS=ST=O$が証明された.
\end{tproof}

\subsection*{p19:問2-(下)-(ハ)}
\addcontentsline{toc}{subsection}{\texorpdfstring{p19:問2-(下)-(ハ)}{p19:問2-(下)-(ハ)}}
\begin{tproof}
  イ),ロ)の文字や結論を用いると,
  \begin{align*}
    T \bm{x} + S \bm{x} & =
    \frac{1}{{a_1}^2+{a_2}^2}
    \begin{pmatrix}
      {a_1}^2 & a_1 a_2 \\
      a_1 a_2 & {a_2}^2
    \end{pmatrix}
    \begin{pmatrix}
      x \\
      y
    \end{pmatrix}
    +
    \frac{1}{{b_1}^2+{b_2}^2}
    \begin{pmatrix}
      {b_1}^2 & b_1 b_2 \\
      b_1 b_2 & {b_2}^2
    \end{pmatrix}
    \begin{pmatrix}
      x \\
      y
    \end{pmatrix}
    \\
                        & = \frac{1}{({a_1}^2+{a_2}^2)({b_1}^2+{b_2}^2)}
    \begin{pmatrix}
      ({a_1}^2+{a_2}^2)({b_1}^2+{b_2}^2) & ({a_1}^2+{a_2}^2)({b_1}^2+{b_2}^2) \\
      ({a_1}^2+{a_2}^2)({b_1}^2+{b_2}^2) & ({a_1}^2+{a_2}^2)({b_1}^2+{b_2}^2)
    \end{pmatrix}
    \begin{pmatrix}
      x \\
      y
    \end{pmatrix}
    \\
                        & =\begin{pmatrix}
                             x \\
                             y
                           \end{pmatrix}
    =\bm{x}
  \end{align*}
  となる.これが証明すべきことであった.
\end{tproof}



\section*{p22:問1}
\addcontentsline{toc}{section}{\texorpdfstring{p22:問1}{p22:問1}}


\subsection*{p22:問1-(イ)}
\addcontentsline{toc}{subsection}{\texorpdfstring{p22:問1-(イ)}{p22:問1-(イ)}}
\begin{tanswer}
  \[
    \begin{pmatrix}
      -1 & 0 & 0  \\
      0  & 1 & 0  \\
      0  & 0 & -1
    \end{pmatrix}
    \begin{pmatrix}
      x \\
      y \\
      z
    \end{pmatrix}
    =\begin{pmatrix}
      -x \\
      y  \\
      -z
    \end{pmatrix}
  \]
  となり,これは$y$軸に関する対象点に移す変換を表す.
\end{tanswer}

\subsection*{p22:問1-(ロ)}
\addcontentsline{toc}{subsection}{\texorpdfstring{p22:問1-(ロ)}{p22:問1-(ロ)}}
\begin{tanswer}
  \[
    \begin{pmatrix}
      1 & 0           & 0            \\
      0 & \cos \alpha & -\sin \alpha \\
      0 & \sin \alpha & \cos \alpha
    \end{pmatrix}
    \begin{pmatrix}
      x \\
      y \\
      z
    \end{pmatrix}
    =
    \begin{pmatrix}
      x                            \\
      y \cos \alpha -z \sin \alpha \\
      y \sin \alpha + z \cos \alpha
    \end{pmatrix}
  \]
  となり,これは$x$軸まわりに角$\alpha$だけ回転する変換を表す.
\end{tanswer}

\subsection*{p22:問1-(ハ)}
\addcontentsline{toc}{subsection}{\texorpdfstring{p22:問1-(ハ)}{p22:問1-(ハ)}}

\begin{tanswer}
  \[
    \begin{pmatrix}
      0 & 1 & 0 \\
      0 & 0 & 1 \\
      1 & 0 & 0
    \end{pmatrix}
    \begin{pmatrix}
      x \\
      y \\
      z
    \end{pmatrix}
    =\begin{pmatrix}
      y \\
      z \\
      x
    \end{pmatrix}
  \]
\end{tanswer}


\setcounter{equation}{0}

\part*{第1章・章末問題}
\addcontentsline{toc}{part}{\texorpdfstring{第1章・章末問題}{第1章・章末問題}}


\section*{p29--30:1}
\addcontentsline{toc}{section}{\texorpdfstring{p29--30:1}{p29--30:1}}

\begin{tproof}
  \[
    P=\left\{\bm{x}=t_1 \bm{x}_1+t_2\bm{x}_2+\dots+t_k\bm{x}_k\; \middle| \;t_i\ \ge 0\;(1\le i\le k),\;\sum_{i=1}^k t_i=1\right\}.
  \]
  であることを示す.
  \begin{enumerate}[(i)]
    \item $ P \subset \left\{\bm{x}=t_1 \bm{x}_1+t_2\bm{x}_2+\dots+t_k\bm{x}_k\; \middle| \;t_i\ \ge 0\;(1\le i\le k),\;\sum_{i=1}^k t_i=1\right\}$
    \item $ P \supset \left\{\bm{x}=t_1 \bm{x}_1+t_2\bm{x}_2+\dots+t_k\bm{x}_k\; \middle| \;t_i\ \ge 0\;(1\le i\le k),\;\sum_{i=1}^k t_i=1\right\}$
  \end{enumerate}
  とする.
  \paragraph{(i)の証明}
  四面体$\mathrm{P_1 P_2 P_3 P_4}$を考える.三角形$\mathrm{P_2 P_3 P_4}$の任意の周および内部の点を$\mathrm{T}$とする.
  $0 \leqq k \leqq 1$,$0 \leqq s \leqq 1$をみたす$k,s \in \mathbb{R}$によって
  \begin{align*}
    \overrightarrow{\mathrm{P_2 T}} & = k \{ s \overrightarrow{\mathrm{P_2 P_3}} + (1-s) \overrightarrow{\mathrm{P_2 P_4}} \} \\
                                    & = ks(\bm{x}_3 -\bm{x}_2) + k(1-s) (\bm{x}_4-\bm{x}_2)                                   \\
                                    & = -k\bm{x}_2 + ks \bm{x}_3 + k(1-s) \bm{x}_4
  \end{align*}
  と表される.

  さて,線分$\mathrm{P_1 T}$上の任意の点を$\mathrm{Q}$とすると,$0 \leqq t \leqq 1$をみたす$t \in \mathbb{R}$によって

  \begin{align*}
    \overrightarrow{\mathrm{P_1 Q}} & =t \overrightarrow{\mathrm{P_1 T}}                                      \\
                                    & = t\overrightarrow{\mathrm{P_2 T}} - t\overrightarrow{\mathrm{P_2 P_1}} \\
                                    & = t (-k\bm{x}_2 + ks \bm{x}_3 + k(1-s) \bm{x}_4)-t(\bm{x}_1 -\bm{x}_2)  \\
                                    & = -t\bm{x}_1 +(t-kt) \bm{x}_2 + kst \bm{x}_3 +kt(1-s) \bm{x}_4
  \end{align*}
  と表されるから,$k=4$のときの求める位置ベクトルは.
  \begin{align*}
    \bm{x} & = \bm{x}_1 + \overrightarrow{\mathrm{P_1 Q}}                     \\
           & = (1-t) \bm{x}_1 +(t-kt)\bm{x}_2 +kst \bm{x}_3 +kt(1-s) \bm{x}_4
  \end{align*}
  となり,
  \[
    (1-t)+ (t-kt)+kst + kt(1-s)=1
  \]
  であるから,$1-t = t_1$,$t-kt =t_2$,$kst = t_3$,$kt(1-s)=t_4$とおくと,
  \[
    \bm{x}= t_1 \bm{x}_1 + t_2 \bm{x}_2 + t_3 \bm{x}_3 + t_4 \bm{x}_4 , \quad t_1, t_2 ,t_3 , t_4 \geqq 0 ,\quad  t_1 +t_2 + t_3 + t_4 =1
  \]
  となり,ここまでで$k=4$の場合が示された.

  ここで,$n \geqq 4$として$k=n$のときに主張が成り立つと仮定する.
  このとき,
  \[
    t_1 \bm{x}_1 + t_2 \bm{x}_2+\dots+ t_n \bm{x}_n
  \]
  は仮定により多面体$\{ \mathrm{P}_n \}$の内部の点であり,これを簡単のために$\bm{X}_n$とおく.

  さて,$\{ \mathrm{P}_n \}$の点と$\mathrm{P}_{n+1}$とを結ぶ線分上の点は,$ 0 \leqq l \leqq 1$をみたす$l \in \mathbb{R}$によって,
  \[
    l \overbrace{(  t_1 \bm{x}_1 + t_2 \bm{x}_2+\dots+ t_n \bm{x}_n)}^{\bm{X}_n}+(1-l) \bm{x}_{n+1} , \quad t_1+t_2+\dots + t_n =1
  \]
  とかける.ここで,
  \[
    l(t_1+t_2+\dots+t_n)+(1-l)=1
  \]
  なので,$\{ \mathrm{P}_n \}$の点と$\mathrm{P}_{n+1}$とを結ぶ線分上の点はこのように表せる.
  よって,$k=n$のときも問題の主張が成り立つ.

  \paragraph{(ii)の証明}
  \[
    \bm{x} = t_1 \bm{x}_1 + t_2 \bm{x}_2 + \dots + t_n \bm{x}_n + t_{n+1} \bm{x}_{n+1} , \quad t_1, t_2 ,\dots , t_n,t_{n+1} \geqq 0 ,\quad  t_1 +t_2 + \dots+t_n + t_{n+1} =1
  \]
  としたとき,
  \begin{align*}
    \bm{x} & =\frac{t_1 \bm{x}_1 + t_2 \bm{x}_2+\dots+ t_n \bm{x}_n}{t_1+t_2+\dots+t_n} \cdot (t_1+t_2+\dots+t_n) +t_{n+1} \bm{x}_{n+1} \\
           & = (t_1 +t_2 + \dots +t_n) \bm{X}_n + t_{n+1} \bm{x}_{n+1}                                                                  \\
           & = (1-t_{n+1}) \bm{X}_n + t_{n+1} \bm{x}_{n+1}                                                                              \\
  \end{align*}
  と変形できる.

  さて
  \[
    \frac{\bm{X}_n }{1-t_{n+1}} = \frac{t_1 \bm{x}_1 + t_2 \bm{x}_2+\dots+ t_n \bm{x}_n}{t_1+t_2+\dots+t_n}
  \]
  であることと
  \begin{align*}
      & \frac{t_1}{t_1+t_2+\dots+t_n}+\frac{t_2 }{t_1+t_2+\dots+t_n} +\dots +\frac{t_n }{t_1+t_2+\dots+t_n} \\
    = & \frac{t_1+t_2+\dots+t_n}{t_1+t_2+\dots+t_n}                                                         \\
    = & 1
  \end{align*}
  であることにより,
  \[
    \frac{\bm{X}_n}{1-t_{n+1}}
  \]
  は,多面体$\{ \mathrm{P}_n \}$の内部の点であり.
  \[
    (1-t_{n+1}) \cdot \frac{\bm{X}_n}{1-t_{n+1}} + t_{n+1} \bm{x}_{n+1}
  \]
  は多面体$\{ \mathrm{P}_n \}$の内部の点と$\mathrm{P}_{n+1}$を結ぶ線分上の点である.

  以上の考察により証明された.
\end{tproof}

\section*{p29--30:2}
\addcontentsline{toc}{section}{\texorpdfstring{p29--30:2}{p29--30:2}}

\begin{tproof}
  2点$\mathrm{P}_1$,$\mathrm{P_2}$を通る直線の方程式を$ax+by+c=0$(ただし$(a,b)=0$)とおく.
  このとき,
  \[
    \begin{cases}
      ax+by+c =0       \\
      ax_1 + by_1 +c=0 \\
      ax_2 + by_2 +c =0
    \end{cases}
  \]
  が成立する.すなわちこれは
  \[
    \begin{pmatrix}
      x   & y   & 1 \\
      x_1 & y_1 & 1 \\
      x_2 & y_2 & 1
    \end{pmatrix}
    \begin{pmatrix}
      a \\
      b \\
      c
    \end{pmatrix}
    = \bm{0}
  \]
  をみたす.これを$a$,$b$,$c$についての連立方程式とみたとき,与条件により自明でない解があり,
  \[
    \begin{vmatrix}
      x   & y   & 1 \\
      x_1 & y_1 & 1 \\
      x_2 & y_2 & 1
    \end{vmatrix}
    =0
  \]
  が成立する.転置行列の行列式はもとの行列の行列式に等しいので,行列式の交代性なども用いて,
  \[
    \begin{vmatrix}
      1 & 1   & 1   \\
      x & x_1 & x_2 \\
      y & y_1 & y_2
    \end{vmatrix}
    =0
  \]
  を得る.これが証明すべきことであった.
\end{tproof}



\section*{p29--30:3}
\addcontentsline{toc}{section}{\texorpdfstring{p29--30:3}{p29--30:3}}

\begin{tanswer}
  点を以下の順で移動させる変換を考える.
  \begin{enumerate}
    \item 原点中心に$-\theta$回転させる.
    \item $x$軸に関して対称移動させる.
    \item 原点中心に$\theta$回転させる.
  \end{enumerate}
  ここで,(1)から(3)までの変換を表す行列をそれぞれ$R_{-\theta}$,$A_{x}$,$R_{\theta}$とすると.
  \begin{align*}
     & R_{-\theta} = \begin{pmatrix} \cos \theta & \sin \theta \\ -\sin \theta & \cos \theta \end{pmatrix} , \\
     & A_{x} = \begin{pmatrix} 1 & 0 \\ 0 & -1 \end{pmatrix} ,                                               \\
     & R_{\theta} = \begin{pmatrix} \cos \theta & -\sin \theta \\ \sin \theta & \cos \theta \end{pmatrix} .
  \end{align*}
  となる.よって,この変換を表す行列は
  \begin{align*}
    R_{\theta} A_x R_{-\theta} & =\begin{pmatrix} \cos \theta & -\sin \theta \\ \sin \theta & \cos \theta \end{pmatrix}
    \begin{pmatrix} 1 & 0 \\ 0 & -1 \end{pmatrix}
    \begin{pmatrix} \cos \theta & \sin \theta \\ -\sin \theta & \cos \theta \end{pmatrix}                                                                                                \\
                               & = \begin{pmatrix} \cos ^2 \theta - \sin ^2 \theta & 2\sin \theta \cos \theta \\ 2\sin \theta \cos \theta & \sin ^2 \theta -\cos ^2 \theta \end{pmatrix} \\
                               & = \begin{pmatrix} \cos 2 \theta & \sin 2 \theta \\ \sin 2\theta & -\cos 2 \theta \end{pmatrix}
  \end{align*}
  である.
\end{tanswer}


\section*{p29--30:4}
\addcontentsline{toc}{section}{\texorpdfstring{p29--30:4}{p29--30:4}}

\begin{tproof}
  以下では,直線$y= \tan \theta$に関する折り返しを$T_{\theta}$とかくことにする.

  さて,直線$ y = \tan (\theta /4) x$に関する折り返しは,
  \[
    T_{\theta/4} = \begin{pmatrix} \cos (\theta /2) & \sin (\theta /2) \\  \sin (\theta /2) & -\cos (\theta /2) \end{pmatrix}
  \]
  で表される.

  また,直線$y = \tan (-\theta /4)x$に関する折り返しは.
  \[
    T_{-\theta/4} = \begin{pmatrix} \cos (\theta /2) & -\sin (\theta/2) \\ -\sin (\theta/2) & -\cos (\theta /2) \end{pmatrix}
  \]
  で表される.

  ここで,
  \begin{align*}
    T_{\theta/4} T_{-\theta/4} & = \begin{pmatrix} \cos (\theta /2) & \sin (\theta /2) \\  \sin (\theta /2) & -\cos (\theta /2) \end{pmatrix} \begin{pmatrix} \cos (\theta /2) & -\sin (\theta/2) \\ -\sin (\theta/2) & -\cos (\theta /2) \end{pmatrix} \\
                               & =\begin{pmatrix} \cos ^2 (\theta /2)-\sin ^2 (\theta/2) & -2\sin (\theta/2) \cos (\theta/2) \\ 2\sin (\theta/2) \cos (\theta/2) & \cos ^2 (\theta/2)-\sin ^2 (\theta/2) \end{pmatrix}                                  \\
                               & = \begin{pmatrix} \cos \theta & -\sin \theta \\ \sin \theta & \cos \theta \end{pmatrix}
  \end{align*}
  となり,これは原点のまわりに$\theta$回転する行列を表す.

  以上の考察により証明された.
\end{tproof}

\section*{p29--30:5}
\addcontentsline{toc}{section}{\texorpdfstring{p29--30:5}{p29--30:5}}

\begin{tanswer}
  任意の点$\mathrm{P}(\bm{p}),~\bm{p} \in \mathbb{R}^3$を平面$(\bm{a},\bm{x})$に対して折り返すことを考える.

  点$\mathrm{P}$から$(\bm{a},\bm{x})$におろした垂線の足は,$t \in \mathbb{R}$を用いて
  \[
    \bm{p} + t \frac{\bm{a}}{(\bm{a},\bm{a})}
  \]
  と表せ,これが平面$(\bm{a},\bm{x})$上にあるので,
  \begin{align*}
                 & (\bm{a},p+t\frac{\bm{a}}{(\bm{a},\bm{a})})=0 \\
    \therefore ~ & t=- (\bm{a},\bm{p})
  \end{align*}
  である.

  また,求める点を$\mathrm{P}' (\bm{p}')$とすると,
  \begin{align*}
    \bm{p}' & = \bm{p}+t \frac{2\bm{a}}{(\bm{a},\bm{a})}               \\
            & = \bm{p}-\frac{2(\bm{a},\bm{p})}{(\bm{a},\bm{a})} \bm{a}
  \end{align*}
  であるから,これはたしかに$V^3$の線型変換を引き起こし,その変換公式は
  \[
    \bm{x} \mapsto \bm{x}-\frac{2(\bm{a},\bm{x})}{(\bm{a},\bm{a})} \bm{a}
  \]
  である,
\end{tanswer}


\section*{p29--30:6}
\addcontentsline{toc}{section}{\texorpdfstring{p29--30:6}{p29--30:6}}

\subsection*{p29--30:6-(イ)}
\addcontentsline{toc}{subsection}{\texorpdfstring{p29--30:6-(イ)}{p29--30:6-(イ)}}

\begin{tanswer}
  $(S)$と$x$軸,$y$軸,$z$軸の交点をそれぞれ$\mathrm{A}$, $\mathrm{B}$,$\mathrm{C}$とする.
  このとき,$\mathrm{A}$,$\mathrm{B}$,$\mathrm{C}$の座標は図のようになり,
  この三角錐の体積を$V$とすると,
  \begin{align*}
    V= & \frac{1}{2} \cdot \frac{1}{3} \cdot\abs{ \det (\overrightarrow{\mathrm{OA}}, \overrightarrow{\mathrm{OB}}, \overrightarrow{\mathrm{OC}})} \\
    =  & \frac{1}{6} \cdot \frac{\abs{d}^3}{\abs{abc}}= \frac{\abs{d}^3}{6\abs{abc}}
  \end{align*}
  である.ここで,$\abs{\det (\overrightarrow{\mathrm{OA}}, \overrightarrow{\mathrm{OB}}, \overrightarrow{\mathrm{OC}})}$が$\overrightarrow{\mathrm{OA}}$,$\overrightarrow{\mathrm{OB}}$,$\overrightarrow{\mathrm{OC}}$の張る平行六面体の体積を表すことを用いた.

\end{tanswer}
%回転角
\tdplotsetmaincoords{70}{135}
\begin{tikzpicture}[tdplot_main_coords]
  %軸
  \draw[->,>=stealth,semithick](0,0,0)--(5,0,0)node[below]{$x$};%x軸
  \draw[->,>=stealth,semithick](0,0,0)--(0,5,0)node[right]{$y$};%y軸
  \draw[->,>=stealth,semithick](0,0,0)--(0,0,5)node[right]{$z$};%y軸
  \draw(0,0)node[below]{O};%原点

  \draw[dashed,semithick](3,0,0)--(0,4,0)--(0,0,7/2)--cycle;
  \fill[color=gray ,opacity=0.1] (3,0,0)--(0,4,0)--(0,0,7/2)--cycle;

  \node at (3,0,0) [above left]{$\mathrm{A} (d/a,0,0)$};
  \node at (0,4,0) [above right]{$\mathrm{B}(0,d/b,0)$};
  \node at (0,0,7/2) [right]{$\mathrm{C}(0,0,d/c)$};
\end{tikzpicture}

\subsection*{p29--30:6-(ロ)}
\addcontentsline{toc}{subsection}{\texorpdfstring{p29--30:6-(ロ)}{p29--30:6-(ロ)}}

\begin{tanswer}
  三角形$\mathrm{ABC}$の体積を$T$,$\mathrm{O}$から平面$ABC$におろした垂線の足を$H$とすると,
  \[
    V = \frac{1}{3} \norm{\overrightarrow{\mathrm{OH}}} \cdot T
  \]
  である.ここで,
  \[
    \norm{\overrightarrow{\mathrm{OH}}}=\frac{\abs{a \cdot 0 + b \cdot 0 + c \cdot 0-d}}{\sqrt{a^2+b^2+c^2}}=\frac{\abs{d}}{\sqrt{a^2+b^2+c^2}}
  \]
  なので,イ)の結果から$V =\frac{\abs{d}^3}{6\abs{abc}}$なのを加味すると,
  \[
    T = \frac{d^2\sqrt{a^2+b^2+c^2}}{2\abs{abc}}
  \]
  である.
\end{tanswer}


\section*{p29--30:7}
\addcontentsline{toc}{section}{\texorpdfstring{p29--30:7}{p29--30:7}}

\subsection*{p29--30:7-(イ)}
\addcontentsline{toc}{subsection}{\texorpdfstring{p29--30:7-(イ)}{p29--30:7-(イ)}}

\begin{tanswer}
  $\bm{a}$,$\bm{b}$,$\bm{c}$が張る平行六面体の体積は,
  \[
    \abs{\det (\bm{a},\bm{b},\bm{c})}
  \]
  で与えられる.

  一方,この平行六面体の$\mathrm{O}$,$\mathrm{B}$,$\mathrm{C}$を含む面の面積は,

  \[
    \norm{\bm{b} \times \bm{c}}
  \]

  で与えられる.

  以上の考察により,求める長さは,
  \[
    \frac{\abs{\det(\bm{a},\bm{b},\bm{c})}}{\norm{\bm{b} \times \bm{c}}}
  \]
  である.
  \scalebox{0.6}[0.6]{
    \begin{tikzpicture}[x=0.75pt,y=0.75pt,yscale=-1,xscale=1]
      %uncomment if require: \path (0,300); %set diagram left start at 0, and has height of 300

      %Straight Lines [id:da9042715539337772] 
      \draw [color={rgb, 255:red, 74; green, 144; blue, 226 }  ,draw opacity=1 ]   (150,250) -- (246.89,105.66) ;
      \draw [shift={(248,104)}, rotate = 123.87] [color={rgb, 255:red, 74; green, 144; blue, 226 }  ,draw opacity=1 ][line width=0.75]    (13.12,-3.95) .. controls (8.34,-1.68) and (3.97,-0.36) .. (0,0) .. controls (3.97,0.36) and (8.34,1.68) .. (13.12,3.95)   ;
      %Straight Lines [id:da29183874211574523] 
      \draw [color={rgb, 255:red, 144; green, 19; blue, 254 }  ,draw opacity=1 ]   (150,250) -- (348,250) ;
      \draw [shift={(350,250)}, rotate = 180] [color={rgb, 255:red, 144; green, 19; blue, 254 }  ,draw opacity=1 ][line width=0.75]    (10.93,-3.29) .. controls (6.95,-1.4) and (3.31,-0.3) .. (0,0) .. controls (3.31,0.3) and (6.95,1.4) .. (10.93,3.29)   ;
      %Straight Lines [id:da3012672762383153] 
      \draw [color={rgb, 255:red, 208; green, 2; blue, 27 }  ,draw opacity=1 ] [dash pattern={on 4.5pt off 4.5pt}]  (150,250) -- (250.21,200.39) ;
      \draw [shift={(252,199.5)}, rotate = 153.66] [color={rgb, 255:red, 208; green, 2; blue, 27 }  ,draw opacity=1 ][line width=0.75]    (10.93,-3.29) .. controls (6.95,-1.4) and (3.31,-0.3) .. (0,0) .. controls (3.31,0.3) and (6.95,1.4) .. (10.93,3.29)   ;
      %Straight Lines [id:da7013310145085282] 
      \draw    (248,104) -- (451,102.5) ;
      %Straight Lines [id:da9417424223698198] 
      \draw    (451,102.5) -- (350,250) ;
      %Straight Lines [id:da10714474623977321] 
      \draw  [dash pattern={on 4.5pt off 4.5pt}]  (247,202.5) -- (452,199.5) ;
      %Straight Lines [id:da9346887424461314] 
      \draw    (452,199.5) -- (350,250) ;
      %Straight Lines [id:da9143571217608703] 
      \draw    (451,102.5) -- (551,51.5) ;
      %Straight Lines [id:da1037466616997732] 
      \draw    (248,104) -- (348,50.5) ;
      %Straight Lines [id:da921854997991198] 
      \draw    (551,51.5) -- (452,199.5) ;
      %Straight Lines [id:da6285253534318218] 
      \draw  [dash pattern={on 4.5pt off 4.5pt}]  (348,50.5) -- (252,199.5) ;
      %Straight Lines [id:da23538555105346293] 
      \draw    (348,50.5) -- (551,51.5) ;
      %Straight Lines [id:da381755385753077] 
      \draw [color={rgb, 255:red, 65; green, 117; blue, 5 }  ,draw opacity=1 ][fill={rgb, 255:red, 65; green, 117; blue, 5 }  ,fill opacity=1 ]   (250,105) -- (250,235) ;
      %Shape: Right Angle [id:dp3981791931061768] 
      \draw  [color={rgb, 255:red, 65; green, 117; blue, 5 }  ,draw opacity=1 ][fill={rgb, 255:red, 255; green, 255; blue, 255 }  ,fill opacity=1 ] (235,235) -- (235,220) -- (250,220) ;

      % Text Node
      \draw (125,245) node [anchor=north west][inner sep=0.75pt]   [align=left] {$\mathrm{O}$};
      % Text Node
      \draw (225,80) node [anchor=north west][inner sep=0.75pt]   [align=left] {$\mathrm{A}$};
      % Text Node
      \draw (275,175) node [anchor=north west][inner sep=0.75pt]   [align=left] {$\mathrm{B}$};
      % Text Node
      \draw (350,260) node [anchor=north west][inner sep=0.75pt]   [align=left] {$\mathrm{C}$};
    \end{tikzpicture}
  }
\end{tanswer}


\subsection*{p29--30:7-(ロ)}
\addcontentsline{toc}{subsection}{\texorpdfstring{p29--30:7-(ロ)}{p29--30:7-(ロ)}}

\begin{tanswer}
  $ \overrightarrow{\mathrm{BA}}$と$\overrightarrow{\mathrm{BC}}$の外積は,
  \[
    \norm{(\bm{a}-\bm{b})\times(\bm{c}-\bm{b})}.
  \]
  これを$\norm{\overrightarrow{\mathrm{BC}}}$で割ればよく,求める長さは
  \[
    \frac{\norm{(\bm{a}-\bm{b})\times(\bm{c}-\bm{b})}}{\norm{\bm{c}-\bm{b}}}.
  \]
\end{tanswer}

\section*{p29--30:8}
\addcontentsline{toc}{section}{\texorpdfstring{p29--30:8}{p29--30:8}}
\begin{tproof}
  \[
    \bm{a}=\begin{pmatrix} a_1 \\ a_2 \\ a_3 \end{pmatrix},\quad \bm{b}=\begin{pmatrix} b_1 \\ b_2 \\ b_3 \end{pmatrix},\quad \bm{c}=\begin{pmatrix} c_1 \\ c_2 \\ c_3 \end{pmatrix}
  \]
  とする.このとき,
  \begin{align*}
      &
    \begin{pmatrix}
      a_1 & a_2 & a_3 \\
      b_1 & b_2 & b_3 \\
      c_1 & c_2 & c_3
    \end{pmatrix}
    \begin{pmatrix}
      a_1 & b_1 & c_1 \\
      a_2 & b_2 & c_2 \\
      a_3 & b_3 & c_3
    \end{pmatrix}
    \\
    = & \begin{pmatrix}
          {a_1}^2 +{a_2}^2 +{a_3}^2   & a_1 b_1 + a_2 b_2 + a_3 b_3 & a_1 c_1 + a_2 c_2 + a_3 c_3 \\
          b_1 a_1 + b_2 a_2 + b_3 a_3 & {b_1}^2 +{b_2}^2 + {b_3}^2  & b_1 c_1 + b_2 c_2 + b_3 c_3 \\
          c_1 a_1 + c_2 a_2 + c_3 a_3 & c_1 b_1 + c_2 b_2 + c_3 b_3 & {c_1}^2 +{c_2}^2 +{c_3}^2
        \end{pmatrix}
    \\
    = & \begin{pmatrix}
          (\bm{a},\bm{a}) & (\bm{a},\bm{b}) & (\bm{a},\bm{c}) \\
          (\bm{b},\bm{a}) & (\bm{b},\bm{b}) & (\bm{b},\bm{c}) \\
          (\bm{c},\bm{a}) & (\bm{c},\bm{b}) & (\bm{c},\bm{c})
        \end{pmatrix}
  \end{align*}
  である.

  一方,
  \begin{align*}
    \det (\bm{a},\bm{b},\bm{c}) & =
    \begin{vmatrix}
      a_1 & b_1 & c_1 \\
      a_2 & b_2 & c_2 \\
      a_3 & b_3 & c_3
    \end{vmatrix}                                                                                        \\
                                & = c_1
    \begin{vmatrix}
      a_2 & b_2 \\
      a_3 & b_3
    \end{vmatrix}
    + c_2
    \begin{vmatrix}
      a_3 & b_3 \\
      a_1 & b_1
    \end{vmatrix}
    + c_3
    \begin{vmatrix}
      a_1 & b_1 \\
      a_2 & b_2
    \end{vmatrix}
    \\
                                & = c_1 (a_2 b_3-b_2 a_3)+c_2 (a_3 b_1 - b_3 a_1)+c_3 (a_1 b_2 -b_1 a_2)   \\
                                & = a_3 (b_1 c_2 - b_2 c_1)+b_3 (c_1 a_2-c_2 a_1)+ c_3 (a_1 b_2 - b_1 a_2) \\
                                & =
    \begin{vmatrix}
      a_1 & a_2 & a_3 \\
      b_1 & b_2 & b_3 \\
      c_1 & c_2 & c_3
    \end{vmatrix}
  \end{align*}
  であるから,これと行列式の積の性質により,
  \[
    \begin{vmatrix}
      (\bm{a},\bm{a}) & (\bm{a},\bm{b}) & (\bm{a},\bm{c}) \\
      (\bm{b},\bm{a}) & (\bm{b},\bm{b}) & (\bm{b},\bm{c}) \\
      (\bm{c},\bm{a}) & (\bm{c},\bm{b}) & (\bm{c},\bm{a})
    \end{vmatrix}
    = {\det (\bm{a},\bm{b},\bm{c})}^2
  \]
  である.
\end{tproof}


\section*{p29--30:9}
\addcontentsline{toc}{section}{\texorpdfstring{p29--30:9}{p29--30:9}}

\begin{tanswer}
  $ \det (\bm{x},\bm{y},\bm{z})$は,$\bm{x}$,$\bm{y}$,$\bm{z}$の張る平行六面体の体積に符号をつけたものに等しい.
  与条件より,$\det (\bm{x},\bm{y},\bm{z})$が最大になるのは,$\bm{x}$.$\bm{y}$,$\bm{z}$の張る図形が立方体のときであり,
  そのとき
  \[
    \det (\bm{x},\bm{y},\bm{z}) =1
  \]
  である.これからただちに$\det (\bm{x},\bm{y},\bm{z})$の最小値が$-1$であることも従う.

  以上により,$\det (\bm{x},\bm{y},\bm{z})$の最大値は$1$,最小値は$-1$である.
\end{tanswer}

\section*{p29--30:10}
\addcontentsline{toc}{section}{\texorpdfstring{p29--30:10}{p29--30:10}}

\subsection*{p29--30:10-(イ)}
\addcontentsline{toc}{subsection}{\texorpdfstring{p29--30:10-(イ)}{p29--30:10-(イ)}}

\begin{tproof}
  単位ベクトル$\bm{e}_1$,$\bm{e}_2$,$\bm{e}_3$を適当にとり,
  \[
    \bm{a} = \alpha_1 \bm{e}_1,\quad \bm{b} = \beta_1 \bm{e}_1+\beta_2 \bm{e}_2,\quad \bm{c}= \gamma_1 \bm{e}_1 + \gamma_2 \bm{e}_2 + \gamma_3 \bm{e}_3
  \]
  とおく.このとき,
  \begin{align*}
    (\bm{a} \times \bm{b}) \times \bm{c} & = \alpha_1 \beta_2 \bm{e}_3 \times (\gamma_1 \bm{e}_1 + \gamma_2 \bm{e}_2 + \gamma_3 \bm{e}_3) \\
                                         & = \alpha_1 \beta_2 \gamma_1 \bm{e}_2 - \alpha_1 \beta_2 \gamma_2 \bm{e}_1                      \\
                                         & = -(\bm{b},\bm{c})\bm{a}+(\bm{a},\bm{c}) \bm{b}
  \end{align*}
  であり,これが証明すべきことであった\footnote{この等式をラグランジュの恒等式とよぶ.}.
\end{tproof}

\subsection*{p29--30:10-(ロ)}
\addcontentsline{toc}{subsection}{\texorpdfstring{p29--30:10-(ロ)}{p29--30:10-(ロ)}}

\begin{tproof}
  イ)の結果により.
  \begin{align*}
     & (\bm{a}\times\bm{b}) \times \bm{c} =  -(\bm{b},\bm{c})\bm{a}+(\bm{a},\bm{c}) \bm{b} ,    \\
     & (\bm{b} \times \bm{c} ) \times \bm{a} = -(\bm{c},\bm{a}) \bm{b} +(\bm{b},\bm{a}) \bm{c}, \\
     & (\bm{c} \times \bm{a} ) \times \bm{b} = -(\bm{a},\bm{b}) \bm{c} +(\bm{c},\bm{b}) \bm{a}.
  \end{align*}
  であるから,
  \[
    (\bm{a}\times\bm{b}) \times \bm{c} + (\bm{b} \times \bm{c} ) \times \bm{a}+(\bm{c} \times \bm{a} ) \times \bm{b} =\bm{0}
  \]
  となる.これが証明すべきことであった.
\end{tproof}