\documentclass[uplatex,dvipdfmx,a4paper,10pt,fleqn]{jsarticle}

\usepackage{cite}

%括弧
\usepackage{delimseasy}

%二段組
\usepackage{multicol}
\setlength{\columnseprule}{.5pt} %中央の線

%見出しのフォント
\renewcommand{\headfont}{\sffamily\bfseries}


%数式
\usepackage{nccmath,amsmath,amssymb}
\usepackage{mathtools}
\usepackage{empheq} %数式の囲いに使う
\usepackage{physics}
\usepackage{bm}
\usepackage[bbsets]{jkmath} %\Nなどをつかえる
\usepackage{amsthm}

\usepackage{titletoc}

\AtEndPreamble{
%ハイパーリンク用
\usepackage{url}
\usepackage{hyperref}
\usepackage{xcolor}
\definecolor{BlueViolet}{RGB}{105,39,255}
\definecolor{mylightgray}{HTML}{DDDDDD}
\definecolor{mydarkgray}{HTML}{777777}
\hypersetup{
    colorlinks=true,
    citecolor=BlueViolet,
    linkcolor=blue!50!black,
    urlcolor=blue!70!black,
  }
\usepackage{bookmark}
}


%tcolorbox系
\usepackage[many]{tcolorbox}
\tcbuselibrary{breakable,skins,theorems}
\newtcolorbox{hosoibox}[1]{colframe=black,colback=white,coltitle=black,colbacktitle=white,boxrule=0.5pt,arc=0mm,enhanced,attach boxed title to top left={xshift=10mm,yshift=-3mm},boxed title style={frame hidden},title=#1}

%leftbar環境に注釈が入れられないことを解消する環境.名前は,tcolorboxの[t]とleftbarの組み合わせ
\newtcolorbox{tbleftline}{blanker,left=5mm,borderline west={1.1mm}{0pt}{mydarkgray}}
\newenvironment{tleftbar}{\begin{tbleftline}\setlength{\parindent}{1\zw}}{\end{tbleftline}}


\usepackage{framed,color}

% leftbar環境の色を変更
\makeatletter
\renewenvironment{leftbar}{%
  \def\FrameCommand{%
    {\color{mydarkgray}\vrule width 3pt}%
    \hspace{10pt}%
    \fboxsep=\FrameSep\colorbox{white}}%
  \MakeFramed{\hsize\hsize\advance\hsize-\width\FrameRestore}%
}{%
  \endMakeFramed%
}
\makeatother

%題名付き四角
\usepackage{ascmac}
\usepackage{fancybox}

%図に使うもの
\usepackage{tikz}
\usetikzlibrary{intersections,calc,arrows.meta,3d,mindmap}
\usepackage{tikz-3dplot}
\usepackage[marginparwidth=0pt,margin=25truemm]{geometry}
\usepackage{bxpapersize}
\usepackage[absolute,overlay]{textpos} %図の配置を好きにする

%画像
\usepackage{wrapfig}
%footnoteの変更
\renewcommand\thefootnote{{\dag}\arabic{footnote}}
\renewcommand{\thempfootnote}{{\dag}\arabic{mpfootnote}}
\interfootnotelinepenalty=10000

\usepackage{oubraces} %overunderbraces

%underbraceの文字数が多いときのためのadunderbrace
\usepackage{ifthen}
\newlength{\wdTempA}
\newlength{\wdTempB}
\newcommand{\adunderbrace}[2]{%
\settowidth{\wdTempA}{$#1$}%
\settowidth{\wdTempB}{${\scriptstyle #2}$}%
\ifthenelse{\wdTempA<\wdTempB}{%
\hspace*{.5\wdTempA}\hspace*{-.5\wdTempB}%
\underbrace{#1}_{#2}%
\hspace*{.5\wdTempA}\hspace*{-.5\wdTempB}%
}{%
\underbrace{#1}_{#2}%
}%
}%
%丸付き文字
\newcommand{\ctext}[1]{\raise0.2ex\hbox{\textcircled{\scriptsize{#1}}}}

%ユーザー定義
\newcommand{\dash}[1]{#1^\prime}
\newcommand{\ddash}[1]{#1^{\prime\prime}}
\newcommand{\dddash}[1]{#1^{\prime\prime\prime}}
\newcommand{\hodash}[2]{#2^{(#1)}}
\renewcommand{\labelenumi}{(\arabic{enumi})}%itemを(数字)に変更
\newcommand{\two}{I\hspace{-1.2pt}I}
\newcommand{\three}{I\hspace{-1.2pt}I\hspace{-1.2pt}I}
\renewcommand{\proofname}{\textgt{証明}}
\renewcommand{\qed}{\unskip\nobreak\quad\qedsymbol}


\renewcommand{\leq}{\leqq}
\renewcommand{\geq}{\geqq}
\renewcommand{\le}{\leqq}
\renewcommand{\ge}{\geqq}


\newcommand\kakko[1]{\noindent\textbf{《#1》}}

\DeclareMathOperator{\sgn}{sgn}
%タイトルページを定義
\usepackage[pagecolor=white,nopagecolor={none}]{pagecolor} % 背景色を変更するためのパッケージ

\newcommand{\tituloum}[5]{\begin{titlepage} 
    \begin{center} 
        \pagecolor{white} % 背景色をBlueVioletに設定
        \color{black} % テキストカラーを白に設定
        
        \vspace*{2\baselineskip}
        
        \rule{\textwidth}{1.6pt}\vspace*{-\baselineskip}\vspace*{2pt} 
        \rule{\textwidth}{0.4pt} 
        
        \vspace{0.75\baselineskip} 
        
        {\huge #1} 
        
        \vspace{0.75\baselineskip}
        
        \rule{\textwidth}{0.4pt}\vspace*{-\baselineskip}\vspace{3.2pt}
        \rule{\textwidth}{1.6pt}
        
        \vspace{2\baselineskip}
        
        #3
        
        \vspace*{3\baselineskip}
        
        
        {\huge #2}
        
        \vspace{0.5\baselineskip}
        
        \textit{#4}
        
        \vfill
        
        \vspace{0.3\baselineskip}
        
        #5
        
    \end{center}
\end{titlepage}}

% 数式のディスプレイスタイル設定
\everymath{\displaystyle}


%箇条書き
\usepackage[shortlabels]{enumitem}

%inputなどのプリアンブルを無視
\usepackage{docmute}

\everymath{\displaystyle}


%新設定2
\definecolor{burgundy}{rgb}{0.5, 0.0, 0.13}
\tcbset{mytheo/.style={fonttitle=\gtfamily\sffamily\bfseries\upshape,
enhanced,colframe=burgundy,colback=burgundy!2!white,colbacktitle=burgundy,
boxed title style={},
attach boxed title to top left={xshift=5mm,yshift*=-\tcboxedtitleheight/2},
before skip=25pt plus 2pt,
after skip=15pt plus 2pt
}}

%Theorem
\newtcbtheorem[number within=subsection]{theorem}{Theorem}%
{mytheo}{th}
\newcommand{\thref}[1]{{\bfseries\sffamily Theorem \ref{th:#1}}}
%Proposition
\newtcbtheorem[use counter from=theorem]{prop}{Proposition}%
{mytheo}{pr}
\newcommand{\prref}[1]{{\bfseries\sffamily Proposition \ref{pr:#1}}}
%Corollary
\newtcbtheorem[use counter from=theorem]{cor}{Corollary}%
{mytheo}{co}
\newcommand{\coref}[1]{{\bfseries\sffamily Corollary \ref{co:#1}}}
% Axiom
\newtcbtheorem[use counter from=theorem]{axiom}{Axiom}%
{mytheo}{ax}
\newcommand{\axref}[1]{{\bfseries\sffamily Axiom \ref{ax:#1}}}
%Definition
\newtcbtheorem[use counter from=theorem]{definition}{Definition}%
{mytheo,
colframe=blue!50!black,colback=blue!50!black!2!white,colbacktitle=blue!50!black,borderline south={2pt}{-2pt}{blue!50!black},}{de}
\newcommand{\deref}[1]{{\bfseries\sffamily Definition \ref{de:#1}}}
%Lemma
\newtcbtheorem[use counter from=theorem]{lemma}{Lemma}%
{mytheo,
colframe=green!50!black,colback=green!50!black!2!white,colbacktitle=green!50!black,borderline south={2pt}{-2pt}{green!50!black},}{le}
\newcommand{\leref}[1]{{\bfseries\sffamily Lemma \ref{le:#1}}}
%Example
\definecolor{charcoal}{rgb}{0.21, 0.27, 0.31}
\newtcbtheorem[use counter from=theorem]{example}{Example}%
{mytheo,
colframe=charcoal,colback=charcoal!2!white,colbacktitle=charcoal,borderline south={2pt}{-2pt}{charcoal},}{ex}
\newcommand{\exref}[1]{{\bfseries\sffamily Example \ref{ex:#1}}}



\usepackage{autobreak}

\usepackage{docmute}
%ヘッダー・フッダー
\usepackage{fancyhdr}
\pagestyle{fancy}
\lhead{}
\chead{齋藤正彦・線型代数入門解答集}
\rhead{\thepage}
\cfoot{}
\begin{document}

\title{齋藤正彦・線型代数入門解答集}
\author{なまちゃん}
\date{\today}
\maketitle
\begin{multicols*}{3}
    \tableofcontents
\end{multicols*}
\addcontentsline{toc}{section}{\texorpdfstring{目次}{目次}}
\newpage
\section*{第1章}
\addcontentsline{toc}{section}{\texorpdfstring{第1章}{第1章}}

\subsection*{p5:問1}
\addcontentsline{toc}{subsection}{\texorpdfstring{p5:問1}{p5:問1}}

\begin{tleftbar}
    \begin{proof}
		線分$\mathrm{PQ}$の中点を$\mathrm{M}$とする.このとき,
		\begin{align*}
		\overrightarrow{\mathrm{OM}} & = \overrightarrow{\mathrm{OP}} + \overrightarrow{\mathrm{PM}} \\
		& = \bm{a} + \frac{\bm{b}-\bm{a}}{2} \\
		& = \frac{\bm{a}+\bm{b}}{2}
		\end{align*}
		である.
	\end{proof}
\end{tleftbar}
\subsection*{p5:問2}
\addcontentsline{toc}{subsection}{\texorpdfstring{p5:問2}{p5:問2}}
\begin{tleftbar}
	\begin{proof}
		三角形$\mathrm{PQR}$の重心を$\mathrm{G}$,$\mathrm{PQ}$の中点を$\mathrm{N}$とする.$\mathrm{G}$は線分$\mathrm{RN}$を$2:1$に内分する点なので,
		\begin{align*}
			\overrightarrow{\mathrm{OG}} &= \overrightarrow{\mathrm{OR}} + \frac{2}{3} \overrightarrow{\mathrm{RN}} \\
			& = \bm{c}+ \frac{2}{3} \left (\frac{\bm{a}+\bm{b}}{2}-\bm{c} \right) \\
			& = \frac{\bm{a}+\bm{b}+\bm{c}}{3}
		\end{align*}
		である.
	\end{proof}
\end{tleftbar}

\newpage

\subsection*{p7:問-(上)}
\addcontentsline{toc}{subsection}{\texorpdfstring{p7:問-(上)}{p7:問-(上)}}

\begin{tleftbar}
    求めるベクトルを,$\bm{x}=(x,y,z)$~(ただし$x^2+y^2 +z^2=1$)とおく.
    このとき,内積の定義により,
    \begin{align*}
        &\bm{x} \cdot 
        \begin{pmatrix}
                1 \\
                1 \\
                1
            \end{pmatrix}
        =x+y+z= 1 \cdot \sqrt{3} \cdot \cos \frac{\pi}{6} =\frac{3}{2} \\
        &\bm{x} \cdot 
            \begin{pmatrix}
                1 \\
                1 \\
                4
            \end{pmatrix}
        =x+y+4z= 1 \cdot 3\sqrt{2} \cdot \cos \frac{\pi}{4} =3
        \end{align*}
        これらの式から,
        \begin{equation*}
            \begin{pmatrix}
            x \\
            y \\
            z  
    \end{pmatrix}
    =
                \begin{pmatrix}
                \frac{2 \pm \sqrt{2}}{4} \\
                \frac{2 \mp \sqrt{2}}{4} \\
                \frac{1}{2} 
    \end{pmatrix}
            \quad (\text{複号同順})
    \end{equation*}
        である.
    \end{tleftbar}
    \subsection*{p7:問-(下)}
    \addcontentsline{toc}{subsection}{\texorpdfstring{p7:問-(下)}{p7:問-(下)}}
\begin{tleftbar}
	[1.4]の結果を利用する.

	求める三角形の面積を$S$とし.
    \[
    \bm{a}=\overrightarrow{\mathrm{P_1 P_2}}=(x_2-x_1,y_2-y_1,z_2-z_1),\quad \bm{b}=\overrightarrow{\mathrm{P_1 P_3}}=(x_3-x_1,y_3-y_1,z_3-z_1)
    \]
    とおく,
	このとき,
	\begin{align*}
		S & = \frac{1}{2} \sqrt{\norm{\overrightarrow{\mathrm{P_1 P_2}}}^2 \norm{\overrightarrow{\mathrm{P_1 P_3}}}^2 - (\overrightarrow{\mathrm{P_1 P_2}}, \overrightarrow{\mathrm{P_1 P_3}})^2} \\
		  & = \frac{1}{2} \sqrt{\norm{\bm{a}}^2 \norm{\bm{b}}^2 - (\bm{a},\bm{b})^2 } \\
		  & = \frac{1}{2} \{ \lbrack (x_2 - x_1)^2+(y_2-y_1)^2+ (z_2 - z_1)^2 \rbrack \lbrack (x_3 - x_1)^2+(y_3-y_1)^2+ (z_3 - z_1)^2 \rbrack\\
		  & \qquad - \lbrack (x_2-x_1)(x_3-x_1)+(y_2 -y_1)(y_3-y_1)+(z_2-z_1)(z_3-z_1)\rbrack ^2\}^{\frac{1}{2}}
	\end{align*}
	である,
\end{tleftbar}
	
\newpage

\subsection*{p8:問1}
\addcontentsline{toc}{subsection}{\texorpdfstring{p8:問1}{p8:問1}}

\begin{tleftbar}
	\begin{description}
        \item[イ] 
    与えられた直線を$l$とする.$l$の方程式に$x=-2$を代入すると,$y=2$となるため,$l$は点$(-2,2)$を通る.
	また,$l$の法線ベクトルのひとつは,
	$
		\begin{pmatrix}
			2 \\
			3
		\end{pmatrix}
		$なので,$l$の方向ベクトルのひとつは,
		$
		\begin{pmatrix}
			-3 \\
			2
		\end{pmatrix}
		$である.よって,$l$のベクトル表示のひとつは,
		$
		\begin{pmatrix}
			x \\
			y
		\end{pmatrix}
		= 
		\begin{pmatrix}
		-2 \\
		2
		\end{pmatrix}
		+t 
			\begin{pmatrix}
				-3 \\
				2
			\end{pmatrix}
			~(-\infty < t < \infty)$である.\\

			\item[ロ]
            与えられた直線を$l'$とする.$l '$の方向ベクトルのひとつは,$
				\begin{pmatrix}
					0 \\
					1
				\end{pmatrix}
			$である.また,$l '$は点$(3,0)$を通るので,そのベクトル表示のひとつは,
			$
				\begin{pmatrix}
					x \\
					y
				\end{pmatrix}
			= 
				\begin{pmatrix}
				3 \\
				0
				\end{pmatrix}
			+t 
				\begin{pmatrix}
					0 \\
					1
                    \end{pmatrix}
                     ~(-\infty < t < \infty)
				$となる.
                \end{description}
		\end{tleftbar}
		
\subsection*{p8:問2}
\addcontentsline{toc}{subsection}{\texorpdfstring{p8:問2}{p8:問2}}
		\begin{tleftbar}
            \begin{description}
			\item[イ]
			与えられたベクトル表示から.
			\begin{align*}
				\begin{cases}
					x=1+2t \\
					y=-1+t
				\end{cases}
			\end{align*}
			であるから,
			\begin{align*}
				\begin{cases}
					t=\frac{x-1}{2} \\
					t=y+1
				\end{cases}
			\end{align*}
			である.これから$t$を消去すると,
			\begin{gather*}
				\frac{x-1}{2} = y+1 \\
				\therefore ~ x-2y-3 =0
			\end{gather*}
			である.

			\item[ロ]
			点$(-1,-2)$を通り,$x$軸に平行な直線を表すから,$y=-2$が求める直線の方程式である.
        \end{description}
		\end{tleftbar}
%
\newpage
%
%

\subsection*{p10:問1}
\addcontentsline{toc}{subsection}{\texorpdfstring{p10:問1}{p10:問1}}

\begin{tleftbar}
\begin{align*}
    \begin{cases}
        x+2y+3z=1 \\
        3x+2y+z=-1
    \end{cases}
\end{align*}
から,
\begin{gather*}
    -2x+2z=2 \\
    \therefore \quad -x+z=1
\end{gather*}
である.このとき,$
    \begin{pmatrix}
        x \\
        z
    \end{pmatrix}
 =
    \begin{pmatrix}
        1 \\
        2
    \end{pmatrix}
,
    \begin{pmatrix}
        2 \\
        3
    \end{pmatrix}
$はこれを満たす.このときの$y$の値を計算すると,それぞれ$-3,-5$なので,結局,与えられた直線は2点$(1,-3,2),(2,-5,3)$を通る.
すなわち,この直線の方向ベクトルのひとつは
\[
\begin{pmatrix}
    2 \\
    -5 \\
    3
\end{pmatrix}
    -
        \begin{pmatrix}
            1 \\
            -3 \\
            2
        \end{pmatrix}
            =
                \begin{pmatrix}
                    1 \\
                    -2 \\
                    1
                \end{pmatrix}
                    \]
である.したがって求めるベクトル表示のひとつは,直線上の任意の位置ベクトルを$\bm{x}$とすると,
\[
    \bm{x} =
        \begin{pmatrix}
            1 \\
            -3 \\
            2
        \end{pmatrix}
            +t
            \begin{pmatrix}
                1 \\
                -2 \\
                1
            \end{pmatrix}
            \]
            と表せる.
\end{tleftbar}

\subsection*{p10:問2}
\addcontentsline{toc}{subsection}{\texorpdfstring{p10:問2}{p10:問2}}

\begin{tleftbar}
    \begin{proof}
        $t$を$0 \le t \le 1$をみたす実数,線分$\mathrm{P_1 P_2}$上の任意の点の位置ベクトルを$\bm{x}$とする.
        このとき,
        \begin{align*}
            \bm{x} & = \overrightarrow{\mathrm{O P_1}}+t\overrightarrow{\mathrm{P_1 P_2}} \\
            & = \bm{x}_1 + t (\bm{x}_2 - \bm{x}_1) \\
             &= (1-t) \bm{x}_1 + t \bm{x}_2
        \end{align*}
        である.$1-t = t_1,~t=t_2$と改めておくと,$t$の定め方から$t_1 \ge 0,~t_2 \ge 0$であり,
        \[
            \bm{x}= t_1 \bm{x}_1 + t_2 \bm{x}_2 ,\quad t_1 + t_2 =1
        \]
        となり,これが証明すべきことであった.
    \end{proof}
\end{tleftbar}

\newpage
\subsection*{p11:問1}
\addcontentsline{toc}{subsection}{\texorpdfstring{p11:問1}{p11:問1}}

\begin{tleftbar}
    与えられた平面を$(S)$とおく.$(S)$は3点$(-1,0,1),~(2,0,-1),~(0,-1,0)$を通るので,
    \begin{equation*}
   \bm{x}_1=
        \begin{pmatrix}
            -1 \\
            0 \\
            1
        \end{pmatrix}
   ,\quad \bm{x}_2=
    \begin{pmatrix}
        2 \\
        0 \\
        -1
    \end{pmatrix}
 \quad 
\bm{x}_3=
    \begin{pmatrix}
        0 \\
        -1 \\
        0
    \end{pmatrix}
\end{equation*}
と改めておくと,
\begin{gather*}
    \bm{x}_2 - \bm{x}_1 =
        \begin{pmatrix}
            3 \\
            0 \\
            -1
        \end{pmatrix}
    ,\quad 
    \bm{x}_3 - \bm{x}_1 =
    \begin{pmatrix}
        1 \\
        -1 \\
        -1
    \end{pmatrix}
\end{gather*}
となり,$\bm{x}_2 - \bm{x}_1$と$\bm{x}_3 - \bm{x}_1$は線型独立なので,求めるベクトル表示のひとつは,
\[
    (S) \colon \bm{x}=
        \begin{pmatrix}
            -1 \\
            0 \\
            1
        \end{pmatrix}
    + t 
        \begin{pmatrix}
            3 \\
            0 \\
            -1
        \end{pmatrix}
    +s
        \begin{pmatrix}
            1 \\
            -1 \\
            -1
        \end{pmatrix}
        ~( -\infty < t,~s<\infty)
        \]
    \end{tleftbar}
%
%
%
%
\subsection*{p12:問2}
\addcontentsline{toc}{subsection}{\texorpdfstring{p12:問2}{p12:問2}}

\begin{tleftbar}
    \begin{align*}
        \begin{cases}
            x=1+t-s \\
            y=2-t -2s \\
            z=0+2t+s
        \end{cases}
    \end{align*}
    から$t$と$s$を消去して,
    \[
        x-y-z=-1
    \]
    これが求める直線の方程式である.
\end{tleftbar}

\subsection*{p12:問3}
\addcontentsline{toc}{subsection}{\texorpdfstring{p12:問3}{p12:問3}}

\begin{tleftbar}
    \begin{proof}
        \[
            \overrightarrow{\mathrm{OP_1}}=\bm{x}_1,\quad \overrightarrow{\mathrm{OP_2}}=\bm{x}_2,\quad \overrightarrow{\mathrm{OP_3}}=\bm{x}_3
        \]
        とする.このとき,三角形$\mathrm{P_1 P_2 P_3}$上の任意の点の位置ベクトルを$\bm{x}$,$s,t$を$0 \le s,t \le 1$を満たす実数とすると,
        \begin{gather*}
            \bm{x}=\bm{x}_1 + s(\bm{x}_2 - \bm{x}_1) + t (\bm{x}_3 - \bm{x}_1) \\
            \therefore \quad \bm{x} = (1-s-t)\bm{x}_1 + s\bm{x}_2 + t \bm{x}_3
        \end{gather*}
        となり,$1-s-t=t_1,~s=t_2,~t=t_3$と改めて書き直すと,$s,t$の定め方より,$0 \le t_1 ,t_2,t_3 \le 1$であり
        \[
            \bm{x} = t_1\bm{x}_1 + t_2\bm{x}_2 + t_3 \bm{x}_3,\quad t_1+t_2+t_3=1
        \]
        となる.これが証明すべきことであった.
    \end{proof}
\end{tleftbar}
%
\newpage
%
\subsection*{p13:問1}
\addcontentsline{toc}{subsection}{\texorpdfstring{p13:問1}{p13:問1}}
%
\begin{tleftbar}
    $(S_1)$,$(S_2)$の法線ベクトルをそれぞれ$\bm{x}_1$,$\bm{x}_2$とおくと,
    \begin{gather*}
        \bm{x}_1 =
            \begin{pmatrix}
                1\\
                1\\
                2
            \end{pmatrix}
        ,
        \bm{x}_2 =
            \begin{pmatrix}
                3\\
                3\\
                0
            \end{pmatrix}
        \end{gather*}
        である.ゆえに,交角を$\theta ~(0 \le \theta \le \frac{\pi}{2})$とすると,
        \[
            \cos \theta = \frac{\bm{x}_1 \cdot \bm{x}_2}{\norm{\bm{x}_1} \norm{\bm{x}_2}}=\frac{3}{3\sqrt{2}}=\frac{1}{\sqrt{2}}
        \]
        であるから,$0 \le \theta \le \frac{\pi}{2}$より$\theta =\frac{\pi}{4}$である.
    \end{tleftbar}
%
\subsection*{p13:問2}
\addcontentsline{toc}{subsection}{\texorpdfstring{p13:問2}{p13:問2}}

\begin{tleftbar}
    \begin{proof}
    平面$\pi_1$,$\pi_2$を考え,$\pi_1$,$\pi_2$の法線ベクトルをそれぞれ$\bm{n}_1$,$\bm{n}_2$とおく.
    \begin{description}
        \item[$\bm{n}_1$と$\bm{n}_2$が平行なとき]
        $\pi_1$に垂直な平面は$\pi_2$にも垂直であり,このような平面を$\pi_3$とすると,
        $\pi_3$は$\bm{n}_1$,$\bm{n}_2$と平行である.よって$\pi_3$と$\pi_1$,$\pi_2$はそれぞれ並行であり,このような平面は確かに存在する.
        \item[$\bm{n}_1$と$\bm{n}_2$が平行でないとき]
        $\bm{n}_1 , \bm{n}_2 \ne \bm{0}$は明らかなので,$\bm{n}_3 \coloneqq \bm{n}_1 \times \bm{n}_2$とすると,
        $\bm{n}_3 \ne \bm{0}$である.よって,$\bm{n}_3$は$\pi_1$,$\pi_2$に垂直である.このとき$n_3$を法線ベクトルとする平面を取ればよい.
    \end{description}
    以上の考察により証明された.
\end{proof}
\end{tleftbar}
\newpage
%
%
%
\subsection*{p18:問}
\addcontentsline{toc}{subsection}{\texorpdfstring{p18:問}{p18:問}}

\begin{tleftbar}
    \begin{proof}
        $A$,$B$,$C$が$2 \times 2$行列の場合を証明する.
        \begin{gather*}
            A=
            \begin{pmatrix}
                a & b \\
                c & d
            \end{pmatrix}
            ,
            B=
            \begin{pmatrix}
                e & f \\
                g & h
            \end{pmatrix}
            ,C=
            \begin{pmatrix}
                i & j \\
                k & l
            \end{pmatrix}
        \end{gather*}
        とし,$A$,$B$,$C$の成分はすべて複素数であるとする.このとき,
        \begin{align*}
            (AB)C & =
            \begin{pmatrix}
                ae+bg & af+bh \\
                ce+dg & cf+dh
            \end{pmatrix}
            \begin{pmatrix}
                i & j \\
                k & l
            \end{pmatrix}
            \\
            & =
            \begin{pmatrix}
                aei +bgi +afk +bhk & aej+bgj+afl+bhl \\
                cei +dgi+cfk +dhk & cej +dgj + cfl +dhl
            \end{pmatrix}
        \end{align*}
            となる.他方
            \begin{align*}
                A(BC)& =
                \begin{pmatrix}
                    a & b \\
                    c & d
                \end{pmatrix}
                \begin{pmatrix}
                    ei + fk & ej +fl \\
                    gi + hk & gj +hl 
                \end{pmatrix}
                \\
                & =
                \begin{pmatrix}
                    aei + afk +bgi +bhk & aej +afl +bgj +bhl \\
                    cei + cfk + dgi +dhk & cej + cfl + dgi +dhl
                \end{pmatrix}
            \end{align*}
            となり,たしかに$(AB)C=A(BC)$である.
        \end{proof}
    \end{tleftbar}
%
\newpage
\subsection*{p19:問1-(上)}
\addcontentsline{toc}{subsection}{\texorpdfstring{p19:問1}{p19:問1}}
%
\begin{tleftbar}
	\begin{proof}
		\[
			\begin{pmatrix}
				-1 & 0\\
				0 & -1
			\end{pmatrix}
				\begin{pmatrix}
					x \\
					y 
				\end{pmatrix}
			=
				\begin{pmatrix}
					-x \\
					-y 
				\end{pmatrix}
			\]
			となり,これは明らかに線型変換である.対応する行列は,$
			\begin{pmatrix}
				-1 & 0 \\
				0 & -1
			\end{pmatrix}
			$である.
		\end{proof}
	\end{tleftbar}

    \subsection*{p19:問2-(上)}
    \addcontentsline{toc}{subsection}{\texorpdfstring{p19:問2}{p19:問2}}
	\begin{tleftbar}
		\begin{proof}
			式(15)より,$2 \times 2$行列$A$,$B$とベクトル$\bm{x}$について,
			\begin{align*}
				T_B (T_A (\bm{x})) &= B(A\bm{x}) \\
				& = (BA) \bm{x} \\
				& = T_{BA} (\bm{x})
			\end{align*}
			である.これが証明すべきことであった.
		\end{proof}
    \end{tleftbar}


    \subsection*{p19:問1-(下)}
    \addcontentsline{toc}{subsection}{\texorpdfstring{p19:問1-(下)}{p19:問1-(下)}}

\begin{tleftbar}
$\bm{x} =
\begin{pmatrix}
x \\
y
\end{pmatrix}
$
とおくと,(17)式より,
\begin{align*}
T \bm{x} &= \frac{ax+by}{a^2+b^2} \bm{a} \\
&=
\begin{pmatrix}
a^2x +aby \\
ab x + b^2y \\
\end{pmatrix}
\\
&=
\begin{pmatrix}
a^2 & ab \\
ab & b^2
\end{pmatrix}
\begin{pmatrix}
x \\
y
\end{pmatrix}
 \\
&= 
\begin{pmatrix}
a^2 & ab \\
ab & b^2
\end{pmatrix}
\bm{x}
\end{align*}
であるから,
\[
T=\begin{pmatrix}
a^2 & ab \\
ab & b^2
\end{pmatrix}
\]
となる.
\end{tleftbar}
%
\newpage

\subsection*{p19:問2-(下)}
\addcontentsline{toc}{subsection}{\texorpdfstring{p19:問2-(下)}{p19:問2-(下)}}
\begin{leftbar}
\begin{description}
    \item[イ] 
    \begin{proof}
    $\bm{a}=
        \begin{pmatrix}
            a_1 \\
            a_2
        \end{pmatrix}
    $,
    $\bm{b}=
        \begin{pmatrix}
            b_1 \\
            b_2
        \end{pmatrix}
    $,$\bm{a} \ne \bm{0}$かつ$\bm{b} \ne \bm{0}$とする.
    このとき,
    \begin{align*}
        T \bm{x} &=\frac{(\bm{a},\bm{x})}{(\bm{a},\bm{a})} \bm{a} \\
        & = \frac{a_1 x + a_2 y}{{a_1}^2+{a_2}^2} 
        \begin{pmatrix}
            a_1 \\
            a_2
        \end{pmatrix}
        \\
        & =
        \frac{1}{{a_1}^2+{a_2}^2}
        \begin{pmatrix}
            {a_1}^2 & a_1 a_2 \\
            a_1 a_2 & {a_2}^2
        \end{pmatrix}
        \begin{pmatrix}
            x \\
            y
        \end{pmatrix}
    \end{align*}
    となる.つまり,$T=\frac{1}{{a_1}^2+{a_2}^2}
    \begin{pmatrix}
        {a_1}^2 & a_1 a_2 \\
        a_1 a_2 & {a_2}^2
    \end{pmatrix}
    $である.このとき,
    \begin{align*}
        T^2 &= \frac{1}{({a_1}^2+{a_2}^2)^2}
        \begin{pmatrix}
            {a_1}^4 + {a_1}^2 {a_2}^2 & {a_1}^3 a_2 + a_1 {a_2}^3 \\
            {a_1}^3 a_2 + a_1 {a_2}^3 & {a_2}^4 + {a_1}^2 {a_2}^2
        \end{pmatrix}
        \\
        &= \frac{1}{{a_1}^2+{a_2}^2}
        \begin{pmatrix}
            {a_1}^2 & a_1 a_2 \\
            a_1 a_2 & {a_2}^2
        \end{pmatrix}
        =T
    \end{align*}
    となり,$T^2=T$である.$S^2=S$も同様にして示される.
\end{proof}
\item[ロ]
    \begin{proof}
        $\bm{a}=
        \begin{pmatrix}
            a_1 \\
            a_2
        \end{pmatrix}
        $,$\bm{b}=
        \begin{pmatrix}
            b_1 \\
            b_2
        \end{pmatrix}
        $とする.このとき,$\bm{a}$と$\bm{b}$が直交することから,
        \begin{gather*}
            \bm{a} \cdot \bm{b}=0 \\
            \therefore ~a_1 b_1 + a_2 b_2 =0
        \end{gather*}
        である.ここで,
        \begin{align*}
            TS & = \frac{1}{({a_1}^2 +{a_2}^2)}
            \begin{pmatrix}
                {a_1}^2 & a_1 a_2 \\
                a_1 a_2 & {a_2}^2
            \end{pmatrix}
            \frac{1}{({b_1}^2 +{b_2}^2)}
            \begin{pmatrix}
                {b_1}^2 & b_1 b_2 \\
                b_1 b_2 & {b_2}^2
            \end{pmatrix}
            \\
            &=\frac{a_1 b_1 + a_2 b_2}{({a_1}^2 +{a_2}^2)({b_1}^2 +{b_2}^2)}
            \begin{pmatrix}
                a_1 b_1 & a_1 b_2 \\
                a_2 b_1 & a_2 b_2
            \end{pmatrix}
            =O \\
            & \qquad \qquad  (\because ~ a_1 b_1 + a_2 b_2 =0)
        \end{align*}
        である.同様に$ST$を計算すると,$ST=O$であることもわかり,これで$TS=ST=O$が証明された.
    \end{proof}
\item[ハ]
    \begin{proof}
    イ),ロ)の文字や結論を用いると,
        \begin{align*}
        T \bm{x} + S \bm{x} & =
        \frac{1}{{a_1}^2+{a_2}^2}
        \begin{pmatrix}
            {a_1}^2 & a_1 a_2 \\
            a_1 a_2 & {a_2}^2
        \end{pmatrix}
        \begin{pmatrix}
            x \\
            y
        \end{pmatrix}
        +
        \frac{1}{{b_1}^2+{b_2}^2}
        \begin{pmatrix}
            {b_1}^2 & b_1 b_2 \\
            b_1 b_2 & {b_2}^2
        \end{pmatrix}
        \begin{pmatrix}
            x \\
            y
        \end{pmatrix}
        \\
        & = \frac{1}{({a_1}^2+{a_2}^2)({b_1}^2+{b_2}^2)}
        \begin{pmatrix}
            ({a_1}^2+{a_2}^2)({b_1}^2+{b_2}^2) & ({a_1}^2+{a_2}^2)({b_1}^2+{b_2}^2) \\
            ({a_1}^2+{a_2}^2)({b_1}^2+{b_2}^2) & ({a_1}^2+{a_2}^2)({b_1}^2+{b_2}^2)
        \end{pmatrix}
        \begin{pmatrix}
            x \\
            y
        \end{pmatrix}
        \\
        & =\begin{pmatrix}
            x \\
            y
        \end{pmatrix}
        =\bm{x}
    \end{align*}
    となる.これが証明すべきことであった.
\end{proof}
\end{description}
\end{leftbar}

\newpage 

\subsection*{p22:問1}
\addcontentsline{toc}{subsection}{\texorpdfstring{p22:問1}{p22:問1}}
\begin{tleftbar}
\begin{description}
    \item[イ] 
    \[
        \begin{pmatrix}
            -1 & 0 & 0 \\
            0 & 1 & 0 \\
            0 & 0 & -1
        \end{pmatrix}
        \begin{pmatrix}
            x \\
            y \\
            z
        \end{pmatrix}
        =\begin{pmatrix}
            -x \\
            y \\
            -z
        \end{pmatrix}
        \]
        となり,これは$y$軸に関する対象点に移す変換を表す.
\item[ロ]
        \[
        \begin{pmatrix}
            1 & 0 & 0 \\
            0 & \cos \alpha & -\sin \alpha \\
            0 & \sin \alpha & \cos \alpha
        \end{pmatrix}
        \begin{pmatrix}
            x \\
            y \\
            z
        \end{pmatrix}
        =
        \begin{pmatrix}
            x \\
            y \cos \alpha -z \sin \alpha \\
            y \sin \alpha + z \cos \alpha 
        \end{pmatrix}
        \]
        となり,これは$x$軸まわりに角$\alpha$だけ回転する変換を表す.
    \item[ハ]
        \[
            \begin{pmatrix}
                0 & 1 & 0 \\
                0 & 0 & 1 \\
                1 & 0 & 0
            \end{pmatrix}
            \begin{pmatrix}
                x \\
                y \\
                z
            \end{pmatrix}
            =\begin{pmatrix}
                y \\
                z \\
                x
            \end{pmatrix}
            \]
        \end{description}
        \end{tleftbar}


        \setcounter{equation}{0}

\section*{第1章・章末問題}
\addcontentsline{toc}{section}{\texorpdfstring{第1章・章末問題}{第1章・章末問題}}


\subsection*{p29-30:1}
\addcontentsline{toc}{subsection}{\texorpdfstring{p29-30:1}{p29-30:1}}

\begin{leftbar}
    \begin{proof}
    四面体$\mathrm{P_1 P_2 P_3 P_4}$を考える.三角形$\mathrm{P_2 P_3 P_4}$の任意の周および内部の点を$\mathrm{T}$とする.
    $0 \leqq k \leqq 1$,$0 \leqq s \leqq 1$をみたす$k,s \in \mathbb{R}$によって
    \begin{align*}
        \overrightarrow{\mathrm{P_2 T}} & = k \{ s \overrightarrow{\mathrm{P_2 P_3}} + (1-s) \overrightarrow{\mathrm{P_2 P_4}} \} \\
        & = ks(\bm{x}_3 -\bm{x}_2) + k(1-s) (\bm{x}_4-\bm{x}_2) \\
        & = -k\bm{x}_2 + ks \bm{x}_3 + k(1-s) \bm{x}_4
    \end{align*}
    と表される.

    さて,線分$\mathrm{P_1 T}$上の任意の点を$\mathrm{Q}$とすると,$0 \leqq t \leqq 1$をみたす$t \in \mathbb{R}$によって

    \begin{align*} 
        \overrightarrow{\mathrm{P_1 Q}} & =t \overrightarrow{\mathrm{P_1 T}}\\
        & = t\overrightarrow{\mathrm{P_2 T}} - t\overrightarrow{\mathrm{P_2 P_1}} \\
        & = t (-k\bm{x}_2 + ks \bm{x}_3 + k(1-s) \bm{x}_4)-t(\bm{x}_1 -\bm{x}_2) \\
        & = -t\bm{x}_1 +(t-kt) \bm{x}_2 + kst \bm{x}_3 +kt(1-s) \bm{x}_4 
    \end{align*} 
    と表されるから,$k=4$のときの求める位置ベクトルは.
    \begin{align*}
        \bm{x} & = \bm{x}_1 + \overrightarrow{\mathrm{P_1 Q}} \\
        & = (1-t) \bm{x}_1 +(t-kt)\bm{x}_2 +kst \bm{x}_3 +kt(1-s) \bm{x}_4 
    \end{align*} 
    となり,
    \[
        (1-t)+ (t-kt)+kst + kt(1-s)=1
    \]
    であるから,$1-t = t_1$,$t-kt =t_2$,$kst = t_3$,$kt(1-s)=t_4$とおくと,
    \[
        \bm{x}= t_1 \bm{x}_1 + t_2 \bm{x}_2 + t_3 \bm{x}_3 + t_4 \bm{x}_4 , \quad t_1, t_2 ,t_3 , t_4 \geqq 0 ,\quad  t_1 +t_2 + t_3 + t_4 =1
    \]
    となり,ここまでで$k=4$の場合が示された.

    ここで,$n \geqq 4$として$k=n$のときに主張が成り立つと仮定する.
    このとき,
    \[
        t_1 \bm{x}_1 + t_2 \bm{x}_2+\dots+ t_n \bm{x}_n
    \]
    は仮定により多面体$\{ \mathrm{P}_n \}$の内部の点であり,これを簡単のために$\bm{X}_n$とおく.

    さて,$\{ \mathrm{P}_n \}$の点と$\mathrm{P}_{n+1}$とを結ぶ線分上の点は,$ 0 \leqq l \leqq 1$をみたす$l \in \mathbb{R}$によって,
    \[
        l \overbrace{(  t_1 \bm{x}_1 + t_2 \bm{x}_2+\dots+ t_n \bm{x}_n)}^{\bm{X}_n}+(1-l) \bm{x}_{n+1} , \quad t_1+t_2+\dots + t_n =1 
    \]
    とかける.ここで,
    \[
        l(t_1+t_2+\dots+t_n)+(1-l)=1
    \]
    なので,$\{ \mathrm{P}_n \}$の点と$\mathrm{P}_{n+1}$とを結ぶ線分上の点はこのように表せる.

    逆に,
    \[
        \bm{x} = t_1 \bm{x}_1 + t_2 \bm{x}_2 + \dots + t_n \bm{x}_n + t_{n+1} \bm{x}_{n+1} , \quad t_1, t_2 ,\dots , t_n,t_{n+1} \geqq 0 ,\quad  t_1 +t_2 + \dots+t_n + t_{n+1} =1
    \]
    としたとき,
    \begin{align*}
        \bm{x} &=\frac{t_1 \bm{x}_1 + t_2 \bm{x}_2+\dots+ t_n \bm{x}_n}{t_1+t_2+\dots+t_n} \cdot (t_1+t_2+\dots+t_n) +t_{n+1} \bm{x}_{n+1} \\
        & = T_n \bm{X}_n + t_{n+1} \bm{x}_{n+1}
    \end{align*}
    と変形できる.ただし$T_n = t_1 + t_2 +\dots+t_n$とおいた.
    
    さて
    \[
        \frac{\bm{X}_n }{T_n} = \frac{t_1 \bm{x}_1 }{T_n}+ \frac{t_2 \bm{x}_2 }{T_n}+\dots +  \frac{t_n \bm{x}_n }{T_n}  \\
    \]
    であることと
    \begin{align*}
        \frac{t_1}{T_n}+\frac{t_2 }{T_n} +\dots +\frac{t_n }{T_n}& = \frac{T_n}{T_n} \\
        & = 1
    \end{align*}
    であることにより,
    \[
        \frac{\bm{X}_n}{T_n}
    \]
    は,多面体$\{ \mathrm{P}_n \}$の内部の点であり.
    \[
        T_n \cdot \frac{\bm{X}_n}{T_n} + t_{n+1} \bm{x}_{n+1}
    \]
    は多面体$\{ \mathrm{P}_n \}$の内部の点と$\mathrm{P}_{n+1}$を結ぶ線分上の点である.

    よって,$k=n$のときも問題の主張が成り立つ.

    以上の考察により証明された.
\end{proof}
\end{leftbar}

\subsection*{p29-30:2}
\addcontentsline{toc}{subsection}{\texorpdfstring{p29-30:2}{p29-30:2}}

\begin{tleftbar}
    \begin{proof}
    2点$\mathrm{P}_1$,$\mathrm{P_2}$を通る直線の方程式を$ax+by+c=0$(ただし$(a,b)=0$)とおく.
    このとき,
    \[
        \begin{cases}
            ax+by+c =0 \\
            ax_1 + by_1 +c=0 \\
            ax_2 + by_2 +c =0
        \end{cases}
    \]
    が成立する.すなわちこれは
    \[
        \begin{pmatrix} 
            x & y & 1 \\
            x_1 & y_1 & 1 \\
            x_2 & y_2 & 1 
        \end{pmatrix}
        \begin{pmatrix}
            a \\
            b \\
            c
        \end{pmatrix}
        = \bm{0}
    \]
    をみたす.これを$a$,$b$,$c$についての連立方程式とみたとき,与条件により自明でない解があり,
    \[
    \begin{vmatrix} 
        x & y & 1 \\
        x_1 & y_1 & 1 \\
        x_2 & y_2 & 1 
    \end{vmatrix}
    =0
    \]
    が成立する.転置行列の行列式はもとの行列の行列式に等しいので,行列式の交代性なども用いて,
    \[ 
    \begin{vmatrix} 
        1 & 1 & 1 \\
        x & x_1 & x_2 \\
        y & y_1 & y_2 
    \end{vmatrix}
    =0
    \]
を得る.これが証明すべきことであった.
\end{proof}
\end{tleftbar}



\subsection*{p29-30:3}
\addcontentsline{toc}{subsection}{\texorpdfstring{p29-30:3}{p29-30:3}}

\begin{tleftbar}
        点を以下の順で移動させる変換を考える.
        \begin{enumerate}
            \item 原点中心に$-\theta$回転させる.
            \item $x$軸に関して対称移動させる.
            \item 原点中心に$\theta$回転させる.
        \end{enumerate}
        ここで,(1)から(3)までの変換を表す行列をそれぞれ$R_{-\theta}$,$A_{x}$,$R_{\theta}$とすると.
        \begin{align*} 
            & R_{-\theta} = \begin{pmatrix} \cos \theta & \sin \theta \\ -\sin \theta & \cos \theta \end{pmatrix} , \\
            & A_{x} = \begin{pmatrix} 1 & 0 \\ 0 & -1 \end{pmatrix} ,\\
            & R_{\theta} = \begin{pmatrix} \cos \theta & -\sin \theta \\ \sin \theta & \cos \theta \end{pmatrix} .
        \end{align*} 
        となる.よって,この変換を表す行列は
        \begin{align*}
            R_{\theta} A_x R_{-\theta} &=\begin{pmatrix} \cos \theta & -\sin \theta \\ \sin \theta & \cos \theta \end{pmatrix}
            \begin{pmatrix} 1 & 0 \\ 0 & -1 \end{pmatrix}
            \begin{pmatrix} \cos \theta & \sin \theta \\ -\sin \theta & \cos \theta \end{pmatrix} \\
            & = \begin{pmatrix} \cos ^2 \theta - \sin ^2 \theta & 2\sin \theta \cos \theta \\ 2\sin \theta \cos \theta & \sin ^2 \theta -\cos ^2 \theta \end{pmatrix} \\
            & = \begin{pmatrix} \cos 2 \theta & \sin 2 \theta \\ \sin 2\theta & -\cos 2 \theta \end{pmatrix}
        \end{align*}
        である.
\end{tleftbar}


\subsection*{p29-30:4}
\addcontentsline{toc}{subsection}{\texorpdfstring{p29-30:4}{p29-30:4}}

\begin{tleftbar}
    \begin{proof}
        以下では,直線$y= \tan \theta$に関する折り返しを$T_{\theta}$とかくことにする.

        さて,直線$ y = \tan (\theta /4) x$に関する折り返しは,
        \[
            T_{\theta/4} = \begin{pmatrix} \cos (\theta /2) & \sin (\theta /2) \\  \sin (\theta /2) & -\cos (\theta /2) \end{pmatrix}
        \]
        で表される.

        また,直線$y = \tan (-\theta /4)x$に関する折り返しは.
        \[
            T_{-\theta/4} = \begin{pmatrix} \cos (\theta /2) & -\sin (\theta/2) \\ -\sin (\theta/2) & -\cos (\theta /2) \end{pmatrix}
        \]
        で表される.

        ここで,
        \begin{align*} 
            T_{\theta/4} T_{-\theta/4} & = \begin{pmatrix} \cos (\theta /2) & \sin (\theta /2) \\  \sin (\theta /2) & -\cos (\theta /2) \end{pmatrix} \begin{pmatrix} \cos (\theta /2) & -\sin (\theta/2) \\ -\sin (\theta/2) & -\cos (\theta /2) \end{pmatrix} \\
            & =\begin{pmatrix} \cos ^2 (\theta /2)-\sin ^2 (\theta/2) & -2\sin (\theta/2) \cos (\theta/2) \\ 2\sin (\theta/2) \cos (\theta/2) & \cos ^2 (\theta/2)-\sin ^2 (\theta/2) \end{pmatrix} \\
            & = \begin{pmatrix} \cos \theta & -\sin \theta \\ \sin \theta & \cos \theta \end{pmatrix}
        \end{align*} 
        となり,これは原点のまわりに$\theta$回転する行列を表す.

        以上の考察により証明された.
    \end{proof}
\end{tleftbar}

\subsection*{p29-30:5}
\addcontentsline{toc}{subsection}{\texorpdfstring{p29-30:5}{p29-30:5}}

\begin{tleftbar}
    任意の点$\mathrm{P}(\bm{p}),~\bm{p} \in \mathbb{R}^3$を平面$(\bm{a},\bm{x})$に対して折り返すことを考える.

    点$\mathrm{P}$から$(\bm{a},\bm{x})$におろした垂線の足は,$t \in \mathbb{R}$を用いて
    \[
        \bm{p} + t \frac{\bm{a}}{(\bm{a},\bm{a})}
    \]
    と表せ,これが平面$(\bm{a},\bm{x})$上にあるので,
    \begin{align*} 
        & (\bm{a},p+t\frac{\bm{a}}{(\bm{a},\bm{a})})=0 \\
        \therefore ~ & t=- (\bm{a},\bm{p}) 
    \end{align*} 
    である.
    
    また,求める点を$\mathrm{P}' (\bm{p}')$とすると,
    \begin{align*}
        \bm{p}' &= \bm{p}+t \frac{2\bm{a}}{(\bm{a},\bm{a})} \\
        & = \bm{p}-\frac{2(\bm{a},\bm{p})}{(\bm{a},\bm{a})} \bm{a}
    \end{align*}
    であるから,これはたしかに$V^3$の線型変換を引き起こし,その変換公式は
    \[
        \bm{x} \mapsto \bm{x}-\frac{2(\bm{a},\bm{x})}{(\bm{a},\bm{a})} \bm{a}
    \]
    である,
\end{tleftbar}

\subsection*{p29-30:7-(1)}
\addcontentsline{toc}{subsection}{\texorpdfstring{p29-30:7-(1)}{p29-30:7-(1)}}

\begin{tleftbar}
$\bm{a}$,$\bm{b}$,$\bm{c}$が張る平行六面体の体積は,
\[
    \abs{\det (\bm{a},\bm{b},\bm{c})}
\]
で与えられる.

一方,この平行六面体の$\mathrm{O}$,$\mathrm{B}$,$\mathrm{C}$を含む面の面積は,

\[
    \norm{\bm{b} \times \bm{c}}
\]

で与えられる.

以上の考察により,求める長さは,
\[
    \frac{\abs{\det(\bm{a},\bm{b},\bm{c})}}{\norm{\bm{b} \times \bm{c}}}
\]
である.
\scalebox{0.6}[0.6]{
\begin{tikzpicture}[x=0.75pt,y=0.75pt,yscale=-1,xscale=1]
    %uncomment if require: \path (0,300); %set diagram left start at 0, and has height of 300
    
    %Straight Lines [id:da9042715539337772] 
    \draw [color={rgb, 255:red, 74; green, 144; blue, 226 }  ,draw opacity=1 ]   (150,250) -- (246.89,105.66) ;
    \draw [shift={(248,104)}, rotate = 123.87] [color={rgb, 255:red, 74; green, 144; blue, 226 }  ,draw opacity=1 ][line width=0.75]    (13.12,-3.95) .. controls (8.34,-1.68) and (3.97,-0.36) .. (0,0) .. controls (3.97,0.36) and (8.34,1.68) .. (13.12,3.95)   ;
    %Straight Lines [id:da29183874211574523] 
    \draw [color={rgb, 255:red, 144; green, 19; blue, 254 }  ,draw opacity=1 ]   (150,250) -- (348,250) ;
    \draw [shift={(350,250)}, rotate = 180] [color={rgb, 255:red, 144; green, 19; blue, 254 }  ,draw opacity=1 ][line width=0.75]    (10.93,-3.29) .. controls (6.95,-1.4) and (3.31,-0.3) .. (0,0) .. controls (3.31,0.3) and (6.95,1.4) .. (10.93,3.29)   ;
    %Straight Lines [id:da3012672762383153] 
    \draw [color={rgb, 255:red, 208; green, 2; blue, 27 }  ,draw opacity=1 ] [dash pattern={on 4.5pt off 4.5pt}]  (150,250) -- (250.21,200.39) ;
    \draw [shift={(252,199.5)}, rotate = 153.66] [color={rgb, 255:red, 208; green, 2; blue, 27 }  ,draw opacity=1 ][line width=0.75]    (10.93,-3.29) .. controls (6.95,-1.4) and (3.31,-0.3) .. (0,0) .. controls (3.31,0.3) and (6.95,1.4) .. (10.93,3.29)   ;
    %Straight Lines [id:da7013310145085282] 
    \draw    (248,104) -- (451,102.5) ;
    %Straight Lines [id:da9417424223698198] 
    \draw    (451,102.5) -- (350,250) ;
    %Straight Lines [id:da10714474623977321] 
    \draw  [dash pattern={on 4.5pt off 4.5pt}]  (247,202.5) -- (452,199.5) ;
    %Straight Lines [id:da9346887424461314] 
    \draw    (452,199.5) -- (350,250) ;
    %Straight Lines [id:da9143571217608703] 
    \draw    (451,102.5) -- (551,51.5) ;
    %Straight Lines [id:da1037466616997732] 
    \draw    (248,104) -- (348,50.5) ;
    %Straight Lines [id:da921854997991198] 
    \draw    (551,51.5) -- (452,199.5) ;
    %Straight Lines [id:da6285253534318218] 
    \draw  [dash pattern={on 4.5pt off 4.5pt}]  (348,50.5) -- (252,199.5) ;
    %Straight Lines [id:da23538555105346293] 
    \draw    (348,50.5) -- (551,51.5) ;
    %Straight Lines [id:da381755385753077] 
    \draw [color={rgb, 255:red, 65; green, 117; blue, 5 }  ,draw opacity=1 ][fill={rgb, 255:red, 65; green, 117; blue, 5 }  ,fill opacity=1 ]   (250,105) -- (250,235) ;
    %Shape: Right Angle [id:dp3981791931061768] 
    \draw  [color={rgb, 255:red, 65; green, 117; blue, 5 }  ,draw opacity=1 ][fill={rgb, 255:red, 255; green, 255; blue, 255 }  ,fill opacity=1 ] (235,235) -- (235,220) -- (250,220) ;
    
    % Text Node
    \draw (125,245) node [anchor=north west][inner sep=0.75pt]   [align=left] {$\mathrm{O}$};
    % Text Node
    \draw (225,80) node [anchor=north west][inner sep=0.75pt]   [align=left] {$\mathrm{A}$};
    % Text Node
    \draw (275,175) node [anchor=north west][inner sep=0.75pt]   [align=left] {$\mathrm{B}$};
    % Text Node
    \draw (350,260) node [anchor=north west][inner sep=0.75pt]   [align=left] {$\mathrm{C}$};
    \end{tikzpicture}
}
\end{tleftbar}


\subsection*{p29-30:8}
\addcontentsline{toc}{subsection}{\texorpdfstring{p29-30:8}{p29-30:8}}
\begin{tleftbar}
    \begin{proof}
    \[
    \bm{a}=\begin{pmatrix} a_1 \\ a_2 \\ a_3 \end{pmatrix},\quad \bm{b}=\begin{pmatrix} b_1 \\ b_2 \\ b_3 \end{pmatrix},\quad \bm{c}=\begin{pmatrix} c_1 \\ c_2 \\ c_3 \end{pmatrix}
    \]
    とする.このとき,
    \begin{align*}
        &
        \begin{pmatrix}
            a_1 & a_2 & a_3 \\
            b_1 & b_2 & b_3 \\
            c_1 & c_2 & c_3
        \end{pmatrix}
        \begin{pmatrix}
            a_1 & b_1 & c_1 \\
            a_2 & b_2 & c_2 \\
            a_3 & b_3 & c_3
        \end{pmatrix}
        \\
         =& \begin{pmatrix}
            {a_1}^2 +{a_2}^2 +{a_3}^2 & a_1 b_1 + a_2 b_2 + a_3 b_3 & a_1 c_1 + a_2 c_2 + a_3 c_3 \\
            b_1 a_1 + b_2 a_2 + b_3 a_3 & {b_1}^2 +{b_2}^2 + {b_3}^2 & b_1 c_1 + b_2 c_2 + b_3 c_3 \\
            c_1 a_1 + c_2 a_2 + c_3 a_3 & c_1 b_1 + c_2 b_2 + c_3 b_3 & {c_1}^2 +{c_2}^2 +{c_3}^2
        \end{pmatrix}
        \\
        =& \begin{pmatrix}
            (\bm{a},\bm{a}) & (\bm{a},\bm{b}) & (\bm{a},\bm{c}) \\
            (\bm{b},\bm{a}) & (\bm{b},\bm{b}) & (\bm{b},\bm{c}) \\
            (\bm{c},\bm{a}) & (\bm{c},\bm{b}) & (\bm{c},\bm{c})
        \end{pmatrix}
    \end{align*}
    である.

    一方,
    \begin{align*}
        \det (\bm{a},\bm{b},\bm{c})&=
        \begin{vmatrix}
            a_1 & b_1 & c_1 \\
            a_2 & b_2 & c_2 \\
            a_3 & b_3 & c_3
        \end{vmatrix} \\
        & = c_1
        \begin{vmatrix}
            a_2 & b_2 \\
            a_3 & b_3 
        \end{vmatrix}
        + c_2
        \begin{vmatrix}
            a_3 & b_3 \\
            a_1 & b_1
        \end{vmatrix}
        + c_3
        \begin{vmatrix}
            a_1 & b_1 \\
            a_2 & b_2
        \end{vmatrix}
        \\
        & = c_1 (a_2 b_3-b_2 a_3)+c_2 (a_3 b_1 - b_3 a_1)+c_3 (a_1 b_2 -b_1 a_2)\\
        & = a_3 (b_1 c_2 - b_2 c_1)+b_3 (c_1 a_2-c_2 a_1)+ c_3 (a_1 b_2 - b_1 a_2)\\
        & =
        \begin{vmatrix}
            a_1 & a_2 & a_3 \\
            b_1 & b_2 & b_3 \\
            c_1 & c_2 & c_3
        \end{vmatrix}
    \end{align*}
    であるから,これと行列式の積の性質により,
    \[
        \begin{vmatrix}
            (\bm{a},\bm{a}) & (\bm{a},\bm{b}) & (\bm{a},\bm{c}) \\
            (\bm{b},\bm{a}) & (\bm{b},\bm{b}) & (\bm{b},\bm{c}) \\
            (\bm{c},\bm{a}) & (\bm{c},\bm{b}) & (\bm{c},\bm{a})
        \end{vmatrix}
        = {\det (\bm{a},\bm{b},\bm{c})}^2
    \]
    である.
    \end{proof}
\end{tleftbar}


\subsection*{p29-30:9}
\addcontentsline{toc}{subsection}{\texorpdfstring{p29-30:9}{p29-30:9}}

\begin{tleftbar}
    $ \det (\bm{x},\bm{y},\bm{z})$は,$\bm{x}$,$\bm{y}$,$\bm{z}$の張る平行六面体の体積に符号をつけたものに等しい.
    与条件より,$\det (\bm{x},\bm{y},\bm{z})$が最大になるのは,$\bm{x}$.$\bm{y}$,$\bm{z}$の張る図形が立方体のときであり,
    そのとき
    \[
        \det (\bm{x},\bm{y},\bm{z}) =1
    \]
    である.これからただちに$\det (\bm{x},\bm{y},\bm{z})$の最小値が$-1$であることも従う.

    以上により,$\det (\bm{x},\bm{y},\bm{z})$の最大値は$1$,最小値は$-1$である.
\end{tleftbar}

\subsection*{p29-30:10}
\addcontentsline{toc}{subsection}{\texorpdfstring{p29-30:10}{p29-30:10}}

\begin{tleftbar}
    \begin{description}
        \item[イ] 
    \begin{proof}
    単位ベクトル$\bm{e}_1$,$\bm{e}_2$,$\bm{e}_3$を適当にとり,
    \[
    \bm{a} = \alpha_1 \bm{e}_1,\quad \bm{b} = \beta_1 \bm{e}_1+\beta_2 \bm{e}_2,\quad \bm{c}= \gamma_1 \bm{e}_1 + \gamma_2 \bm{e}_2 + \gamma_3 \bm{e}_3
    \]
    とおく.このとき,
        \begin{align*}
            (\bm{a} \times \bm{b}) \times \bm{c} &= \alpha_1 \beta_2 \bm{e}_3 \times (\gamma_1 \bm{e}_1 + \gamma_2 \bm{e}_2 + \gamma_3 \bm{e}_3) \\
            & = \alpha_1 \beta_2 \gamma_1 \bm{e}_2 - \alpha_1 \beta_2 \gamma_2 \bm{e}_1 \\
            & = -(\bm{b},\bm{c})\bm{a}+(\bm{a},\bm{c}) \bm{b}
        \end{align*}
        であり,これが証明すべきことであった\footnote{この等式をラグランジュの恒等式とよぶ.}.
        \item[ロ]
        イ)の結果により.
        \begin{align*} 
                & (\bm{a}\times\bm{b}) \times \bm{c} =  -(\bm{b},\bm{c})\bm{a}+(\bm{a},\bm{c}) \bm{b} ,\\
                & (\bm{b} \times \bm{c} ) \times \bm{a} = -(\bm{c},\bm{a}) \bm{b} +(\bm{b},\bm{a}) \bm{c}, \\
                & (\bm{c} \times \bm{a} ) \times \bm{b} = -(\bm{a},\bm{b}) \bm{c} +(\bm{c},\bm{b}) \bm{a}.
        \end{align*} 
        であるから,
        \[
            (\bm{a}\times\bm{b}) \times \bm{c} + (\bm{b} \times \bm{c} ) \times \bm{a}+(\bm{c} \times \bm{a} ) \times \bm{b} =\bm{0}
        \]
        となる.これが証明すべきことであった.
    \end{proof}
\end{description}
    \end{tleftbar}


\section*{第2章}
\addcontentsline{toc}{section}{\texorpdfstring{第2章}{第2章}}


\subsection*{p34:問1}
\addcontentsline{toc}{subsection}{\texorpdfstring{p34:問1}{p34:問1}}

\begin{tleftbar}
\begin{proof}
後半二つの主張は明らか.また,二つ目の主張は一つ目の主張と同様にして示すことができるので,一つ目のみ示すことにする.


$A=(a_{pq})$を$k \times l$行列,$B= (b_{qr})$,$C=(c_{qr})$を$l \times m$行列とする.示したい式の両辺がともに定義され,ともに$k \times m$行列であることはよい.行列$B+C$の$(q,r)$成分は$b_{qr}+c_{qr}$であるから,左辺の$(p,r)$成分は,
\[
\sum_{q=1}^{l}a_{pq}\left(b_{qr}+c_{qr}\right)=\sum_{q=1}^{l}a_{pq}b_{qr}+\sum_{q=1}^{l}a_{pq}c_{qr}
\]
とかける.この等号の右辺は$AB$の$(p,r)$成分と$AC$の$(p,r)$成分の和である.これより,主張が示された.
\end{proof}
\end{tleftbar}



\subsection*{p40:問}
\addcontentsline{toc}{subsection}{\texorpdfstring{p40:問}{p40:問}}

\begin{tleftbar}
    \begin{description}
        \item[イ] 
        \[
            A_{11} = \begin{pmatrix} 1 & -1 \\ 0 & -2 \end{pmatrix},\quad A_{22} = \begin{pmatrix} -2 & 3 \\ 1 & 1 \end{pmatrix} ,\quad B_{11} = \begin{pmatrix} 2 & 1 \\ 0 & 1\end{pmatrix} ,\quad B_{22}= \begin{pmatrix} 1 & 1 \\ 2 & -3 \end{pmatrix}
        \]
        とおくと,
        \begin{align*} 
            (\text{与式})&= \begin{pmatrix} A_{11} & O \\ O & A_{22} \end{pmatrix} \begin{pmatrix} B_{11} & O \\ O & B_{22} \end{pmatrix} \\
            & = \begin{pmatrix} A_{11} B_{11} & O \\ O & A_{22} B_{22} \end{pmatrix} \\
            & = \begin{pmatrix} 2 & 0 & 0 & 0 \\ 0 & -2 & 0 & 0 \\ 0 & 0 & 4 & -11 \\ 0 & 0 & 3 & -2 \end{pmatrix}
        \end{align*}
        である.
    \end{description}
\end{tleftbar}

\subsection*{p41:問1}
\addcontentsline{toc}{subsection}{\texorpdfstring{p41:問1}{p41:問1}}
%

\begin{tleftbar}
    \begin{enumerate}
        \item \mbox{ }
    $ X =\begin{pmatrix} x_{11} & x_{12} \\ x_{21} & x_{22} \end{pmatrix}$とする.このとき,
    \[
        AX =\begin{pmatrix} x_{11}+2x_{21} &  x_{12}+2x_{22}  \\ 2x_{11} + 4x_{21} &2x_{12} + 4x_{22} \end{pmatrix}
    \]
    となり,これが$E_2$と等しくなるためには
    \[
        \begin{cases}
            x_{11}+2x_{21} =1 \\
            x_{12}+2x_{22}  =0 \\
            2x_{11} + 4x_{21}=0 \\
            2x_{12} + 4x_{22}=1
        \end{cases}
    \]
    となることが必要かつ十分であるが,これを満たす$x_{11},x_{12},x_{21},x_{22} \in \mathbb{C}$は存在しない.よって前半の主張が示された.\par 
    後半について示す.$Y =\begin{pmatrix} y_{11} & y_{12} \\ y_{21} & y_{22} \end{pmatrix}$とする.このとき,
    \[
        YA = \begin{pmatrix} y_{11} +2y_{12} & 2y_{11}+4y_{12} \\ y_{21}+2y_{22} & 2y_{21}+4y_{22}  \end{pmatrix}
    \]
    となり,これが$E_2$と等しくなるためには
    \[
        \begin{cases}
            y_{11} +2y_{12}=1 \\
            2y_{11}+4y_{12}  =0 \\
            y_{21}+2y_{22} =0 \\
            2y_{21}+4y_{22} =1
        \end{cases}
    \]
    となることが必要かつ十分であるが,これを満たす$y_{11},y_{12},y_{21},y_{22} \in \mathbb{C}$は存在しない.よって後半の主張も示された.\qed
    \item \mbox{ }
        $X,Y$を(1)で定義したものとする.このとき,
        \[
            AX = \begin{pmatrix} x_{11}+2x_{21} & x_{12}+2x_{22} \\ 0 & 0 \end{pmatrix}
        \]
        となり,これが$B$と等しくならないことは明らか.\par 
        後半について,
        \[
            YA =\begin{pmatrix} x_{11} & 2 x_{11} \\  x_{21}&  2x_{21} \end{pmatrix}
        \]
        となり,これが$B$と等しくなるためには$x_{11}=1$,$x_{21}=2$となることが必要かつ十分であるが,$x_{12}$,$x_{22}$については任意の複素数である.以上の議論により,このような$Y$は無限に存在する.\qed 
        \item  \mbox{ }
        $A$の第$k$列の成分が全て$0$であるとする.ただしここで$ 1 \le k \le n,~ k \in \mathbb{N}$であるとする.
        $XA=E$をみたす$X$が存在すると仮定する.このとき,$X$は明らかに$n \times n$行列であり,積$XA$は定義される.
        いま$X=(x_{jk})$,$A=(a_{kj})$,$ 1 \le j ,k \le n$と表す.このとき,
        \[
           (XA \text{の}(j,j)\text{成分}) = \sum_{k=1}^{n} x_{jk} a_{kj} =0
        \]
        となり,これは$XA =E$に矛盾する.よってこのような$X$は存在しないことが示された.\qed
    \end{enumerate}
    \end{tleftbar}


    \subsection*{p42:問1}
    \addcontentsline{toc}{subsection}{\texorpdfstring{p42:問1}{p42:問1}}

    \begin{leftbar}
        \begin{enumerate}
            \item 
        まず,
        \[
            \overline{A} \ \overline{A^{-1}} = \overline{A A^{-1}}=E,\quad \overline{A^{-1}} \ \overline{A} =\overline{A^{-1} A}=E
        \]
        より,$\overline{A}$は正則で,逆行列は$\overline{A^{-1}}$である.さらに,
        \[
            {}^t A {}^t A^{-1} ={}^t (A^{-1} A)=E,\quad {}^t A^{-1} {}^t A = {}^t (A A^{-1})=E
        \]
        であるから,${}^t A$は正則であり,逆行列は${}^t A^{-1}$である.
        \item 
        \[
            A \coloneqq \begin{pmatrix} a & b \\ c & d \end{pmatrix},\quad A' \coloneqq \begin{pmatrix} x & y \\ z & w \end{pmatrix}
        \]
        とする.このとき,
        \[
            A A' = \begin{pmatrix} a x + b z & ay + bw \\ cx + dz & cy +dw \end{pmatrix}
        \]
        である.$AA'=E$となる条件は,$x$,$y$,$z$,$w$についてのふたつの連立方程式
        \[
            \begin{cases}
                ax+bz =1 \\
                cx+dz =0
            \end{cases}
            ,\quad 
            \begin{cases}
                ay+bw=0\\
                cy+dw=1
            \end{cases}
        \]
        が解を持つことで,その条件は$ad-bc \ne 0$である.そのときの解は,
        \[
            (x,y,z,w)=  (\frac{d}{ad-bc},-\frac{b}{ad-bc},-\frac{c}{ad-bc},\frac{a}{ad-bc})
        \]
        である.これを用いて$A'A$を計算すると,$A' A =E$となり.たしかに$A'$は$A$の逆行列である.
    
        以上の議論により,$ad - bc \ne 0$となることが必要十分条件である.
        \item 計算する.
    \begin{description}
        \item[イ] (2)の結果により,
        \[
            \frac{1}{3 \cdot 3 - 2 \cdot 4} \begin{pmatrix} 3 & -2 \\ -4 & 3 \end{pmatrix} =  \begin{pmatrix} 3 & -2 \\ -4 & 3 \end{pmatrix}
        \]
        が求める逆行列である.
        \item[ロ] まず,
        \[
            X= \begin{pmatrix} x_{11} & x_{12} & x_{13} \\x_{21} & x_{22} & x_{23} \\x_{31} & x_{32} & x_{33}  \end{pmatrix}
        \]
        としたときに
        \begin{align*} 
           XA &=  \begin{pmatrix} x_{11} & x_{12} & x_{13} \\x_{21} & x_{22} & x_{23} \\x_{31} & x_{32} & x_{33}  \end{pmatrix}
            \begin{pmatrix} 1 & 2 & -1 \\ 0& 1 & 3 \\ 0 & 0 & 1 \end{pmatrix}  \\
            & = \begin{pmatrix} x_{11} & 2x_{11} + x_{12} & -x_{11} +3x_{12} +x_{13} \\ x_{21} & 2x_{21} + x_{22} & -x_{21} +3x_{22} +x_{23} \\  x_{31} & 2x_{31} + x_{32} & -x_{31} +3x_{32} +x_{33} \end{pmatrix} \\
        \end{align*} 
        であるから,これに関して
        \[
            \begin{pmatrix} x_{11} & 2x_{11} + x_{12} & -x_{11} +3x_{12} +x_{13} \\ x_{21} & 2x_{21} + x_{22} & -x_{21} +3x_{22} +x_{23} \\  x_{31} & 2x_{31} + x_{32} & -x_{31} +3x_{32} +x_{33} \end{pmatrix} = \begin{pmatrix} 1 & 0 & 0 \\ 0 & 1 & 0 \\ 0 & 0 & 1 \end{pmatrix}
        \]
        となれば,行列$X$が求める逆行列である.

        計算すると
        \[
            X = \begin{pmatrix} 1 & -2 & 7 \\ 0 & 1 & -3 \\ 0 & 0 & 1 \end{pmatrix}
        \]
        であり,これが求める逆行列であった.
        \item[ハ] まず,
        \[
            X= \begin{pmatrix} x_{11} & x_{12} & x_{13} & x_{14}\\x_{21} & x_{22} & x_{23} & x_{24} \\x_{31} & x_{32} & x_{33} & x_{34} \\x_{41} & x_{42} & x_{43} & x_{44} \end{pmatrix}
        \]
        としたとき,
        \begin{align*} 
            XA &= \begin{pmatrix} x_{11} & x_{12} & x_{13} & x_{14}\\x_{21} & x_{22} & x_{23} & x_{24} \\x_{31} & x_{32} & x_{33} & x_{34} \\x_{41} & x_{42} & x_{43} & x_{44} \end{pmatrix}
            \begin{pmatrix} 0 & 0 & 0 & 1 \\ 0 & 0 & 1 & 0 \\ 0 & 1 & 0 & 0 \\ 1 & 0 & 0 & 0 \end{pmatrix} \\
            & = \begin{pmatrix} x_{14} & x_{13} & x_{12} & x_{11} \\x_{24} & x_{23} & x_{22} & x_{21} \\x_{34} & x_{33} & x_{32} & x_{31} \\x_{44} & x_{43} & x_{42} & x_{41} \end{pmatrix}
        \end{align*} 
        であるから,これに関して
        \[
            \begin{pmatrix} x_{14} & x_{13} & x_{12} & x_{11} \\x_{24} & x_{23} & x_{22} & x_{21} \\x_{34} & x_{33} & x_{32} & x_{31} \\x_{44} & x_{43} & x_{42} & x_{41} \end{pmatrix} = \begin{pmatrix} 1 & 0 & 0 & 0 \\ 0 & 1 & 0 & 0 \\ 0 & 0 & 1 & 0 \\ 0 & 0 & 0 & 1 \end{pmatrix}
        \]
        となれば,行列$X$が求める逆行列である.

        計算すると,
        \[
            X = \begin{pmatrix} 0 & 0 & 0 & 1 \\ 0 & 0 & 1 & 0 \\ 0 & 1 & 0 & 0 \\ 1 & 0 & 0 & 0 \end{pmatrix} 
        \]
        であり.これが求める逆行列であった.
    \end{description}
        \end{enumerate}
    \end{leftbar}

    \newpage 

    \subsection*{p52:問}
    \addcontentsline{toc}{subsection}{\texorpdfstring{p52:問}{p52:問}}

\begin{tleftbar}
    \begin{align*} 
        &
        \left( 
            \begin{array}{ccc|ccc}
            1 & 3 & 2 & 1 & 0 & 0\\ 
            2 & 6 & 3 & 0 & 1 & 0 \\ 
            -2 & -5 & -2 & 0 & 0 & 1
            \end{array}
            \right) \\
           \xrightarrow{\text{第$1$行の$(-2)$倍,第$1$行の$2$倍をそれぞれ第$2$行,第$3$行に加える}} &
            \left( \begin{array}{ccc|ccc}
                1 & 3 & 2 & 1 & 0 & 0\\ 
                0 & 0 & -1 & -2 & 1 & 0 \\ 
                0 & 1 & 2 & 2 & 0 & 1
                \end{array}
                \right)\\
                \xrightarrow{\text{第$2$行と第$3$行を交換する}} &
                \left( \begin{array}{ccc|ccc}
                    1 & 3 & 2 & 1 & 0 & 0\\ 
                    0 & 1 & 2 & 2 & 0 & 1 \\
                    0 & 0 & -1 & -2 & 1 & 0 
                    \end{array}
                    \right)\\
                   \xrightarrow{\text{第$2$行の$(-3)$倍を第$1$行に加え,第$3$行の$(-4)$倍を第$1$行に加える}} &
                    \left( \begin{array}{ccc|ccc}
                    1 & 0 & 0 & 3 & -4 & -3\\ 
                    0 & 1 & 2 & 2 & 0 & 1 \\
                    0 & 0 & -1 & -2 & 1 & 0 
                        \end{array}
                        \right) \\
                        \xrightarrow{\text{第$3$行の$2$倍を第$2$行に加え,第$3$行を$(-1)$倍する}} &
                         \left( \begin{array}{ccc|ccc}
                            1 & 0 & 0 & 3 & -4 & -3\\ 
                            0 & 1 & 0 & -2 & 2 & 1 \\
                            0 & 0 & 1 & 2 & -1 & 0 
                             \end{array}
                             \right)
                            \end{align*}
        よって,求める逆行列は
        \[
            \begin{pmatrix} 
                3 & -4 & -3 \\
                -2 & 2 & 1 \\
                2 & -1 & 0 
            \end{pmatrix}
        \]
        である.
    \end{tleftbar}

    \newpage 


    \subsection*{p58:問}
    \addcontentsline{toc}{subsection}{\texorpdfstring{p58:問}{p58:問}}

    \begin{tleftbar}
        \begin{description}
            \item[イ] 与えられた連立方程式について,拡大係数行列を考えて基本変形を施すと,
            \[
                \begin{pmatrix} 1 & 0 & 1 & -4/3 \\ 0 & 1 & 1 & 8/3 \\ 0 & 0 & 0 & 0 \end{pmatrix}
            \]
            となる.つまり,解は存在し,ひとつの任意定数を含む.任意定数を$ x_3 =\alpha$とすると,
            \[
                x_1 = -\frac{4}{3} - \alpha,\quad x_2 = \frac{8}{3} - \alpha,\quad x_3 = \alpha
            \]
            とかける.ベクトルの形で表すと,
            \[
                \begin{pmatrix} x_1 \\ x_2 \\ x_3 \end{pmatrix} = \begin{pmatrix} -4/3 \\ 8/3 \\ 0 \end{pmatrix} + \alpha \begin{pmatrix} -1 \\ -1 \\ 1 \end{pmatrix}
            \]
            である.
            \item[ロ] 与えられた連立方程式について,拡大係数行列を考えて基本変形を施すと,
            \[
                \begin{pmatrix} 1 & 0 & 0 & 0 \\ 0 & 1& 0 & 1 \\ 0 & 0 & 1 & 0 \\ 0 & 0 & 0 & -1 \end{pmatrix}
            \]
            となるが,$0 = -1$とはならないため,この連立方程式は解を持たない.
            \item[ハ] 与えられた連立方程式について,拡大係数行列を考えて基本変形を施すと,
            \[
                \begin{pmatrix} 1 & 0 & 0 & 2 & 6 & 6 \\ 0 & 1& 0 & -2 & -11 & -9 \\ 0 & 0 & 1 & 0 & 19 & 14 \end{pmatrix}
            \]
            となる.ただしここで第3列と第4列を入れ替えた.
            
            つまり,解は存在し,ふたつの任意定数を含む,任意定数を$x_3 = \alpha$,$x_5 = \beta$とすると,この連立方程式の解は
            \[
                x_1 = 6-2\alpha -2 \beta , \quad x_2 = -9 + 2\alpha +11 \beta , \quad x_3 = \alpha , \quad x_4 = 14-19\beta,\quad x_5 =\beta
            \]
            とかける.ベクトルの形で表すと,
            \[
                \begin{pmatrix} x_1 \\ x_2 \\ x_3 \\ x_4 \\ x_5 \end{pmatrix} = \begin{pmatrix} 6 \\ -9 \\ 0 \\ 14 \\ 0 \end{pmatrix} + \alpha \begin{pmatrix} -2 \\ 2 \\ 1 \\ 0 \\ 0 \end{pmatrix} + \beta \begin{pmatrix} -2 \\ 11 \\ 0 \\ -19 \\ 1 \end{pmatrix}
            \]
            である.
        \end{description}
    \end{tleftbar}

    \newpage 

    \subsection*{p62-63:問1}
    \addcontentsline{toc}{subsection}{\texorpdfstring{p62:問1}{p62:問1}}

\begin{tleftbar}
    \begin{proof}
        定義に従って計算すると,
        \begin{align*}
            \norm{\bm{x}+\bm{y} }^2 +\norm{\bm{x}-\bm{y} }^2 & = (\bm{x}+\bm{y},\bm{x}+\bm{y})+(\bm{x}-\bm{y},\bm{x}-\bm{y}) \\
            & =(\bm{x},\bm{x})+(\bm{x},\bm{y})+(\bm{y},\bm{x})+(\bm{y},\bm{y})+(\bm{x},\bm{x})-(\bm{x},\bm{y})-(\bm{y},\bm{x})+(\bm{y},\bm{y}) \\
            & = 2((\bm{x},\bm{x})+(\bm{y},\bm{y})) \\
            & =2 (\norm{\bm{x}}^2+\norm{\bm{y}}^2)
        \end{align*}
        となり,これが証明すべきことであった.
    \end{proof}
\end{tleftbar}
\subsection*{p62-63:問2}
\addcontentsline{toc}{subsection}{\texorpdfstring{p62:問2}{p62:問2}}
\begin{tleftbar}
    \begin{proof}
        \[
        \norm{\bm{x}+\bm{y}}^2=(\bm{x},\bm{x})+(\bm{x},\bm{y})+(\bm{y},\bm{x})+(\bm{y},\bm{y})
        \]
        である.ここで,$\bm{x}$と$\bm{y}$が直交することから,
        \[
            (\bm{x},\bm{y})+(\bm{y},\bm{x})=(\bm{x},\bm{y})+\overline{(\bm{x},\bm{y})}=0
            \]
            であり,これを用いると
        \[
            \norm{\bm{x}+\bm{y}}^2 =(\bm{x},\bm{x})+(\bm{y},\bm{y})=\norm{\bm{x}}^2 +\norm{\bm{y}}^2
        \]
        となる.$\bm{x},~\bm{y}$がともに実ベクトルのときは$(\bm{x},\bm{y})=0$であるから確かに逆が成り立つが,たとえば$\bm{x}=
        \begin{pmatrix}
            2 \\
            0
        \end{pmatrix}
        ,~
        \bm{y}=
        \begin{pmatrix}
            2i \\
            0
        \end{pmatrix}
        $とすれば,等式は成り立つが$\bm{x}$と$\bm{y}$は直交しないため,逆は成り立たない.
    \end{proof}
\end{tleftbar}

\subsection*{p62-63:問3}
\addcontentsline{toc}{subsection}{\texorpdfstring{p62-63:問3}{p62-63:問3}}

\begin{tleftbar}
    \begin{proof}
        $\bm{x},\bm{y} \in \mathbb{R}^n$のとき,
        \begin{align*} 
            \norm{\bm{x}+\bm{y}}^2 - \norm{\bm{x}-\bm{y}}^2 & = (\bm{x}+\bm{y},\bm{x}+\bm{y}) - (\bm{x}-\bm{y},\bm{x}-\bm{y}) \\
            & = \norm{\bm{x}}^2 +(\bm{x},\bm{y})+(\bm{y},\bm{x})+ \norm{\bm{y}}^2 - (\norm{\bm{x}}^2 -(\bm{x},\bm{y})-(\bm{y},\bm{x})+\norm{\bm{y}}^2) \\
            & = \norm{\bm{x}}^2 + 2(\bm{x},\bm{y}) + \norm{\bm{y}}^2 - (\norm{\bm{x}}^2 - 2(\bm{x},\bm{y}) + \norm{\bm{y}}^2) \\
            & = 4(\bm{x},\bm{y})
        \end{align*}
        であるから,この両辺を$4$で割るとただちに主張が従う.
        
        また,$\bm{x},\bm{y} \in \mathbb{C}$のときにはこの等式が成り立たないことがある.
        $\bm{x}={}^t (2i,0)$,$\bm{y}={}^t (2,0)$とすると,
        \begin{align*}
        \frac{\norm{\bm{x}+\bm{y}}^2 - \norm{\bm{x}-\bm{y}}^2 }{4} &= \frac{4-4}{4} \\
        & =0
        \end{align*}
        であるが,
        \begin{align*}
        (\bm{x},\bm{y})& =(2,0) \begin{pmatrix} -2i  \\ 0 \end{pmatrix} \\
        &= -4i
        \end{align*}
        となり,確かにこれが反例になっている.
        \end{proof}
    \end{tleftbar}

    \newpage 

    \subsection*{p65:問1}
    \addcontentsline{toc}{subsection}{\texorpdfstring{p65:問1}{p65:問1}}

    \begin{tleftbar} 
        \[
            A \coloneqq \begin{pmatrix} a & b \\ c & d\end{pmatrix} \in \mathbb{R}^{2 \times 2}
        \]
        とおく.このとき,
        \begin{align*}
            A {}^t A &= \begin{pmatrix} a & b \\ c & d \end{pmatrix} \begin{pmatrix} a & c \\ b & d\end{pmatrix} \\
            & = \begin{pmatrix} a^2 +b^2 & ac+bd \\ ac+bd & c^2 + d^2 \end{pmatrix}
        \end{align*}
        となり,これが$E$に等しいので,
        \[
            \begin{cases}
                a^2 + b^2 =1 ,\\
                c^2 + d^2 =1 ,\\
                ac+bd =0 
            \end{cases}
        \]
        となる.このことから$ 0 \leqq \theta < 2\pi $,$ 0 \leqq \phi < 2\pi$として
        \begin{align*}
            & a = \cos \theta , \quad b = \sin \theta,\\
            & c = \cos \phi , \quad d = \sin \phi 
        \end{align*}
        とおくと,
        \begin{align*} 
            ac+bd & = \cos \theta \cos \phi + \sin \theta \sin \phi \\
            & = \cos (\theta - \phi)
        \end{align*}
        となり,これが$0$に等しいので.
        \begin{align*}
            & \theta -\phi = \pi /2 , 3\pi /2 , \\
            \therefore & \phi = \theta - \pi /2 , \theta - 3\pi/2.
        \end{align*}
        これより,任意の二次直交行列は$ 0 \leqq \theta < 2\pi $,$ 0 \leqq \phi < 2\pi$として
        \[
            \begin{pmatrix} \cos \theta & -\sin \theta \\ \sin \theta & \cos \theta \end{pmatrix},\quad \begin{pmatrix} \cos \theta & \sin \theta \\ \sin \theta & -\cos \theta \end{pmatrix}
        \]
        のいずれかで表される.
    \end{tleftbar}

    \subsection*{p65:問2}
\addcontentsline{toc}{subsection}{\texorpdfstring{p65:問2}{p65:問2}}

\begin{tleftbar}
    \begin{proof}
    \[
        (A^\ast A)^{\ast} = A^{\ast} A^{\ast \ast} = A^{\ast} A, \quad (A A^{\ast})^\ast = A^{\ast \ast} A^\ast =A A^{\ast}
    \]
    であるから,$A^\ast A$,$A A^{\ast}$はエルミート行列である.

    また,任意の$n \times 1$ベクトル$\bm{x}$に対して,
    \begin{align*} 
        (\bm{x}, A^{\ast}A \bm{x}) & = (A^{\ast \ast} \bm{x}, A \bm{x}) \\
        & = (A \bm{x}, A \bm{x}) \\
        & = \norm{A \bm{x}}^2 \geqq 0
    \end{align*}
    であり,また,$\bm{x}$として第$i$成分のみ1でほかの成分は$0$のベクトル$\bm{e}_i$をとると,
    \[
        (\bm{e}_i, A^{\ast}A \bm{e}_i) = \text{$A^{\ast}A$の$(i,i)$成分}
    \]
    となる.先の不等式よりこれは$0$または正なので,$A^{\ast}A$の対角成分は$0$または正である.同様にして$AA^{\ast}$の対角成分も$0$または正である.

    以上のことが証明すべきことであった.
\end{proof}
\end{tleftbar}

    \newpage 
\section*{第2章・章末問題}
\addcontentsline{toc}{section}{\texorpdfstring{第2章・章末問題}{第2章・章末問題}}


\subsection*{p70-73:1-イ)}
\addcontentsline{toc}{subsection}{\texorpdfstring{p70-73:1-イ)}{p70-73:1-イ)}}

\begin{leftbar}
        \begin{align*} 
        &
        \left( 
            \begin{array}{cccc|cccc}
            3 & 3 & -5 & -6 & 1 & 0 & 0 & 0\\ 
            1 & 2 & -3 & -1 & 0 & 1 & 0 & 0 \\ 
            2 & 3 & -5 & -3 & 0 & 0 & 1 & 0 \\
            -1 & 0 & 2 & 2 & 0 & 0 & 0 & 1 
            \end{array}
            \right) \\
           \xrightarrow{\text{第$1$行と第$2$行を交換する}} &
            \left( \begin{array}{cccc|cccc}
                1 & 2 & -3 & -1 & 0 & 1 & 0 & 0 \\ 
                3 & 3 & -5 & -6 & 1 & 0 & 0 & 0\\ 
                2 & 3 & -5 & -3 & 0 & 0 & 1 & 0 \\
                -1 & 0 & 2 & 2 & 0 & 0 & 0 & 1 
                \end{array}
                \right)\\
           \xrightarrow{\text{$(1,1)$をかなめとして左から第1列を掃き出す}} &
           \left( \begin{array}{cccc|cccc}
            1 & 2 & -3 & -1 & 0 & 1 & 0 & 0 \\ 
            0 & -3 & 4 & -3 & 1 & -3 & 0 & 0\\ 
            0 & -1 & 1 & -1 & 0 & -2 & 1 & 0 \\
            0 & 2 & -1 & 1 & 0 & 1 & 0 & 1 
            \end{array}
            \right)\\
            \xrightarrow{\text{第$3$行を$(-1)$倍して第2行と交換し,$(2,2)$をかなめとして左から第$2$列を掃き出す}} &
           \left( \begin{array}{cccc|cccc}
            1 & 0 & -1 & -3 & 0 & -3 & 2 & 0 \\ 
            0 & 1 & -1 & 1 & 0 & 2 & -1 & 0 \\
            0 & 0 & 1 & 0 & 1 & 3 & -3 & 0\\ 
            0 & 0 & 1 & -1 & 0 & -3 & 2 & 1 
            \end{array}
            \right)\\
            \xrightarrow{\text{$(3,3)$をかなめとして左から第$3$列を掃き出す}} &
            \left( \begin{array}{cccc|cccc}
                1 & 0 & 0 & -3 & 1 & 0 & -1 & 0 \\ 
                0 & 1 & 0 & 1 & 1 & 5 & -4 & 0 \\
                0 & 0 & 1 & 0 & 1 & 3 & -3 & 0\\ 
                0 & 0 & 0 & -1 & -1 & -6 & 5 & 1 
                \end{array}
                \right)\\
            \xrightarrow{\text{第4行を$(-1)$倍して,$(4,4)$をかなめとして第$4$列を掃き出す}} &
            \left( \begin{array}{cccc|cccc}
                1 & 0 & 0 & 0 & 4 & 18 & -16 & -3 \\ 
                0 & 1 & 0 & 0 & 0 & -1 & 1 & 1 \\
                0 & 0 & 1 & 0 & 1 & 3 & -3 & 0\\ 
                0 & 0 & 0 & 1 & 1 & 6 & -5 & -1
                \end{array}
                \right)
            \end{align*} 
            よって,求める逆行列は,
            \[
                \begin{pmatrix}
                    4 & 18 & -16 & -3 \\ 0 & -1 & 1 & 1\\ 1 & 3 & -3 & 0 \\ 1 & 6 & -5 & -1 
                \end{pmatrix}
            \]
            である.
            \end{leftbar}

            \newpage

\subsection*{p70-73:1-ロ)}
\addcontentsline{toc}{subsection}{\texorpdfstring{p70-73:1-ロ)}{p70-73:1-ロ)}}
        \begin{leftbar}
        \begin{align*} 
        &
        \left( 
            \begin{array}{cccc|cccc}
            1 & 2 & 0 & -1 & 1 & 0 & 0 & 0\\ 
            -3 & -5 & 1 & 2 & 0 & 1 & 0 & 0 \\ 
            1 & 3 & 2 & -2 & 0 & 0 & 1 & 0 \\
            0 & 2 & 1 & -1 & 0 & 0 & 0 & 1 
            \end{array}
            \right) \\
           \xrightarrow{\text{$(1,1)$をかなめとして第$1$列を掃き出す}} &
            \left( \begin{array}{cccc|cccc}
            1 & 2 & 0 & -1 & 1 & 0 & 0 & 0\\ 
            0 & 1 & 1 & -1 & 3 & 1 & 0 & 0 \\
            0 & 1 & 2 & -1 & -1 & 0 & 1 & 0 \\
            0 & 2 & 1 & -1 & 0 & 0 & 0 & 1 
            \end{array}
            \right) \\
           \xrightarrow{\text{$(2,2)$をかなめとして第$2$列を掃き出す}} &
           \left( \begin{array}{cccc|cccc}
            1 & 0 & -2 & 1 & -5 & -2 & 0 & 0\\ 
            0 & 1 & 1 & -1 & 3 & 1 & 0 & 0 \\
            0 & 0 & 1 & 0 & -4 & -1 & 1 & 0 \\
            0 & 0 & -1 & 1 & -6 & -2 & 0 & 1 
            \end{array}
            \right) \\
           \xrightarrow{\text{$(3,3)$をかなめとして第$3$列を掃き出す}} &
           \left( \begin{array}{cccc|cccc}
            1 & 0 & 0 & 1 & -13 & -4 & 2 & 0\\ 
            0 & 1 & 0 & -1 & 7 & 2 & -1 & 0 \\
            0 & 0 & 1 & 0 & -4 & -1 & 1 & 0 \\
            0 & 0 & 0 & 1 & -10 & -3 & 1 & 1 
            \end{array}
            \right) \\
           \xrightarrow{\text{$(3,3)$をかなめとして第$3$列を掃き出す}} &
           \left( \begin{array}{cccc|cccc}
            1 & 0 & 0 & 0 & -3 & -1 & 1 & -1\\ 
            0 & 1 & 0 & 0 & -3 & -1 & 0 & 1 \\
            0 & 0 & 1 & 0 & -4 & -1 & 1 & 0 \\
            0 & 0 & 0 & 1 & -10 & -3 & 1 & 1 
            \end{array}
            \right) \\
        \end{align*} 
        よって,求める逆行列は
        \[
            \begin{pmatrix} -3 & -1 & 1& -1 \\ -3 & -1 & 0 & 1 \\ -4 & -1 & 1 & 0 \\ -10 & -3 & 1& 1 \end{pmatrix}
        \]
        である.
    \end{leftbar}

    \newpage 


\subsection*{p70-73:2-イ)}
\addcontentsline{toc}{subsection}{\texorpdfstring{p70-73:2-イ)}{p70-73:2-イ)}}

\begin{tleftbar}
    与えられた連立方程式について,拡大係数行列を考えて基本変形を施すと,
    \[
        \begin{pmatrix} 1 & 0 & 0 & 2 & 2 & 0 \\ 0 & 1& 0 & 0 & 0 & -3/5 \\ 0 & 0 & 1 & -1 & 0 & 1/5 \\ 0 & 0 & 0 & 0 & 0 & 0 \end{pmatrix}
    \]
    となる.ただしここで第2列と第5列を入れ替えた.

    つまり,解は存在し,2つの任意定数を含む.任意定数を$x_4 = \alpha$,$x_2 = \beta $とすると,
    \[
        x_1 = -2 \alpha - 2 \beta , \quad x_2 =\beta , \quad x_3 = \alpha + \frac{1}{5} , \quad x_4 = \alpha , \quad x_5 = -\frac{3}{5}
    \]
    となる.ベクトルの形で表すと
    \[
        \begin{pmatrix} x_1 \\ x_2 \\ x_3 \\ x_4 \\ x_5 \end{pmatrix}= \begin{pmatrix} 0 \\ 0 \\ 1/5 \\ 0 \\ -3/5 \end{pmatrix} + \alpha \begin{pmatrix} -2 \\ 0 \\ 1 \\ 1 \\ 0 \end{pmatrix} + \beta \begin{pmatrix} -2 \\ 1 \\ 0 \\ 0 \\ 0 \end{pmatrix}
    \]
    となる.
\end{tleftbar}

\subsection*{p70-73:2-ロ)}
\addcontentsline{toc}{subsection}{\texorpdfstring{p70-73:2-ロ)}{p70-73:2-ロ)}}

\begin{tleftbar}
    与えられた連立方程式について,拡大係数行列を考えて基本変形を施すと,
    \[
        \begin{pmatrix} 1 & 0 & 0  & 2 & -1 \\ 0 & 1& 0  & -1 & 1 \\ 0 & 0 & 1 & -1 & -1 \\ 0 & 0 & 0 & 0 & 0 \end{pmatrix}
    \]
    となる.

    つまり,解は存在し,一つの任意定数を含む.任意定数を$x_4 = \alpha$とすると,
    \[
        x_1 = -1 -2\alpha   , \quad x_2 =1+\alpha  , \quad x_3 =-1+ \alpha  , \quad x_4 = \alpha 
    \]
    となる.ベクトルの形で表すと
    \[
        \begin{pmatrix} x_1 \\ x_2 \\ x_3 \\ x_4  \end{pmatrix}= \begin{pmatrix} -1 \\ 1 \\ -1 \\ 0 \end{pmatrix} +\alpha \begin{pmatrix} -2 \\ 1\\ 1 \\ 1\end{pmatrix}
    \]
    となる.
\end{tleftbar}

\newpage 

\subsection*{p70-73:2-ハ)}
\addcontentsline{toc}{subsection}{\texorpdfstring{p70-73:2-ハ)}{p70-73:2-ハ)}}

\begin{tleftbar}
    与えられた連立方程式について,拡大係数行列を考えて基本変形を施すと,
    \[
        \begin{pmatrix} 1 & 0 & 0  & 0 & 3 \\ 0 & 1& 0  & 0 & 4 \\ 0 & 0 & 1 & 0 & 1 \\ 0 & 0 & 0 & 1 & 1 \end{pmatrix}
    \]
    となる.

    つまり,解は存在し,
    \[
        x_1 = 3 , \quad x_2 =4 , \quad x_3 = 1 , \quad x_4 = 1
    \]
    となる.ベクトルの形で表すと
    \[
        \begin{pmatrix} x_1 \\ x_2 \\ x_3 \\ x_4 \end{pmatrix}= \begin{pmatrix} 3 \\ 4 \\ 1 \\ 1 \end{pmatrix} 
    \]
    となる.
\end{tleftbar}


\subsection*{p70-73:2-ニ)}
\addcontentsline{toc}{subsection}{\texorpdfstring{p70-73:2-ニ)}{p70-73:2-ニ)}}

\begin{tleftbar}
    与えられた連立方程式について,拡大係数行列を考えて基本変形を施すと,
    \[
        \begin{pmatrix} 1 & 0 & 0  & 2 & -2 & 0  \\ 0 & 1& 0  & 0 & 24 & 4 \\ 0 & 0 & 1 & 0 & 10 & 3 \end{pmatrix}
    \]
    となる.ただしここで第2列と第4列を入れ替えた.

    つまり,解は存在し,$2$つの任意定数を含む.$x_2 = \alpha$,$x_5 = \beta$とすると,
    \[
        x_1 = -2\alpha + 2\beta  , \quad x_2 =\alpha  , \quad x_3 = 3 -10\beta  , \quad x_4 = 4 -24 \beta ,\quad x_5 =\beta 
    \]
    となる.ベクトルの形で表すと
    \[
        \begin{pmatrix} x_1 \\ x_2 \\ x_3 \\ x_4 \\ x_5 \end{pmatrix}= \begin{pmatrix} 0 \\ 0 \\ 3 \\ 3 \\ 0 \end{pmatrix} +\alpha \begin{pmatrix} -2 \\ 1\\ 0 \\ 0 \\ 0 \end{pmatrix} + \beta \begin{pmatrix} 2 \\ 0 \\ -10 \\ -24 \\ 1 \end{pmatrix}
    \]
    となる.
\end{tleftbar}


\newpage
    \subsection*{p70-73:3-イ)}
    \addcontentsline{toc}{subsection}{\texorpdfstring{p70-73:3-イ)}{p70-73:3-イ)}}

    \begin{leftbar}
    \begin{align*} 
    &
    \left( 
        \begin{array}{ccc|ccc}
        1 & 3 & 2 & 1 & 0 & 0 \\
        -1 & -2 & -1 & 0 & 1 & 0 \\
        2 & 4 & 3 &  0 & 0 & 1
        \end{array}
        \right) \\
       \xrightarrow{\text{第$1$列を掃き出す}} &
    \left( 
        \begin{array}{ccc|ccc}
        1 & 3 & 2 & 1 & 0 & 0 \\
        0 & 1 & 1 & 1 & 1 & 0 \\
        0 & -2 & -1 &  -2 & 0 & 1
        \end{array}
        \right) \\
       \xrightarrow{\text{第$2$列を掃き出す}} &
       \left( 
           \begin{array}{ccc|ccc}
           1 & 0 & -1 & -2 & -3 & 0 \\
           0 & 1 & 1 & 1 & 1 & 0 \\
           0 & 0 & 1 &  0 & 2 & 1 
           \end{array}
           \right) \\
       \xrightarrow{\text{第$3$列を掃き出す}} &
       \left( 
           \begin{array}{ccc|ccc}
            1 & 0 & 0 & -2 & -1 & 1 \\
            0 & 1 & 0 & 1 & -1 & -1 \\
            0 & 0 & 1 &  0 & 2 & 1 
           \end{array}
           \right) 
        \end{align*}
        である.ゆえに
        \[
            P^{-1} = \begin{pmatrix} -2 & -1 & 1\\ 1 & -1 & -1 \\ 0 & 2 & 1 \end{pmatrix}
        \]
        である.だから
        \[
            P^{-1} A P = \begin{pmatrix} 2 & 0 & 0 \\ 0 & 1 & 0 \\ 0 & 0 & 0 \end{pmatrix}
        \]
    \end{leftbar}


    \subsection*{p70-73:3-ロ)}
    \addcontentsline{toc}{subsection}{\texorpdfstring{p70-73:3-ロ)}{p70-73:3-ロ)}}



    \begin{tleftbar}
        $P^{-1} A P = B$とおくと,
        \begin{align*}
            A^n &= P B^n P^{-1} \\
            & = \begin{pmatrix} 1 & 3 & 2 \\ -1 & -2 & -1 \\ 2 & 4 & 3 \end{pmatrix} \begin{pmatrix} 2^n & 0 & 0 \\ 0 & 1^n & 0 \\ 0 & 0 & 0^n \end{pmatrix} \begin{pmatrix} -2 & -1 & 1\\ 1 & -1 & -1 \\ 0 & 2 & 1 \end{pmatrix} \\
            & = \begin{pmatrix} -2^{n+1}+3 & -2^n -3 & 2^n -3 \\ 2^{n+1}-2 & 2^n +2 & -2^n -2 \\ -2^{n+2}+4 & -2^{n+1} -4 & 2^{n+1} -4 \end{pmatrix}
        \end{align*}
        となる.
    \end{tleftbar}
    \newpage 


\subsection*{p70-73:4}
\addcontentsline{toc}{subsection}{\texorpdfstring{p70-73:4}{p70-73:4}}

\begin{leftbar}
    与えられた行列を$A$とする.

    $A$の第1列に,第2列から第$n$列までを足して変形すると,

    \[
        \begin{pmatrix} 
            (n-1)x+1 & x & x&  \cdots & x \\
            (n-1)x+1 & 1 & x & \cdots & x \\
            (n-1)x+1 & x & 1 &  \cdots & x \\
            \vdots & \vdots & \vdots & \ddots & \vdots \\
            (n-1)x+1 & x & x& \cdots & 1 
        \end{pmatrix}
    \]
    となる.

    ここで,この行列の第$2$行から第$n$行までの各行から第$1$行を引くと,

    \[
    \begin{pmatrix} 
        (n-1)x+1 & x & x&  \cdots & x \\
        0 & 1-x & 0 & \cdots & 0 \\
        0  & 0 & 1-x &  \cdots & 0 \\
        \vdots & \vdots & \vdots & \ddots & \vdots \\
        0 & 0 & 0& \cdots & 1 -x 
    \end{pmatrix}
    \]
    となるので,行列$A$の階数は,$x=1$のとき$1$,$ x= -1/(n-1)$のとき$n-1$,それ以外の場合は$n$である.
\end{leftbar}


    \newpage 

    \subsection*{p70-73:5}
    \addcontentsline{toc}{subsection}{\texorpdfstring{p70-73:5}{p70-73:5}}
    

    \begin{leftbar}
        \begin{proof}
        $A$が正則でないと仮定すると,
        \[
           A \bm{x} = \bm{0}
        \]
        をみたす$\bm{x} \in \mathbb{C}^n$が存在する.

        また,$\bm{x} ={} ^t (x_1,x_2,\dots,x_n)$とし,$x_1,x_2,\dots,x_n$の中で絶対値が最大のものを$x_p$とする.
        
        $A \bm{x}$の$p$行を考えると,

        \begin{align*}
             & a_{p1} x_1 + a_{p2} x_2 + \dots + a_{pp} x_p + \dots + a_{pn} x_n = 0 \\
             \therefore & ~ x_p = -(a_{p1}x_1 + a_{p2} x_2+ \dots + a_{pn} x_n) =- \sum_{\substack{i \ne p \\ i \in \{1,2,\dots,n\}}} a_{pi} x_i 
        \end{align*}
        となる.

        ここで,
        \begin{align*} 
            \abs{x_p} & \leqq \sum_{\substack{i \ne p \\ i \in \{1,2,\dots,n\}}} \abs{a_{pi}} \abs{x_i} \\
            & < \sum_{\substack{i \ne p \\ i \in \{1,2,\dots,n\}}} \frac{1}{n-1} \abs{x_i} \\
            & < \frac{n-1}{n-1} \cdot \abs{x_p} =\abs{x_p}
        \end{align*} 
        と計算でき,$ \abs{x_p} < \abs{x_p}$となり,これは矛盾である.

        よって,先の過程が誤りであり,このとき$A$は正則である.
    \end{proof}
\end{leftbar}
    \newpage 

    \subsection*{p70-73:6}
    \addcontentsline{toc}{subsection}{\texorpdfstring{p70-73:6}{p70-73:6}}

    \begin{leftbar}
        \begin{description}
            \item[イ] 計算すると,
            \[
                A A^{k-1} = A^{k-1} A = E
            \]
            なので,$A$は正則である.
            \item[ロ] $A$が正則であるとすると,$A^{-1}$が存在して,
            \begin{align*}
                & A^{-1} A^{2} = A^{-1} A \\
                & A = E 
            \end{align*} 
            となるが,これは矛盾であるため,$A$は正則でない.
            \item[ハ] $A$が正則であるとすると,
            \begin{align*} 
                E & = (A^{-1} A)^{k} \\
                & = A^{-k} A^{k} \\
                & = O 
            \end{align*} 
            となるが,これは矛盾であるため,$A$は正則でない.
            \item[二] $k$を用いて,$A^k$を考えると
            \[
                E = (E-A)(E+A+A^2+\dots+A^{k-1})
            \]
            であり,逆からかけても同じであるため,$E-A$は正則であり,
            \[
                (E-A)^{-1}=E+A+A^2+\dots+A^{k-1}
            \]
            である.

            また,
            \[
                E=(E+A)(E-A+A^2-\dots+A^{k-1})
            \]
            であり,逆からかけても同じなので,$E+A$は正則であり,
            \[
                (E+A)^{-1} = E-A+A^2-\dots+A^{k-1}
            \]
            である.
        \end{description}
    \end{leftbar}

    \newpage 

\subsection*{p70-73:7}
\addcontentsline{toc}{subsection}{\texorpdfstring{p70-73:7}{p70-73:7}}

\begin{tleftbar}
\begin{proof}
$X=(x_{ij}),~Y=(y_{ij})$とする.
ここで,$XY$の$(i,i)$成分は$\sum_{j=1}^{n} x_{ij} y_{ji}$であるから,
\begin{equation*}
\tr (XY)  =  \sum_{i=1}^{n} \left( \sum_{j=1}^{n} x_{ij} y_{ji} \right)
\end{equation*}
となる.$YX$については,同様の議論により,
\begin{align*}
\tr  (YX) & =  \sum_{i=1}^{n}\left( \sum_{j=1}^{n}  y_{ij} x_{ji} \right) \\
 & =  \sum_{i=1}^{n} \left( \sum_{j=1}^{n}  x_{ji} y_{ij} \right)
\end{align*}
である.ここで,$i$と$j$をおきかえれば,
\begin{equation}
\tr  (YX) = \sum_{j=1}^{n} \left( \sum_{i=1}^{n}  x_{ij} y_{ji} \right)
\end{equation}
となる.これより,
\begin{equation}
\tr (XY) = \tr  (YX)
\end{equation}
を得て,これとトレースの線型性により$\tr (XY-YX) =0$であるが,$\tr  (E_n) =n \neq 0$であるため,これは矛盾である.

ゆえに,$XY-YX=E_n$となる$n$次行列$X,~Y$は存在しないことが示された.
\end{proof}
\end{tleftbar}

\newpage 

\subsection*{p70-73:8}
\addcontentsline{toc}{subsection}{\texorpdfstring{p70-73:8}{p70-73:8}}

\begin{leftbar}
    \begin{proof}
        行列$B$の階数を$r$とすると,$m$次正則行列$P$,$n$次正則行列$Q$によって,
        \[
            P B Q = F_{m,n} (r)
        \]
        と表せる.

        これにより,
        \[
            ABQ = A P^{-1} F_{m,n} (r) 
        \]
        とかける.$A_{11}$を$r$次の行列として,
        \[
            A P^{-1} = \begin{pmatrix} A_{11} & A_{12} \\ A_{21} & A_{22} \end{pmatrix}, \quad F_{m,n} (r) = \begin{pmatrix} E_r & O \\ O & O \end{pmatrix}
        \]
        とかくと,
        \begin{align*}
           A P^{-1} F_{m,n} (r)&= A P^{-1} Q \\
           & = \begin{pmatrix} A_{11}& O \\ A_{21} & O \end{pmatrix}
        \end{align*}
        とかけ,$A_{11}$の定義により,$ABQ$の階数は$r$以下となる.いま$Q$は基本行列の積なので, $AB$の階数も$r$以下である.

        行列$A$についても同様に示せる.
        
        以上の議論により,行列$AB$の階数は行列$A$,行列$B$の階数以下であることが証明された.

    \end{proof}
\end{leftbar}

\newpage 


\subsection*{p70-73:9}
\addcontentsline{toc}{subsection}{\texorpdfstring{p70-73:9}{p70-73:9}}

\begin{tleftbar}
    $3$つの平面が$1$本の直線を共有する必要十分条件は,
    与式を$x$,$y$,$z$に関する方程式とみたときに,解が存在して$1$つの任意定数を含むことである.

    これは
    \[
        \begin{cases} 
            r(A)=2 \\
            r(A)=r(\tilde{A})
        \end{cases}
    \]
    と同値であり,したがって,
    \[
        r(A)=r(\tilde{A})=2
    \]
    が必要十分条件である.
    \end{tleftbar}

\newpage 


\subsection*{p70-73:11}
\addcontentsline{toc}{subsection}{\texorpdfstring{p70-73:11}{p70-73:11}}

\begin{leftbar}
    \begin{description}
        \item[イ] ${}^t PP =E$を加味して$(P \pm E)$の転置行列を考えると
        \[
           {}^t (P \pm E) = {}^t P \pm {}^t PP = {}^t P {}^t (E \pm P)
        \]
        となり,これを用いると,

        \begin{align*} 
            {}^t A & ={}^t  \{ (P-E)(P+E)^{-1} \} \\
            & = {}^t (P+E)^{-1} {}^t (P-E) \\
            & = (E+{}^t P)^{-1} {}^t P {}^t P (E-P) \\
            & = \{ {}^t P (P+E)  \}^{-1} {}^t P (E-P) \\
            & = (P+E)^{-1} {}^t P^{-1} {}^t P (E-P) \\
            & = (P+E)^{-1} (E-P) \\
            & = -(P+E)^{-1} \{ (P+E)-2E \} \\
            & = -(P+E)^{-1} (P+E) +2E (P+E)^{-1} \\
            & = -(P+E) (P+E)^{-1} +2E (P+E)^{-1} \\
            & = (-(P+E)+2E) (P+E)^{-1} \\
            & = -(P-E) (P+E)^{-1} = -A
        \end{align*} 
        となり,これが証明すべきことであった. \qed
        \item[ロ] 計算すると,
        \begin{align*} 
            E-A & = E-(P-E) (P+E)^{-1} \\
            & = (P+E)(P+E)^{-1} - (P-E) (P+E)^{-1} \\
            & = \{ (P+E)-(P-E) \} (P+E)^{-1} \\
            & = 2(P+E)^{-1}
        \end{align*} 
        と変形でき,いま$(P+E)$が正則だから,$2(P+E)^{-1}$も正則であり,
        \[
            (E-A)^{-1} = \frac{1}{2} (P+E)
        \]
        である.\qed
        \item[ハ] まず,
        \begin{align*}
            E+A &=(P+E)(P+E)^{-1}+(P-E) (P+E)^{-1} \\
            & = \{ (P+E)+(P-E) \} (P+E)^{-1} \\
            & = 2P (P+E)^{-1}
        \end{align*}
        であるから,これを用いると
        \[
            (E+A)(E-A)^{-1} = 2P (P+E)^{-1} \frac{1}{2} (P+E) =P
        \]
        となり,これが証明すべきことであった.\qed 
    \end{description}
\end{leftbar}


\newpage 


\subsection*{p70-73:13-イ)}
\addcontentsline{toc}{subsection}{\texorpdfstring{p70-73:13-イ)}{p70-73:13-イ)}}

\begin{tleftbar}
    まず,
\begin{align*} 
    [ [X,Y],Z ] & = [XY-YX,Z] \\
    & = (XY-YX)Z -Z(XY-YX) \\
    & = XYZ -YXZ -ZXY +ZYX.
\end{align*}
同様に計算すると,
\begin{align*} 
    & [[Y,Z],X] = YZX -ZYX -XYZ +XZY, \\
    & [ [Z,X],Y] = ZXY -XZY -YZX +YXZ.
\end{align*} 
よって,
\[
    [[X,Y],Z] +[[Y,Z],X]+[[Z,X],Y]=O
\]
である.
\end{tleftbar}

\subsection*{p70-73:13-ロ)}
\addcontentsline{toc}{subsection}{\texorpdfstring{p70-73:13-ロ)}{p70-73:13-ロ)}}

\begin{tleftbar}
    \begin{proof}
        $X$,$Y$は交代行列だから,
        \[
          X=- {}^t X ,\quad Y = -{}^t Y .
        \]
        これを用いると,
        \begin{align*} 
            [X,Y] & = XY -YX \\
            & = (-{}^t X) (-{}^t Y) - (-{}^t Y)(-{}^t X) \\
            & = {}^t (YX) - {}^t (XY) \\
            & = -{}^t (XY-YX) \\
            & = -{}^t [X,Y]
        \end{align*} 
        となる.よってこのとき$[X,Y]$は交代行列である.
    \end{proof}
    \end{tleftbar}
\newpage 

    \subsection*{p70-73:13-ハ)}
    \addcontentsline{toc}{subsection}{\texorpdfstring{p70-73:13-ハ)}{p70-73:13-ハ)}}

    \begin{leftbar}
        \begin{proof}
            以下では
        \[
            Y = \begin{pmatrix} 0 & -z ' & y ' \\ z' & 0 & -x' \\ -y' & x ' & 0 \end{pmatrix},\quad \bm{y} = \begin{pmatrix} x' \\ y ' \\ z ' \end{pmatrix}
        \]
        とおく.
        \begin{description}
            \item[【$X+Y$と$\bm{x}+\bm{y}$について】]
        \[
            X + Y  = \begin{pmatrix} 0 & -z & y \\ z & 0 & -x \\ -y & x & 0 \end{pmatrix}+ \begin{pmatrix} 0 & -z ' & y ' \\ z' & 0 & -x' \\ -y' & x ' & 0 \end{pmatrix} = \begin{pmatrix} 0 & -(z+z)' & y+y' \\ z+z & 0 & -(x+x') \\ -(y+y') & x+x' & 0 \end{pmatrix}
        \]
        であり,なおかつ
        \[
            \bm{x}+\bm{y} =\begin{pmatrix} x \\ y \\ z \end{pmatrix}+ \begin{pmatrix} x' \\ y' \\ z' \end{pmatrix} = \begin{pmatrix} x+x' \\ y+y' \\ z+z' \end{pmatrix}
        \]
        であるから,たしかに$X+Y$と$\bm{x}+\bm{y}$は対応する.
        \item[【$cX$と$c\bm{x}$について】]
        \[
            cX =c\begin{pmatrix} 0 & -z & y \\ z & 0 & -x \\ -y & x & 0 \end{pmatrix} =\begin{pmatrix} 0 & -cz & cy \\ cz & 0 & -cx \\ -cy & cx & 0 \end{pmatrix}
        \]
        であり,なおかつ
        \[
            c\bm{x} = c\begin{pmatrix} x \\ y \\ z \end{pmatrix} = \begin{pmatrix} cx \\ cy \\ cz \end{pmatrix}
        \]
        であるから,たしかに$cX$と$c\bm{x}$は対応する.
        \item [【\text{$[X,Y]$と$\bm{x} \times \bm{y}$について】}]
        \begin{align*} 
            [X,Y] &=\begin{pmatrix} 0 & -z & y \\ z & 0 & -x \\ -y & x & 0 \end{pmatrix}\begin{pmatrix} 0 & -z ' & y ' \\ z' & 0 & -x' \\ -y' & x ' & 0 \end{pmatrix}-\begin{pmatrix} 0 & -z ' & y ' \\ z' & 0 & -x' \\ -y' & x ' & 0 \end{pmatrix} \begin{pmatrix} 0 & -z & y \\ z & 0 & -x \\ -y & x & 0 \end{pmatrix} \\
            & = \begin{pmatrix} 0 & -z'x+x'z & y'x-x'y \\ z'x-x'z & 0 & -y'z+z'y \\ -y'x+x'y & z'y-y'z & 0 \end{pmatrix}
        \end{align*} 
        であり,なおかつ
        \[
            \bm{x}\times \bm{y} = \begin{pmatrix} x \\ y \\ z \end{pmatrix} \times \begin{pmatrix} x' \\ y ' \\ z' \end{pmatrix} = \begin{pmatrix} yz'-zy' \\ zx'-xz' \\ xy'-yx' \end{pmatrix}
        \]
        であるから,たしかに$[X,Y]$と$\bm{x} \times \bm{y}$は対応する.
        \item[【$X\bm{y}$と$\bm{x}\times\bm{y}$について】]
        \[
            \begin{pmatrix} 0 & -z & y \\ z & 0 & -x \\ -y & x & 0 \end{pmatrix} \begin{pmatrix} x' \\ y' \\ z' \end{pmatrix} = \begin{pmatrix} -zy'+yz' \\ zx'-xz' \\ -yx'+xy' \end{pmatrix}
        \]
        であり,なおかつ
   
        \[
            \bm{x}\times \bm{y} = \begin{pmatrix} x \\ y \\ z \end{pmatrix} \times \begin{pmatrix} x' \\ y ' \\ z' \end{pmatrix} = \begin{pmatrix} yz'-zy' \\ zx'-xz' \\ xy'-yx' \end{pmatrix}
        \]
        であるから,たしかに$X\bm{y}$と$\bm{x} \times \bm{y}$は対応する.
    \end{description}
    \end{proof}
    \end{leftbar}
    

    \subsection*{p70-73:13-ニ)}
    \addcontentsline{toc}{subsection}{\texorpdfstring{p70-73:13-ニ)}{p70-73:13-ニ)}}

    \begin{tleftbar}
        \begin{proof}
            ハ)で証明したことから,$[X,Y]$には$\bm{x} \times \bm{y}$が対応する.

            また,イ)で証明したことより,$[[X,Y],Z] +[[Y,Z],X]+[[Z,X],Y]=O$であり,
            この左辺には$(\bm{x}\times\bm{y}) \times \bm{z} + (\bm{y}\times\bm{z}) \times \bm{x} + (\bm{z}\times\bm{x}) \times \bm{y}$が対応し,右辺には$\bm{0}$が対応する.

            以上の考察により
            \[
                (\bm{x}\times\bm{y}) \times \bm{z} + (\bm{y}\times\bm{z}) \times \bm{x} + (\bm{z}\times\bm{x}) \times \bm{y} =\bm{0}
            \]
            であることが示された.
        \end{proof}
    \end{tleftbar}
    
\newpage


\subsection*{p70-73:14}
\addcontentsline{toc}{subsection}{\texorpdfstring{p70-73:14}{p70-73:14}}

\begin{leftbar}
\begin{proof} 
    二つに分けて証明する,
\begin{description}
    \item[イ)$\Longrightarrow$ロ)] \mbox{} \\
    $A$が正則であると仮定すると,$A^{-1}$が存在し,
    \[
        \bm{x} = A^{-1} (A\bm{x})
    \]
    と変形できるから,$A \bm{x}$が非負ベクトルであれば,$\bm{x}$も非負ベクトルである.
    \item[ロ)$\Longrightarrow$イ)] \mbox{} \\
    まず,$ A \bm{x} =\bm{0}$である$\bm{x}$が存在すると仮定する.このとき,$A (-\bm{x}) =\bm{0}$であるから,
    $A (-\bm{x})$も非負ベクトルであり,条件から$ \bm{x}$,$-\bm{x}$は非負ベクトルである.
    したがって$\bm{x}=\bm{0}$となり,$A$は正則である.

    また,非負ベクトル$\bm{x}$を任意にとると,
    \[
        \bm{x} = A (A^{-1} \bm{x})
    \]
    も非負ベクトルであり,条件から$A^{-1} \bm{x}$も非負ベクトルである.
    ここで,$A^{-1}$が非負行列でないと仮定すると,ある単位ベクトル$\bm{e}_j$について,
    $A^{-1} \bm{e}_j $が非負ベクトルでないことになり,$\bm{x}$が非負ベクトルであることに反する.
    これより$A^{-1}$は非負行列である.
\end{description}
以上の議論により証明された.
\end{proof}
\end{leftbar}


\newpage 



\subsection*{p70-73:15}
\addcontentsline{toc}{subsection}{\texorpdfstring{p70-73:15}{p70-73:15}}

\begin{leftbar}
    \begin{description}
        \item[イ] まず,$A=(a_{ij})$,$\bm{f} = {}^t (f_1 , f_2,\dots,f_j) ={}^t (1,1,\dots,1)$とおくと,
        $A \bm{f}$の第$i$行の成分は
        \begin{align*}
            \sum_{j=1}^{n} a_{ij} f_j &= \sum_{j=1}^{n} a_{ij} \\
            &=1 
        \end{align*}
        であるから,$\bm{f}$の定義とあわせて,
        \[
            A \bm{f} =\bm{f}
        \]
        が成り立つ.\qed 
        \item[ロ] $C =AB=(c_{ij})$とすると,$C$の$(i,k)$成分は
        \[
            c_{ik}  =\sum_{j=1}^{n} a_{ij} b_{jk}
        \]
        である.これにより,
        \begin{align*} 
            \sum_{k=1}^{n} c_{ik} & = \sum_{k=1}^{n} \left (\sum_{j=1}^{n} a_{ij} b_{jk}\right) \\
            & = \sum_{j=1}^{n} a_{ij} \sum_{k=1}^{n} b_{jk} \\
            & = \sum_{j=1}^{n} a_{ij} \cdot 1 \\
            & = 1
        \end{align*} 
        であるから,$C$すなわち$AB$は確率行列である.\qed
        \item[ハ] $ A \bm{x}=\alpha \bm{x}$において,$\bm{x}$の成分で絶対値が最大のものを$x_p$とする.
    
        このとき,$ A \bm{x} = \alpha \bm{x}$の第$p$行成分の絶対値を考えると,
        \begin{align*} 
            \abs{\alpha} \abs{x_p} &\leqq \sum_{j=1}^{n} a_{pj} \abs{x_j} \\
            & \leqq \sum_{j=1}^{n} a_{pj} \abs{x_p} \\
            & = \abs{x_p}
        \end{align*} 
        であるから,
        \[
            \abs{\alpha} \abs{x_p} \leqq \abs{x_p}
        \]
        を得るので,
        \[
            \abs{\alpha} \leqq 1
        \]
        となり,これが証明すべきことであった.\qed
        \end{description}
    \end{leftbar}

    \newpage 


\section*{第3章}
\addcontentsline{toc}{section}{\texorpdfstring{第3章}{第3章}}

\subsection*{p77:問2}
\addcontentsline{toc}{subsection}{\texorpdfstring{p77:問2}{p77:問2}}

\begin{tleftbar}
    \begin{proof}
$S_n$の偶置換全体の集合を$A_n$,偶置換全体の集合を$B_n$とする.
置換は必ず奇置換か偶置換のいずれかであるから, 
\begin{align*} 
    & S_n = A_n \cup B_n , \\
    &A_n \cap B_n = \varnothing
\end{align*} 
となる.

ここで,
\[
    \tau = \begin{pmatrix} 1 & 2 & 3 \\ 2 & 1 & 3 \end{pmatrix}
\]
とすると,$\tau$は奇置換であり,$\sigma \in  A_n$のとき,$ \tau \sigma \in B_n$である.
同様に,$ \rho  \in B_n$のとき,$\tau^{-1} \rho = \tau \rho \in A_n$である.
これらにより,全単射
\[
    A_n \ni \sigma \mapsto \tau \sigma \in B_n
\]
が存在し,偶置換と奇置換は同数あり,その個数は$n! /2$である.
\end{proof}
\end{tleftbar}


\subsection*{p77:問3}
\addcontentsline{toc}{subsection}{\texorpdfstring{p77:問3}{p77:問3}}

\begin{leftbar}
    $m \in \mathbb{N}$とする.
    \begin{enumerate}[(I)]
        \item $n=2m$とかけるとき,この置換を互換の積で表すと,
        \[
            (1,2m)(2,2m-1) \dotsm (m,m+1)
        \] 
        となるため,置換の符号は$(-1)^m$,すなわち
        \[
            (-1)^{\frac{n}{2}}
        \]
        となる.
        \item $n=2m-1$とかけるとき,この置換を互換の積で表すと,
        \[
            (1,2m-1)(2,2m-2) \dotsm (m-1,m+1)
        \] 
        となるため,置換の符号は$(-1)^{m-1}$,すなわち
        \[
            (-1)^{\frac{n-1}{2}}
        \]
        となる.
    \end{enumerate}
\end{leftbar}

\newpage 


\subsection*{p79:問}
\addcontentsline{toc}{subsection}{\texorpdfstring{p79:問}{p77:問}}

\begin{leftbar}
    \begin{description}
        \item[イ)] まず
        \[
            \sigma = \begin{pmatrix} 1 & 2 & \cdots & n \\ n & n-1 & \cdots & 1 \end{pmatrix}
        \]
        とすると,$m \in \mathbb{N}$として,
        \[
            \sgn \sigma =
            \begin{cases} 
                (-1)^\frac{n}{2} & \text{($n=2m$のとき)}  \\
                (-1)^\frac{n-1}{2} & \text{($n=2m-1$のとき)}
            \end{cases}
        \]
        となる.
        また,
        \[
            \text{(与式)}  = \sum_{\sigma \in S_n} \sgn \sigma \cdot a_1 a_2 \dotsm a_n 
        \]
        だから,
        \[
            \text{(与式)}  = 
            \begin{cases} 
             (-1)^\frac{n}{2}   a_1 a_2 \dotsm a_n & \text{($n=2m$のとき)} \\
             (-1)^\frac{n-1}{2} a_1 a_2 \dotsm a_n & \text{($n=2m-1$のとき)}
            \end{cases}
        \]
        である.
        \item[ロ)]計算すると,
        \begin{align*} 
            \text{(与式)} & = a^3 + b^3 + c^3 -abc -bca -cab \\
            & = a^3 + b^3 +c^3 -3abc 
        \end{align*}
        となる.
    \end{description}
\end{leftbar}

\newpage 

\subsection*{p83:問}
\addcontentsline{toc}{subsection}{\texorpdfstring{p83:問}{p83:問}}

\begin{leftbar}
    \begin{proof}
    $(n,n)$行列$A$,$X$を
    \[
        A = (\bm{a}_1 ,\bm{a}_2, \dots ,\bm{a}_n) , \quad X = (\bm{x}_1,\bm{x}_2,\dots,\bm{x}_n)
    \]
    とする.このとき,$AX$は定義され,
    \[
        AX = (A\bm{x}_1 , A\bm{x}_2 , \dots ,A\bm{x}_n)
    \]
    と表せる.ここで,$A \bm{x}_j$を単位ベクトルの線型結合で表すと,
    \begin{align*} 
        A \bm{x}_j & = A {}^t (x_{1j} \bm{e}_1 , x_{2j} \bm{e}_2,\ldots ,x_{nj} \bm{e}_n ) \\
        &=A (x_{1j} \bm{e}_1 + x_{2j} \bm{e}_2+ \dots + x_{nj} \bm{e}_n) \\
        & = x_{1j} \bm{a}_1 + x_{2j} \bm{a}_2 + x_{nj} \bm{a}_n \\
        & = \sum_{ i =1}^{n} x_{ij} \bm{a}_{i}
    \end{align*} 
    となる.これにより,$\abs{AX}$は,多重線型性を用いて,
    \begin{align*} 
        \abs{AX} & = \abs{\sum_{i_1 =1}^{n} x_{i_1 1} \bm{a}_{i_1} , \sum_{i_2 =1}^{n} x_{i_2 2} \bm{a}_{i_2},\ldots , \sum_{i_n =1}^{n} x_{i_n n} \bm{a}_{i_n} } \\
        & = \sum_{i_1 =1}^{n} \sum_{i_2 = 1}^{n} \dots \sum_{i_n =1}^{n} x_{i_1 1} x_{i_2 2} \dots x_{i_n n} \abs{\bm{a}_{i_1},\bm{a}_{i_2},\dots,\bm{a}_{i_n}} 
    \end{align*} 
    と変形できる.ここで,
    \[
        \sigma = \begin{pmatrix} 1 & 2 & \cdots & n \\ i_1 & i_2 & \cdots & i_n \end{pmatrix}
    \]
    とおくと,
    \[
        \abs{\bm{a}_{\sigma (1)},\bm{a}_{\sigma(2)},\dots,\bm{a}_{\sigma(n)}} = \sgn \sigma \abs{A}
    \]
    \begin{align*} 
        \abs{AX} = & \sum_{\sigma \in S_n} x_{\sigma (1)1} x_{\sigma(2)2} \dots x_{\sigma(n)n} \cdot \sgn \sigma  \abs{A} \\
        & = \sum_{\sigma \in S_n } \sgn \sigma \cdot x_{\sigma(1)1} x_{\sigma(2)2} \dots x_{\sigma (n)n} \abs{A} \\
        & = \abs{{}^t X} \abs{A} \\
        & = \abs{A} \abs{X}
    \end{align*} 
    を得る.これが証明すべきことであった.
\end{proof}
\end{leftbar}

\newpage 

\subsection*{p83:問-(イ)}
\addcontentsline{toc}{subsection}{\texorpdfstring{p83:問-(イ)}{p83:問-(イ)}}

\begin{tleftbar}
    多重線型性などを用いて変形すると,
    \begin{align*} 
        (\text{与式}) & = -
        \begin{vmatrix} 
            2 & -5 & 3 & 10 \\
            1 & 0 & 2 & -3 \\
            5 & 3 & -2 & 2 \\
            -3 & -2 & 4 & 2 
        \end{vmatrix} 
        = - 
        \begin{vmatrix}
            0 & -5 & -1 & 16 \\
            1 & 0 & 2 & -3 \\
            0 & 3 & -12 & 17 \\
            0 & -2 & 10 & -7 
        \end{vmatrix} 
        = 
        \begin{vmatrix}
            1 & 0 & 2 & -3 \\
            0 & -5 & -1 & 16 \\
            0 & 3 & -12 & 17 \\
            0 & -2 & 10 & -7 
        \end{vmatrix} \\
        & = 1 \cdot (-1)^{1+1} 
        \begin{vmatrix} 
            -5 & -1 & 16 \\
            3 & -12 & 17 \\
            -2 & 10 & -7 
        \end{vmatrix} 
        =539 
    \end{align*} 
    となるので,この行列式の値は$539$である.
    \end{tleftbar}


    \subsection*{p83:問-(ロ)}
    \addcontentsline{toc}{subsection}{\texorpdfstring{p83:問-(ロ)}{p83:問-(ロ)}}
    
    \begin{tleftbar}
        多重線型性などを用いて変形すると,
        \begin{align*} 
            (\text{与式}) & = -
            \begin{vmatrix} 
                2 & 3 & 5 & -4 \\
                1 & -7 & -8 & 6 \\
                3 & 10 & 6 & 1\\
                5 & 2 & 4 & 3 
            \end{vmatrix} 
            = - 
            \begin{vmatrix} 
                0 & 17 & 21 & -16 \\
                1 & -7 & -8 & 6 \\
                0 & 31 & 30 & -17 \\
                0 & 37 & 44 & -27 
            \end{vmatrix} 
            = 
            \begin{vmatrix} 
                1 & -7 & -8 & 6 \\
                0 & 17 & 21 & -16 \\
                0 & 31 & 30 & -17 \\
                0 & 37 & 44 & -27 
            \end{vmatrix}
            \\
            & = 1 \cdot (-1)^{1+1} 
            \begin{vmatrix} 
                17 & 21 & -16 \\
                31 & 30 & -17 \\
                37 & 44 & -27 
            \end{vmatrix}
        \end{align*} 
        となる.ここで,第$2$列に第$1$列の$-1$倍を加え,第$3$列に第$1$列を加えると,
        \[
            (\text{与式}) =
             \begin{vmatrix} 
                17 & 4 & 1\\
                31 & -1 & 14 \\
                37 & 7 & 10 
             \end{vmatrix}
        \]
        を得る.ここで,第$1$列に第$3$列の$-2$倍を加えると,
        \[
            (\text{与式})= 
            \begin{vmatrix} 
                15 & 4 & 1\\
                3 & -1 & 14 \\
                17 & 7 & 10 
            \end{vmatrix} 
            = -750
        \]
        となるため,この行列式の値は$-750$である.
        \end{tleftbar}

    \newpage 
\section*{第3章・章末問題}
\addcontentsline{toc}{section}{\texorpdfstring{第3章・章末問題}{第3章・章末問題}}


\subsection*{p90-91:3}
\addcontentsline{toc}{subsection}{\texorpdfstring{p90-91:3}{p90-91:3}}

\begin{leftbar}
    \begin{description}
        \item[イ] 与えられた行列式に対して多重線型性を用いると,
        \begin{align*} 
            \text{(与式)} & = 
            \begin{vmatrix}
                A+B & A+B \\
                B & A 
            \end{vmatrix}
            \\
            & = \begin{vmatrix}
                A+B &O \\
                B & A-B 
            \end{vmatrix}
            \\
            & = \abs{A+B} \cdot \abs{A-B}
        \end{align*} 
        となり,これが証明すべきことであった.\qed 
        \item [ロ] 与えられた行列式に対して多重線型性を用いると,
        \begin{align*} 
            \text{(与式)} & = 
            \begin{vmatrix} 
                A+iB & iA-B \\
                B & A 
            \end{vmatrix}
            \\
            & = \begin{vmatrix}
                A+iB & O \\
                B & A-iB 
            \end{vmatrix}
            \\
            & = \det (A+iB) \cdot \det(A-iB)
        \end{align*} 
        となり,いま$A$,$B$は実行列なので,
        \begin{align*} 
            \det (A+iB) \cdot \det(A-iB) & = \det (A+iB) \cdot \overline{\det (A+iB)} \\
            & = \abs{\det(A+iB)}^2
        \end{align*} 
        である.
    \end{description}
\end{leftbar}

\newpage 


\subsection*{p90-91:4}
\addcontentsline{toc}{subsection}{\texorpdfstring{p90-91:4}{p90-91:4}}

\begin{leftbar}
    \begin{proof}
    $\alpha ^n =1$をみたす$\alpha \in \mathbb{C}$をひとつ固定する.
    さて,与えられた行列式の第$j$行を$\alpha^{j-1}$倍して第$1$列に足す操作を行うと,この行列式は
    \[
        \begin{vmatrix}
            \sum_{i=0}^{n-1} \alpha^i x_i & x_1 & x_2 & \cdots & x_{n-1} \\
            \alpha \sum_{i=0}^{n-1} \alpha^i x_i & x_0 & x_1 & \cdots & x_{n-2} \\
            \alpha^2 \sum_{i=0}^{n-1} \alpha^i x_i & x_{n-1} & x_0 & \cdots & x_{n-3} \\
            \vdots & \vdots & \vdots & \vdots & \vdots  \\
            \alpha^{n-1} \sum_{i=0}^{n-1} \alpha^i x_i & x_{n-2} & x_{n-3} & \cdots & x_0 
        \end{vmatrix}
    \]
    と変形できる.これにより,この行列式は
    \[
        \sum_{i=0}^{n-1} \alpha^i x_i = x_0 + \alpha x_1 + \alpha^2 x_2 + \dots +\alpha^{n-1} x_{n-1}
    \]
    を因数にもつ.すべての$\alpha$に関してこのことがいえるから,因数定理により,この行列式は
    \[
        \prod_{\alpha^n=1} (x_0 + \alpha x_1 + \alpha^2 x_2 + \dots +\alpha^{n-1} x_{n-1})
    \]
を因数にもつ.これは$n$次式であり,なおかつ$x_0$の係数は$1$であることより,結果として
\[
   \begin{vmatrix} 
    x_0 & x_1 & x_2 & \cdots & x_{n-1} \\
    x_{n-1} & x_0 & x_1 & \cdots & x_{n-2} \\
    \vdots & \vdots & \vdots & \vdots & \vdots \\
    x_1 & x_2 & x_3 & \cdots & x_0 
   \end{vmatrix} 
   =  \prod_{\alpha^n=1} (x_0 + \alpha x_1 + \alpha^2 x_2 + \dots +\alpha^{n-1} x_{n-1})
\]
である.これが証明すべきことであった.
\end{proof}
\end{leftbar}


\subsection*{p90-91:5}
\addcontentsline{toc}{subsection}{\texorpdfstring{p90-91:5}{p90-91:5}}

\begin{leftbar} 
    前問において,$n=4$,$x_1 = i$,$x_2 = 1$,$x_3=-i$とした場合を考えればよいので,$\alpha = \pm 1 , \pm i$により,
    \begin{align*} 
        \text{(与式)} & = \prod_{\alpha^4=1} (x+ \alpha i +\alpha^2 -  \alpha^3i ) \\
        & = (x+i+1-i) (x-i+1+i)(x-1-1-1)(x+1-1+1) \\
        & =(x+1)^3(x-3)
    \end{align*} 
    となる.
\end{leftbar}

\newpage 

\subsection*{p90-91:6}
\addcontentsline{toc}{subsection}{\texorpdfstring{p90-91:6}{p90-91:6}}

\begin{tleftbar}
    \begin{proof}
        $i \in \{ 1,2,\ldots,n \}$のもとで,$n$個の点を$(x_i , y_i ) \in \mathbb{R}^2$とする.このとき,
        \[
            \begin{cases} 
                a_0 + a_1 x_1 + a_2 {x_1}^2 + \dots + a_n {x_1}^{n-1} = y_1 \\
                a_0 + a_1 x_2 + a_2 {x_2}^2 + \dots + a_n {x_2}^{n-1} = y_2\\
                \vdots \\
                a_0 + a_1 x_n + a_2 {x_n}^2 + \dots + a_n {x_n}^{n-1} = y_3
            \end{cases}
        \]
        である,これを行列の形に表すと,
        \[
            \begin{pmatrix} 
                1 & x_1 & {x_1}^2 & \cdots & {x_1}^{n-1} \\
                1 & x_2 & {x_2}^2 & \cdots & {x_2}^{n-1} \\
                \vdots & \vdots & \vdots & \ddots & \vdots \\
                1 & x_n & {x_n}^2 & \cdots & {x_n}^{n-1}
            \end{pmatrix}
            \begin{pmatrix}
                a_0 \\
                a_1 \\
                \vdots \\
                a_n
            \end{pmatrix}
            =
            \begin{pmatrix}
                y_1 \\
                y_2 \\
                \vdots \\
                y_n
            \end{pmatrix}
        \]
        となる.

        ここで,
        \[
            A \coloneqq 
        \begin{pmatrix} 
            1 & x_1 & {x_1}^2 & \cdots & {x_1}^{n-1} \\
            1 & x_2 & {x_2}^2 & \cdots & {x_2}^{n-1} \\
            \vdots & \vdots & \vdots & \ddots & \vdots \\
            1 & x_n & {x_n}^2 & \cdots & {x_n}^{n-1}
        \end{pmatrix}
        \]
        とおくと,$\abs{{}^t A}$はヴァンデルモンドの行列式である.

        行列式の値は,行列の転置に対して不変なので,
        \[
            \abs{A}= \prod_{i < j} (x_j - x_i)
        \]
        となり,条件によりこの値は$0$でない.ゆえに先の連立方程式はただ一つの解をもつ.

        以上の考察によって,これら$n$個の点を通る直線がただ一つ存在することが示された.
    \end{proof}
\end{tleftbar}

\newpage 

\subsection*{p90-91:9}
\addcontentsline{toc}{subsection}{\texorpdfstring{p90-91:9}{p90-91:9}}

\begin{tleftbar}
    \begin{proof}
    3点$\mathrm{P}_1$,$\mathrm{P_2}$,$\mathrm{P_3}$を通る平面の方程式を$ax+by+cz+d=0$とおく.
    このとき,
    \[
        \begin{cases}
            ax+by+cz + d =0 \\
            ax_1 + by_1 +cz_1 +d =0 \\
            ax_2 + by_2 +cz_2 +d =0 \\
            ax_3 + by_3 +cz_3 +d =0
        \end{cases}
    \]
    が成立する.すなわちこれは
    \[
        \begin{pmatrix} 
            x & y & z & 1 \\
            x_1 & y_1 & z_1& 1 \\
            x_2 & y_2 &z_2 & 1 \\
            x_3 & y_3 &z_3 & 1 
        \end{pmatrix}
        \begin{pmatrix}
            a \\
            b \\
            c \\
            d 
        \end{pmatrix}
        = \bm{0}
    \]
    をみたす.これを$a$,$b$,$c$,$d$についての連立方程式とみたとき,与条件により自明でない解があり,
    \[
    \begin{vmatrix} 
        x & y & z & 1 \\
        x_1 & y_1 & z_1& 1 \\
        x_2 & y_2 &z_2 & 1 \\
        x_3 & y_3 &z_3 & 1 
    \end{vmatrix}
    =0
    \]
    が成立する.これが証明すべきことであった.
\end{proof}
\end{tleftbar}

\subsection*{p90-91:11}
\addcontentsline{toc}{subsection}{\texorpdfstring{p90-91:11}{p90-91:11}}
\begin{tleftbar}
    \begin{description}
        \item[イ] ${}^t A_\sigma$は$(j,\sigma(j))$成分が$1$でそれ以外が$0$である行列である.
        いま$A = (\bm{a}_1,\bm{a}_2,\ldots,\bm{a}_n)$とすると,
        \begin{align*}
            {}^t A_\sigma A_\sigma &= 
            \begin{pmatrix} 
                (\bm{a}_1,\bm{a}_1) & (\bm{a}_1,\bm{a}_2) & \cdots & (\bm{a}_1,\bm{a}_n) \\ 
                (\bm{a}_2,\bm{a}_1) & (\bm{a}_2,\bm{a}_2) & \cdots & (\bm{a}_2,\bm{a}_n) \\
                \vdots & \vdots & \ddots & \vdots \\
                (\bm{a}_n,\bm{a}_1) & (\bm{a}_n,\bm{a}_2) & \cdots & (\bm{a}_n,\bm{a}_n)
            \end{pmatrix}
            \\
            & = \begin{pmatrix} 1 & 0 & \cdots & 0 \\ 0 & 1 & \cdots & 0 \\ \vdots & \vdots & \ddots & \vdots \\ 0 & 0 & \cdots & 1 \end{pmatrix} \\
            &= E
        \end{align*}
        となり,$A$は直交行列である.
    \end{description}
\end{tleftbar}

\newpage 


\section*{第4章}
\addcontentsline{toc}{section}{\texorpdfstring{第4章}{第4章}}



\subsection*{p93:問}
\addcontentsline{toc}{subsection}{\texorpdfstring{p93:問}{p239:問}}

\begin{leftbar}
    \begin{proof} 
        $\abs{A \cup B}$について,$ \abs{A \cap B}$は$A$と$B$の共通部分の元の個数を考えているので,
        \begin{align*} 
            & \abs{A \cup B} = \abs{A}+\abs{B} - \abs{A \cap B} \\
            & \therefore ~ \abs{A}+\abs{B} = \abs{A \cup B} + \abs{A \cap B}
        \end{align*} 
        である.これが証明すべきことであった.
    \end{proof}
    \end{leftbar}

    \newpage 


\subsection*{p94:問}
\addcontentsline{toc}{subsection}{\texorpdfstring{p94:問}{p94:問}}

\begin{tleftbar}
    3つのことを証明する.
    \begin{description}
    \item [反射律について] 明らかに,$A$に基本変形を施して$A$自身にすることができる.
    \item [対称律について] $P$を$(m,m)$型の基本行列,$Q$を$(n,n)$型の基本行列として,
    \[
      A = P B Q
    \]
    とかくと,$P$,$Q$は正則なので, $P^{-1}$,$Q^{-1}$が存在し,
    \[
      B= P^{-1} A Q^{-1}
    \]
    とかける.よって,対称律が成り立つことが示された.
    \item[推移律について] $P_1$,$P_2$を$(m,m)$型の基本行列,$Q_1$,$Q_2$を$(n,n)$型の基本行列として,
    \[
      A = P_1 B Q_1 , \quad B = P_2 C Q_2
    \]
    とかく.このとき,$P_1$,$Q_1$は正則だから,${P_1}^{-1}$,${Q_1}^{-1}$が存在し,
    \[
      B = {P_1}^{-1} A {Q_1}^{-1}
    \]
    となる.これにより,
    \[
      {P_1}^{-1} A {Q_1}^{-1} =P_2 C Q_2
    \]
    となり,同様の議論によって
    \[
      A = P_1 P_2 C Q_2 Q_1
    \]
    となり,推移律も成り立つことが示された.\qed 
  \end{description}
  さて,行列$A$に基本変形を施すと,$A$の階数を$r$として$F_{m,n} (r)$が得られることと,
  $r$は$0$から$\min \{m,n\}$までの整数値を取り得るので,商集合の元の個数は
  \[
    \min \{ m , n \} +1
  \]
  となる.
\end{tleftbar}


\newpage 


\subsection*{p106-107:問1}
\addcontentsline{toc}{subsection}{\texorpdfstring{p106-107:問1}{p106-107:問1}}
\begin{leftbar}
求める$E\to F$の取り替え行列を$P=(p_{ij})$とし,
\begin{align*}
&\bm{e}_1=
\begin{pmatrix}
1 \\
0 \\
1
\end{pmatrix}
,\quad 
\bm{e}_2=
\begin{pmatrix}
2 \\
1 \\
0
\end{pmatrix}
,\quad 
\bm{e}_3=
\begin{pmatrix}
1 \\
1 \\
1
\end{pmatrix}
,\\
&
\bm{f}_1=
\begin{pmatrix}
3 \\
-1 \\
4
\end{pmatrix}
,\quad 
\bm{f}_2=
\begin{pmatrix}
4 \\
1 \\
8
\end{pmatrix}
,\quad 
\bm{f}_3=
\begin{pmatrix}
3 \\
-2 \\
6
\end{pmatrix}
\end{align*}
とする.ここで,
\[
\bm{f}_i=\sum^{3}_{j=1}p_{ji}\bm{e}_{j}=p_{1i}\bm{e}_1+p_{2i}\bm{e}_2+p_{3i}\bm{e}_3
\]
であり,$i=1,2,3$の場合についての連立方程式を作ると
\begin{align*}
&f_1=p_{11}\bm{e}_1+p_{21}\bm{e}_2+p_{31}\bm{e}_3 \\
&f_2=p_{12}\bm{e}_1+p_{22}\bm{e}_2+p_{32}\bm{e}_3 \\
&f_3=p_{13}\bm{e}_1+p_{23}\bm{e}_2+p_{33}\bm{e}_3
\end{align*}
これを解くことにより
\begin{align*}
&p_{11}=\frac{9}{2},\quad p_{21}=-\frac{1}{2},\quad p_{31}=-\frac{1}{2},\\
&p_{12}=5,\quad p_{22}=-2,\quad p_{32}=3,\\
&p_{13}=\frac{13}{2},\quad p_{23}=-\frac{3}{2},\quad p_{33}=-\frac{1}{2}
\end{align*} 
なので,
\[
P=
\begin{pmatrix}
9/2 & 5  & 13/2 \\
-1/2 & -2 & -3/2 \\
-1/2 & 3 & -1/2
\end{pmatrix}
\]
である.また
\begin{equation*}
(\bm{f}_1,\bm{f}_2,\ldots ,\bm{f}_n)=(\bm{e}_1,\bm{e}_2,\ldots ,\bm{e}_n)P
\end{equation*}
であるから
\begin{align*}
    &
\begin{pmatrix}
3 & 4 & 3 \\
-1 & 1 & -2 \\
4 & 8 & 6 
\end{pmatrix}
=
\begin{pmatrix}
1 & 2 & 1 \\
0 & 1 & 1 \\
1 & 0 & 1 
\end{pmatrix}
P \\ 
&P=
\begin{pmatrix}
3 & 4 & 3 \\
-1 & 1 & -2 \\
4 & 8 & 6 \\
\end{pmatrix}
\begin{pmatrix}
1 & 2 & 1 \\
0 & 1 & 1 \\
1 & 0 & 1 \\
\end{pmatrix}
^{-1}
\end{align*}
から求めることもできる.
\end{leftbar}

\newpage 

\subsection*{p106-107:問2}
\addcontentsline{toc}{subsection}{\texorpdfstring{p106-107:問2}{p106-107:問2}}

\begin{leftbar}
まず,
\begin{equation*}
\bm{f}_i=\sum^{2}_{j=1}p_{ji}\bm{e}_{j}=p_{1i}\bm{e}_1+p_{2i}\bm{e}_2
\end{equation*}
である,これにより
\begin{align*}
& \bm{f}_1=p_{11}\bm{e}_1+p_{21}\bm{e}_2, \\
& \bm{f}_2=p_{12}\bm{e}_1+p_{22}\bm{e}_2 
\end{align*}
であるから,
\begin{align*}
&
\begin{pmatrix}
0 \\
1 \\
-1 
\end{pmatrix}
=
p_{11}
\begin{pmatrix}
1 \\
-1 \\
0 
\end{pmatrix}
+p_{21}
\begin{pmatrix}
1 \\
0 \\
-1 
\end{pmatrix}
,\\ 
&
\begin{pmatrix}
1 \\
1 \\
-2 
\end{pmatrix}
=
p_{12}
\begin{pmatrix}
1 \\
-1 \\
0 
\end{pmatrix}
+p_{22}
\begin{pmatrix}
1 \\
0 \\
-1 
\end{pmatrix}
\end{align*}
となり,これにより
\[
p_{11}=-1,\quad p_{21}=1,\quad p_{12}=-1,\quad p_{22}=2
\]
であるから,基底の取り替え$E \to F$の行列は
\begin{align*}
P=
\begin{pmatrix}
-1 & -1 \\
1 & 2 
\end{pmatrix}
\end{align*}
である.
\end{leftbar}

\newpage 
\subsection*{p107-108:問1}
\addcontentsline{toc}{subsection}{\texorpdfstring{p107-108:問1}{p107-108:問1}}

\begin{leftbar}
    \begin{description}
        \item[イ] この集合を$W_1$とおくと,$W_1$は$\mathbb{C}^n$の部分空間をなす.
        \[
            \bm{x} =\bm{0} \in  W_1
        \] 
        であるから,$W_1 \ne \varnothing$である.
        
        また,

        \[
            \bm{v} = {}^t ( v_{1} , v_{2} , \ldots ,v_{n} ) , \quad \bm{w} = {}^t (w_{1} , w_{2} , \ldots , w_{n} )
        \]
        とおくと,
        \[
         \bm{v}+\bm{w}={}^t (v_1+w_1,v_2+w_2,\ldots,v_n + w_n)
        \]
        となり,これに加えて
        \[
           ( v_1+w_1)+(v_2+w_2) + \dots + (v_n + w_n) =(v_1+v_2+\dots + v_n ) + (w_1+w_2+\dots + w_n )=0+0=0
        \]
        となるから,$\bm{v}+\bm{w} \in W_1$である.
        さらに,$a \in \mathbb{C}$をとると,
        \[
            a \bm{v}={}^t ( av_{1} ,av_{2} , \ldots , av_{n} )
        \]
        であり,
        \[
           av_1 + av_2+ \dots + av_n = a (v_1 + v_2 + \dots + v_n) = a \cdot 0 =0
        \]
        であるから,このとき$ a \bm{v} \in W_1$である.
        
        以上により,$W_1$は$\mathbb{C}^n$の線型部分空間をなす.\qed 
        \item[ロ] この集合を$W_2$とおくと,$W_2$は$\mathbb{C}^n$の部分空間をなす.
        \[
            \bm{x} = \bm{0} \in W_2
        \]
        であるから,$W_2 \ne \varnothing$である.
        
        また,
        \[
            \bm{v} ={}^t (v_{p+1} , v_{p+2} , \ldots , v_{n} ) , \quad \bm{w} ={}^t ( w_{p+1} , w_{p+2} , \ldots , w_{n} )
        \]
        とおくと,
        \[
            \bm{v}+ \bm{w} = {}^t (v_{p+1}+w_{p+1},v_{p+2}+w_{p+2},\ldots,v_n + w_n)
        \]
        であり,
        \begin{align*}
           &( v_{p+1}+w_{p+1}) +(v_{p+2}+w_{p+2})+ \dots + (v_n+w_n) \\
           =& (v_{p+1}+v_{p+2}+\dots+v_n)+(w_{p+1}+w_{p+2}+\dots+w_n) \\
           =&0+0\\
           =&0
        \end{align*}
        となるため,このとき$\bm{v}+\bm{w} \in W_2$である.

        また,
        \[
            a\bm{v} = {}^t (av_{p+1},av_{p+2},\ldots,av_n)
        \]
        であり,
        \[
            av_{p+1} + av_{p+2}+\dots + av_n =a (v_{p+1}+v_{p+2}+\dots+v_n) = a\cdot 0 =0
        \]
        であるため,このとき$a \bm{v} \in W_2$である.

        以上により,$W_2$は$\mathbb{C}^n$の線型部分空間をなす.\qed
        \item [ハ] これは部分空間をなさない.
        \[
            \bm{v}= {}^t (1,0,0,\ldots , 0),\quad \bm{w}={}^t (0,1,0,\ldots ,0) 
        \]
            とすると
            \[
                \bm{v} + \bm{w}={}^t (1,1,0,\ldots,0)
            \]
        となり,与えられた条件式に当てはめると
        \[
            1^2+1^2+0^2 +\dots + 0^2 =2 \ne 1
        \]
        であるから,この集合は$\mathbb{C}^n$の部分空間でない.
        \item [ニ] この集合を$W_3$とおくと,$W_3$は$\mathbb{C}^n$の部分空間をなす.
        
        $\bm{x}=\bm{0}$とすると,
        \[
        (\bm{a},\bm{x}) =0
        \] 
        であるため,$W_3 \ne \varnothing$である.

        さて,$\bm{v}$,$\bm{w}$が条件を満たすとすると,内積の定義から
        \[
        (\bm{a},\bm{v}+\bm{w})=(\bm{a},\bm{v})+(\bm{a},\bm{w})=0
        \]
        である.また,$ c \in \mathbb{C}$とすると,
        \[
        (\bm{a},c\bm{v})=c(\bm{a},\bm{v})=0
        \]
        である.

        以上により,$W_3$は$\mathbb{C}^n$の線型部分空間をなす.\qed
    \end{description}
\end{leftbar}


\subsection*{p107-108:問2}
\addcontentsline{toc}{subsection}{\texorpdfstring{p107-108:問2}{p107-108:問2}}

\begin{leftbar}
    \begin{description}
        \item[イ] この集合を$W_1$とおくと,$W_1$は$\mathbb{K}^n$の線型部分空間とならない.
        
        たとえば
        \[
            A = \begin{pmatrix} 1 & 0 \\ 0 & 0 \end{pmatrix},\quad B = \begin{pmatrix} 0 & 0 \\ 0 & 1 \end{pmatrix}
        \]
        とおくと,$ A , B \in W_1$であるが,
        \[
            A + B = \begin{pmatrix} 1 & 0 \\ 0 & 1 \end{pmatrix}
        \]
        となり,$A+B$は正則行列である.よって$W_1$は$\mathbb{K}^n$の線型部分空間とならない.\qed 
        \item[ロ] この集合を$W_2$とおくと,$W_2$は$\mathbb{K}^n$の線型部分空間となる.
        
        $X =O$としたとき,$A O = OB$が成り立つのは明らかなので,$ W_2 \ne \varnothing$である.

        また, $X,Y \in W_2$とすると,
        \[
            A(X+Y)=(X+Y)B 
        \]
        が成立し,さらに$ a\in \mathbb{K}$とすると,
        \[
            A(aX)=(aX)B
        \]
        が成立する.

        以上により, $W_2$は$\mathbb{K}^n$の線型部分空間である.\qed 
        \item[ハ] この集合を$W_3$とおくと,これは$\mathbb{K}^n$の線型部分空間とならない.
        
        たとえば
        \[
            A = \begin{pmatrix} -1 & 1 \\ -1 & 1 \end{pmatrix} , \quad B = \begin{pmatrix} 0 & 1 \\ 0 & 0 \end{pmatrix}
        \]
        とおくと,
        \[
            A^2 = O , \quad B^2 =O
        \]
        となり,$A ,B \in W_3$であるが,

        \[
            A+B = \begin{pmatrix} -1 & 2\\-1 & 1\end{pmatrix}
        \]
        となり,これは冪零行列とならない.よって$W_3$は$\mathbb{K}^n$の線型部分空間とならない.\qed 
        \item[ニ]この集合を$W_4$とおくと,これは$\mathbb{K}^n$の線型部分空間とならない.
        
        たとえば,
        \[ 
            A= \begin{pmatrix} 1 & 0 \\ 0 & 1 \\ \end{pmatrix}
        \]
        とおき,$ 1/2 \in \mathbb{K}$をとると,
        \[
            \frac{1}{2} A = \begin{pmatrix} 1/2 & 0 \\ 0 & 1/2 \end{pmatrix}
        \]
        となり,これは$W_4$の元ではない.よって$W_4$は$\mathbb{K}^n$の線型部分空間とならない.
    \end{description}
\end{leftbar}
\newpage 

\subsection*{p122:問}
\addcontentsline{toc}{subsection}{\texorpdfstring{p122:問}{p122:問}}

\begin{tleftbar}
    まず,
    \[
        \bm{a}_1 = \begin{pmatrix} 1 \\ -1 \\ 0 \end{pmatrix},\quad \bm{a}_2 =\begin{pmatrix} 1 \\ 0 \\ -1 \end{pmatrix},\quad \bm{a}_3 = \begin{pmatrix} 1 \\ 2 \\ 3 \end{pmatrix}
    \]
    とおく.正規直交基底のひとつを$\bm{e}_1$とすると,$\norm{\bm{a}_1}=\sqrt{2}$により,
    \[
    \bm{e}_1 = \frac{1}{\norm{\bm{a}_1}} \bm{a}_1 = \frac{1}{\sqrt{2}} \begin{pmatrix} 1 \\ -1 \\ 0 \end{pmatrix}
    \]
    となる.また,
    \[
        \bm{a}_2 ' = \bm{a}_2 - (\bm{a}_2,\bm{e}_1)\bm{e}_1 
    \]
    とすると,
    \[
        \bm{a}_2 ' = \begin{pmatrix} 1 \\ 0 \\ -1 \end{pmatrix}-\left \{ \begin{pmatrix} 1 \\ 0 \\ -1 \end{pmatrix} \cdot \frac{1}{\sqrt{2}} \begin{pmatrix} 1 \\ -1 \\ 0 \end{pmatrix} \right \} \frac{1}{\sqrt{2}} \begin{pmatrix} 1 \\ -1 \\ 0 \end{pmatrix} =\frac{1}{2} \begin{pmatrix} 1 \\ 1\\ -2 \end{pmatrix}
    \]
    である.これを用いると,
    \[
        \bm{e}_2 = \frac{1}{\norm{\bm{a}_2'}} \bm{a}_2 ' =\frac{1}{\sqrt{6}/2} \cdot \frac{1}{2} \begin{pmatrix} 1 \\ 1\\ -2 \end{pmatrix} = \frac{1}{\sqrt{6}} \begin{pmatrix} 1 \\ 1\\ -2 \end{pmatrix} 
    \]
    となる.また,
    \[
        \bm{a}_3 ' = \bm{a}_3 - (\bm{a}_3,\bm{e}_1)\bm{e}_1 -(\bm{a}_3,\bm{e}_2)\bm{e}_2
    \]
    とすると,
    \[
        \bm{a}_3'  = \begin{pmatrix} 1 \\ 2 \\ 3 \end{pmatrix} - \left \{ \begin{pmatrix} 1 \\ 2\\ 3 \end{pmatrix} \cdot \frac{1}{\sqrt{2}} \begin{pmatrix} 1 \\ -1 \\ 0 \end{pmatrix} \right \} \frac{1}{\sqrt{2}} \begin{pmatrix} 1 \\ -1 \\ 0 \end{pmatrix}- \left \{ \begin{pmatrix} 1 \\ 2 \\ 3 \end{pmatrix} \cdot  \frac{1}{\sqrt{6}} \begin{pmatrix} 1 \\ 1\\ -2 \end{pmatrix} \right \}  \frac{1}{\sqrt{6}} \begin{pmatrix} 1 \\ 1\\ -2 \end{pmatrix} =\begin{pmatrix} 2 \\ 2 \\ 2 \end{pmatrix}
    \]
    となり,
    \[
        \bm{e}_3 = \frac{1}{\norm{\bm{a}_3 ' }} \bm{a}_3 ' = \frac{1}{\sqrt{3}} \begin{pmatrix} 1 \\ 1 \\ 1 \end{pmatrix}
    \]
    となる.

    以上の考察により,求める正規直交基底は
    \[
        \langle \frac{1}{\sqrt{2}} \begin{pmatrix} 1 \\ -1 \\ 0 \end{pmatrix} , \frac{1}{\sqrt{6}} \begin{pmatrix} 1 \\ 1\\ -2 \end{pmatrix} ,  \frac{1}{\sqrt{3}} \begin{pmatrix} 1 \\ 1 \\ 1 \end{pmatrix} \rangle 
    \]
    である.
\end{tleftbar}
\newpage 
\section*{第4章・章末問題}
\addcontentsline{toc}{section}{\texorpdfstring{第4章・章末問題}{第4章・章末問題}}


\subsection*{p127-130:1}
\addcontentsline{toc}{subsection}{\texorpdfstring{p127-130:1}{p127-130:1}}

\begin{tleftbar}
$s,t,u,v \in \mathbb{R}$とし.
\[
s\bm{a}_1+t\bm{a}_2=u\bm{a}_3+v\bm{a}_4
\]
とおく.これにより,
\begin{align*}
    &
\begin{pmatrix}
1 & -1 & 0 & 1 \\
2 & 1 & -1 & 9 \\
0 & 3 & 5 & 1 \\
4 & -3 & 2 & 4 
\end{pmatrix}
\begin{pmatrix}
s \\
t \\
u \\
v 
\end{pmatrix}
=\bm{o} ,\\
& 
\begin{pmatrix}
1 & 0 & 0 & 3 \\
0 & 1 & 0 & 2 \\
0 & 0 & 1 & -1 \\
0 & 0 & 0 & 0 
\end{pmatrix}
\begin{pmatrix}
s \\
t \\
u \\
v 
\end{pmatrix}
=\bm{o} ,\\
&
\begin{pmatrix}
s \\
t \\
u \\
v 
\end{pmatrix}
=a
\begin{pmatrix}
-3 \\
-2 \\
1 \\
1 
\end{pmatrix}
. \quad \text{($a$は任意の定数)}
\end{align*}
とかけるので,$W_1 \cap W_2$の次元は$1$であり,その基底は
\[
    s \bm{a}_1+ t \bm{a}_2 =-a \begin{pmatrix} 1 \\ 8 \\ 6 \\ 6 \end{pmatrix}
\]
により,
\[
    \langle 
        \begin{pmatrix}
        1 \\
        8 \\
        6 \\
        6
        \end{pmatrix}
    \rangle 
\]
である.
\end{tleftbar}
\newpage 


\subsection*{p127-130:2}
\addcontentsline{toc}{subsection}{\texorpdfstring{p127-130:2}{p127-130:2}}

\begin{tleftbar}
    $ W_1$に関して,$x_3 =s$,$x_4 =t$とおくと,
    \[
        \begin{pmatrix} x_1 \\ x_2 \\ x_3 \\ x_4 \end{pmatrix} = s \begin{pmatrix} 1 \\ -1 \\ 1 \\ 0 \end{pmatrix} + t \begin{pmatrix} -9 \\ 3 \\ 0 \\ 1 \end{pmatrix}
    \]
    とかけるため,$\dim W_1 = 2$であり,その基底は
    \[
       \langle  \begin{pmatrix} 1 \\ -1 \\ 1 \\ 0 \end{pmatrix} , \begin{pmatrix} -9 \\ 3 \\ 0 \\ 1 \end{pmatrix} \rangle 
    \]
    である.$W_2$に関しても同様にして,$\dim W_2 = 2$であり,その基底は
    \[
        \langle \begin{pmatrix} 0 \\ -1 \\ 1 \\ 0 \end{pmatrix} , \begin{pmatrix} -1 \\ 3 \\ 0 \\ 1 \end{pmatrix} \rangle
    \]
    である.したがって$W_1 + W_2$は
    \[
     \begin{pmatrix} 1 \\ -1 \\ 1 \\ 0 \end{pmatrix} ,\quad  \begin{pmatrix} -9 \\ 3 \\ 0 \\ 1 \end{pmatrix},\quad \begin{pmatrix} 0 \\ -1 \\ 1 \\ 0 \end{pmatrix} ,\quad  \begin{pmatrix} -1 \\ 3 \\ 0 \\ 1 \end{pmatrix}
    \]
    によって生成される.

    ここで,
    \[
        x \begin{pmatrix} 1 \\ -1 \\ 1 \\ 0 \end{pmatrix} + y \begin{pmatrix} -9 \\ 3 \\ 0 \\ 1 \end{pmatrix} + z \begin{pmatrix} 0 \\ -1 \\ 1 \\ 0 \end{pmatrix} + w \begin{pmatrix} -1 \\ 3 \\ 0 \\ 1 \end{pmatrix} =\bm{o}
    \]
    とすると,
    \[
        \begin{pmatrix} 1 & -9 & 0 & -1 \\ -1 & 3 & -1 & 3 \\ 1 & 0 & 1 & 0 \\ 0 & 1 & 0 & 1 \end{pmatrix} \begin{pmatrix} x \\ y \\ z \\ w \end{pmatrix} = \bm{o}
    \]
    となり,これに基本変形を施すと,
    \[
        \begin{pmatrix} 1 & 0 & 0 & 8 \\ 0 & 1 & 0 & -8 \\0 & 0 & 1 & 1 \\ 0 & 0 & 0 & 0 \end{pmatrix} \begin{pmatrix} x \\ z \\ y \\ w \end{pmatrix} = \bm{o}
    \]
    となる.
    したがって,$W_1 + W_2$の次元は$3$であり,その基底は
    \[
        \langle \begin{pmatrix} 1 \\ -1 \\ 1 \\ 0 \end{pmatrix} , \begin{pmatrix} -9 \\ 3 \\ 0 \\ 1 \end{pmatrix} , \begin{pmatrix} 0 \\ -1 \\ 1 \\ 0 \end{pmatrix} \rangle
    \]
    である.
\end{tleftbar} 


\section*{附録\three}
\addcontentsline{toc}{section}{\texorpdfstring{附録\three}{附録\three}}


\subsection*{p249:問}
\addcontentsline{toc}{subsection}{\texorpdfstring{p249:問}{p249:問}}

\begin{tleftbar}
\begin{description}
    \item[イ] 
	\begin{proof}
		体$K$の単位元について,$0=0+0$であるから,
		\begin{align*}
			&a 0=a(0+0)=a0 + a0\\
			&\therefore ~ a0 = a0 + a0
		\end{align*}
		$K$は加法について可換群であるから,$a0$の逆元$-a0$が$K$に存在する.これを用いると,
\begin{align*}
	&a0 + (-a0) = a0 + a0 + (-a0) \\
	&\therefore ~ 0 = a0 + a0 +(-a0)
\end{align*}
 ここで,
 \begin{align*}
	a0 + a0 +(-a0)&=a0+ \{a0+(-a0)\} \\
	& = a0 + 0 \\
	& = a0
 \end{align*}
となるから,$0=a0$である.$0=0a$についても同様.
\end{proof}
\item[ロ]
\begin{proof}
        $a \ne 0$とする.このとき,$a$の逆元$a^{-1} \in K$が存在し,$ab=0$の両辺に$a^{-1}$をかけると,
        \begin{align*}
            &a^{-1} (ab) = a^{-1} 0 \\
            &(a^{-1}a)b =0 \\
            &1b =0 \\
            &\therefore~ b=0
        \end{align*}
        である.これと$b \ne 0$を仮定したときの同様の考察により,$ab=0$のとき,$a=0$または$b=0$である.
    \end{proof}
    \end{description}
\end{tleftbar}

\begin{thebibliography}{9}
	\bibitem{saito} 齋藤正彦『線型代数入門』,東京大学出版会,1966
\end{thebibliography}

\end{document}