\part*{第2章}
\addcontentsline{toc}{part}{\texorpdfstring{第2章}{第2章}}


\section*{p34:問1}
\addcontentsline{toc}{section}{\texorpdfstring{p34:問1}{p34:問1}}

\begin{tproof}
  後半二つの主張は明らか.また,二つ目の主張は一つ目の主張と同様にして示すことができるので,一つ目のみ示すことにする.


  $A=(a_{pq})$を$k \times l$行列,$B= (b_{qr})$,$C=(c_{qr})$を$l \times m$行列とする.示したい式の両辺がともに定義され,ともに$k \times m$行列であることはよい.行列$B+C$の$(q,r)$成分は$b_{qr}+c_{qr}$であるから,左辺の$(p,r)$成分は,
  \[
    \sum_{q=1}^{l}a_{pq}\left(b_{qr}+c_{qr}\right)=\sum_{q=1}^{l}a_{pq}b_{qr}+\sum_{q=1}^{l}a_{pq}c_{qr}
  \]
  とかける.この等号の右辺は$AB$の$(p,r)$成分と$AC$の$(p,r)$成分の和である.これより,主張が示された.
\end{tproof}



\section*{p40:問}
\addcontentsline{toc}{section}{\texorpdfstring{p40:問}{p40:問}}

\subsection*{p40:問-(イ)}
\addcontentsline{toc}{subsection}{\texorpdfstring{p40:問-(イ)}{p40:問-(イ)}}

\begin{tanswer}
  \[
    A_{11} = \begin{pmatrix} 1 & -1 \\ 0 & -2 \end{pmatrix},\quad A_{22} = \begin{pmatrix} -2 & 3 \\ 1 & 1 \end{pmatrix} ,\quad B_{11} = \begin{pmatrix} 2 & 1 \\ 0 & 1\end{pmatrix} ,\quad B_{22}= \begin{pmatrix} 1 & 1 \\ 2 & -3 \end{pmatrix}
  \]
  とおくと,
  \begin{align*}
    (\text{与式}) & = \begin{pmatrix} A_{11} & O \\ O & A_{22} \end{pmatrix} \begin{pmatrix} B_{11} & O \\ O & B_{22} \end{pmatrix} \\
                & = \begin{pmatrix} A_{11} B_{11} & O \\ O & A_{22} B_{22} \end{pmatrix}                                          \\
                & = \begin{pmatrix} 2 & 0 & 0 & 0 \\ 0 & -2 & 0 & 0 \\ 0 & 0 & 4 & -11 \\ 0 & 0 & 3 & -2 \end{pmatrix}
  \end{align*}
  である.
\end{tanswer}

\section*{p41:問1}
\addcontentsline{toc}{section}{\texorpdfstring{p41:問1}{p41:問1}}
%

\begin{tanswer}
  \begin{enumerate}
    \item \mbox{ }
          $ X =\begin{pmatrix} x_{11} & x_{12} \\ x_{21} & x_{22} \end{pmatrix}$とする.このとき,
          \[
            AX =\begin{pmatrix} x_{11}+2x_{21} &  x_{12}+2x_{22}  \\ 2x_{11} + 4x_{21} &2x_{12} + 4x_{22} \end{pmatrix}
          \]
          となり,これが$E_2$と等しくなるためには
          \[
            \begin{cases}
              x_{11}+2x_{21} =1   \\
              x_{12}+2x_{22}  =0  \\
              2x_{11} + 4x_{21}=0 \\
              2x_{12} + 4x_{22}=1
            \end{cases}
          \]
          となることが必要かつ十分であるが,これを満たす$x_{11},x_{12},x_{21},x_{22} \in \mathbb{C}$は存在しない.よって前半の主張が示された.\par
          後半について示す.$Y =\begin{pmatrix} y_{11} & y_{12} \\ y_{21} & y_{22} \end{pmatrix}$とする.このとき,
          \[
            YA = \begin{pmatrix} y_{11} +2y_{12} & 2y_{11}+4y_{12} \\ y_{21}+2y_{22} & 2y_{21}+4y_{22}  \end{pmatrix}
          \]
          となり,これが$E_2$と等しくなるためには
          \[
            \begin{cases}
              y_{11} +2y_{12}=1   \\
              2y_{11}+4y_{12}  =0 \\
              y_{21}+2y_{22} =0   \\
              2y_{21}+4y_{22} =1
            \end{cases}
          \]
          となることが必要かつ十分であるが,これを満たす$y_{11},y_{12},y_{21},y_{22} \in \mathbb{C}$は存在しない.よって後半の主張も示された.\qed
    \item \mbox{ }
          $X,Y$を(1)で定義したものとする.このとき,
          \[
            AX = \begin{pmatrix} x_{11}+2x_{21} & x_{12}+2x_{22} \\ 0 & 0 \end{pmatrix}
          \]
          となり,これが$B$と等しくならないことは明らか.\par
          後半について,
          \[
            YA =\begin{pmatrix} x_{11} & 2 x_{11} \\  x_{21}&  2x_{21} \end{pmatrix}
          \]
          となり,これが$B$と等しくなるためには$x_{11}=1$,$x_{21}=2$となることが必要かつ十分であるが,$x_{12}$,$x_{22}$については任意の複素数である.以上の議論により,このような$Y$は無限に存在する.\qed
    \item  \mbox{ }
          $A$の第$k$列の成分が全て$0$であるとする.ただしここで$ 1 \le k \le n,~ k \in \mathbb{N}$であるとする.
          $XA=E$をみたす$X$が存在すると仮定する.このとき,$X$は明らかに$n \times n$行列であり,積$XA$は定義される.
          いま$X=(x_{jk})$,$A=(a_{kj})$,$ 1 \le j ,k \le n$と表す.このとき,
          \[
            (XA \text{の}(j,j)\text{成分}) = \sum_{k=1}^{n} x_{jk} a_{kj} =0
          \]
          となり,これは$XA =E$に矛盾する.よってこのような$X$は存在しないことが示された.\qed
  \end{enumerate}
\end{tanswer}


\section*{p42:問1}
\addcontentsline{toc}{section}{\texorpdfstring{p42:問1}{p42:問1}}

\begin{tproof}
  まず,
  \[
    \overline{A} \ \overline{A^{-1}} = \overline{A A^{-1}}=E,\quad \overline{A^{-1}} \ \overline{A} =\overline{A^{-1} A}=E
  \]
  より,$\overline{A}$は正則で,逆行列は$\overline{A^{-1}}$である.さらに,
  \[
    {}^t A {}^t A^{-1} ={}^t (A^{-1} A)=E,\quad {}^t A^{-1} {}^t A = {}^t (A A^{-1})=E
  \]
  であるから,${}^t A$は正則であり,逆行列は${}^t A^{-1}$である.
\end{tproof}

\section*{p42:問2}
\addcontentsline{toc}{section}{\texorpdfstring{p42:問2}{p42:問2}}

\begin{tanswer}
  \[
    A \coloneqq \begin{pmatrix} a & b \\ c & d \end{pmatrix},\quad A' \coloneqq \begin{pmatrix} x & y \\ z & w \end{pmatrix}
  \]
  とする.このとき,
  \[
    A A' = \begin{pmatrix} a x + b z & ay + bw \\ cx + dz & cy +dw \end{pmatrix}
  \]
  である.$AA'=E$となる条件は,$x$,$y$,$z$,$w$についてのふたつの連立方程式
  \[
    \begin{cases}
      ax+bz =1 \\
      cx+dz =0
    \end{cases}
    ,\quad
    \begin{cases}
      ay+bw=0 \\
      cy+dw=1
    \end{cases}
  \]
  が解を持つことで,その条件は$ad-bc \ne 0$である.そのときの解は,
  \[
    (x,y,z,w)=  (\frac{d}{ad-bc},-\frac{b}{ad-bc},-\frac{c}{ad-bc},\frac{a}{ad-bc})
  \]
  である.これを用いて$A'A$を計算すると,$A' A =E$となり.たしかに$A'$は$A$の逆行列である.

  以上の議論により,$ad - bc \ne 0$となることが必要十分条件である.
\end{tanswer}

\section*{p42:問3}
\addcontentsline{toc}{section}{\texorpdfstring{p42:問3}{p42:問3}}

\subsection*{p42:問3-(イ)}
\addcontentsline{toc}{subsection}{\texorpdfstring{p42:問3-(イ)}{p42:問3-(イ)}}

\begin{tanswer}
  (2)の結果により,
  \[
    \frac{1}{3 \cdot 3 - 2 \cdot 4} \begin{pmatrix} 3 & -2 \\ -4 & 3 \end{pmatrix} =  \begin{pmatrix} 3 & -2 \\ -4 & 3 \end{pmatrix}
  \]
  が求める逆行列である.
\end{tanswer}

\subsection*{p42:問3-(ロ)}
\addcontentsline{toc}{subsection}{\texorpdfstring{p42:問3-(ロ)}{p42:問3-(ロ)}}

\begin{tanswer}
  まず,
  \[
    X= \begin{pmatrix} x_{11} & x_{12} & x_{13} \\x_{21} & x_{22} & x_{23} \\x_{31} & x_{32} & x_{33}  \end{pmatrix}
  \]
  としたときに
  \begin{align*}
    XA & =  \begin{pmatrix} x_{11} & x_{12} & x_{13} \\x_{21} & x_{22} & x_{23} \\x_{31} & x_{32} & x_{33}  \end{pmatrix}
    \begin{pmatrix} 1 & 2 & -1 \\ 0& 1 & 3 \\ 0 & 0 & 1 \end{pmatrix}                                                                                                                                          \\
       & = \begin{pmatrix} x_{11} & 2x_{11} + x_{12} & -x_{11} +3x_{12} +x_{13} \\ x_{21} & 2x_{21} + x_{22} & -x_{21} +3x_{22} +x_{23} \\  x_{31} & 2x_{31} + x_{32} & -x_{31} +3x_{32} +x_{33} \end{pmatrix} \\
  \end{align*}
  であるから,これに関して
  \[
    \begin{pmatrix} x_{11} & 2x_{11} + x_{12} & -x_{11} +3x_{12} +x_{13} \\ x_{21} & 2x_{21} + x_{22} & -x_{21} +3x_{22} +x_{23} \\  x_{31} & 2x_{31} + x_{32} & -x_{31} +3x_{32} +x_{33} \end{pmatrix} = \begin{pmatrix} 1 & 0 & 0 \\ 0 & 1 & 0 \\ 0 & 0 & 1 \end{pmatrix}
  \]
  となれば,行列$X$が求める逆行列である.
  計算すると
  \[
    X = \begin{pmatrix} 1 & -2 & 7 \\ 0 & 1 & -3 \\ 0 & 0 & 1 \end{pmatrix}
  \]
  であり,これが求める逆行列であった.
\end{tanswer}

\subsection*{p42:問3-(ハ)}
\addcontentsline{toc}{subsection}{\texorpdfstring{p42:問3-(ハ)}{p42:問3-(ハ)}}

\begin{tanswer}
  まず,
  \[
    X= \begin{pmatrix} x_{11} & x_{12} & x_{13} & x_{14}\\x_{21} & x_{22} & x_{23} & x_{24} \\x_{31} & x_{32} & x_{33} & x_{34} \\x_{41} & x_{42} & x_{43} & x_{44} \end{pmatrix}
  \]
  としたとき,
  \begin{align*}
    XA & = \begin{pmatrix} x_{11} & x_{12} & x_{13} & x_{14}\\x_{21} & x_{22} & x_{23} & x_{24} \\x_{31} & x_{32} & x_{33} & x_{34} \\x_{41} & x_{42} & x_{43} & x_{44} \end{pmatrix}
    \begin{pmatrix} 0 & 0 & 0 & 1 \\ 0 & 0 & 1 & 0 \\ 0 & 1 & 0 & 0 \\ 1 & 0 & 0 & 0 \end{pmatrix}                                                                                     \\
       & = \begin{pmatrix} x_{14} & x_{13} & x_{12} & x_{11} \\x_{24} & x_{23} & x_{22} & x_{21} \\x_{34} & x_{33} & x_{32} & x_{31} \\x_{44} & x_{43} & x_{42} & x_{41} \end{pmatrix}
  \end{align*}
  であるから,これに関して
  \[
    \begin{pmatrix} x_{14} & x_{13} & x_{12} & x_{11} \\x_{24} & x_{23} & x_{22} & x_{21} \\x_{34} & x_{33} & x_{32} & x_{31} \\x_{44} & x_{43} & x_{42} & x_{41} \end{pmatrix} = \begin{pmatrix} 1 & 0 & 0 & 0 \\ 0 & 1 & 0 & 0 \\ 0 & 0 & 1 & 0 \\ 0 & 0 & 0 & 1 \end{pmatrix}
  \]
  となれば,行列$X$が求める逆行列である.

  計算すると,
  \[
    X = \begin{pmatrix} 0 & 0 & 0 & 1 \\ 0 & 0 & 1 & 0 \\ 0 & 1 & 0 & 0 \\ 1 & 0 & 0 & 0 \end{pmatrix}
  \]
  であり.これが求める逆行列であった.
\end{tanswer}



\section*{p52:問}
\addcontentsline{toc}{section}{\texorpdfstring{p52:問}{p52:問}}

\begin{tanswer}
  \begin{align*}
                                                                        &
    \left(
    \begin{array}{ccc|ccc}
        1  & 3  & 2  & 1 & 0 & 0 \\
        2  & 6  & 3  & 0 & 1 & 0 \\
        -2 & -5 & -2 & 0 & 0 & 1
      \end{array}
    \right)                                                               \\
    \xrightarrow{\text{第$1$行の$(-2)$倍,第$1$行の$2$倍をそれぞれ第$2$行,第$3$行に加える}}   &
    \left( \begin{array}{ccc|ccc}
               1 & 3 & 2  & 1  & 0 & 0 \\
               0 & 0 & -1 & -2 & 1 & 0 \\
               0 & 1 & 2  & 2  & 0 & 1
             \end{array}
    \right)                                                               \\
    \xrightarrow{\text{第$2$行と第$3$行を交換する}}                               &
    \left( \begin{array}{ccc|ccc}
               1 & 3 & 2  & 1  & 0 & 0 \\
               0 & 1 & 2  & 2  & 0 & 1 \\
               0 & 0 & -1 & -2 & 1 & 0
             \end{array}
    \right)                                                               \\
    \xrightarrow{\text{第$2$行の$(-3)$倍を第$1$行に加え,第$3$行の$(-4)$倍を第$1$行に加える}} &
    \left( \begin{array}{ccc|ccc}
               1 & 0 & 0  & 3  & -4 & -3 \\
               0 & 1 & 2  & 2  & 0  & 1  \\
               0 & 0 & -1 & -2 & 1  & 0
             \end{array}
    \right)                                                               \\
    \xrightarrow{\text{第$3$行の$2$倍を第$2$行に加え,第$3$行を$(-1)$倍する}}            &
    \left( \begin{array}{ccc|ccc}
               1 & 0 & 0 & 3  & -4 & -3 \\
               0 & 1 & 0 & -2 & 2  & 1  \\
               0 & 0 & 1 & 2  & -1 & 0
             \end{array}
    \right)
  \end{align*}
  よって,求める逆行列は
  \[
    \begin{pmatrix}
      3  & -4 & -3 \\
      -2 & 2  & 1  \\
      2  & -1 & 0
    \end{pmatrix}
  \]
  である.
\end{tanswer}




\section*{p58:問}
\addcontentsline{toc}{section}{\texorpdfstring{p58:問}{p58:問}}

\subsection*{p58:問-(イ)}
\addcontentsline{toc}{subsection}{\texorpdfstring{p58:問-(イ)}{p58:問-(イ)}}

\begin{tanswer}
  与えられた連立方程式について,拡大係数行列を考えて基本変形を施すと,
  \[
    \begin{pmatrix} 1 & 0 & 1 & -4/3 \\ 0 & 1 & 1 & 8/3 \\ 0 & 0 & 0 & 0 \end{pmatrix}
  \]
  となる.つまり,解は存在し,ひとつの任意定数を含む.任意定数を$ x_3 =\alpha$とすると,
  \[
    x_1 = -\frac{4}{3} - \alpha,\quad x_2 = \frac{8}{3} - \alpha,\quad x_3 = \alpha
  \]
  とかける.ベクトルの形で表すと,
  \[
    \begin{pmatrix} x_1 \\ x_2 \\ x_3 \end{pmatrix} = \begin{pmatrix} -4/3 \\ 8/3 \\ 0 \end{pmatrix} + \alpha \begin{pmatrix} -1 \\ -1 \\ 1 \end{pmatrix}
  \]
  である.
\end{tanswer}


\subsection*{p58:問-(ロ)}
\addcontentsline{toc}{subsection}{\texorpdfstring{p58:問-(ロ)}{p58:問-(ロ)}}

\begin{tanswer}
  与えられた連立方程式について,拡大係数行列を考えて基本変形を施すと,
  \[
    \begin{pmatrix} 1 & 0 & 0 & 0 \\ 0 & 1& 0 & 1 \\ 0 & 0 & 1 & 0 \\ 0 & 0 & 0 & -1 \end{pmatrix}
  \]
  となるが,$0 = -1$とはならないため,この連立方程式は解を持たない.
\end{tanswer}

\subsection*{p58:問-(ハ)}
\addcontentsline{toc}{subsection}{\texorpdfstring{p58:問-(ハ)}{p58:問-(ハ)}}

\begin{tanswer}
  与えられた連立方程式について,拡大係数行列を考えて基本変形を施すと,
  \[
    \begin{pmatrix} 1 & 0 & 0 & 2 & 6 & 6 \\ 0 & 1& 0 & -2 & -11 & -9 \\ 0 & 0 & 1 & 0 & 19 & 14 \end{pmatrix}
  \]
  となる.ただしここで第3列と第4列を入れ替えた.

  つまり,解は存在し,ふたつの任意定数を含む,任意定数を$x_3 = \alpha$,$x_5 = \beta$とすると,この連立方程式の解は
  \[
    x_1 = 6-2\alpha -2 \beta , \quad x_2 = -9 + 2\alpha +11 \beta , \quad x_3 = \alpha , \quad x_4 = 14-19\beta,\quad x_5 =\beta
  \]
  とかける.ベクトルの形で表すと,
  \[
    \begin{pmatrix} x_1 \\ x_2 \\ x_3 \\ x_4 \\ x_5 \end{pmatrix} = \begin{pmatrix} 6 \\ -9 \\ 0 \\ 14 \\ 0 \end{pmatrix} + \alpha \begin{pmatrix} -2 \\ 2 \\ 1 \\ 0 \\ 0 \end{pmatrix} + \beta \begin{pmatrix} -2 \\ 11 \\ 0 \\ -19 \\ 1 \end{pmatrix}
  \]
  である.
\end{tanswer}



\section*{p62--63:問1}
\addcontentsline{toc}{section}{\texorpdfstring{p62--63:問1}{p62--63:問1}}

\begin{tproof}
  定義に従って計算すると,
  \begin{align*}
    \norm{\bm{x}+\bm{y} }^2 +\norm{\bm{x}-\bm{y} }^2 & = (\bm{x}+\bm{y},\bm{x}+\bm{y})+(\bm{x}-\bm{y},\bm{x}-\bm{y})                                                                    \\
                                                     & =(\bm{x},\bm{x})+(\bm{x},\bm{y})+(\bm{y},\bm{x})+(\bm{y},\bm{y})+(\bm{x},\bm{x})-(\bm{x},\bm{y})-(\bm{y},\bm{x})+(\bm{y},\bm{y}) \\
                                                     & = 2((\bm{x},\bm{x})+(\bm{y},\bm{y}))                                                                                             \\
                                                     & =2 (\norm{\bm{x}}^2+\norm{\bm{y}}^2)
  \end{align*}
  となり,これが証明すべきことであった.
\end{tproof}
\section*{p62-63:問2}
\addcontentsline{toc}{section}{\texorpdfstring{p62:問2}{p62:問2}}
\begin{tproof}
  \[
    \norm{\bm{x}+\bm{y}}^2=(\bm{x},\bm{x})+(\bm{x},\bm{y})+(\bm{y},\bm{x})+(\bm{y},\bm{y})
  \]
  である.ここで,$\bm{x}$と$\bm{y}$が直交することから,
  \[
    (\bm{x},\bm{y})+(\bm{y},\bm{x})=(\bm{x},\bm{y})+\overline{(\bm{x},\bm{y})}=0
  \]
  であり,これを用いると
  \[
    \norm{\bm{x}+\bm{y}}^2 =(\bm{x},\bm{x})+(\bm{y},\bm{y})=\norm{\bm{x}}^2 +\norm{\bm{y}}^2
  \]
  となる.$\bm{x},~\bm{y}$がともに実ベクトルのときは$(\bm{x},\bm{y})=0$であるから確かに逆が成り立つが,たとえば$\bm{x}=
    \begin{pmatrix}
      2 \\
      0
    \end{pmatrix}
    ,~
    \bm{y}=
    \begin{pmatrix}
      2i \\
      0
    \end{pmatrix}
  $とすれば,等式は成り立つが$\bm{x}$と$\bm{y}$は直交しないため,逆は成り立たない.
\end{tproof}

\section*{p62--63:問3}
\addcontentsline{toc}{section}{\texorpdfstring{p62--63:問3}{p62--63:問3}}

\begin{tproof}
  $\bm{x},\bm{y} \in \mathbb{R}^n$のとき,
  \begin{align*}
    \norm{\bm{x}+\bm{y}}^2 - \norm{\bm{x}-\bm{y}}^2 & = (\bm{x}+\bm{y},\bm{x}+\bm{y}) - (\bm{x}-\bm{y},\bm{x}-\bm{y})                                                                          \\
                                                    & = \norm{\bm{x}}^2 +(\bm{x},\bm{y})+(\bm{y},\bm{x})+ \norm{\bm{y}}^2 - (\norm{\bm{x}}^2 -(\bm{x},\bm{y})-(\bm{y},\bm{x})+\norm{\bm{y}}^2) \\
                                                    & = \norm{\bm{x}}^2 + 2(\bm{x},\bm{y}) + \norm{\bm{y}}^2 - (\norm{\bm{x}}^2 - 2(\bm{x},\bm{y}) + \norm{\bm{y}}^2)                          \\
                                                    & = 4(\bm{x},\bm{y})
  \end{align*}
  であるから,この両辺を$4$で割るとただちに主張が従う.

  また,$\bm{x},\bm{y} \in \mathbb{C}$のときにはこの等式が成り立たないことがある.
  $\bm{x}={}^t (2i,0)$,$\bm{y}={}^t (2,0)$とすると,
  \begin{align*}
    \frac{\norm{\bm{x}+\bm{y}}^2 - \norm{\bm{x}-\bm{y}}^2 }{4} & = \frac{4-4}{4} \\
                                                               & =0
  \end{align*}
  であるが,
  \begin{align*}
    (\bm{x},\bm{y}) & =(2,0) \begin{pmatrix} -2i  \\ 0 \end{pmatrix} \\
                    & = -4i
  \end{align*}
  となり,確かにこれが反例になっている.
\end{tproof}



\section*{p65:問1}
\addcontentsline{toc}{section}{\texorpdfstring{p65:問1}{p65:問1}}

\begin{tanswer}
  \[
    A \coloneqq \begin{pmatrix} a & b \\ c & d\end{pmatrix} \in \mathbb{R}^{2 \times 2}
  \]
  とおく.このとき,
  \begin{align*}
    A {}^t A & = \begin{pmatrix} a & b \\ c & d \end{pmatrix} \begin{pmatrix} a & c \\ b & d\end{pmatrix} \\
             & = \begin{pmatrix} a^2 +b^2 & ac+bd \\ ac+bd & c^2 + d^2 \end{pmatrix}
  \end{align*}
  となり,これが$E$に等しいので,
  \[
    \begin{cases}
      a^2 + b^2 =1 , \\
      c^2 + d^2 =1 , \\
      ac+bd =0
    \end{cases}
  \]
  となる.このことから$ 0 \leqq \theta < 2\pi $,$ 0 \leqq \phi < 2\pi$として
  \begin{align*}
     & a = \cos \theta , \quad b = \sin \theta, \\
     & c = \cos \phi , \quad d = \sin \phi
  \end{align*}
  とおくと,
  \begin{align*}
    ac+bd & = \cos \theta \cos \phi + \sin \theta \sin \phi \\
          & = \cos (\theta - \phi)
  \end{align*}
  となり,これが$0$に等しいので.
  \begin{align*}
               & \theta -\phi = \pi /2 , 3\pi /2 ,         \\
    \therefore & \phi = \theta - \pi /2 , \theta - 3\pi/2.
  \end{align*}
  これより,任意の二次直交行列は$ 0 \leqq \theta < 2\pi $,$ 0 \leqq \phi < 2\pi$として
  \[
    \begin{pmatrix} \cos \theta & -\sin \theta \\ \sin \theta & \cos \theta \end{pmatrix},\quad \begin{pmatrix} \cos \theta & \sin \theta \\ \sin \theta & -\cos \theta \end{pmatrix}
  \]
  のいずれかで表される.
\end{tanswer}

\section*{p65:問2}
\addcontentsline{toc}{section}{\texorpdfstring{p65:問2}{p65:問2}}

\begin{tproof}
  \[
    (A^\ast A)^{\ast} = A^{\ast} A^{\ast \ast} = A^{\ast} A, \quad (A A^{\ast})^\ast = A^{\ast \ast} A^\ast =A A^{\ast}
  \]
  であるから,$A^\ast A$,$A A^{\ast}$はエルミート行列である.

  また,任意の$n \times 1$ベクトル$\bm{x}$に対して,
  \begin{align*}
    (\bm{x}, A^{\ast}A \bm{x}) & = (A^{\ast \ast} \bm{x}, A \bm{x}) \\
                               & = (A \bm{x}, A \bm{x})             \\
                               & = \norm{A \bm{x}}^2 \geqq 0
  \end{align*}
  であり,また,$\bm{x}$として第$i$成分のみ1でほかの成分は$0$のベクトル$\bm{e}_i$をとると,
  \[
    (\bm{e}_i, A^{\ast}A \bm{e}_i) = \text{$A^{\ast}A$の$(i,i)$成分}
  \]
  となる.先の不等式よりこれは$0$または正なので,$A^{\ast}A$の対角成分は$0$または正である.同様にして$AA^{\ast}$の対角成分も$0$または正である.

  以上のことが証明すべきことであった.
\end{tproof}


\part*{第2章・章末問題}
\addcontentsline{toc}{part}{\texorpdfstring{第2章・章末問題}{第2章・章末問題}}

\section*{p70--73:1}
\addcontentsline{toc}{section}{\texorpdfstring{p70--73:1}{p70--73:1}}

\subsection*{p70--73:1-(イ)}
\addcontentsline{toc}{subsection}{\texorpdfstring{p70--73:1-(イ)}{p70--73:1-(イ)}}

\begin{tanswer}
  \begin{align*}
                                                                            &
    \left(
    \begin{array}{cccc|cccc}
        3  & 3 & -5 & -6 & 1 & 0 & 0 & 0 \\
        1  & 2 & -3 & -1 & 0 & 1 & 0 & 0 \\
        2  & 3 & -5 & -3 & 0 & 0 & 1 & 0 \\
        -1 & 0 & 2  & 2  & 0 & 0 & 0 & 1
      \end{array}
    \right)                                                                   \\
    \xrightarrow{\text{第$1$行と第$2$行を交換する}}                                   &
    \left( \begin{array}{cccc|cccc}
               1  & 2 & -3 & -1 & 0 & 1 & 0 & 0 \\
               3  & 3 & -5 & -6 & 1 & 0 & 0 & 0 \\
               2  & 3 & -5 & -3 & 0 & 0 & 1 & 0 \\
               -1 & 0 & 2  & 2  & 0 & 0 & 0 & 1
             \end{array}
    \right)                                                                   \\
    \xrightarrow{\text{$(1,1)$をかなめとして左から第1列を掃き出す}}                          &
    \left( \begin{array}{cccc|cccc}
               1 & 2  & -3 & -1 & 0 & 1  & 0 & 0 \\
               0 & -3 & 4  & -3 & 1 & -3 & 0 & 0 \\
               0 & -1 & 1  & -1 & 0 & -2 & 1 & 0 \\
               0 & 2  & -1 & 1  & 0 & 1  & 0 & 1
             \end{array}
    \right)                                                                   \\
    \xrightarrow{\text{第$3$行を$(-1)$倍して第2行と交換し,$(2,2)$をかなめとして左から第$2$列を掃き出す}} &
    \left( \begin{array}{cccc|cccc}
               1 & 0 & -1 & -3 & 0 & -3 & 2  & 0 \\
               0 & 1 & -1 & 1  & 0 & 2  & -1 & 0 \\
               0 & 0 & 1  & 0  & 1 & 3  & -3 & 0 \\
               0 & 0 & 1  & -1 & 0 & -3 & 2  & 1
             \end{array}
    \right)                                                                   \\
    \xrightarrow{\text{$(3,3)$をかなめとして左から第$3$列を掃き出す}}                        &
    \left( \begin{array}{cccc|cccc}
               1 & 0 & 0 & -3 & 1  & 0  & -1 & 0 \\
               0 & 1 & 0 & 1  & 1  & 5  & -4 & 0 \\
               0 & 0 & 1 & 0  & 1  & 3  & -3 & 0 \\
               0 & 0 & 0 & -1 & -1 & -6 & 5  & 1
             \end{array}
    \right)                                                                   \\
    \xrightarrow{\text{第4行を$(-1)$倍して,$(4,4)$をかなめとして第$4$列を掃き出す}}             &
    \left( \begin{array}{cccc|cccc}
               1 & 0 & 0 & 0 & 4 & 18 & -16 & -3 \\
               0 & 1 & 0 & 0 & 0 & -1 & 1   & 1  \\
               0 & 0 & 1 & 0 & 1 & 3  & -3  & 0  \\
               0 & 0 & 0 & 1 & 1 & 6  & -5  & -1
             \end{array}
    \right)
  \end{align*}
  よって,求める逆行列は,
  \[
    \begin{pmatrix}
      4 & 18 & -16 & -3 \\ 0 & -1 & 1 & 1\\ 1 & 3 & -3 & 0 \\ 1 & 6 & -5 & -1
    \end{pmatrix}
  \]
  である.
\end{tanswer}



\subsection*{p70--73:1-(ロ)}
\addcontentsline{toc}{subsection}{\texorpdfstring{p70--73:1-(ロ)}{p70--73:1-(ロ)}}

\begin{tanswer}
  \begin{align*}
                                                  &
    \left(
    \begin{array}{cccc|cccc}
        1  & 2  & 0 & -1 & 1 & 0 & 0 & 0 \\
        -3 & -5 & 1 & 2  & 0 & 1 & 0 & 0 \\
        1  & 3  & 2 & -2 & 0 & 0 & 1 & 0 \\
        0  & 2  & 1 & -1 & 0 & 0 & 0 & 1
      \end{array}
    \right)                                         \\
    \xrightarrow{\text{$(1,1)$をかなめとして第$1$列を掃き出す}} &
    \left( \begin{array}{cccc|cccc}
               1 & 2 & 0 & -1 & 1  & 0 & 0 & 0 \\
               0 & 1 & 1 & -1 & 3  & 1 & 0 & 0 \\
               0 & 1 & 2 & -1 & -1 & 0 & 1 & 0 \\
               0 & 2 & 1 & -1 & 0  & 0 & 0 & 1
             \end{array}
    \right)                                         \\
    \xrightarrow{\text{$(2,2)$をかなめとして第$2$列を掃き出す}} &
    \left( \begin{array}{cccc|cccc}
               1 & 0 & -2 & 1  & -5 & -2 & 0 & 0 \\
               0 & 1 & 1  & -1 & 3  & 1  & 0 & 0 \\
               0 & 0 & 1  & 0  & -4 & -1 & 1 & 0 \\
               0 & 0 & -1 & 1  & -6 & -2 & 0 & 1
             \end{array}
    \right)                                         \\
    \xrightarrow{\text{$(3,3)$をかなめとして第$3$列を掃き出す}} &
    \left( \begin{array}{cccc|cccc}
               1 & 0 & 0 & 1  & -13 & -4 & 2  & 0 \\
               0 & 1 & 0 & -1 & 7   & 2  & -1 & 0 \\
               0 & 0 & 1 & 0  & -4  & -1 & 1  & 0 \\
               0 & 0 & 0 & 1  & -10 & -3 & 1  & 1
             \end{array}
    \right)                                         \\
    \xrightarrow{\text{$(3,3)$をかなめとして第$3$列を掃き出す}} &
    \left( \begin{array}{cccc|cccc}
               1 & 0 & 0 & 0 & -3  & -1 & 1 & -1 \\
               0 & 1 & 0 & 0 & -3  & -1 & 0 & 1  \\
               0 & 0 & 1 & 0 & -4  & -1 & 1 & 0  \\
               0 & 0 & 0 & 1 & -10 & -3 & 1 & 1
             \end{array}
    \right)                                         \\
  \end{align*}
  よって,求める逆行列は
  \[
    \begin{pmatrix} -3 & -1 & 1& -1 \\ -3 & -1 & 0 & 1 \\ -4 & -1 & 1 & 0 \\ -10 & -3 & 1& 1 \end{pmatrix}
  \]
  である.
\end{tanswer}



\section*{p70--73:2}
\addcontentsline{toc}{section}{\texorpdfstring{p70--73:2}{p70--73:2}}

\subsection*{p70--73:2-(イ)}
\addcontentsline{toc}{subsection}{\texorpdfstring{p70--73:2-(イ)}{p70--73:2-(イ)}}

\begin{tanswer}
  与えられた連立方程式について,拡大係数行列を考えて基本変形を施すと,
  \[
    \begin{pmatrix} 1 & 0 & 0 & 2 & 2 & 0 \\ 0 & 1& 0 & 0 & 0 & -3/5 \\ 0 & 0 & 1 & -1 & 0 & 1/5 \\ 0 & 0 & 0 & 0 & 0 & 0 \end{pmatrix}
  \]
  となる.ただしここで第2列と第5列を入れ替えた.

  つまり,解は存在し,2つの任意定数を含む.任意定数を$x_4 = \alpha$,$x_2 = \beta $とすると,
  \[
    x_1 = -2 \alpha - 2 \beta , \quad x_2 =\beta , \quad x_3 = \alpha + \frac{1}{5} , \quad x_4 = \alpha , \quad x_5 = -\frac{3}{5}
  \]
  となる.ベクトルの形で表すと
  \[
    \begin{pmatrix} x_1 \\ x_2 \\ x_3 \\ x_4 \\ x_5 \end{pmatrix}= \begin{pmatrix} 0 \\ 0 \\ 1/5 \\ 0 \\ -3/5 \end{pmatrix} + \alpha \begin{pmatrix} -2 \\ 0 \\ 1 \\ 1 \\ 0 \end{pmatrix} + \beta \begin{pmatrix} -2 \\ 1 \\ 0 \\ 0 \\ 0 \end{pmatrix}
  \]
  となる.
\end{tanswer}

\subsection*{p70--73:2-(ロ)}
\addcontentsline{toc}{subsection}{\texorpdfstring{p70--73:2-(ロ)}{p70--73:2-(ロ)}}

\begin{tanswer}
  与えられた連立方程式について,拡大係数行列を考えて基本変形を施すと,
  \[
    \begin{pmatrix} 1 & 0 & 0  & 2 & -1 \\ 0 & 1& 0  & -1 & 1 \\ 0 & 0 & 1 & -1 & -1 \\ 0 & 0 & 0 & 0 & 0 \end{pmatrix}
  \]
  となる.

  つまり,解は存在し,一つの任意定数を含む.任意定数を$x_4 = \alpha$とすると,
  \[
    x_1 = -1 -2\alpha   , \quad x_2 =1+\alpha  , \quad x_3 =-1+ \alpha  , \quad x_4 = \alpha
  \]
  となる.ベクトルの形で表すと
  \[
    \begin{pmatrix} x_1 \\ x_2 \\ x_3 \\ x_4  \end{pmatrix}= \begin{pmatrix} -1 \\ 1 \\ -1 \\ 0 \end{pmatrix} +\alpha \begin{pmatrix} -2 \\ 1\\ 1 \\ 1\end{pmatrix}
  \]
  となる.
\end{tanswer}



\subsection*{p70--73:2-(ハ)}
\addcontentsline{toc}{subsection}{\texorpdfstring{p70--73:2-(ハ)}{p70--73:2-(ハ)}}

\begin{tanswer}
  与えられた連立方程式について,拡大係数行列を考えて基本変形を施すと,
  \[
    \begin{pmatrix} 1 & 0 & 0  & 0 & 3 \\ 0 & 1& 0  & 0 & 4 \\ 0 & 0 & 1 & 0 & 1 \\ 0 & 0 & 0 & 1 & 1 \end{pmatrix}
  \]
  となる.

  つまり,解は存在し,
  \[
    x_1 = 3 , \quad x_2 =4 , \quad x_3 = 1 , \quad x_4 = 1
  \]
  となる.ベクトルの形で表すと
  \[
    \begin{pmatrix} x_1 \\ x_2 \\ x_3 \\ x_4 \end{pmatrix}= \begin{pmatrix} 3 \\ 4 \\ 1 \\ 1 \end{pmatrix}
  \]
  となる.
\end{tanswer}


\subsection*{p70--73:2-(ニ)}
\addcontentsline{toc}{subsection}{\texorpdfstring{p70--73:2-(ニ)}{p70--73:2-(ニ)}}

\begin{tanswer}
  与えられた連立方程式について,拡大係数行列を考えて基本変形を施すと,
  \[
    \begin{pmatrix} 1 & 0 & 0  & 2 & -2 & 0  \\ 0 & 1& 0  & 0 & 24 & 4 \\ 0 & 0 & 1 & 0 & 10 & 3 \end{pmatrix}
  \]
  となる.ただしここで第2列と第4列を入れ替えた.

  つまり,解は存在し,$2$つの任意定数を含む.$x_2 = \alpha$,$x_5 = \beta$とすると,
  \[
    x_1 = -2\alpha + 2\beta  , \quad x_2 =\alpha  , \quad x_3 = 3 -10\beta  , \quad x_4 = 4 -24 \beta ,\quad x_5 =\beta
  \]
  となる.ベクトルの形で表すと
  \[
    \begin{pmatrix} x_1 \\ x_2 \\ x_3 \\ x_4 \\ x_5 \end{pmatrix}= \begin{pmatrix} 0 \\ 0 \\ 3 \\ 3 \\ 0 \end{pmatrix} +\alpha \begin{pmatrix} -2 \\ 1\\ 0 \\ 0 \\ 0 \end{pmatrix} + \beta \begin{pmatrix} 2 \\ 0 \\ -10 \\ -24 \\ 1 \end{pmatrix}
  \]
  となる.
\end{tanswer}




\section*{p70--73:3}
\addcontentsline{toc}{section}{\texorpdfstring{p70--73:3}{p70--73:3}}

\subsection*{p70--73:3-(イ)}
\addcontentsline{toc}{subsection}{\texorpdfstring{p70--73:3-(イ)}{p70--73:3-(イ)}}

\begin{tanswer}
  \begin{align*}
                                    &
    \left(
    \begin{array}{ccc|ccc}
        1  & 3  & 2  & 1 & 0 & 0 \\
        -1 & -2 & -1 & 0 & 1 & 0 \\
        2  & 4  & 3  & 0 & 0 & 1
      \end{array}
    \right)                           \\
    \xrightarrow{\text{第$1$列を掃き出す}} &
    \left(
    \begin{array}{ccc|ccc}
        1 & 3  & 2  & 1  & 0 & 0 \\
        0 & 1  & 1  & 1  & 1 & 0 \\
        0 & -2 & -1 & -2 & 0 & 1
      \end{array}
    \right)                           \\
    \xrightarrow{\text{第$2$列を掃き出す}} &
    \left(
    \begin{array}{ccc|ccc}
        1 & 0 & -1 & -2 & -3 & 0 \\
        0 & 1 & 1  & 1  & 1  & 0 \\
        0 & 0 & 1  & 0  & 2  & 1
      \end{array}
    \right)                           \\
    \xrightarrow{\text{第$3$列を掃き出す}} &
    \left(
    \begin{array}{ccc|ccc}
        1 & 0 & 0 & -2 & -1 & 1  \\
        0 & 1 & 0 & 1  & -1 & -1 \\
        0 & 0 & 1 & 0  & 2  & 1
      \end{array}
    \right)
  \end{align*}
  である.ゆえに
  \[
    P^{-1} = \begin{pmatrix} -2 & -1 & 1\\ 1 & -1 & -1 \\ 0 & 2 & 1 \end{pmatrix}
  \]
  である.だから
  \[
    P^{-1} A P = \begin{pmatrix} 2 & 0 & 0 \\ 0 & 1 & 0 \\ 0 & 0 & 0 \end{pmatrix}
  \]
\end{tanswer}


\subsection*{p70--73:3-(ロ)}
\addcontentsline{toc}{subsection}{\texorpdfstring{p70--73:3-(ロ)}{p70--73:3-(ロ)}}


\begin{tanswer}
  $P^{-1} A P = B$とおくと,
  \begin{align*}
    A^n & = P B^n P^{-1}                                                                                                                                                                                                      \\
        & = \begin{pmatrix} 1 & 3 & 2 \\ -1 & -2 & -1 \\ 2 & 4 & 3 \end{pmatrix} \begin{pmatrix} 2^n & 0 & 0 \\ 0 & 1^n & 0 \\ 0 & 0 & 0^n \end{pmatrix} \begin{pmatrix} -2 & -1 & 1\\ 1 & -1 & -1 \\ 0 & 2 & 1 \end{pmatrix} \\
        & = \begin{pmatrix} -2^{n+1}+3 & -2^n -3 & 2^n -3 \\ 2^{n+1}-2 & 2^n +2 & -2^n -2 \\ -2^{n+2}+4 & -2^{n+1} -4 & 2^{n+1} -4 \end{pmatrix}
  \end{align*}
  となる.
\end{tanswer}



\section*{p70--73:4}
\addcontentsline{toc}{section}{\texorpdfstring{p70--73:4}{p70--73:4}}

\begin{tanswer}
  与えられた行列を$A$とする.

  $A$の第1列に,第2列から第$n$列までを足して変形すると,

  \[
    \begin{pmatrix}
      (n-1)x+1 & x      & x      & \cdots & x      \\
      (n-1)x+1 & 1      & x      & \cdots & x      \\
      (n-1)x+1 & x      & 1      & \cdots & x      \\
      \vdots   & \vdots & \vdots & \ddots & \vdots \\
      (n-1)x+1 & x      & x      & \cdots & 1
    \end{pmatrix}
  \]
  となる.

  ここで,この行列の第$2$行から第$n$行までの各行から第$1$行を引くと,

  \[
    \begin{pmatrix}
      (n-1)x+1 & x      & x      & \cdots & x      \\
      0        & 1-x    & 0      & \cdots & 0      \\
      0        & 0      & 1-x    & \cdots & 0      \\
      \vdots   & \vdots & \vdots & \ddots & \vdots \\
      0        & 0      & 0      & \cdots & 1 -x
    \end{pmatrix}
  \]
  となるので,行列$A$の階数は,$x=1$のとき$1$,$ x= -1/(n-1)$のとき$n-1$,それ以外の場合は$n$である.
\end{tanswer}




\section*{p70--73:5}
\addcontentsline{toc}{section}{\texorpdfstring{p70--73:5}{p70--73:5}}


\begin{tproof}
  $A$が正則でないと仮定すると,
  \[
    A \bm{x} = \bm{0}
  \]
  をみたす$\bm{x} \in \mathbb{C}^n$が存在する.

  また,$\bm{x} ={} ^t (x_1,x_2,\dots,x_n)$とし,$x_1,x_2,\dots,x_n$の中で絶対値が最大のものを$x_p$とする.

  $A \bm{x}$の$p$行を考えると,

  \begin{align*}
               & a_{p1} x_1 + a_{p2} x_2 + \dots + a_{pp} x_p + \dots + a_{pn} x_n = 0            \\
    \therefore & ~ x_p = -(a_{p1}x_1 + a_{p2} x_2+ \dots + a_{pn} x_n) =- \sum_{\substack{i \ne p \\ i \in \{1,2,\dots,n\}}} a_{pi} x_i
  \end{align*}
  となる.

  ここで,
  \begin{align*}
    \abs{x_p} & \leqq \sum_{\substack{i \ne p                \\ i \in \{1,2,\dots,n\}}} \abs{a_{pi}} \abs{x_i} \\
              & < \sum_{\substack{i \ne p                    \\ i \in \{1,2,\dots,n\}}} \frac{1}{n-1} \abs{x_i} \\
              & < \frac{n-1}{n-1} \cdot \abs{x_p} =\abs{x_p}
  \end{align*}
  と計算でき,$ \abs{x_p} < \abs{x_p}$となり,これは矛盾である.

  よって,先の過程が誤りであり,このとき$A$は正則である.
\end{tproof}


\section*{p70--73:6}
\addcontentsline{toc}{section}{\texorpdfstring{p70--73:6}{p70--73:6}}

\subsection*{p70--73:6-(イ)}
\addcontentsline{toc}{subsection}{\texorpdfstring{p70--73:6-(イ)}{p70--73:6-(イ)}}
\begin{tproof}
  \[
    A A^{k-1} = A^{k-1} A = E
  \]
  なので,$A$は正則である.
\end{tproof}

\subsection*{p70--73:6-(ロ)}
\addcontentsline{toc}{subsection}{\texorpdfstring{p70--73:6-(ロ)}{p70--73:6-(ロ)}}


\begin{tproof}
  $A$が正則であるとすると,$A^{-1}$が存在して,
  \begin{align*}
     & A^{-1} A^{2} = A^{-1} A \\
     & A = E
  \end{align*}
  となるが,これは矛盾であるため,$A$は正則でない.
\end{tproof}

\subsection*{p70--73:6-(ハ)}
\addcontentsline{toc}{subsection}{\texorpdfstring{p70--73:6-(ハ)}{p70--73:6-(ハ)}}

\begin{tproof}
  $A$が正則であるとすると,
  \begin{align*}
    E & = (A^{-1} A)^{k} \\
      & = A^{-k} A^{k}   \\
      & = O
  \end{align*}
  となるが,これは矛盾であるため,$A$は正則でない.
\end{tproof}

\subsection*{p70--73:6-(ニ)}
\addcontentsline{toc}{subsection}{\texorpdfstring{p70--73:6-(ニ)}{p70--73:6-(ニ)}}

\begin{tproof}
  $k$を用いて,$A^k$を考えると
  \[
    E = (E-A)(E+A+A^2+\dots+A^{k-1})
  \]
  であり,逆からかけても同じであるため,$E-A$は正則であり,
  \[
    (E-A)^{-1}=E+A+A^2+\dots+A^{k-1}
  \]
  である.

  また,
  \[
    E=(E+A)(E-A+A^2-\dots+A^{k-1})
  \]
  であり,逆からかけても同じなので,$E+A$は正則であり,
  \[
    (E+A)^{-1} = E-A+A^2-\dots+A^{k-1}
  \]
  である.
\end{tproof}



\section*{p70--73:7}
\addcontentsline{toc}{section}{\texorpdfstring{p70--73:7}{p70--73:7}}

\begin{tproof}
  $X=(x_{ij}),~Y=(y_{ij})$とする.
  ここで,$XY$の$(i,i)$成分は$\sum_{j=1}^{n} x_{ij} y_{ji}$であるから,
  \begin{equation*}
    \tr (XY)  =  \sum_{i=1}^{n} \left( \sum_{j=1}^{n} x_{ij} y_{ji} \right)
  \end{equation*}
  となる.$YX$については,同様の議論により,
  \begin{align*}
    \tr  (YX) & =  \sum_{i=1}^{n}\left( \sum_{j=1}^{n}  y_{ij} x_{ji} \right)  \\
              & =  \sum_{i=1}^{n} \left( \sum_{j=1}^{n}  x_{ji} y_{ij} \right)
  \end{align*}
  である.ここで,$i$と$j$をおきかえれば,
  \begin{equation}
    \tr  (YX) = \sum_{j=1}^{n} \left( \sum_{i=1}^{n}  x_{ij} y_{ji} \right)
  \end{equation}
  となる.これより,
  \begin{equation}
    \tr (XY) = \tr  (YX)
  \end{equation}
  を得て,これとトレースの線型性により$\tr (XY-YX) =0$であるが,$\tr  (E_n) =n \neq 0$であるため,これは矛盾である.

  ゆえに,$XY-YX=E_n$となる$n$次行列$X,~Y$は存在しないことが示された.
\end{tproof}



\section*{p70--73:8}
\addcontentsline{toc}{section}{\texorpdfstring{p70--73:8}{p70--73:8}}

\begin{tproof}
  行列$B$の階数を$r$とすると,$m$次正則行列$P$,$n$次正則行列$Q$によって,
  \[
    P B Q = F_{m,n} (r)
  \]
  と表せる.

  これにより,
  \[
    ABQ = A P^{-1} F_{m,n} (r)
  \]
  とかける.$A_{11}$を$r$次の行列として,
  \[
    A P^{-1} = \begin{pmatrix} A_{11} & A_{12} \\ A_{21} & A_{22} \end{pmatrix}, \quad F_{m,n} (r) = \begin{pmatrix} E_r & O \\ O & O \end{pmatrix}
  \]
  とかくと,
  \begin{align*}
    A P^{-1} F_{m,n} (r) & = A P^{-1} Q                                            \\
                         & = \begin{pmatrix} A_{11}& O \\ A_{21} & O \end{pmatrix}
  \end{align*}
  とかけ,$A_{11}$の定義により,$ABQ$の階数は$r$以下となる.いま$Q$は基本行列の積なので, $AB$の階数も$r$以下である.

  行列$A$についても同様に示せる.

  以上の議論により,行列$AB$の階数は行列$A$,行列$B$の階数以下であることが証明された.

\end{tproof}




\section*{p70--73:9}
\addcontentsline{toc}{section}{\texorpdfstring{p70--73:9}{p70--73:9}}

\begin{tanswer}
  $3$つの平面が$1$本の直線を共有する必要十分条件は,
  与式を$x$,$y$,$z$に関する方程式とみたときに,解が存在して$1$つの任意定数を含むことである.

  これは
  \[
    \begin{cases}
      r(A)=2 \\
      r(A)=r(\tilde{A})
    \end{cases}
  \]
  と同値であり,したがって,
  \[
    r(A)=r(\tilde{A})=2
  \]
  が必要十分条件である.
\end{tanswer}




\section*{p70--73:10}
\addcontentsline{toc}{section}{\texorpdfstring{p70--73:10}{p70--73:10}}

\subsection*{p70--73:10-(イ)}
\addcontentsline{toc}{subsection}{\texorpdfstring{p70--73:10-(イ)}{p70--73:10-(イ)}}

\begin{tproof}
  $AX=E$をみたす$n$次正則行列$X$が存在するとする.このとき,
  \[
    X = \begin{pmatrix} x_{11} & x_{12} \\ x_{21} & x_{22} \end{pmatrix}
  \]
  とおくと,
  \begin{align*}
    AX & = \begin{pmatrix} a & -b \\ b & a \end{pmatrix} \begin{pmatrix} x_{11} & x_{12} \\ x_{21} & x_{22} \end{pmatrix} \\
       & = \begin{pmatrix} ax_{11}-bx_{21} & ax_{12}-bx_{22} \\ bx_{11}+ax_{21} & bx_{12}+ax_{22} \end{pmatrix}
  \end{align*}
  となり,これが$E$に等しいので,
  \[
    \begin{cases}
      ax_{11}-bx_{21} =1 \\
      ax_{12}-bx_{22} =0 \\
      bx_{11}+ax_{21} =0 \\
      bx_{12}+ax_{22} =1
    \end{cases}
  \]
  となり,これを変形すると,
  \[
    \begin{cases}
      (a^2+b^2) x_{11} = a \\
      (a^2+b^2) x_{12} = b \\
      (a^2+b^2) x_{21}=-b  \\
      (a^2+b^2) x_{22} = a
    \end{cases}
  \]
  となるから,このような$x_{11}$,$x_{12}$,$x_{21}$,$x_{22}$が存在する必要十分条件は
  \[
    a^2 + b^2 \ne 0
  \]
  である.このことから直ちに主張が従う.
\end{tproof}




\subsection*{p70--73:10-(ロ)}
\addcontentsline{toc}{subsection}{\texorpdfstring{p70--73:10-(ロ)}{p70--73:10-(ロ)}}

\begin{tproof}
  \[
    A' = \begin{pmatrix} a' & -b' \\ b' & a' \end{pmatrix} , \quad \alpha ' = a' + b' i
  \]
  とおく.

  和については
  \[
    A+A' = \begin{pmatrix} a+a' & -(b+b') \\ b+b' & a+a' \end{pmatrix} , \quad \alpha + \alpha ' = (a+a') + (b+b') i
  \]
  となり,このときたしかに$A+A'$と$\alpha + \alpha '$が一対一に対応する.

  積については
  \[
    A A' = \begin{pmatrix} a a' - b b' & - (a b' + b a') \\ a b' + b a' & a a' - b b' \end{pmatrix} , \quad \alpha \alpha ' = (a a' - b b') + (a b' + b a') i
  \]
  となり,たしかに$AA'$と$\alpha \alpha '$が一対一に対応する.

  逆数については
  \[
    A^{-1} = \frac{1}{a^2+b^2} \begin{pmatrix} a & b \\ -b & a \end{pmatrix},\quad \frac{1}{\alpha} = \frac{1}{a^2+b^2}(a -bi )
  \]
  となり,たしかに$A^{-1}$と$1/\alpha$が一対一に対応する.

  以上の考察により証明された.
\end{tproof}




\subsection*{p70--73:10-(ハ)}
\addcontentsline{toc}{subsection}{\texorpdfstring{p70--73:10-(ハ)}{p70--73:10-(ハ)}}

\begin{tproof}
  \[
    \alpha = r (\cos \theta + i \sin \theta)
  \]
  と表せるとすると,
  \[
    a+bi = r (\cos \theta + i \sin \theta )
  \]
  であるから,
  \[
    a= r \cos \theta , \quad r = r \sin \theta
  \]
  とかけ,このとき
  \[
    \begin{pmatrix} a & -b \\ b & a \end{pmatrix} = r \begin{pmatrix} \cos \theta & -\sin \theta \\ \sin \theta & \cos \theta \end{pmatrix}
  \]
  となる.これが証明すべきことであった.
\end{tproof}



\section*{p70--73:11}
\addcontentsline{toc}{section}{\texorpdfstring{p70--73:11}{p70--73:11}}

\subsection*{p70--73:11-(イ)}
\addcontentsline{toc}{subsection}{\texorpdfstring{p70--73:11-(イ)}{p70--73:11-(イ)}}

\begin{tproof}
  ${}^t PP =E$を加味して$(P \pm E)$の転置行列を考えると
  \[
    {}^t (P \pm E) = {}^t P \pm {}^t PP = {}^t P {}^t (E \pm P)
  \]
  となり,これを用いると,

  \begin{align*}
    {}^t A & ={}^t  \{ (P-E)(P+E)^{-1} \}            \\
           & = {}^t (P+E)^{-1} {}^t (P-E)            \\
           & = (E+{}^t P)^{-1} {}^t P {}^t P (E-P)   \\
           & = \{ {}^t P (P+E)  \}^{-1} {}^t P (E-P) \\
           & = (P+E)^{-1} {}^t P^{-1} {}^t P (E-P)   \\
           & = (P+E)^{-1} (E-P)                      \\
           & = -(P+E)^{-1} \{ (P+E)-2E \}            \\
           & = -(P+E)^{-1} (P+E) +2E (P+E)^{-1}      \\
           & = -(P+E) (P+E)^{-1} +2E (P+E)^{-1}      \\
           & = (-(P+E)+2E) (P+E)^{-1}                \\
           & = -(P-E) (P+E)^{-1} = -A
  \end{align*}
  となり,これが証明すべきことであった.
\end{tproof}


\subsection*{p70--73:11-(ロ)}
\addcontentsline{toc}{subsection}{\texorpdfstring{p70--73:11-(ロ)}{p70--73:11-(ロ)}}

\begin{tproof}

  計算すると,
  \begin{align*}
    E-A & = E-(P-E) (P+E)^{-1}                 \\
        & = (P+E)(P+E)^{-1} - (P-E) (P+E)^{-1} \\
        & = \{ (P+E)-(P-E) \} (P+E)^{-1}       \\
        & = 2(P+E)^{-1}
  \end{align*}
  と変形でき,いま$(P+E)$が正則だから,$2(P+E)^{-1}$も正則であり,
  \[
    (E-A)^{-1} = \frac{1}{2} (P+E)
  \]
  である.
\end{tproof}


\subsection*{p70--73:11-(ハ)}
\addcontentsline{toc}{subsection}{\texorpdfstring{p70--73:11-(ハ)}{p70--73:11-(ハ)}}

\begin{tproof}
  まず,
  \begin{align*}
    E+A & =(P+E)(P+E)^{-1}+(P-E) (P+E)^{-1} \\
        & = \{ (P+E)+(P-E) \} (P+E)^{-1}    \\
        & = 2P (P+E)^{-1}
  \end{align*}
  であるから,これを用いると
  \[
    (E+A)(E-A)^{-1} = 2P (P+E)^{-1} \frac{1}{2} (P+E) =P
  \]
  となり,これが証明すべきことであった.
\end{tproof}


\section*{p70--73:12}
\addcontentsline{toc}{section}{\texorpdfstring{p70--73:12}{p70--73:12}}

\begin{tproof}
  以下の3つの命題が同値であることを示す.
  \begin{enumerate}[(1)]
    \item $A$は正規行列である.すなわち$A^\ast A = A A^\ast$である.
    \item 任意の$\bm{x} \in \mathbb{C}^n$について,$\norm{A \bm{x}}=\norm{A^\ast \bm{x}}$が成立する.
    \item 任意の$\bm{x},\bm{y} \in \mathbb{C}^n$について,$(A\bm{x}, A\bm{y}) = (A^\ast \bm{x}, A^\ast\bm{y})$が成立する.
  \end{enumerate}
  \begin{description}
    \item[(1) $\Longrightarrow$ (2)] $\bm{x} \in \mathbb{C}^n$を任意にとる.このとき
          \begin{align*}
            \norm{A \bm{x}}^2 & = (A \bm{x}, A \bm{x})                                      \\
                              & = (\bm{x}, A^\ast A\bm{x})                                  \\
                              & = (\bm{x}, A A^\ast \bm{x}) \quad (\because \text{正規行列の定義}) \\
                              & = (A^\ast \bm{x}, A^\ast \bm{x}) = \norm{A^\ast \bm{x}}^2
          \end{align*}
          であるから,$\norm{A \bm{x}}=\norm{A^\ast \bm{x}}$が成立する.
    \item[(2) $\Longrightarrow$ (3)] $\bm{x},\bm{y} \in \mathbb{C}^n$を任意にとる.このとき
          \begin{align*}
            \norm{A(\bm{x}+\bm{y})}^2 & = (A(\bm{x}+\bm{y}), A(\bm{x}+\bm{y}))                                                     \\
                                      & = \norm{A\bm{x}}^2  + (A\bm{x}, A\bm{y}) + (A\bm{y}, A\bm{x}) +\norm{A\bm{y}}^2            \\
                                      & = \norm{A\bm{x}}^2  + (A\bm{x}, A\bm{y}) + \overline{(A\bm{x}, A\bm{y})} +\norm{A\bm{y}}^2
          \end{align*}
          であり,同様に計算すると
          \[
            \norm{A^\ast(\bm{x}+\bm{y})}^2 = \norm{A^\ast \bm{x}}^2  + (A^\ast \bm{x}, A^\ast\bm{y}) + \overline{(A^\ast \bm{x}, A^\ast\bm{y})} +\norm{A^\ast\bm{y}}^2
          \]
          を得る.$\norm{A(\bm{x}+\bm{y})}=\norm{A^\ast(\bm{x}+\bm{y})}$,$\norm{A\bm{x}}=\norm{A^\ast \bm{x}}$,$\norm{A\bm{y}}=\norm{A^\ast \bm{y}}$を仮定したので,
          \[
            (A\bm{x}, A\bm{y}) + \overline{(A\bm{x}, A\bm{y})} = (A^\ast \bm{x}, A^\ast\bm{y}) + \overline{(A^\ast \bm{x}, A^\ast\bm{y})}
          \]
          となり,これは$ \Re (A\bm{x}, A\bm{y}) = \Re (A^\ast \bm{x}, A^\ast\bm{y})$であることを表す.

          また,$\bm{x}$を$i\bm{x}$におきかえることで,$ \Im (A\bm{x}, A\bm{y}) = \Im (A^\ast \bm{x}, A^\ast\bm{y})$も示される.ゆえにこのとき
          \[
            (A\bm{x}, A\bm{y}) = (A^\ast \bm{x}, A^\ast\bm{y}).
          \]
    \item[(3) $\Longrightarrow$ (1)] 任意の$\bm{x},\bm{y} \in \mathbb{C}^n$に対して,
          \[
            (\bm{x}, (A^\ast A-A A^\ast)\bm{y}) =0
          \]
          である.いま$\bm{y}$をいったん固定すると,$\bm{x}$は任意なので,$\bm{x}=(A^\ast A-A A^\ast)\bm{y}$とすることができ,
          \begin{align*}
             & \norm{(A^\ast A-A A^\ast)\bm{y}}^2  = 0,       \\
             & \because ~ (A^\ast A-A A^\ast)\bm{y} = \bm{0}.
          \end{align*}
          $\bm{y}$は任意だから,$A^\ast A=A A^\ast$となり,$A$は正規行列である.
  \end{description}
  以上の議論により証明された.
\end{tproof}




\section*{p70--73:13}
\addcontentsline{toc}{section}{\texorpdfstring{p70--73:13}{p70--73:13}}

\subsection*{p70--73:13-(イ)}
\addcontentsline{toc}{subsection}{\texorpdfstring{p70--73:13-(イ)}{p70--73:13-(イ)}}

\begin{tanswer}
  まず,
  \begin{align*}
    [ [X,Y],Z ] & = [XY-YX,Z]           \\
                & = (XY-YX)Z -Z(XY-YX)  \\
                & = XYZ -YXZ -ZXY +ZYX.
  \end{align*}
  同様に計算すると,
  \begin{align*}
     & [[Y,Z],X] = YZX -ZYX -XYZ +XZY,  \\
     & [ [Z,X],Y] = ZXY -XZY -YZX +YXZ.
  \end{align*}
  よって,
  \[
    [[X,Y],Z] +[[Y,Z],X]+[[Z,X],Y]=O
  \]
  である.
\end{tanswer}


\subsection*{p70--73:13-(ロ)}
\addcontentsline{toc}{subsection}{\texorpdfstring{p70--73:13-(ロ)}{p70--73:13-(ロ)}}

\begin{tproof}
  $X$,$Y$は交代行列だから,
  \[
    X=- {}^t X ,\quad Y = -{}^t Y .
  \]
  これを用いると,
  \begin{align*}
    [X,Y] & = XY -YX                                   \\
          & = (-{}^t X) (-{}^t Y) - (-{}^t Y)(-{}^t X) \\
          & = {}^t (YX) - {}^t (XY)                    \\
          & = -{}^t (XY-YX)                            \\
          & = -{}^t [X,Y]
  \end{align*}
  となる.よってこのとき$[X,Y]$は交代行列である.
\end{tproof}



\subsection*{p70--73:13-(ハ)}
\addcontentsline{toc}{subsection}{\texorpdfstring{p70--73:13-(ハ)}{p70--73:13-(ハ)}}

\begin{tproof}
  以下では
  \[
    Y = \begin{pmatrix} 0 & -z ' & y ' \\ z' & 0 & -x' \\ -y' & x ' & 0 \end{pmatrix},\quad \bm{y} = \begin{pmatrix} x' \\ y ' \\ z ' \end{pmatrix}
  \]
  とおく.
  \begin{description}
    \item[【$X+Y$と$\bm{x}+\bm{y}$について】]
          \[
            X + Y  = \begin{pmatrix} 0 & -z & y \\ z & 0 & -x \\ -y & x & 0 \end{pmatrix}+ \begin{pmatrix} 0 & -z ' & y ' \\ z' & 0 & -x' \\ -y' & x ' & 0 \end{pmatrix} = \begin{pmatrix} 0 & -(z+z)' & y+y' \\ z+z & 0 & -(x+x') \\ -(y+y') & x+x' & 0 \end{pmatrix}
          \]
          であり,なおかつ
          \[
            \bm{x}+\bm{y} =\begin{pmatrix} x \\ y \\ z \end{pmatrix}+ \begin{pmatrix} x' \\ y' \\ z' \end{pmatrix} = \begin{pmatrix} x+x' \\ y+y' \\ z+z' \end{pmatrix}
          \]
          であるから,たしかに$X+Y$と$\bm{x}+\bm{y}$は対応する.
    \item[【$cX$と$c\bm{x}$について】]
          \[
            cX =c\begin{pmatrix} 0 & -z & y \\ z & 0 & -x \\ -y & x & 0 \end{pmatrix} =\begin{pmatrix} 0 & -cz & cy \\ cz & 0 & -cx \\ -cy & cx & 0 \end{pmatrix}
          \]
          であり,なおかつ
          \[
            c\bm{x} = c\begin{pmatrix} x \\ y \\ z \end{pmatrix} = \begin{pmatrix} cx \\ cy \\ cz \end{pmatrix}
          \]
          であるから,たしかに$cX$と$c\bm{x}$は対応する.
    \item [【\text{$[X,Y]$と$\bm{x} \times \bm{y}$について】}]
          \begin{align*}
            [X,Y] & =\begin{pmatrix} 0 & -z & y \\ z & 0 & -x \\ -y & x & 0 \end{pmatrix}\begin{pmatrix} 0 & -z ' & y ' \\ z' & 0 & -x' \\ -y' & x ' & 0 \end{pmatrix}-\begin{pmatrix} 0 & -z ' & y ' \\ z' & 0 & -x' \\ -y' & x ' & 0 \end{pmatrix} \begin{pmatrix} 0 & -z & y \\ z & 0 & -x \\ -y & x & 0 \end{pmatrix} \\
                  & = \begin{pmatrix} 0 & -z'x+x'z & y'x-x'y \\ z'x-x'z & 0 & -y'z+z'y \\ -y'x+x'y & z'y-y'z & 0 \end{pmatrix}
          \end{align*}
          であり,なおかつ
          \[
            \bm{x}\times \bm{y} = \begin{pmatrix} x \\ y \\ z \end{pmatrix} \times \begin{pmatrix} x' \\ y ' \\ z' \end{pmatrix} = \begin{pmatrix} yz'-zy' \\ zx'-xz' \\ xy'-yx' \end{pmatrix}
          \]
          であるから,たしかに$[X,Y]$と$\bm{x} \times \bm{y}$は対応する.
    \item[【$X\bm{y}$と$\bm{x}\times\bm{y}$について】]
          \[
            \begin{pmatrix} 0 & -z & y \\ z & 0 & -x \\ -y & x & 0 \end{pmatrix} \begin{pmatrix} x' \\ y' \\ z' \end{pmatrix} = \begin{pmatrix} -zy'+yz' \\ zx'-xz' \\ -yx'+xy' \end{pmatrix}
          \]
          であり,なおかつ

          \[
            \bm{x}\times \bm{y} = \begin{pmatrix} x \\ y \\ z \end{pmatrix} \times \begin{pmatrix} x' \\ y ' \\ z' \end{pmatrix} = \begin{pmatrix} yz'-zy' \\ zx'-xz' \\ xy'-yx' \end{pmatrix}
          \]
          であるから,たしかに$X\bm{y}$と$\bm{x} \times \bm{y}$は対応する.
  \end{description}
\end{tproof}




\subsection*{p70--73:13-(ニ)}
\addcontentsline{toc}{subsection}{\texorpdfstring{p70--73:13-(ニ)}{p70--73:13-(ニ)}}
\begin{tproof}
  ハ)で証明したことから,$[X,Y]$には$\bm{x} \times \bm{y}$が対応する.

  また,イ)で証明したことより,$[[X,Y],Z] +[[Y,Z],X]+[[Z,X],Y]=O$であり,
  この左辺には$(\bm{x}\times\bm{y}) \times \bm{z} + (\bm{y}\times\bm{z}) \times \bm{x} + (\bm{z}\times\bm{x}) \times \bm{y}$が対応し,右辺には$\bm{0}$が対応する.

  以上の考察により
  \[
    (\bm{x}\times\bm{y}) \times \bm{z} + (\bm{y}\times\bm{z}) \times \bm{x} + (\bm{z}\times\bm{x}) \times \bm{y} =\bm{0}
  \]
  であることが示された.
\end{tproof}




\section*{p70--73:14}
\addcontentsline{toc}{section}{\texorpdfstring{p70--73:14}{p70--73:14}}

\begin{tproof}
  二つに分けて証明する,
  \begin{description}
    \item[イ)$\Longrightarrow$ロ)] \mbox{} \\
          $A$が正則であると仮定すると,$A^{-1}$が存在し,
          \[
            \bm{x} = A^{-1} (A\bm{x})
          \]
          と変形できるから,$A \bm{x}$が非負ベクトルであれば,$\bm{x}$も非負ベクトルである.
    \item[ロ)$\Longrightarrow$イ)] \mbox{} \\
          まず,$ A \bm{x} =\bm{0}$であると仮定する.このとき,$A (-\bm{x}) =\bm{0}$であるから,
          $A (-\bm{x})$も非負ベクトルであり,条件から$ \bm{x}$,$-\bm{x}$は非負ベクトルである.
          したがって$\bm{x}=\bm{0}$となり,$A$は正則である.

          また,非負ベクトル$\bm{x}$を任意にとると,
          \[
            \bm{x} = A (A^{-1} \bm{x})
          \]
          も非負ベクトルであり,条件から$A^{-1} \bm{x}$も非負ベクトルである.
          ここで,$A^{-1}$が非負行列でないと仮定すると,ある単位ベクトル$\bm{e}_j$について,
          $A^{-1} \bm{e}_j $が非負ベクトルでないことになり,$\bm{x}$が非負ベクトルであることに反する.
          これより$A^{-1}$は非負行列である.
  \end{description}
  以上の議論により証明された.
\end{tproof}






\section*{p70--73:15}
\addcontentsline{toc}{section}{\texorpdfstring{p70--73:15}{p70--73:15}}


\subsection*{p70--73:15-(イ)}
\addcontentsline{toc}{subsection}{\texorpdfstring{p70--73:15-(イ)}{p70--73:15-(イ)}}
\begin{tproof}
  まず,$A=(a_{ij})$,$\bm{f} = {}^t (f_1 , f_2,\dots,f_j) ={}^t (1,1,\dots,1)$とおくと,
  $A \bm{f}$の第$i$行の成分は
  \begin{align*}
    \sum_{j=1}^{n} a_{ij} f_j & = \sum_{j=1}^{n} a_{ij} \\
                              & =1
  \end{align*}
  であるから,$\bm{f}$の定義とあわせて,
  \[
    A \bm{f} =\bm{f}
  \]
  が成り立つ.
\end{tproof}


\subsection*{p70--73:15-(ロ)}
\addcontentsline{toc}{subsection}{\texorpdfstring{p70--73:15-(ロ)}{p70--73:15-(ロ)}}

\begin{tproof}
  $C =AB=(c_{ij})$とすると,$C$の$(i,k)$成分は
  \[
    c_{ik}  =\sum_{j=1}^{n} a_{ij} b_{jk}
  \]
  である.これにより,
  \begin{align*}
    \sum_{k=1}^{n} c_{ik} & = \sum_{k=1}^{n} \left (\sum_{j=1}^{n} a_{ij} b_{jk}\right) \\
                          & = \sum_{j=1}^{n} a_{ij} \sum_{k=1}^{n} b_{jk}               \\
                          & = \sum_{j=1}^{n} a_{ij} \cdot 1                             \\
                          & = 1
  \end{align*}
  であるから,$C$すなわち$AB$は確率行列である.
\end{tproof}

\subsection*{p70--73:15-(ハ)}
\addcontentsline{toc}{subsection}{\texorpdfstring{p70--73:15-(ハ)}{p70--73:15-(ハ)}}

\begin{tproof}
  $ A \bm{x}=\alpha \bm{x}$において,$\bm{x}$の成分で絶対値が最大のものを$x_p$とする.

  このとき,$ A \bm{x} = \alpha \bm{x}$の第$p$行成分の絶対値を考えると,
  \begin{align*}
    \abs{\alpha} \abs{x_p} & \leqq \sum_{j=1}^{n} a_{pj} \abs{x_j} \\
                           & \leqq \sum_{j=1}^{n} a_{pj} \abs{x_p} \\
                           & = \abs{x_p}
  \end{align*}
  であるから,
  \[
    \abs{\alpha} \abs{x_p} \leqq \abs{x_p}
  \]
  を得るので,
  \[
    \abs{\alpha} \leqq 1
  \]
  となり,これが証明すべきことであった.
\end{tproof}
