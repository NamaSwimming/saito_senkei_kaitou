\part*{第4章}
\addcontentsline{toc}{part}{\texorpdfstring{第4章}{第4章}}



\section*{p93:問}
\addcontentsline{toc}{section}{\texorpdfstring{p93:問}{p239:問}}

\begin{tproof}
  $\abs{A \cup B}$について,$ \abs{A \cap B}$は$A$と$B$の共通部分の元の個数を考えているので,
  \begin{align*}
     & \abs{A \cup B} = \abs{A}+\abs{B} - \abs{A \cap B}              \\
     & \therefore ~ \abs{A}+\abs{B} = \abs{A \cup B} + \abs{A \cap B}
  \end{align*}
  である.これが証明すべきことであった.
\end{tproof}




\section*{p94:問}
\addcontentsline{toc}{section}{\texorpdfstring{p94:問}{p94:問}}

\begin{tanswer}
  3つのことを証明する.
  \begin{description}
    \item [反射律について] 明らかに,$A$に基本変形を施して$A$自身にすることができる.
    \item [対称律について] $P$を$(m,m)$型の基本行列,$Q$を$(n,n)$型の基本行列として,
          \[
            A = P B Q
          \]
          とかくと,$P$,$Q$は正則なので, $P^{-1}$,$Q^{-1}$が存在し,
          \[
            B= P^{-1} A Q^{-1}
          \]
          とかける.よって,対称律が成り立つことが示された.
    \item[推移律について] $P_1$,$P_2$を$(m,m)$型の基本行列,$Q_1$,$Q_2$を$(n,n)$型の基本行列として,
          \[
            A = P_1 B Q_1 , \quad B = P_2 C Q_2
          \]
          とかく.このとき,$P_1$,$Q_1$は正則だから,${P_1}^{-1}$,${Q_1}^{-1}$が存在し,
          \[
            B = {P_1}^{-1} A {Q_1}^{-1}
          \]
          となる.これにより,
          \[
            {P_1}^{-1} A {Q_1}^{-1} =P_2 C Q_2
          \]
          となり,同様の議論によって
          \[
            A = P_1 P_2 C Q_2 Q_1
          \]
          となり,推移律も成り立つことが示された.\qed
  \end{description}
  さて,行列$A$に基本変形を施すと,$A$の階数を$r$として$F_{m,n} (r)$が得られることと,
  $r$は$0$から$\min \{m,n\}$までの整数値を取り得るので,商集合の元の個数は
  \[
    \min \{ m , n \} +1
  \]
  となる.
\end{tanswer}





\section*{p106--107:問1}
\addcontentsline{toc}{section}{\texorpdfstring{p106--107:問1}{p106--107:問1}}
\begin{tanswer}
  求める$E\to F$の取り替え行列を$P=(p_{ij})$とし,
  \begin{align*}
     & \bm{e}_1=
    \begin{pmatrix}
      1 \\
      0 \\
      1
    \end{pmatrix}
    ,\quad
    \bm{e}_2=
    \begin{pmatrix}
      2 \\
      1 \\
      0
    \end{pmatrix}
    ,\quad
    \bm{e}_3=
    \begin{pmatrix}
      1 \\
      1 \\
      1
    \end{pmatrix}
    ,            \\
     &
    \bm{f}_1=
    \begin{pmatrix}
      3  \\
      -1 \\
      4
    \end{pmatrix}
    ,\quad
    \bm{f}_2=
    \begin{pmatrix}
      4 \\
      1 \\
      8
    \end{pmatrix}
    ,\quad
    \bm{f}_3=
    \begin{pmatrix}
      3  \\
      -2 \\
      6
    \end{pmatrix}
  \end{align*}
  とする.ここで,
  \[
    \bm{f}_i=\sum^{3}_{j=1}p_{ji}\bm{e}_{j}=p_{1i}\bm{e}_1+p_{2i}\bm{e}_2+p_{3i}\bm{e}_3
  \]
  であり,$i=1,2,3$の場合についての連立方程式を作ると
  \begin{align*}
     & f_1=p_{11}\bm{e}_1+p_{21}\bm{e}_2+p_{31}\bm{e}_3 \\
     & f_2=p_{12}\bm{e}_1+p_{22}\bm{e}_2+p_{32}\bm{e}_3 \\
     & f_3=p_{13}\bm{e}_1+p_{23}\bm{e}_2+p_{33}\bm{e}_3
  \end{align*}
  これを解くことにより
  \begin{align*}
     & p_{11}=\frac{9}{2},\quad p_{21}=-\frac{1}{2},\quad p_{31}=-\frac{1}{2}, \\
     & p_{12}=5,\quad p_{22}=-2,\quad p_{32}=3,                                \\
     & p_{13}=\frac{13}{2},\quad p_{23}=-\frac{3}{2},\quad p_{33}=-\frac{1}{2}
  \end{align*}
  なので,
  \[
    P=
    \begin{pmatrix}
      9/2  & 5  & 13/2 \\
      -1/2 & -2 & -3/2 \\
      -1/2 & 3  & -1/2
    \end{pmatrix}
  \]
  である.また
  \begin{equation*}
    (\bm{f}_1,\bm{f}_2,\ldots ,\bm{f}_n)=(\bm{e}_1,\bm{e}_2,\ldots ,\bm{e}_n)P
  \end{equation*}
  であるから
  \begin{align*}
     &
    \begin{pmatrix}
      3  & 4 & 3  \\
      -1 & 1 & -2 \\
      4  & 8 & 6
    \end{pmatrix}
    =
    \begin{pmatrix}
      1 & 2 & 1 \\
      0 & 1 & 1 \\
      1 & 0 & 1
    \end{pmatrix}
    P     \\
     & P=
    \begin{pmatrix}
      1 & 2 & 1 \\
      0 & 1 & 1 \\
      1 & 0 & 1
    \end{pmatrix}
    ^{-1}
    \begin{pmatrix}
      3  & 4 & 3  \\
      -1 & 1 & -2 \\
      4  & 8 & 6
    \end{pmatrix}
  \end{align*}
  から求めることもできる.
\end{tanswer}



\section*{p106--107:問2}
\addcontentsline{toc}{section}{\texorpdfstring{p106--107:問2}{p106--107:問2}}

\begin{tanswer}
  まず,
  \begin{equation*}
    \bm{f}_i=\sum^{2}_{j=1}p_{ji}\bm{e}_{j}=p_{1i}\bm{e}_1+p_{2i}\bm{e}_2
  \end{equation*}
  である,これにより
  \begin{align*}
     & \bm{f}_1=p_{11}\bm{e}_1+p_{21}\bm{e}_2, \\
     & \bm{f}_2=p_{12}\bm{e}_1+p_{22}\bm{e}_2
  \end{align*}
  であるから,
  \begin{align*}
     &
    \begin{pmatrix}
      0 \\
      1 \\
      -1
    \end{pmatrix}
    =
    p_{11}
    \begin{pmatrix}
      1  \\
      -1 \\
      0
    \end{pmatrix}
    +p_{21}
    \begin{pmatrix}
      1 \\
      0 \\
      -1
    \end{pmatrix}
    ,  \\
     &
    \begin{pmatrix}
      1 \\
      1 \\
      -2
    \end{pmatrix}
    =
    p_{12}
    \begin{pmatrix}
      1  \\
      -1 \\
      0
    \end{pmatrix}
    +p_{22}
    \begin{pmatrix}
      1 \\
      0 \\
      -1
    \end{pmatrix}
  \end{align*}
  となり,これにより
  \[
    p_{11}=-1,\quad p_{21}=1,\quad p_{12}=-1,\quad p_{22}=2
  \]
  であるから,基底の取り替え$E \to F$の行列は
  \begin{align*}
    P=
    \begin{pmatrix}
      -1 & -1 \\
      1  & 2
    \end{pmatrix}
  \end{align*}
  である.
\end{tanswer}


\section*{p107--108:問1}
\addcontentsline{toc}{section}{\texorpdfstring{p107--108:問1}{p107--108:問1}}


\subsection*{p107--108:問1-(イ)}
\addcontentsline{toc}{subsection}{\texorpdfstring{p107--108:問1-(イ)}{p107--108:問1-(イ)}}


\begin{tanswer}
  この集合を$W_1$とおくと,$W_1$は$\mathbb{C}^n$の部分空間をなす.
  \[
    \bm{x} =\bm{0} \in  W_1
  \]
  であるから,$W_1 \ne \varnothing$である.

  また,

  \[
    \bm{v} = {}^t ( v_{1} , v_{2} , \ldots ,v_{n} ) , \quad \bm{w} = {}^t (w_{1} , w_{2} , \ldots , w_{n} )
  \]
  とおくと,
  \[
    \bm{v}+\bm{w}={}^t (v_1+w_1,v_2+w_2,\ldots,v_n + w_n)
  \]
  となり,これに加えて
  \[
    ( v_1+w_1)+(v_2+w_2) + \dots + (v_n + w_n) =(v_1+v_2+\dots + v_n ) + (w_1+w_2+\dots + w_n )=0+0=0
  \]
  となるから,$\bm{v}+\bm{w} \in W_1$である.
  さらに,$a \in \mathbb{C}$をとると,
  \[
    a \bm{v}={}^t ( av_{1} ,av_{2} , \ldots , av_{n} )
  \]
  であり,
  \[
    av_1 + av_2+ \dots + av_n = a (v_1 + v_2 + \dots + v_n) = a \cdot 0 =0
  \]
  であるから,このとき$ a \bm{v} \in W_1$である.

  以上により,$W_1$は$\mathbb{C}^n$の線型部分空間をなす.
\end{tanswer}


\subsection*{p107--108:問1-(ロ)}
\addcontentsline{toc}{subsection}{\texorpdfstring{p107--108:問1-(ロ)}{p107--108:問1-(ロ)}}

\begin{tanswer}
  この集合を$W_2$とおくと,$W_2$は$\mathbb{C}^n$の部分空間をなす.
  \[
    \bm{x} = \bm{0} \in W_2
  \]
  であるから,$W_2 \ne \varnothing$である.

  また,
  \[
    \bm{v} ={}^t (v_{p+1} , v_{p+2} , \ldots , v_{n} ) , \quad \bm{w} ={}^t ( w_{p+1} , w_{p+2} , \ldots , w_{n} )
  \]
  とおくと,
  \[
    \bm{v}+ \bm{w} = {}^t (v_{p+1}+w_{p+1},v_{p+2}+w_{p+2},\ldots,v_n + w_n)
  \]
  であり,
  \begin{align*}
      & ( v_{p+1}+w_{p+1}) +(v_{p+2}+w_{p+2})+ \dots + (v_n+w_n) \\
    = & (v_{p+1}+v_{p+2}+\dots+v_n)+(w_{p+1}+w_{p+2}+\dots+w_n)  \\
    = & 0+0                                                      \\
    = & 0
  \end{align*}
  となるため,このとき$\bm{v}+\bm{w} \in W_2$である.

  また,
  \[
    a\bm{v} = {}^t (av_{p+1},av_{p+2},\ldots,av_n)
  \]
  であり,
  \[
    av_{p+1} + av_{p+2}+\dots + av_n =a (v_{p+1}+v_{p+2}+\dots+v_n) = a\cdot 0 =0
  \]
  であるため,このとき$a \bm{v} \in W_2$である.

  以上により,$W_2$は$\mathbb{C}^n$の線型部分空間をなす.
\end{tanswer}

\subsection*{p107--108:問1-(ハ)}
\addcontentsline{toc}{subsection}{\texorpdfstring{p107--108:問1-(ハ)}{p107--108:問1-(ハ)}}

\begin{tanswer}
  これは部分空間をなさない.
  \[
    \bm{v}= {}^t (1,0,0,\ldots , 0),\quad \bm{w}={}^t (0,1,0,\ldots ,0)
  \]
  とすると
  \[
    \bm{v} + \bm{w}={}^t (1,1,0,\ldots,0)
  \]
  となり,与えられた条件式に当てはめると
  \[
    1^2+1^2+0^2 +\dots + 0^2 =2 \ne 1
  \]
  であるから,この集合は$\mathbb{C}^n$の部分空間でない.
\end{tanswer}


\subsection*{p107--108:問1-(ニ)}
\addcontentsline{toc}{subsection}{\texorpdfstring{p107--108:問1-(ニ)}{p107--108:問1-(ニ)}}

\begin{tanswer}
  この集合を$W_3$とおくと,$W_3$は$\mathbb{C}^n$の部分空間をなす.

  $\bm{x}=\bm{0}$とすると,
  \[
    (\bm{a},\bm{x}) =0
  \]
  であるため,$W_3 \ne \varnothing$である.

  さて,$\bm{v}$,$\bm{w}$が条件を満たすとすると,内積の定義から
  \[
    (\bm{a},\bm{v}+\bm{w})=(\bm{a},\bm{v})+(\bm{a},\bm{w})=0
  \]
  である.また,$ c \in \mathbb{C}$とすると,
  \[
    (\bm{a},c\bm{v})=c(\bm{a},\bm{v})=0
  \]
  である.

  以上により,$W_3$は$\mathbb{C}^n$の線型部分空間をなす.
\end{tanswer}


\section*{p107--108:問2}
\addcontentsline{toc}{section}{\texorpdfstring{p107--108:問2}{p107--108:問2}}


\subsection*{p107--108:問2-(イ)}
\addcontentsline{toc}{subsection}{\texorpdfstring{p107--108:問2-(イ)}{p107--108:問2-(イ)}}

\begin{tanswer}この集合を$W_1$とおくと,$W_1$は$\mathbb{K}^n$の線型部分空間とならない.

  たとえば
  \[
    A = \begin{pmatrix} 1 & 0 \\ 0 & 0 \end{pmatrix},\quad B = \begin{pmatrix} 0 & 0 \\ 0 & 1 \end{pmatrix}
  \]
  とおくと,$ A , B \in W_1$であるが,
  \[
    A + B = \begin{pmatrix} 1 & 0 \\ 0 & 1 \end{pmatrix}
  \]
  となり,$A+B$は正則行列である.よって$W_1$は$\mathbb{K}^n$の線型部分空間とならない.
\end{tanswer}


\subsection*{p107--108:問2-(ロ)}
\addcontentsline{toc}{subsection}{\texorpdfstring{p107--108:問2-(ロ)}{p107--108:問2-(ロ)}}

\begin{tanswer}
  この集合を$W_2$とおくと,$W_2$は$\mathbb{K}^n$の線型部分空間となる.

  $X =O$としたとき,$A O = OB$が成り立つのは明らかなので,$ W_2 \ne \varnothing$である.

  また, $X,Y \in W_2$とすると,
  \[
    A(X+Y)=(X+Y)B
  \]
  が成立し,さらに$ a\in \mathbb{K}$とすると,
  \[
    A(aX)=(aX)B
  \]
  が成立する.

  以上により, $W_2$は$\mathbb{K}^n$の線型部分空間である.
\end{tanswer}

\subsection*{p107--108:問2-(ハ)}
\addcontentsline{toc}{subsection}{\texorpdfstring{p107--108:問2-(ハ)}{p107--108:問2-(ハ)}}

\begin{tanswer}
  この集合を$W_3$とおくと,これは$\mathbb{K}^n$の線型部分空間とならない.

  たとえば
  \[
    A = \begin{pmatrix} -1 & 1 \\ -1 & 1 \end{pmatrix} , \quad B = \begin{pmatrix} 0 & 1 \\ 0 & 0 \end{pmatrix}
  \]
  とおくと,
  \[
    A^2 = O , \quad B^2 =O
  \]
  となり,$A ,B \in W_3$であるが,

  \[
    A+B = \begin{pmatrix} -1 & 2\\-1 & 1\end{pmatrix}
  \]
  となり,これは冪零行列とならない.よって$W_3$は$\mathbb{K}^n$の線型部分空間とならない.
\end{tanswer}

\subsection*{p107--108:問2-(ニ)}
\addcontentsline{toc}{subsection}{\texorpdfstring{p107--108:問2-(ニ)}{p107--108:問2-(ニ)}}

\begin{tanswer}
  この集合を$W_4$とおくと,これは$\mathbb{K}^n$の線型部分空間とならない.

  たとえば,
  \[
    A= \begin{pmatrix} 1 & 0 \\ 0 & 1 \\ \end{pmatrix}
  \]
  とおき,$ 1/2 \in \mathbb{K}$をとると,
  \[
    \frac{1}{2} A = \begin{pmatrix} 1/2 & 0 \\ 0 & 1/2 \end{pmatrix}
  \]
  となり,これは$W_4$の元ではない.よって$W_4$は$\mathbb{K}^n$の線型部分空間とならない.
\end{tanswer}


\section*{p122:問}
\addcontentsline{toc}{section}{\texorpdfstring{p122:問}{p122:問}}

\begin{tanswer}
  まず,
  \[
    \bm{a}_1 = \begin{pmatrix} 1 \\ -1 \\ 0 \end{pmatrix},\quad \bm{a}_2 =\begin{pmatrix} 1 \\ 0 \\ -1 \end{pmatrix},\quad \bm{a}_3 = \begin{pmatrix} 1 \\ 2 \\ 3 \end{pmatrix}
  \]
  とおく.正規直交基底のひとつを$\bm{e}_1$とすると,$\norm{\bm{a}_1}=\sqrt{2}$により,
  \[
    \bm{e}_1 = \frac{1}{\norm{\bm{a}_1}} \bm{a}_1 = \frac{1}{\sqrt{2}} \begin{pmatrix} 1 \\ -1 \\ 0 \end{pmatrix}
  \]
  となる.また,
  \[
    \bm{a}_2 ' = \bm{a}_2 - (\bm{a}_2,\bm{e}_1)\bm{e}_1
  \]
  とすると,
  \[
    \bm{a}_2 ' = \begin{pmatrix} 1 \\ 0 \\ -1 \end{pmatrix}-\left \{ \begin{pmatrix} 1 \\ 0 \\ -1 \end{pmatrix} \cdot \frac{1}{\sqrt{2}} \begin{pmatrix} 1 \\ -1 \\ 0 \end{pmatrix} \right \} \frac{1}{\sqrt{2}} \begin{pmatrix} 1 \\ -1 \\ 0 \end{pmatrix} =\frac{1}{2} \begin{pmatrix} 1 \\ 1\\ -2 \end{pmatrix}
  \]
  である.これを用いると,
  \[
    \bm{e}_2 = \frac{1}{\norm{\bm{a}_2'}} \bm{a}_2 ' =\frac{1}{\sqrt{6}/2} \cdot \frac{1}{2} \begin{pmatrix} 1 \\ 1\\ -2 \end{pmatrix} = \frac{1}{\sqrt{6}} \begin{pmatrix} 1 \\ 1\\ -2 \end{pmatrix}
  \]
  となる.また,
  \[
    \bm{a}_3 ' = \bm{a}_3 - (\bm{a}_3,\bm{e}_1)\bm{e}_1 -(\bm{a}_3,\bm{e}_2)\bm{e}_2
  \]
  とすると,
  \[
    \bm{a}_3'  = \begin{pmatrix} 1 \\ 2 \\ 3 \end{pmatrix} - \left \{ \begin{pmatrix} 1 \\ 2\\ 3 \end{pmatrix} \cdot \frac{1}{\sqrt{2}} \begin{pmatrix} 1 \\ -1 \\ 0 \end{pmatrix} \right \} \frac{1}{\sqrt{2}} \begin{pmatrix} 1 \\ -1 \\ 0 \end{pmatrix}- \left \{ \begin{pmatrix} 1 \\ 2 \\ 3 \end{pmatrix} \cdot  \frac{1}{\sqrt{6}} \begin{pmatrix} 1 \\ 1\\ -2 \end{pmatrix} \right \}  \frac{1}{\sqrt{6}} \begin{pmatrix} 1 \\ 1\\ -2 \end{pmatrix} =\begin{pmatrix} 2 \\ 2 \\ 2 \end{pmatrix}
  \]
  となり,
  \[
    \bm{e}_3 = \frac{1}{\norm{\bm{a}_3 ' }} \bm{a}_3 ' = \frac{1}{\sqrt{3}} \begin{pmatrix} 1 \\ 1 \\ 1 \end{pmatrix}
  \]
  となる.

  以上の考察により,求める正規直交基底は
  \[
    \langle \frac{1}{\sqrt{2}} \begin{pmatrix} 1 \\ -1 \\ 0 \end{pmatrix} , \frac{1}{\sqrt{6}} \begin{pmatrix} 1 \\ 1\\ -2 \end{pmatrix} ,  \frac{1}{\sqrt{3}} \begin{pmatrix} 1 \\ 1 \\ 1 \end{pmatrix} \rangle
  \]
  である.
\end{tanswer}



\section*{p124:問-1)}
\addcontentsline{toc}{section}{\texorpdfstring{p124:問-1)}{p124:問-1)}}

\begin{tproof}
  任意に $\bm{x} \in W$ をとる.$W^{\perp}$ は「$W$ の任意のベクトルと直交するベクトル全体の集合」であるから,$\bm{x} \in W$ に対しては,任意の $\bm{y} \in W^{\perp}$ において $(\bm{x}, \bm{y})=0$ が成り立つ.

  ゆえに,任意の $\bm{x} \in W$ は $W^{\perp}$ の元全てと直交することになり,したがって $\bm{x} \in (W^{\perp})^{\perp}$ が従う.このことから
  \[
    W \subset (W^{\perp})^{\perp}
  \]
  が得られる.
  また,定理[4.7]から
  \begin{align*}
     & \dim W +\dim W^{\perp} =\dim (W +W^{\perp} )+\dim (W \cap W^{\perp} ), \\
     & \dim W +\dim W^{\perp} =n.
  \end{align*}
  ここで定理[6.4]から$\mathbb{R}^n$の計量空間$V$は$W\dot{+}W^{\perp}$と表されること,[4.8]から,この直和の共通部分は$\{ \bm{o} \}$のみであることを用いた.
\end{tproof}



\part*{第4章・章末問題}
\addcontentsline{toc}{part}{\texorpdfstring{第4章・章末問題}{第4章・章末問題}}


\section*{p127--130:1}
\addcontentsline{toc}{section}{\texorpdfstring{p127--130:1}{p127--130:1}}

\begin{tanswer}
  $s,t,u,v \in \mathbb{R}$とし.
  \[
    s\bm{a}_1+t\bm{a}_2=u\bm{a}_3+v\bm{a}_4
  \]
  とおく.これにより,
  \begin{align*}
     &
    \begin{pmatrix}
      1 & -1 & 0  & 1 \\
      2 & 1  & -1 & 9 \\
      0 & 3  & 5  & 1 \\
      4 & -3 & 2  & 4
    \end{pmatrix}
    \begin{pmatrix}
      s \\
      t \\
      u \\
      v
    \end{pmatrix}
    =\bm{o} , \\
     &
    \begin{pmatrix}
      1 & 0 & 0 & 3  \\
      0 & 1 & 0 & 2  \\
      0 & 0 & 1 & -1 \\
      0 & 0 & 0 & 0
    \end{pmatrix}
    \begin{pmatrix}
      s \\
      t \\
      u \\
      v
    \end{pmatrix}
    =\bm{o} , \\
     &
    \begin{pmatrix}
      s \\
      t \\
      u \\
      v
    \end{pmatrix}
    =a
    \begin{pmatrix}
      -3 \\
      -2 \\
      1  \\
      1
    \end{pmatrix}
    . \quad \text{($a$は任意の定数)}
  \end{align*}
  とかけるので,$W_1 \cap W_2$の次元は$1$であり,その基底は
  \[
    s \bm{a}_1+ t \bm{a}_2 =-a \begin{pmatrix} 1 \\ 8 \\ 6 \\ 6 \end{pmatrix}
  \]
  により,
  \[
    \langle
    \begin{pmatrix}
      1 \\
      8 \\
      6 \\
      6
    \end{pmatrix}
    \rangle
  \]
  である.
\end{tanswer}



\section*{p127--130:2}
\addcontentsline{toc}{section}{\texorpdfstring{p127--130:2}{p127--130:2}}

\begin{tanswer}
  $ W_1$に関して,$x_3 =s$,$x_4 =t$とおくと,
  \[
    \begin{pmatrix} x_1 \\ x_2 \\ x_3 \\ x_4 \end{pmatrix} = s \begin{pmatrix} 1 \\ -1 \\ 1 \\ 0 \end{pmatrix} + t \begin{pmatrix} -9 \\ 3 \\ 0 \\ 1 \end{pmatrix}
  \]
  とかけるため,$\dim W_1 = 2$であり,その基底は
  \[
    \langle  \begin{pmatrix} 1 \\ -1 \\ 1 \\ 0 \end{pmatrix} , \begin{pmatrix} -9 \\ 3 \\ 0 \\ 1 \end{pmatrix} \rangle
  \]
  である.$W_2$に関しても同様にして,$\dim W_2 = 2$であり,その基底は
  \[
    \langle \begin{pmatrix} 0 \\ -1 \\ 1 \\ 0 \end{pmatrix} , \begin{pmatrix} -1 \\ 3 \\ 0 \\ 1 \end{pmatrix} \rangle
  \]
  である.したがって$W_1 + W_2$は
  \[
    \begin{pmatrix} 1 \\ -1 \\ 1 \\ 0 \end{pmatrix} ,\quad  \begin{pmatrix} -9 \\ 3 \\ 0 \\ 1 \end{pmatrix},\quad \begin{pmatrix} 0 \\ -1 \\ 1 \\ 0 \end{pmatrix} ,\quad  \begin{pmatrix} -1 \\ 3 \\ 0 \\ 1 \end{pmatrix}
  \]
  によって生成される.

  ここで,
  \[
    x \begin{pmatrix} 1 \\ -1 \\ 1 \\ 0 \end{pmatrix} + y \begin{pmatrix} -9 \\ 3 \\ 0 \\ 1 \end{pmatrix} + z \begin{pmatrix} 0 \\ -1 \\ 1 \\ 0 \end{pmatrix} + w \begin{pmatrix} -1 \\ 3 \\ 0 \\ 1 \end{pmatrix} =\bm{o}
  \]
  とすると,
  \[
    \begin{pmatrix} 1 & -9 & 0 & -1 \\ -1 & 3 & -1 & 3 \\ 1 & 0 & 1 & 0 \\ 0 & 1 & 0 & 1 \end{pmatrix} \begin{pmatrix} x \\ y \\ z \\ w \end{pmatrix} = \bm{o}
  \]
  となり,これに基本変形を施すと,
  \[
    \begin{pmatrix} 1 & 0 & 0 & 8 \\ 0 & 1 & 0 & -8 \\0 & 0 & 1 & 1 \\ 0 & 0 & 0 & 0 \end{pmatrix} \begin{pmatrix} x \\ z \\ y \\ w \end{pmatrix} = \bm{o}
  \]
  となる.
  したがって,$W_1 + W_2$の次元は$3$であり,その基底は
  \[
    \langle \begin{pmatrix} 1 \\ -1 \\ 1 \\ 0 \end{pmatrix} , \begin{pmatrix} -9 \\ 3 \\ 0 \\ 1 \end{pmatrix} , \begin{pmatrix} 0 \\ -1 \\ 1 \\ 0 \end{pmatrix} \rangle
  \]
  である.
\end{tanswer}



\section*{p127--130:5}
\addcontentsline{toc}{section}{\texorpdfstring{p127--130:5}{p127--130:5}}

\begin{tanswer}
  $A$,$B$の定める線型写像をそれぞれ$T_A$,$T_B$とする.$ \bm{x} \in \Im (T_A + T_B)$を任意にとると,
  ある$\bm{y} \in \mathbb{R}^n$が存在して,
  \begin{alignat*}{2}
    \bm{x} & = (T_A+T_B)(\bm{y}) \quad           &  &                                       \\
           & = T_A (\bm{y}) + T_B (\bm{y}) \quad &  & \text{($\because$~$T_A$と$T_B$は線型写像)}.
  \end{alignat*}
  よって,
  \begin{equation}
    \Im (T_A + T_B) \subset \Im T_A + \Im T_B \quad \tag{$\ast$}
    \label{eq:p127--130:5_ast}
  \end{equation}
  これにより,
  \begin{alignat*}{2}
    \rank (T_A + T_B) & = \dim (\Im (T_A + T_B))\quad                  &  & \text{($\because$~階数の定義)}                       \\
                      & \leqq \dim (\Im (T_A) + \Im (T_B))\quad        &  & \text{($\because$ ~\eqref{eq:p127--130:5_ast})} \\
                      & \leqq \dim (\Im (T_A)) + \dim (\Im (T_B))\quad &  & \text{($\because$~定理[4.7])}                     \\
                      & = \rank (T_A)+\rank( T_B).\quad                &  & \text{($\because$~階数の定義)}
  \end{alignat*}
  これを書き換えると
  \[
    \rank (A+B) \leqq \rank (A) + \rank (B).
  \]
  これが証明すべきことであった.
\end{tanswer}



\section*{p127--130:6}
\addcontentsline{toc}{section}{\texorpdfstring{p127--130:6}{p127--130:6}}


\subsection*{p127--130:6-(イ)}
\addcontentsline{toc}{subsection}{\texorpdfstring{p127--130:6-(イ)}{p127--130:6-(イ)}}

\begin{tproof}
  $BA$は$n$次の正方行列である.ここで,
  \begin{align*}
    \rank (BA) & \leqq \min \{ \rank B, \rank A \} \\
               & =  m <n
  \end{align*}
  であるから,$\rank (BA)<n$である.よって$BA$は正則行列でない.
\end{tproof}


\subsection*{p127--130:6-(ロ)}
\addcontentsline{toc}{subsection}{\texorpdfstring{p127--130:6-(ロ)}{p127--130:6-(ロ)}}

\begin{tproof}
  $AB$ が正則であるとする.$m = \rank AB \leq \min\{\rank A,\rank B \}$ であるから,
  $m \leq \rank A$ かつ $m \leq \rank B$ である.一方 $\rank A \leq m,\ \rank B \leq m$ でもあるから
  $m = \rank A = \rank B$ である.

  行列 $X$ の定める線形写像を $T_X$ と書くことにする
  ($T_A \colon  \mathbb{C}^n \to \mathbb{C}^m$,$ T_B \colon  \mathbb{C}^m \to \mathbb{C}^n$ である).

  \begin{align*}
    \rank AB & = \dim(T_{AB}(\mathbb{C}^m))           \\
             & = \dim((T_A \circ T_B)(\mathbb{C}^m)).
  \end{align*}

  $W = T_B(\mathbb{C}^m)$ とおく.$T_A$ の定義域を $W$ に制限した写像 $T_A \upharpoonright W : W \to \mathbb{C}^n$ について次元定理を適用すると,

  \[
    \dim W = \dim((T_A \upharpoonright W)(W)) + \dim((T_A \upharpoonright W)^{-1}(\{o_m\})).
  \]

  ここで,

  \begin{align*}
    (T_A \upharpoonright W)^{-1}(\{o_m\}) & = \{ x \in W : (T_A \upharpoonright W)(x) = o_m \} \\
                                          & = \{ x \in W : T_A(x) = o_m \}                     \\
                                          & = \{ x \in \mathbb{C}^n : T_A(x) = o_m \} \cap W   \\
                                          & = {T_A}^{-1}(\{o_m\}) \cap W.
  \end{align*}

  であるから,

  \[
    m = \dim((T_A \circ T_B)(\mathbb{C}^m)) + \dim({T_A}^{-1}(\{0_m\}) \cap W).
  \]

  となる.よって,

  \[
    \rank AB = m - \dim({T_A}^{-1}(\{o_m\}) \cap W).
  \]

  だから $\dim({T_A}^{-1}(\{o_m\}) \cap W) = 0$ である.以上より,

  \begin{enumerate}[(I)]
    \item $m = \rank A = \rank B.$ \label{enu:p127--130:6-I}
    \item $\dim({T_A}^{-1}(\{o_m\}) \cap T_B(\mathbb{C}^m)) = 0.$\label{enu:p127--130:6-II}
  \end{enumerate}

  は必要条件である.一方,\ref{enu:p127--130:6-I}かつ \ref{enu:p127--130:6-II} を仮定すると,$\rank AB = m$ であるから $AB$ は正則でもある.

  よって\ref{enu:p127--130:6-I}かつ\ref{enu:p127--130:6-II}が必要十分条件である.
\end{tproof}


\section*{p127--130:7}
\addcontentsline{toc}{section}{\texorpdfstring{p127--130:7}{p127--130:7}}
\begin{tproof}
  $M_n(\mathbb{K})$ の基底 $\langle \bm{e}_{11}, \bm{e}_{12}, \dots, \bm{e}_{1n}, \bm{e}_{21}, \dots, \bm{e}_{nn} \rangle$ を,
  $\bm{e}_{ij}$の$(i,j)$成分が$1$で,その他の成分は$0$であるものとして定義する.

  $X = (x_{ji}) \in M_n(\mathbb{K})$ を取り,
  $A = (a_{ij})$を$a_{ij} = T \bm{e}_{ji}$であるものとすれば,
  \begin{align*}
    \tr (AX) & = \tr \left(
    \begin{pmatrix}
        T \bm{e}_{11} & T \bm{e}_{12} & \dots  & T \bm{e}_{1n} \\
        T \bm{e}_{21} & T \bm{e}_{22} & \dots  & T \bm{e}_{2n} \\
        \vdots        & \vdots        & \ddots & \vdots        \\
        T \bm{e}_{n1} & T \bm{e}_{n2} & \dots  & T \bm{e}_{nn}
      \end{pmatrix}
    \begin{pmatrix}
        x_{11} & x_{21} & \dots  & x_{n1} \\
        x_{12} & x_{22} & \dots  & x_{n2} \\
        \vdots & \vdots & \ddots & \vdots \\
        x_{1n} & x_{2n} & \dots  & x_{nn}
      \end{pmatrix}
    \right)                                                                  \\
             & = \tr \left(
    \begin{pmatrix}
        x_{11}T \bm{e}_{11} + x_{12}T \bm{e}_{12} + \cdots + x_{1n}T \bm{e}_{1n} &                                                             \\
                                                                                 & \ddots &                                                    \\
                                                                                 &        & x_{n1}T \bm{e}_{n1} + \cdots + x_{nn}T \bm{e}_{nn}
      \end{pmatrix}
    \right)                                                                  \\
             & = \sum_{1 \leq i,j \leq n} x_{ij} T \bm{e}_{i,j}              \\
             & = T \left(\sum_{1 \leq i,j \leq n } x_{ij} \bm{e}_{ij}\right) \\
             & = T(X)
  \end{align*}
  となり,上記のように$A$をとればよい.
\end{tproof}



\section*{p127--130:8}
\addcontentsline{toc}{section}{\texorpdfstring{p127--130:8}{p127--130:8}}

\begin{tanswer}
  例7をふまえ,
  \[
    (f, g) = \int_{-\pi}^{\pi} f(x) g(x) \, dx
  \]
  とする.このとき,
  \begin{align*}
    \norm{f-g}^2 & = \int_{-\pi}^{\pi} \abs{f(x) - g(x)}^2 \, dx                                                                       \\
                 & = \int_{-\pi}^{\pi} (f(x)-g(x))^2 \, dx                                                                             \\
                 & = \int_{-\pi}^{\pi} \abs{f(x)}^2 \,dx - \int_{-\pi}^{\pi} 2 f(x) g(x) \, dx + \int_{-\pi}^{\pi} \abs{g(x)}^2 \,  dx
  \end{align*}
  である.さらに,第2項と第3項に関連して,
  \begin{align*}
     & \int_{-\pi}^{\pi} f(x) g(x)  \, dx = a_0 \int_{-\pi}^{\pi} f(x) \,  dx + \sum_{k=1}^{n} \left( a_k \int_{-\pi}^{\pi} f(x) \cos kx \, dx + b_k \int_{-\pi}^{\pi} f(x) \sin kx \, dx \right), \\
     & \int_{-\pi}^{\pi} \abs{g(x)}^2 \,dx = 2\pi {a_0}^2 + \pi \sum_{k=1}^{n} ({a_k}^2 + {b_k}^2).
  \end{align*}
  これらを用いると,
  \begin{align*}
    \int_{-\pi}^{\pi} \abs{f(x) - g(x)}^2 \, dx
     & = 2\pi {a_0}^2 + \pi \sum_{k=1}^{n} ({a_k}^2 + {b_k}^2)                                                     \\
     & \quad - 2 \left( a_0 \int_{-\pi}^{\pi} f(x) \, dx
    + \sum_{k=1}^{n} \left( a_k \int_{-\pi}^{\pi} f(x) \cos kx \, dx
    + b_k \int_{-\pi}^{\pi} f(x) \sin kx \, dx \right) \right)                                                     \\
     & \quad + \int_{-\pi}^{\pi} \abs{f(x)}^2 \, dx                                                                \\
     & = 2\pi \left( a_0 - \frac{1}{2\pi} \int_{-\pi}^{\pi} f(x) \, dx \right)^2
    + \pi \sum_{k=1}^{n} \left\{ \left( a_k - \frac{1}{\pi} \int_{-\pi}^{\pi} f(x) \cos kx \, dx \right)^2 \right. \\
     & \quad + \left. \left( b_k - \frac{1}{\pi} \int_{-\pi}^{\pi} f(x) \sin kx \, dx \right)^2 \right\} + R.
  \end{align*}

  ただし,
  \[
    R= \int_{-\pi}^{\pi} f(x)^2 \, dx - 2\pi \left( \frac{1}{2\pi} \int_{-\pi}^{\pi} f(x) \, dx \right)^2 - \pi \sum_{k=1}^{n} \left( \left( \frac{1}{\pi} \int_{-\pi}^{\pi} f(x) \cos kx \, dx \right)^2 + \left( \frac{1}{\pi} \int_{-\pi}^{\pi} f(x) \sin kx \, dx \right)^2 \right).
  \]
  ï
  したがって,$\norm{f - g}^2$ を最小にする $g(x)$ は,

  \[
    g(x) = \frac{1}{2\pi} \int_{-\pi}^{\pi} f(x) \, dx + \sum_{k=1}^{n} \left(\frac{1}{\pi}  \left(\int_{-\pi}^{\pi} f(x) \cos kx \, dx \right) \cos kx + \frac{1}{\pi} \left(\int_{-\pi}^{\pi} f(x) \sin kx\,  dx \right) \sin kx \right).
  \]
\end{tanswer}




\section*{p127--130:12}
\addcontentsline{toc}{section}{\texorpdfstring{p127--130:12}{p127--130:12}}


\subsection*{p127--130:12-(イ)}
\addcontentsline{toc}{subsection}{\texorpdfstring{p127--130:12-(イ)}{p127--130:12-(イ)}}


\begin{tproof}
  ペクトル空間 $V$ に対して,$V$ の線型汎函数全体の集合を $V^*$ とする.

  $V$ の基底 $E= \langle \bm{e}_1, \bm{e}_2, \dots, \bm{e}_n\rangle $ に対して,
  $V^*$の元$\bm{f}_i$を$\bm{f}_i(\bm{e}_j)=\delta_{ij}$とする.
  $E^* = \langle \bm{f}_1, \bm{f}_2, \dots, \bm{f}_n\rangle$ は $V^*$ の基底である.

  任意の $\bm{f}_i  \in V^*$ が線型結合で表されることを示す.
  \[
    (c_1 \bm{f}_1 + \dots + c_n \bm{f}_n) (\bm{x}) = \bm{0}
  \]
  とする.
  ここで $\bm{x} = \bm{e}_i$ ~($1 \leqq i \leqq n$)を代入すると,
  $\bm{f}_i (\bm{e}_j) = \delta_{ij}$となり,$c_i =0$と併せると
  \[
    c_1=c_2 = \dots =c_n = 0
  \]
  となり線型独立である.

  次に,$V^\ast$ の元が $\bm{f}_1, \bm{f}_2, \dots, \bm{f}_n$ の線型結合で表されることを示す.

  $V^\ast$の元$\bm{f}$ が $V$ の元$\bm{x}=x_1 \bm{e}_1 + \dots + x_n \bm{e}_n$に対して
  $\bm{f}(\bm{e}_j) = a_i$~ ($1 \leqq i \leqq n$)とすると,
  \begin{alignat*}{2}
    \bm{f} (\bm{x}) & = \sum_{i=1}^{n} x_i \bm{f}(\bm{e}_i)                                            & \quad & (\text{$\because$~$f$の線型性})                           \\
                    & = \sum_{i=1}^{n} a_i x_i                                                         &       &                                                       \\
                    & = \sum_{i=1}^{n} a_i \bm{f}_i (x_1 \bm{e}_1 + x_2 \bm{e}_2 +\dots+ x_n \bm{e}_n) & \quad & (\text{$\because~ \bm{f}_i (\bm{e}_j)=\delta_{i,j}$}) \\
                    & = \left( \sum_{i=1}^{n} a_i \bm{f}_i \right) (\bm{x})                            &       &
  \end{alignat*}
  と$\bm{f}_1 , \bm{f}_2 , \dots ,\bm{f}_n$の線型結合として表される.

  以上により,$E^\ast$は$V^\ast$の基底である.
\end{tproof}


\subsection*{p127--130:12-(ロ)}
\addcontentsline{toc}{subsection}{\texorpdfstring{p127--130:12-(ロ)}{p127--130:12-(ロ)}}

\begin{tproof}
  $ W^\ast$の元$\bm{f}= c_1 \bm{f}_1' + \dots + c_n \bm{f}_n'$をとる.
  $V$の元$\bm{x} =\sum_{k=1}^{n} x_k \bm{e}_k$に対して
  \begin{alignat*}{2}
    (T^\ast \bm{f}') (\bm{x}) & = \bm{f} \circ T(\bm{x})                                                                                                & \quad &                                              \\
                              & = \sum_{k=1}^{m} c_k \bm{f}_k ' \circ T \left( \sum_{l=1}^{n} x_l \bm{e}_l \right)                                      & \quad & (\text{$\because$~$\bm{f}'$の線型性})            \\
                              & = \sum_{k=1}^{m} c_k \bm{f}_k ' \left( \sum_{l=1}^{n} x_l T (\bm{e}_l) \right)                                          & \quad & (\text{$\because$~$T$の線型性})                  \\
                              & = \sum_{k=1}^{m} c_k \sum_{l=1}^{n} x_l \bm{f}_k ' (a_{1l} \bm{e}_1 ' + a_{2l} \bm{e}_2 ' + \dots + a_{nl} \bm{e}_n ' ) & \quad &                                              \\
                              & = \sum_{k=1}^{m} c_k \sum_{l=1}^{n} x_l a_{kl}                                                                          & \quad & (\text{$\because$~双対基底の定義と$\bm{f}_k '$の線型性}) \\
                              & = \sum_{k=1}^{m} c_K \sum_{l=1}^{n} c_{kl} \bm{f}_l ( x_1 \bm{e}_1 + \dots + x_n \bm{e}_n)                              &       & (\text{$\because$~双対基底の定義と$\bm{f}_l '$の線型性}) \\
                              & = \left( \sum_{l=1}^{n} \left( \sum_{k=1}^{m} c_k a_{kl} \right) \bm{f}_l\right) (\bm{x})
  \end{alignat*}
  より,基底$E^\ast$,$F^\ast$に関する$T^\ast$の表現行列$B= (b_{ij})$は
  \[
    \begin{pmatrix} \sum_{k=1}^{m} c_k a_{k1} \\ \vdots \\ \sum_{k=1}^{m} c_k a_{kn} \end{pmatrix}
    =
    \begin{pmatrix} l_{11} & l_{12} & \dots & l_{1m} \\ \vdots & & & \vdots \\ l_{n1} & l_{n2} & \dots & l_{nm} \end{pmatrix}
    \begin{pmatrix} c_1 \\ \vdots \\ c_n \end{pmatrix}
  \]
  より,$b_{ij}=a_{ji}$となり,$B= {}^t A$である.
\end{tproof}



\section*{p127--130:9}
\addcontentsline{toc}{section}{\texorpdfstring{p127--130:9}{p127--130:9}}
\begin{tproof}
  この写像を$\varphi$とする.
  まず,$\varphi$が線型写像であることを示す.

  $\bm{x} , \bm{y} \in V$ と $c \in \mathbb{R}$ に対して,$ \forall \bm{f} \in V^\ast$で

  \begin{align*}
    ( \varphi(\bm{x}+\bm{y})) (\bm{f}) & = \bm{f} (\bm{x}+\bm{y})                                  \\
                                       & =\bm{f} (\bm{x}) + \bm{f} (\bm{y})                        \\
                                       & = (\varphi(\bm{x})) (\bm{f}) + (\varphi(\bm{y})) (\bm{f}) \\
                                       & = ( \varphi(\bm{x}) + \varphi(\bm{y})) (\bm{f})
  \end{align*}
  \begin{align*}
    ( \varphi(c \bm{x})) (\bm{f}) & = \bm{f} (c \bm{x})            \\
                                  & = c \bm{f} (\bm{x})            \\
                                  & = c (\varphi(\bm{x})) (\bm{f}) \\
                                  & = (c \varphi(\bm{x})) (\bm{f})
  \end{align*}
  であるから,$\varphi$は線型写像である.

  次に,$V$の基底$E= \langle \bm{e}_1 , \bm{e}_2, \dots , \bm{e}_n \rangle$に対して,
  $\bm{e}_i ' = \varphi ( \bm{e}_i)$ ~(ただし$1 \leqq i \leqq n$)としたとき,
  $(E^\ast )^\ast= \langle \bm{e}_1 ' , \bm{e}_2 ' , \dots , \bm{e}_n ' \rangle$が$(V^\ast)^\ast$の基底であることを示す.

  $ E^\ast = \langle \bm{f}_1 , \bm{f}_2 , \dots , \bm{f}_n \rangle$を$E$の双対基底とする.
  $ c_1 \bm{e}_1 ' + c_2 \bm{e}_2 ' + \dots + c_n \bm{e}_n ' = \bm{0}$となるとき,
  $\varphi$は線型写像で$\varphi (c_1 \bm{e}_1 + c_2 \bm{e}_2 + \dots + c_n \bm{e}_n) = c_1 \bm{e}_1 ' + c_2 \bm{e}_2 ' + \dots + c_n \bm{e}_n '$であるので,
  \begin{align*}
    (c_1 \bm{e}_1 ' + c_2 \bm{e}_2 ' + \dots + c_n \bm{e}_n ') (\bm{f}_i) & = \bm{f}_i (c_1 \bm{e}_1 + c_2 \bm{e}_2 + \dots + c_n \bm{e}_n) \\
                                                                          & = \sum_{k=1}^{n} c_k \bm{f}_i (\bm{e}_k)                        \\
                                                                          & = \sum_{k=1}^{n} c_k \delta_{ik}                                \\
                                                                          & = c_i=0
  \end{align*}
  となり,$c_1 = c_2 = \dots = c_n = 0$であるから,
  $e_1 ' , e_2 ' , \dots , e_n '$は線型独立であり,
  $\dim (V^\ast)^\ast = n $より$(E^\ast)^\ast$は基底である.
  とくに$\varphi$の階数は$n$となる.適当な基底での$\varphi$の表現行列$A$に対してp.117の(3)により,$r(A)=n$となり,
  これは$\varphi$が全単射対応を与えることを示す.
\end{tproof}





\section*{p127--130:10}
\addcontentsline{toc}{section}{\texorpdfstring{p127--130:10}{p127--130:10}}


\subsection*{p127--130:10-(イ)}
\addcontentsline{toc}{subsection}{\texorpdfstring{p127--130:10-(イ)}{p127--130:10-(イ)}}

\begin{tproof}
  (1),(2)で双線型性,(3)で対称性,(4)で正値性を証明する.
  \begin{enumerate}[(1)]
    \item
          \begin{align*}
            (f,g_1+g_2)_p & = \int_{a}^{b} p(x) f(x) \{ g_1(x)+g_2(x) \} \, dx                           \\
                          & = \int_{a}^{b} p(x) f(x) g_1 (x) \, dx + \int_{a}^{b} p(x) f(x) g_2(x) \, dx \\
                          & = (f,g_1)_p + (f,g_2)_p.
          \end{align*}
          また,
          \begin{align*} (f_1 + f_2 , g)_p & = \int_{a}^{b} p(x) \{ f_1(x)+f_2(x) \} g(x) \, dx                            \\
                                 & = \int_{a}^{b} p(x) f_1 (x) g(x) \, dx + \int_{a}^{b} p(x) f_2 (x) g(x) \, dx \\
                                 & = (f_1,g)_p + (f_2,g)_p.
          \end{align*}
    \item $c$は任意の実数とする.
          \begin{align*}
            (cf,g)_p & = \int_{a}^{b} p(x) \{ cf(x) \} g(x) \, dx \\
                     & = c \int_{a}^{b} p(x) f(x) g(x) \, dx      \\
                     & = c(f,g)_p .
          \end{align*}
          また,
          \begin{align*}
            (f,cg)_p & = \int_{a}^{b} p(x) f(x) \{ cg(x) \} \, dx \\
                     & = c \int_{a}^{b} p(x) f(x) g(x) \, dx      \\
                     & = c(f,g)_p .
          \end{align*}
    \item
          \begin{align*}
            (f,g)_p & = \int_{a}^{b} p(x) f(x) g(x) \, dx \\
                    & = \int_{a}^{b} p(x) g(x) f(x) \, dx \\
                    & =(g,f)_p .
          \end{align*}
    \item
          \begin{alignat*}{2}
            (f,f)_p & = \int_{a}^{b} p(x) f(x) f(x) \, dx   &       &                                 \\
                    & =\int_{a}^{b} p(x) \{ f(x) \}^2 \, dx &       &                                 \\
                    & >0.                                   & \quad & \text{($\because$~ $p(x)$は常に正)}
          \end{alignat*}
          等号が成立するのは$f(x)=0$のとき.
  \end{enumerate}
  (1)から(4)の考察により,$(f,g)_p$は内積の定義をみたす.
\end{tproof}

