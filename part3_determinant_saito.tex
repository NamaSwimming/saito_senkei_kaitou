
\part*{第3章}
\addcontentsline{toc}{part}{\texorpdfstring{第3章}{第3章}}


\section*{p77:問1}
\addcontentsline{toc}{section}{\texorpdfstring{p77:問1}{p77:問1}}

\kakko{3文字の置換}

\begin{tanswer}
  3文字の置換は$3!=6$個ある.それを互換の数によって分類する.
  \begin{description}
    \item[0つ] 1個のみ.偶置換かつ恒等置換.
          \[
            \begin{pmatrix} 1 & 2 & 3 \\ 1 & 2 & 3 \end{pmatrix}.
          \]
    \item [1つ] 3個.奇置換.
          \[
            \begin{pmatrix} 1 & 2 & 3\\ 1 & 3 & 2 \end{pmatrix},\quad
            \begin{pmatrix} 1 & 2 & 3\\ 3 & 2 & 1 \end{pmatrix},\quad
            \begin{pmatrix} 1 & 2 & 3\\ 2 & 1 & 3 \end{pmatrix}.
          \]
    \item [2つ] 2個.偶置換.
          \[
            \begin{pmatrix} 1 & 2 & 3\\ 2 & 3 & 1 \end{pmatrix},\quad
            \begin{pmatrix} 1 & 2 & 3\\ 3 & 1 & 2 \end{pmatrix}.
          \]
  \end{description}
\end{tanswer}

\kakko{4文字の置換}

\begin{tanswer}
  4文字の置換は$4!=24$通りある.それを互換の数によって分類する.
  \begin{description}
    \item[0つ] 1 個.偶置換(恒等置換).
          \[
            \begin{pmatrix} 1 & 2 & 3 & 4 \\ 1 & 2 & 3 & 4 \end{pmatrix}.
          \]
    \item[1つ] 6 個.奇置換.
          \begin{align*}
             & \begin{pmatrix} 1 & 2 & 3 & 4 \\ 1 & 2 & 4 & 3 \end{pmatrix},\quad
            \begin{pmatrix} 1 & 2 & 3 & 4 \\ 1 & 3 & 2 & 4 \end{pmatrix},\quad
            \begin{pmatrix} 1 & 2 & 3 & 4 \\ 1 & 4 & 3 & 2 \end{pmatrix},         \\
             & \begin{pmatrix} 1 & 2 & 3 & 4 \\ 2 & 1 & 3 & 4 \end{pmatrix},\quad
            \begin{pmatrix} 1 & 2 & 3 & 4 \\ 3 & 2 & 1 & 4 \end{pmatrix},\quad
            \begin{pmatrix} 1 & 2 & 3 & 4 \\ 4 & 2 & 3 & 1 \end{pmatrix}.
          \end{align*}

    \item[2つ] 11 個.偶置換.
          \begin{align*}
             & \begin{pmatrix} 1 & 2 & 3 & 4 \\ 1 & 3 & 4 & 2 \end{pmatrix},\quad
            \begin{pmatrix} 1 & 2 & 3 & 4 \\ 1 & 4 & 2 & 3 \end{pmatrix},\quad
            \begin{pmatrix} 1 & 2 & 3 & 4 \\ 2 & 1 & 4 & 3 \end{pmatrix},         \\
             & \begin{pmatrix} 1 & 2 & 3 & 4 \\ 2 & 3 & 1 & 4 \end{pmatrix},\quad
            \begin{pmatrix} 1 & 2 & 3 & 4 \\ 2 & 4 & 3 & 1 \end{pmatrix},\quad
            \begin{pmatrix} 1 & 2 & 3 & 4 \\ 3 & 1 & 2 & 4 \end{pmatrix},         \\
             & \begin{pmatrix} 1 & 2 & 3 & 4 \\ 3 & 2 & 4 & 1 \end{pmatrix},\quad
            \begin{pmatrix} 1 & 2 & 3 & 4 \\ 3 & 4 & 1 & 2 \end{pmatrix},\quad
            \begin{pmatrix} 1 & 2 & 3 & 4 \\ 4 & 1 & 3 & 2 \end{pmatrix},         \\
             & \begin{pmatrix} 1 & 2 & 3 & 4 \\ 4 & 2 & 1 & 3 \end{pmatrix},\quad
            \begin{pmatrix} 1 & 2 & 3 & 4 \\ 4 & 3 & 2 & 1 \end{pmatrix}.
          \end{align*}

    \item[3つ] 6 個.奇置換
          \begin{align*}
             & \begin{pmatrix} 1 & 2 & 3 & 4 \\ 2 & 3 & 4 & 1 \end{pmatrix},\quad
            \begin{pmatrix} 1 & 2 & 3 & 4 \\ 2 & 4 & 1 & 3 \end{pmatrix},\quad
            \begin{pmatrix} 1 & 2 & 3 & 4 \\ 3 & 1 & 4 & 2 \end{pmatrix},         \\
             & \begin{pmatrix} 1 & 2 & 3 & 4 \\ 3 & 4 & 2 & 1 \end{pmatrix},\quad
            \begin{pmatrix} 1 & 2 & 3 & 4 \\ 4 & 1 & 2 & 3 \end{pmatrix},\quad
            \begin{pmatrix} 1 & 2 & 3 & 4 \\ 4 & 3 & 1 & 2 \end{pmatrix}.
          \end{align*}
  \end{description}
\end{tanswer}

\section*{p77:問2}
\addcontentsline{toc}{section}{\texorpdfstring{p77:問2}{p77:問2}}

\begin{tproof}
  $S_n$の偶置換全体の集合を$A_n$,偶置換全体の集合を$B_n$とする.
  置換は必ず奇置換か偶置換のいずれかであるから,
  \begin{align*}
     & S_n = A_n \cup B_n ,       \\
     & A_n \cap B_n = \varnothing
  \end{align*}
  となる.

  ここで,
  \[
    \tau = \begin{pmatrix} 1 & 2 & 3 \\ 2 & 1 & 3 \end{pmatrix}
  \]
  とすると,$\tau$は奇置換であり,$\sigma \in  A_n$のとき,$ \tau \sigma \in B_n$である.
  同様に,$ \rho  \in B_n$のとき,$\tau^{-1} \rho = \tau \rho \in A_n$である.
  これらにより,全単射
  \[
    A_n \ni \sigma \mapsto \tau \sigma \in B_n
  \]
  が存在し,偶置換と奇置換は同数あり,その個数は$n! /2$である.
\end{tproof}


\section*{p77:問3}
\addcontentsline{toc}{section}{\texorpdfstring{p77:問3}{p77:問3}}

\begin{tanswer}
  $m \in \mathbb{N}$とする.
  \begin{enumerate}[(I)]
    \item $n=2m$とかけるとき,この置換を互換の積で表すと,
          \[
            (1,2m)(2,2m-1) \dotsm (m,m+1)
          \]
          となるため,置換の符号は$(-1)^m$,すなわち
          \[
            (-1)^{\frac{n}{2}}
          \]
          となる.
    \item $n=2m-1$とかけるとき,この置換を互換の積で表すと,
          \[
            (1,2m-1)(2,2m-2) \dotsm (m-1,m+1)
          \]
          となるため,置換の符号は$(-1)^{m-1}$,すなわち
          \[
            (-1)^{\frac{n-1}{2}}
          \]
          となる.
  \end{enumerate}
\end{tanswer}




\section*{p79:問}
\addcontentsline{toc}{section}{\texorpdfstring{p79:問}{p77:問}}


\subsection*{p79:問-(イ)}
\addcontentsline{toc}{subsection}{\texorpdfstring{p79:問-(イ)}{p77:問-(イ)}}

\begin{tanswer}
  \[
    \sigma = \begin{pmatrix} 1 & 2 & \cdots & n \\ n & n-1 & \cdots & 1 \end{pmatrix}
  \]
  とすると,$m \in \mathbb{N}$として,
  \[
    \sgn \sigma =
    \begin{cases}
      (-1)^\frac{n}{2}   & \text{($n=2m$のとき)}   \\
      (-1)^\frac{n-1}{2} & \text{($n=2m-1$のとき)}
    \end{cases}
  \]
  となる.
  また,
  \[
    \text{(与式)}  = \sum_{\sigma \in S_n} \sgn \sigma \cdot a_1 a_2 \dotsm a_n
  \]
  だから,
  \[
    \text{(与式)}  =
    \begin{cases}
      (-1)^\frac{n}{2}   a_1 a_2 \dotsm a_n & \text{($n=2m$のとき)}   \\
      (-1)^\frac{n-1}{2} a_1 a_2 \dotsm a_n & \text{($n=2m-1$のとき)}
    \end{cases}
  \]
  である.
\end{tanswer}

\subsection*{p79:問-(ロ)}
\addcontentsline{toc}{subsection}{\texorpdfstring{p79:問-(ロ)}{p77:問-(ロ)}}

\begin{tanswer}
  計算すると,
  \begin{align*}
    \text{(与式)} & = a^3 + b^3 + c^3 -abc -bca -cab \\
                & = a^3 + b^3 +c^3 -3abc
  \end{align*}
  となる.
\end{tanswer}



\section*{p83:問}
\addcontentsline{toc}{section}{\texorpdfstring{p83:問}{p83:問}}

\begin{tproof}
  $(n,n)$行列$A$,$X$を
  \[
    A = (\bm{a}_1 ,\bm{a}_2, \dots ,\bm{a}_n) , \quad X = (\bm{x}_1,\bm{x}_2,\dots,\bm{x}_n)
  \]
  とする.このとき,$AX$は定義され,
  \[
    AX = (A\bm{x}_1 , A\bm{x}_2 , \dots ,A\bm{x}_n)
  \]
  と表せる.ここで,$A \bm{x}_j$を単位ベクトルの線型結合で表すと,
  \begin{align*}
    A \bm{x}_j & = A {}^t (x_{1j} \bm{e}_1 , x_{2j} \bm{e}_2,\ldots ,x_{nj} \bm{e}_n ) \\
               & =A (x_{1j} \bm{e}_1 + x_{2j} \bm{e}_2+ \dots + x_{nj} \bm{e}_n)       \\
               & = x_{1j} \bm{a}_1 + x_{2j} \bm{a}_2 + x_{nj} \bm{a}_n                 \\
               & = \sum_{ i =1}^{n} x_{ij} \bm{a}_{i}
  \end{align*}
  となる.これにより,$\abs{AX}$は,多重線型性を用いて,
  \begin{align*}
    \abs{AX} & = \abs{\sum_{i_1 =1}^{n} x_{i_1 1} \bm{a}_{i_1} , \sum_{i_2 =1}^{n} x_{i_2 2} \bm{a}_{i_2},\ldots , \sum_{i_n =1}^{n} x_{i_n n} \bm{a}_{i_n} }        \\
             & = \sum_{i_1 =1}^{n} \sum_{i_2 = 1}^{n} \dots \sum_{i_n =1}^{n} x_{i_1 1} x_{i_2 2} \dots x_{i_n n} \abs{\bm{a}_{i_1},\bm{a}_{i_2},\dots,\bm{a}_{i_n}}
  \end{align*}
  と変形できる.ここで,
  \[
    \sigma = \begin{pmatrix} 1 & 2 & \cdots & n \\ i_1 & i_2 & \cdots & i_n \end{pmatrix}
  \]
  とおくと,
  \[
    \abs{\bm{a}_{\sigma (1)},\bm{a}_{\sigma(2)},\dots,\bm{a}_{\sigma(n)}} = \sgn \sigma \abs{A}
  \]
  \begin{align*}
    \abs{AX} = & \sum_{\sigma \in S_n} x_{\sigma (1)1} x_{\sigma(2)2} \dots x_{\sigma(n)n} \cdot \sgn \sigma  \abs{A}   \\
               & = \sum_{\sigma \in S_n } \sgn \sigma \cdot x_{\sigma(1)1} x_{\sigma(2)2} \dots x_{\sigma (n)n} \abs{A} \\
               & = \abs{{}^t X} \abs{A}                                                                                 \\
               & = \abs{A} \abs{X}
  \end{align*}
  を得る.これが証明すべきことであった.
\end{tproof}



\section*{p83:問}
\addcontentsline{toc}{section}{\texorpdfstring{p83:問}{p83:問}}

\subsection*{p83:問-(イ)}
\addcontentsline{toc}{subsection}{\texorpdfstring{p83:問-(イ)}{p83:問-(イ)}}

\begin{tanswer}
  多重線型性などを用いて変形すると,
  \begin{align*}
    (\text{与式}) & = -
    \begin{vmatrix}
      2  & -5 & 3  & 10 \\
      1  & 0  & 2  & -3 \\
      5  & 3  & -2 & 2  \\
      -3 & -2 & 4  & 2
    \end{vmatrix}
    = -
    \begin{vmatrix}
      0 & -5 & -1  & 16 \\
      1 & 0  & 2   & -3 \\
      0 & 3  & -12 & 17 \\
      0 & -2 & 10  & -7
    \end{vmatrix}
    =
    \begin{vmatrix}
      1 & 0  & 2   & -3 \\
      0 & -5 & -1  & 16 \\
      0 & 3  & -12 & 17 \\
      0 & -2 & 10  & -7
    \end{vmatrix}                  \\
                & = 1 \cdot (-1)^{1+1}
    \begin{vmatrix}
      -5 & -1  & 16 \\
      3  & -12 & 17 \\
      -2 & 10  & -7
    \end{vmatrix}
    =539
  \end{align*}
  となるので,この行列式の値は$539$である.
\end{tanswer}


\subsection*{p83:問-(ロ)}
\addcontentsline{toc}{subsection}{\texorpdfstring{p83:問-(ロ)}{p83:問-(ロ)}}

\begin{tanswer}
  多重線型性などを用いて変形すると,
  \begin{align*}
    (\text{与式}) & = -
    \begin{vmatrix}
      2 & 3  & 5  & -4 \\
      1 & -7 & -8 & 6  \\
      3 & 10 & 6  & 1  \\
      5 & 2  & 4  & 3
    \end{vmatrix}
    = -
    \begin{vmatrix}
      0 & 17 & 21 & -16 \\
      1 & -7 & -8 & 6   \\
      0 & 31 & 30 & -17 \\
      0 & 37 & 44 & -27
    \end{vmatrix}
    =
    \begin{vmatrix}
      1 & -7 & -8 & 6   \\
      0 & 17 & 21 & -16 \\
      0 & 31 & 30 & -17 \\
      0 & 37 & 44 & -27
    \end{vmatrix}
    \\
                & = 1 \cdot (-1)^{1+1}
    \begin{vmatrix}
      17 & 21 & -16 \\
      31 & 30 & -17 \\
      37 & 44 & -27
    \end{vmatrix}
  \end{align*}
  となる.ここで,第$2$列に第$1$列の$-1$倍を加え,第$3$列に第$1$列を加えると,
  \[
    (\text{与式}) =
    \begin{vmatrix}
      17 & 4  & 1  \\
      31 & -1 & 14 \\
      37 & 7  & 10
    \end{vmatrix}
  \]
  を得る.ここで,第$1$列に第$3$列の$-2$倍を加えると,
  \[
    (\text{与式})=
    \begin{vmatrix}
      15 & 4  & 1  \\
      3  & -1 & 14 \\
      17 & 7  & 10
    \end{vmatrix}
    = -750
  \]
  となるため,この行列式の値は$-750$である.
\end{tanswer}


\part*{第3章・章末問題}
\addcontentsline{toc}{part}{\texorpdfstring{第3章・章末問題}{第3章・章末問題}}


\section*{p90--91:1}
\addcontentsline{toc}{section}{\texorpdfstring{p90--91:1}{p90--91:1}}


\subsection*{p90--91:1-(イ)}
\addcontentsline{toc}{subsection}{\texorpdfstring{p90--91:1-(イ)}{p90--91:1-(イ)}}

\begin{tanswer}
  $ k \in \{ 2,3 ,\ldots,n \}$として,第$1$列に第$k$列の$x^k$倍を加えると,
  \[
    \begin{vmatrix}
      0                                              & -1      & 0       & \cdots & 0      & 0      \\
      0                                              & x       & -1      & \cdots & 0      & 0      \\
      0                                              & 0       & x       & \cdots & 0      & 0      \\
      \vdots                                         & \vdots  & \vdots  & \ddots & \vdots & \vdots \\
      0                                              & 0       & 0       & \cdots & x      & -1     \\
      a_n + a_{n-1} x + \dots +a_1 x^{n-1} + a_0 x^n & a_{n-1} & a_{n-2} & \cdots & a_1    & a_0
    \end{vmatrix}.
  \]
  第$1$列で余因子展開すると,
  \begin{align*}
    (\text{与式}) & = (-1)^{n+2} (a_n + a_{n-1} x + \dots +a_1 x^{n-1} + a_0 x^n)
    \begin{vmatrix}
      -1     & 0      & \cdots & 0           \\
      x      & -1     & \cdots & 0           \\
      0      & x      & \cdots & 0           \\
      \vdots & \vdots & \ddots & \vdots &    \\
      0      & 0      & \cdots & x      & -1 \\
    \end{vmatrix}                                                                                                                           \\
                & = (-1)^{n+2}(a_n + a_{n-1} x + \dots +a_1 x^{n-1} + a_0 x^n) \cdot (-1)^{n-2} \begin{vmatrix} -1 & 0 \\ x & -1 \end{vmatrix} \quad (\text{$\ast$}) \\
                & = (-1)^{n+2} (a_n + a_{n-1} x + \dots +a_1 x^{n-1} + a_0 x^n) \cdot (-1)^{n-2}                                                                     \\
                & = (-1)^{2n} (a_n + a_{n-1} x + \dots +a_1 x^{n-1} + a_0 x^n)                                                                                       \\
                & = a_n + a_{n-1} x + \dots +a_1 x^{n-1} + a_0 x^n
  \end{align*}
  となる.ただし($\ast$)では同様の余因子展開を繰り返した.

  以上の計算により,
  \[
    (\text{与式}) = a_n + a_{n-1} x + \dots +a_1 x^{n-1} + a_0 x^n.
  \]
\end{tanswer}


\subsection*{p90--91:1-(ロ)}
\addcontentsline{toc}{subsection}{\texorpdfstring{p90--91:1-(ロ)}{p90--91:1-(ロ)}}

\begin{tanswer}
  与式の第$2$列から第$n$列までを第$1$列に足すと,
  \[
    \begin{vmatrix}
      x+\sum_{k=1}^{n} a_k & a_1    & a_2     & \cdots & a_n    \\
      x+\sum_{k=1}^{n} a_k & x      & a_2     & \cdots & a_n    \\
      x+\sum_{k=1}^{n} a_k & a_2    & x       & \cdots & a_n    \\
      \vdots               & \vdots & \vdots  & \ddots & \vdots \\
      x+\sum_{k=1}^{n} a_k & a_n    & a_{n-1} & \cdots & x
    \end{vmatrix}
  \]
  である.第$2$行から第$n$行のそれぞれから第$1$行を引くと,
  \[
    \begin{vmatrix}
      x+\sum_{k=1}^{n} a_k & a_1      & a_2         & \cdots & a_n    \\
      0                    & x-a_1    & 0           & \cdots & 0      \\
      0                    & a_2 -a_1 & x-a_2       & \cdots & 0      \\
      \vdots               & \vdots   & \vdots      & \ddots & \vdots \\
      0                    & a_n-a_1  & a_{n-1}-a_2 & \cdots & x-a_n
    \end{vmatrix}
  \]
  である.第$1$列で余因子展開すると,
  \begin{align*}
     &
    (x+\sum_{k=1}^{n} a_k)
    \begin{vmatrix}
      x-a_1    & 0           & \cdots & 0      \\
      a_2 -a_1 & x-a_2       & \cdots & 0      \\
      \vdots   & \vdots      & \ddots & \vdots \\
      a_n-a_1  & a_{n-1}-a_2 & \cdots & x-a_n
    \end{vmatrix}
    \\
     & = (x+\sum_{k=1}^{n} a_k) (x-a_1)(x-a_2) \dotsm (x-a_n) \\
     & = (x+\sum_{k=1}^{n} a_k) \prod_{k=1}^{n} (x-a_k).
  \end{align*}
\end{tanswer}


\subsection*{p90--91:1-(ハ)}
\addcontentsline{toc}{subsection}{\texorpdfstring{p90--91:1-(ハ)}{p90--91:1-(ハ)}}

\begin{tanswer}
  この形の$n\times n $行列を$A_n$とすると,
  \begin{align*}
    A_{n+2} & =
    \begin{vmatrix}
      1+x^2  & x      & 0      & \cdots & 0      \\
      x      & 1+x^2  & x      & \cdots & 0      \\
      0      & x      & 1+x^2  & \cdots & 0      \\
      \vdots & \vdots & \vdots & \ddots & \vdots \\
      0      & 0      & 0      & \cdots & 1+x^2
    \end{vmatrix} \\
            & = (1+x^2)
    \begin{vmatrix}
      1+x^2  & x      & \cdots & 0      \\
      x      & 1+x^2  & \cdots & 0      \\
      \vdots & \vdots & \ddots & \vdots \\
      0      & 0      & \cdots & 1+x^2
    \end{vmatrix}
    -x \begin{vmatrix}
         x      & 0      & \cdots & 0      \\
         x      & 1+x^2  & \cdots & 0      \\
         \vdots & \vdots & \ddots & \vdots \\
         0      & 0      & \cdots & 1+x^2
       \end{vmatrix}
    \\
            & = (1+x^2) A_{n+1} -x^2 A_n
  \end{align*}
  となる.

  よって,
  \[
    A_{n+2}-A_{n+1}=x^2 (A_{n+1}-A_n)\quad ( n \geqq 2) .
  \]
  これと$ A_1=1+x^2$,$ A_2 = 1+x^2 + x^4$により,$n \geqq 2$のとき

  \begin{align*}
    A_{n+1}-A_{n} & = x^2 (A_n-A_{n-1})                   \\
                  & = x^4 (A_{n-1} - A_{n-2})             \\
                  & = \dots = x^{2(n-1)}(A_2-A_1)         \\
                  & = x^{2(n-1)} ((1+x^2 + x^4) -(1+x^2)) \\
                  & = x^{2(n+1)}
  \end{align*}
  であるから,$n \geqq 2$のとき,
  \begin{align*}
    A_n & = A_1 +\sum_{k=1}^{n-1} x^{2k+2}                                \\
        & = 1+x^2 + \frac{x^4(1-x^{2(n-1)})}{1-x^2}                       \\
        & = 1+x^2 + \frac{x^4 (1-x^2)(1+x^2 + \dots + x^{2(n-2)})}{1-x^2} \\
        & = 1+x^2 + x^4 + x^6 + \dots + x^{2n}
  \end{align*}
  となる.この$ A_n$を用いると,$ A_1 = 1+x^2 $,$ A_2 = 1+x^2+x^4$であるから,
  与えられた行列式の値は,$n \times n$行列の場合
  \[
    1+x^2 + x^4 + x^6 + \dots + x^{2n}
  \]
  である.
\end{tanswer}



\subsection*{p90--91:1-(ニ)}
\addcontentsline{toc}{subsection}{\texorpdfstring{p90--91:1-(ニ)}{p90--91:1-(ニ)}}

\begin{tanswer}
  第$1$列で余因子展開すると
  \begin{align*}
    (\text{与式}) & = -a^2\begin{vmatrix} a^2 & b^2 & 1 \\ c^2 & 0 & 1 \\ 1 & 1 & 0 \end{vmatrix}+b^2 \begin{vmatrix} a^2 & b^2 & 1 \\ 0 & c^2 & 1 \\ 1 & 1 & 0 \end{vmatrix} - \begin{vmatrix} a^2 & b^2 & 1 \\ 0 & c^2 & 1 \\ c^2 & 0 & 1 \end{vmatrix} \\
                & = -a^2 (b^2+c^2 -a^2) +b^2 (b^2-c^2-a^2)- (a^2c^2+b^2c^2-c^4)                                                                                                                                                                         \\
                & =a^4+b^4+c^4 -2a^2b^2 - 2b^2 c^2 -2c^2 a^2 .
  \end{align*}
\end{tanswer}



\section*{p90--91:2}
\addcontentsline{toc}{section}{\texorpdfstring{p90--91:2}{p90--91:2}}

\subsection*{p90--91:2-(イ)}
\addcontentsline{toc}{subsection}{\texorpdfstring{p90--91:2-(イ)}{p90--91:2-(イ)}}

\begin{tproof}
  余因子展開を用いると,
  \begin{align*}
    (\text{与式}) & = -a \begin{vmatrix} a & b & c \\ -d & 0 &f \\ -e & -f & 0 \end{vmatrix} +b \begin{vmatrix} a & b & c \\ 0 & d & e \\ -e & -f & 0 \end{vmatrix} -c \begin{vmatrix} a & b & c \\ 0 & d & e \\ -d & 0 & f \end{vmatrix} \\
                & = -a (-cdf +bfe-af^2)+b(be^2-cde-adf)-c(-adf+bde-cd^2)                                                                                                                                                                \\
                & = af (cd-be+af) -be(-be+cd+af) +cf(af-be-cd)                                                                                                                                                                          \\
                & = (af-be+cd)^2.
  \end{align*}
  となり,これが証明すべきことであった.
\end{tproof}

\subsection*{p90--91:2-(ロ)}
\addcontentsline{toc}{subsection}{\texorpdfstring{p90--91:2-(ロ)}{p90--91:2-(ロ)}}

\begin{tproof}
  $A$を$n$次行列とする.${}^t A = -A$であるから,
  \[
    \abs{{}^t A}=(-1)^n\abs{A}.
  \]
  ここで,$n$は奇数であるから,
  \[
    \abs{{}^t A}=-\abs{A}.
  \]
  また,行列式の転置に関する不変性により,$\abs{{}^t A}=\abs{A}$なので,
  \begin{align*}
     & \abs{A}=-\abs{A} ,     \\
     & \therefore ~ \abs{A}=0
  \end{align*}
  となり,これが証明すべきことであった.
\end{tproof}


\section*{p90--91:3}
\addcontentsline{toc}{section}{\texorpdfstring{p90--91:3}{p90--91:3}}


\subsection*{p90--91:3-(イ)}
\addcontentsline{toc}{subsection}{\texorpdfstring{p90--91:3-(イ)}{p90--91:3-(イ)}}

\begin{tproof}
  与えられた行列式に対して多重線型性を用いると,
  \begin{align*}
    \text{(与式)} & =
    \begin{vmatrix}
      A+B & A+B \\
      B   & A
    \end{vmatrix}
    \\
                & = \begin{vmatrix}
                      A+B & O   \\
                      B   & A-B
                    \end{vmatrix}
    \\
                & = \abs{A+B} \cdot \abs{A-B}
  \end{align*}
  となり,これが証明すべきことであった.
\end{tproof}

\subsection*{p90--91:3-(ロ)}
\addcontentsline{toc}{subsection}{\texorpdfstring{p90--91:3-(ロ)}{p90--91:3-(ロ)}}
\begin{tanswer}
  与えられた行列式に対して多重線型性を用いると,
  \begin{align*}
    \text{(与式)} & =
    \begin{vmatrix}
      A+iB & iA-B \\
      B    & A
    \end{vmatrix}
    \\
                & = \begin{vmatrix}
                      A+iB & O    \\
                      B    & A-iB
                    \end{vmatrix}
    \\
                & = \det (A+iB) \cdot \det(A-iB)
  \end{align*}
  となり,いま$A$,$B$は実行列なので,
  \begin{align*}
    \det (A+iB) \cdot \det(A-iB) & = \det (A+iB) \cdot \overline{\det (A+iB)} \\
                                 & = \abs{\det(A+iB)}^2
  \end{align*}
  である.
\end{tanswer}




\section*{p90--91:4}
\addcontentsline{toc}{section}{\texorpdfstring{p90--91:4}{p90--91:4}}

\begin{tproof}
  $\alpha ^n =1$をみたす$\alpha \in \mathbb{C}$をひとつ固定する.
  さて,与えられた行列式の第$j$行を$\alpha^{j-1}$倍して第$1$列に足す操作を行うと,この行列式は
  \[
    \begin{vmatrix}
      \sum_{i=0}^{n-1} \alpha^i x_i              & x_1     & x_2     & \cdots & x_{n-1} \\
      \alpha \sum_{i=0}^{n-1} \alpha^i x_i       & x_0     & x_1     & \cdots & x_{n-2} \\
      \alpha^2 \sum_{i=0}^{n-1} \alpha^i x_i     & x_{n-1} & x_0     & \cdots & x_{n-3} \\
      \vdots                                     & \vdots  & \vdots  & \vdots & \vdots  \\
      \alpha^{n-1} \sum_{i=0}^{n-1} \alpha^i x_i & x_{n-2} & x_{n-3} & \cdots & x_0
    \end{vmatrix}
  \]
  と変形できる.よって,この行列式は
  \[
    \sum_{i=0}^{n-1} \alpha^i x_i = x_0 + \alpha x_1 + \alpha^2 x_2 + \dots +\alpha^{n-1} x_{n-1}
  \]
  を因数にもつ.すべての$\alpha$に関してこのことがいえるから,因数定理により,この行列式は
  \[
    \prod_{\alpha^n=1} (x_0 + \alpha x_1 + \alpha^2 x_2 + \dots +\alpha^{n-1} x_{n-1})
  \]
  を因数にもつ.これは$n$次式であり,なおかつ$x_0$の係数は$1$であることより,結果として
  \[
    \begin{vmatrix}
      x_0     & x_1    & x_2    & \cdots & x_{n-1} \\
      x_{n-1} & x_0    & x_1    & \cdots & x_{n-2} \\
      \vdots  & \vdots & \vdots & \vdots & \vdots  \\
      x_1     & x_2    & x_3    & \cdots & x_0
    \end{vmatrix}
    =  \prod_{\alpha^n=1} (x_0 + \alpha x_1 + \alpha^2 x_2 + \dots +\alpha^{n-1} x_{n-1})
  \]
  である.これが証明すべきことであった.
\end{tproof}


\section*{p90--91:5}
\addcontentsline{toc}{section}{\texorpdfstring{p90--91:5}{p90--91:5}}

\begin{tanswer}
  前問において,$n=4$,$x_1 = i$,$x_2 = 1$,$x_3=-i$とした場合を考えればよいので,$\alpha = \pm 1 , \pm i$により,
  \begin{align*}
    \text{(与式)} & = \prod_{\alpha^4=1} (x+ \alpha i +\alpha^2 -  \alpha^3i ) \\
                & = (x+i+1-i) (x-i+1+i)(x-1-1-1)(x+1-1+1)                    \\
                & =(x+1)^3(x-3)
  \end{align*}
  となる.
\end{tanswer}



\section*{p90--91:6}
\addcontentsline{toc}{section}{\texorpdfstring{p90--91:6}{p90--91:6}}

\begin{tproof}
  $i \in \{ 1,2,\ldots,n \}$のもとで,$n$個の点を$(x_i , y_i ) \in \mathbb{R}^2$とする.このとき,
  \[
    \begin{cases}
      a_0 + a_1 x_1 + a_2 {x_1}^2 + \dots + a_n {x_1}^{n-1} = y_1 \\
      a_0 + a_1 x_2 + a_2 {x_2}^2 + \dots + a_n {x_2}^{n-1} = y_2 \\
      \vdots                                                      \\
      a_0 + a_1 x_n + a_2 {x_n}^2 + \dots + a_n {x_n}^{n-1} = y_3
    \end{cases}
  \]
  である,これを行列の形に表すと,
  \[
    \begin{pmatrix}
      1      & x_1    & {x_1}^2 & \cdots & {x_1}^{n-1} \\
      1      & x_2    & {x_2}^2 & \cdots & {x_2}^{n-1} \\
      \vdots & \vdots & \vdots  & \ddots & \vdots      \\
      1      & x_n    & {x_n}^2 & \cdots & {x_n}^{n-1}
    \end{pmatrix}
    \begin{pmatrix}
      a_0    \\
      a_1    \\
      \vdots \\
      a_n
    \end{pmatrix}
    =
    \begin{pmatrix}
      y_1    \\
      y_2    \\
      \vdots \\
      y_n
    \end{pmatrix}
  \]
  となる.

  ここで,
  \[
    A \coloneqq
    \begin{pmatrix}
      1      & x_1    & {x_1}^2 & \cdots & {x_1}^{n-1} \\
      1      & x_2    & {x_2}^2 & \cdots & {x_2}^{n-1} \\
      \vdots & \vdots & \vdots  & \ddots & \vdots      \\
      1      & x_n    & {x_n}^2 & \cdots & {x_n}^{n-1}
    \end{pmatrix}
  \]
  とおくと,$\abs{{}^t A}$はヴァンデルモンドの行列式である.

  行列式の値は,行列の転置に対して不変なので,
  \[
    \abs{A}= \prod_{i < j} (x_j - x_i)
  \]
  となり,条件によりこの値は$0$でない.ゆえに先の連立方程式はただ一つの解をもつ.

  以上の考察によって,これら$n$個の点を通る直線がただ一つ存在することが示された.
\end{tproof}



\section*{p90--91:7}
\addcontentsline{toc}{section}{\texorpdfstring{p90--91:7}{p90--91:7}}


\begin{tanswer}
  与えられた行列式の係数行列式を$A$, $A$の第$j$列を${}^t (1, 0, 0 ,0 )$で置き換えた行列を$A_j$とする.
  また,
  \[
    T = \begin{pmatrix} a & -b\\ b & a \end{pmatrix},\quad S = \begin{pmatrix} c & -d\\ d & c \end{pmatrix}
  \]
  とする.このとき,
  \begin{align*}
    \begin{vmatrix} T & -T \\  S & S \end{vmatrix} & = \begin{vmatrix} 2T & - T \\ O & S \end{vmatrix} \\
                                                   & = \abs{2T} \abs{S}                                \\
                                                   & = 4(a^2+b^2) (c^2+d^2).
  \end{align*}
  クラメールの公式により
  \[
    \begin{cases}
      x  = \abs{A_1}/\abs{A}, \\
      y = \abs{A_2}/\abs{A},  \\
      z = \abs{A_3}/\abs{A},  \\
      u = \abs{A_4}/\abs{A}.
    \end{cases}
  \]
  さて,
  \begin{align*}
    \abs{A_1} & = \begin{vmatrix} 1 & -b & -a & b \\ 0 & a & -b & -a \\ 0 & -d & c & d \\ 0 & c & d & c \end{vmatrix} \\
              & = 1 \cdot \begin{vmatrix} a & -b & -a \\ -d & c & d \\ c & d & c \end{vmatrix}                        \\
              & = 2a(c^2+d^2).
  \end{align*}
  よって,
  \[
    x = \frac{\abs{A_1}}{\abs{A}} = \frac{2a(c^2+d^2)}{4(a^2+b^2)(c^2+d^2)} = \frac{a}{2(a^2+b^2)}.
  \]
  同様にして$y$,$z$,$u$を求めると
  \[
    x= \frac{a}{2(a^2+b^2)},\quad y = -\frac{b}{2(a^2+b^2)},\quad z = -\frac{a}{2(a^2+a^2)},\quad u = \frac{b}{2(a^2+b^2)}.
  \]
\end{tanswer}




\section*{p90--91:8}
\addcontentsline{toc}{section}{\texorpdfstring{p90--91:8}{p90--91:8}}

\begin{tanswer}
  与えられた行列式は,
  \[
    \begin{cases}
      a_1 x + b_1 y =c_1 \\
      a_2 x + b_2 y =c_2 \\
      a_3 x + b_3 y =c_3
    \end{cases}
  \]
  の拡大係数行列の行列式を表す.基本変形を施すと,この行列式は
  \begin{align}
     & \begin{pmatrix}
         1 & 0 & g_1 \\
         0 & 1 & g_2 \\
         0 & 0 & g_3
       \end{pmatrix}
    \label{eq:8-0}
    ,                 \\
     & \begin{pmatrix}
         1 & t & h_1 \\
         0 & 0 & h_2 \\
         0 & 0 & h_3
       \end{pmatrix}
    \label{eq:8-1}
  \end{align}
  の場合に変形できる.

  \begin{enumerate}[(I)]
    \item \eqref{eq:8-0}の場合,行列式が$0$となる条件は$g_3 =0$である,このとき,上の連立方程式の解$(x,y)$は存在し一意に定まる.これは$3$直線が$1$点で交わることを表す.
    \item \eqref{eq:8-1}の場合,行列式は常に$0$であり,このとき,3直線はすべて平行であるか一致するかである.
  \end{enumerate}

  以上の考察により,与えられた行列式が$0$であるのは
  \begin{enumerate}[(i)]
    \item 3直線が$1$点で交わる
    \item 3直線が平行である
    \item 3直線が一致する
  \end{enumerate}
  のいずれかの場合である.
\end{tanswer}





\section*{p90--91:9}
\addcontentsline{toc}{section}{\texorpdfstring{p90--91:9}{p90--91:9}}

\begin{tproof}
  3点$\mathrm{P}_1$,$\mathrm{P_2}$,$\mathrm{P_3}$を通る平面の方程式を$ax+by+cz+d=0$とおく.
  このとき,
  \[
    \begin{cases}
      ax+by+cz + d =0         \\
      ax_1 + by_1 +cz_1 +d =0 \\
      ax_2 + by_2 +cz_2 +d =0 \\
      ax_3 + by_3 +cz_3 +d =0
    \end{cases}
  \]
  が成立する.すなわちこれは
  \[
    \begin{pmatrix}
      x   & y   & z   & 1 \\
      x_1 & y_1 & z_1 & 1 \\
      x_2 & y_2 & z_2 & 1 \\
      x_3 & y_3 & z_3 & 1
    \end{pmatrix}
    \begin{pmatrix}
      a \\
      b \\
      c \\
      d
    \end{pmatrix}
    = \bm{0}
  \]
  をみたす.これを$a$,$b$,$c$,$d$についての連立方程式とみたとき,与条件により自明でない解があり,
  \[
    \begin{vmatrix}
      x   & y   & z   & 1 \\
      x_1 & y_1 & z_1 & 1 \\
      x_2 & y_2 & z_2 & 1 \\
      x_3 & y_3 & z_3 & 1
    \end{vmatrix}
    =0
  \]
  が成立する.これが証明すべきことであった.
\end{tproof}



\section*{p90--91:10}
\addcontentsline{toc}{section}{\texorpdfstring{p90--91:10}{p90--91:10}}

\begin{tproof}
  必要性・十分性をそれぞれ証明する.
  \begin{enumerate}
    \item $A$が正則かつ$A^{-1}$が整数行列であると仮定し,$\det A=\pm 1$であることを示す.

          $A$は整数行列であり,その行列式は,各要素の和と積でかけているから$\det A \in \mathbb{Z}$である.同様にして$\det (A^{-1}) \in \mathbb{Z}$である.逆行列の行列式は,
          \[
            \det (A^{-1})=\frac{1}{\det A}
          \]
          であり,つまり$\det A,1/\det A \in \mathbb{Z}$である.これを満たす整数は$\pm 1$だけである.
    \item $\det A=\pm 1$であることを仮定し,$A$が正則かつ$A^{-1}$が整数行列であることを示す.
          $\det A \neq 0$より$A$の正則性がわかる.また,$A$の余因子行列を$\tilde{A}$とすると,余因子はAの各要素の和と積によって表現される.つまり,余因子は整数であるから$\tilde{A}$は整数行列である. また
          \[
            A^{-1}=\frac{1}{\det A}\tilde{A}
          \]
          となる.$\det A=\pm 1$であり, 余因子は整数であるから,$A^{-1}$は整数行列である.
  \end{enumerate}
  以上の議論により証明された.
\end{tproof}



\section*{p90--91:11}
\addcontentsline{toc}{section}{\texorpdfstring{p90--91:11}{p90--91:11}}

\subsection*{p90--91:11-(イ)}
\addcontentsline{toc}{subsection}{\texorpdfstring{p90--91:11-(イ)}{p90--91:11-(イ)}}

\begin{tproof}
  ${}^t A_\sigma$は$(j,\sigma(j))$成分が$1$でそれ以外が$0$である行列である.
  いま$A = (\bm{a}_1,\bm{a}_2,\ldots,\bm{a}_n)$とすると,
  \begin{align*}
    {}^t A_\sigma A_\sigma & =
    \begin{pmatrix}
      (\bm{a}_1,\bm{a}_1) & (\bm{a}_1,\bm{a}_2) & \cdots & (\bm{a}_1,\bm{a}_n) \\
      (\bm{a}_2,\bm{a}_1) & (\bm{a}_2,\bm{a}_2) & \cdots & (\bm{a}_2,\bm{a}_n) \\
      \vdots              & \vdots              & \ddots & \vdots              \\
      (\bm{a}_n,\bm{a}_1) & (\bm{a}_n,\bm{a}_2) & \cdots & (\bm{a}_n,\bm{a}_n)
    \end{pmatrix}
    \\
                           & = \begin{pmatrix} 1 & 0 & \cdots & 0 \\ 0 & 1 & \cdots & 0 \\ \vdots & \vdots & \ddots & \vdots \\ 0 & 0 & \cdots & 1 \end{pmatrix} \\
                           & = E
  \end{align*}
  となり,$A$は直交行列である.
\end{tproof}

\subsection*{p90--91:11-(ロ)}
\addcontentsline{toc}{subsection}{\texorpdfstring{p90--91:11-(ロ)}{p90--91:11-(ロ)}}

\begin{tproof}
  置換$\sigma$,$\tau$に関して
  \[
    a_{ij} =
    \begin{cases}
      1 & \text{if } i = \sigma(j) \\
      0 & \text{otherwise}
    \end{cases},
    \quad
    b_{ij}=
    \begin{cases}
      1 & \text{if } i = \tau(j) \\
      0 & \text{otherwise}
    \end{cases}
  \]
  と定義する.
  このとき,
  \[
    (A_\sigma A_\tau)_{ik} = \sum_{j=1}^n a_{ij} b_{jk}.
  \]
  $b_{jk}$が$1$になるのは$j= \tau(k)$のときなので,
  \[
    (A_\sigma A_\tau)_{ik} = a_{i, \tau(k)}.
  \]
  さらに,$a_{i, \tau(k)}$が$1$になるのは$i = \sigma(\tau(k))$のときなので,
  \[
    (A_\sigma A_\tau)_{ik} =
    \begin{cases}
      1 & \text{if } i=\sigma \tau(k) \\
      0 & \text{otherwise}
    \end{cases}.
  \]
  これは$A_{\sigma \tau}$の定義そのものなので,
  \[
    A_\sigma A_\tau = A_{\sigma \tau}.
  \]
\end{tproof}


\subsection*{p90--91:11-(ハ)}
\addcontentsline{toc}{subsection}{\texorpdfstring{p90--91:11-(ハ)}{p90--91:11-(ハ)}}

\begin{tproof}
  $ \tau = \sigma^{-1}$とおくと,
  \begin{align*}
    A_\sigma & = \sum_{ \tau \in S_n} \sgn  (\tau) a_{1 \tau(1)} a_{2 \tau(2)} \cdots a_{n \tau(n)}               \\
             & = \sum_{ \tau \in S_n} \sgn  (\tau) a_{\tau^{-1}(1)1} a_{\tau^{-1} (2)2} \cdots a_{\tau^{-1} (n)n} \\
             & = \sum_{\sigma \in S_n} \sgn (\sigma^{-1}) a_{\sigma(1)1} a_{\sigma(2)2} \cdots a_{\sigma(n)n}     \\
             & = \sum_{\sigma \in S_n} \sgn (\sigma) a_{\sigma(1)1} a_{\sigma(2)2} \cdots a_{\sigma(n)n}          \\
             & = \sgn (\sigma).
  \end{align*}
  これにより
  \[
    \sgn (\sigma) = \pm 1 \iff A_\sigma = \pm 1.
  \]
\end{tproof}
